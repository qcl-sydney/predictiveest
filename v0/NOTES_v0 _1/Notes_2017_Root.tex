\documentclass[pra, reprint]{revtex4-1}
\usepackage{amssymb,amsmath}
\usepackage{graphicx, import}
% \usepackage{caption}  # not compatible with Revtek
% \usepackage{subcaption}  # not compatible with Revtek
\usepackage{placeins}
\usepackage{hyperref}
\usepackage[nameinlink]{cleveref}

\usepackage{nomencl}
\makenomenclature
% First time - invoke makeindex via commandline: makeindex <filename>.nlo -s nomencl.ist -o <filename>.nls
 

\graphicspath{ {./tex/fig_main/svg2pdf/}{/home/riddhisw/Documents/2017/Scripts_Git/v0/FIGS_v0/}}

% Clever Referencing
\crefname{equation}{Eq.}{Eqs.}
\crefname{figure}{FIG.}{FIGS.}
\crefname{table}{Table}{Tables} 
\crefname{section}{Section}{Sections}
\crefname{chapter}{Chapter}{Chapters}
\crefname{appendix}{Appendix}{Appendices} 

% Define state  and bayes prediction risk
\newcommand{\state}[0]{f} %mathmode only
\newcommand{\normpr}[0]{\text{N.} \langle (\state_n - \hat{\state}_n)^2 \rangle_{f, \mathcal{D}}} %mathmode
 
% Define Dirac and Pauli notation
\newcommand{\ket}[1]{| #1 \rangle} %mathmode only
\newcommand{\bra}[1]{\langle #1 |} %mathmode only 
\newcommand{\op}[1]{\hat{#1}} %mathmode only - operator hat
\newcommand{\p}[1]{\hat{\sigma}_{#1}} %mathmode only - pauli operators
\newcommand{\idn}[0]{\op{\mathcal{I}}} %mathmode only - identity
\newcommand{\pj}[1]{\ket{#1}\bra{#1}} %mathmode only - projection operator
\newcommand{\pjs}[2]{\bra{#1}#2\rangle} %mathmoe only - state projection

% Define commmon linear operators
\newcommand{\dd}[0]{(t)} %mathmode only - time dependence
\newcommand{\dr}[1]{\frac{d #1}{dt}} %mathmode only - time derivative 
\newcommand{\ex}[1]{\mathbb{E}[#1]} %mathmode only - expectation values
\newcommand{\norm}[1]{||#1||} %mathmode only - norm

% Define apriori and aposteriori estimates for $x$ and $P$ in Kalman Filtering
\newcommand{\amx}[1]{\hat{x}_{#1}(-)} %mathmode only - apriori x at index [1]
\newcommand{\apx}[1]{\hat{x}_{#1}(+)} %mathmode only - aposteriori x at index [1]
\newcommand{\amp}[1]{\hat{P}_{#1}(-)} %mathmode only - apriori P at index [1]
\newcommand{\app}[1]{\hat{P}_{#1}(+)} %mathmode only - aposteriori P at index [1]

%Define format for defintions
\newtheorem{defn}{Definition}
\newtheorem{thm}{Theorem} 
\newtheorem{azm}{Assumption} 

% Define paragraph spacing and line spacing
% \setlength{\parindent}{4em}
\setlength{\parskip}{20em} 
% \renewcommand{\baselinestretch}{1.5}

\begin{document}
\title{Machine learning tools for predictive estimation of a single qubit under dephasing noise}

\author{Riddhi Swaroop Gupta} 
\email{riddhi.sw@gmail.com}
\affiliation{ARC Centre of Excellence for Engineered Quantum Systems, School of Physics, The University of Sydney, New South Wales 2006, Australia}

\author{Michael J. Biercuk}
\affiliation{ARC Centre of Excellence for Engineered Quantum Systems, School of Physics, The University of Sydney, New South Wales 2006, Australia}


% \begin{abstract}
% Quantum computing hardware must preserve coherence of quantum systems over long operating procedures. Even if a quantum system is reset repeatedly during a procedure, the presence of non-Markovian environmental noise can correlate observations which would otherwise be independent. 
% In this work, machine learning (ML) procedures extract correlations from measurements performed on a qubit subject to non-Markovian dephasing noise. We numerically investigate the performance of various ML algorithms in predicting qubit state evolution beyond the data. 
% A comparison of achievable prediction horizons, model robustness, and noise filtering capabilities for Kalman Filters (KF) and a Gaussian Process Regression (GPR) algorithm is provided. 
% In the absence of an analytical model describing qubit evolution, stochastic qubit dynamics are represented via autoregressive processes or state-space resonators.
% We find an autoregressive KF is model-robust in contrast to resonator-based KF, and we extend the former to use only binary data. Contrarily, a GPR algorithm using an infinite basis of oscillators enables interpolation but not forward prediction. These results apply to predictive estimation schemes for any two-level system under arbitrary non-Markovian dephasing.
% \end{abstract}

\begin{abstract}
Decoherence remains a major challenge in quantum computing hardware and a variety of physical-layer controls provide opportunities to mitigate the impact of this phenomenon. In particular, laboratory-based systems typically suffer from the presence of non-Markovian noise processes and this opens an opportunity for using feedback and feedforward correction strategies exploiting underlying noise correlations. In this work, we use a numerical record of projective qubit measurements to investigate the performance of various machine learning algorithms in performing state estimation (retrodiction) and forward prediction of future qubit state evolution. Our approaches involve the construction of a dynamical model capturing qubit dynamics via autoregressive or Fourier-type protocols. A comparison of achievable prediction horizons, model robustness, and noise filtering capabilities for Kalman Filters (KF) and a Gaussian Process Regression (GPR) algorithm is provided. We demonstrate superior performance from the autoregressive KF relative to Fourier-based KF approaches. Further, a GPR algorithm with an infinite basis of oscillators permits only retrodiction based on the data but not forward prediction. 
\end{abstract}

\maketitle

% \listoffigures
% \listoftables
% \begin{widetext} 
% \tableofcontents
% 
\section{Figures and Captions}

\begin{figure}[h!]
    \includegraphics[scale=1]{Predive_control_Fig_overview_17_one} 
    \caption{ \label{fig:main:Predive_control_Fig_overview_17_one} Physical Setting: In (a), we define the Hamiltonian for stochastic qubit dynamics under arbitrary environmental dephasing using a covariance stationary, non Markovian detuning $\delta \omega(t)$ with an arbitrary power spectral density. A sequence of Ramsey experiments with fixed wait time $\tau$ yield single shot outcomes $\{ d_n \}$, with likelihood $P(d_n|\state_n, \tau,n)$,  conditioned on a mean-square ergodic sequence of true phases, $\{ f_n\}$, with $n \in [-N_T, 0]$ indexing time during data collection. Our objective is to maximise forward time $n \in [0, N_P]$ for which an algorithm uses measurement data to predict a future qubit state and incurs a lower Bayes prediction risk relative to predicting the mean value of the dephasing noise [dark gray shaded]. In (b), single shot outcomes are processed to yield noisy accumulated phase estimates, $\{ y_n\}$, corrupted by measurement noise $\{v_n\}$. The choice of $\{d_n\}$ or $\{y_n\}$ as datasets for predictive estimation corressponds to non-linear or linear measurement records in (b) and (c).}
\end{figure} 

\clearpage \newpage

\begin{widetext}
\begin{figure*}
    \includegraphics[scale=1]{Predive_control_Fig_overview_17_two} 
    \caption{ \label{fig:main:Predive_control_Fig_overview_17_two} Predictive Methodologies: (a) In GPR, a prior  distribution over true phase sequences $P(\state), \state \equiv \{ \state_n \}$ is constrained by a linear Bayesian likelihood of observed data, $\{ y_n\}$. The prior encodes dephasing noise correlations by defining covariance relations for the $i, j$-th time points using  $\Sigma_\state^{i, j}$ and optimising over its free parameters during training. The moments of the resulting predictive distribution $P(\state^*|y)$ are interpreted as pointwise predictions and their pointwise uncertainties when evaluated for $n>0$.  (b) In KF, the Kalman state and its variance correspond to moments of a Gaussian distribution propagated in time via $\Phi$, and filtered via the Kalman gain, $\gamma$ at timestep $n$. The design of $\Phi$ deterministically colors a white noise process $\{w_n \}$ and `encodes' an apriori structure for learning dephasing noise correlations. Prediction proceeds by propagating forwards with $\gamma_n=0, n>0$. Additive white Gaussian measurement noise $v_n$ corrupts all measurement records.}
\end{figure*}
\end{widetext}

\clearpage \newpage 

\begin{figure} [h]
    \includegraphics[scale=1]{Predive_control_Fig_overview_17_three}
    \caption{\label{Predive_control_Fig_overview_17_three} Apriori Structure for $\Phi$ : All Kalman $\Phi$ variants are mean square approximations to covariance stationary, mean square ergodic $f$. AKF and QKF define $\Phi$ as a weight sum of $q$ past measurements driven by process noise, $w$ [top]. Kalman $\Phi$ for LKFFB represents a collection of $J$ osscilators driven by process noise, $w$, where frequency of oscillators must span dephasing noise bandwidth. The instantaneous amplitude and phase of each basis oscillator can be derived from the Kalman state estimate $x_{j, n}$ at any $n$. Predictions combine learned amplitudes and phases for each basis oscillator and sum contributions over all $J$ [bottom].}
\end{figure}

\clearpage \newpage


\begin{figure}
    \includegraphics[scale=1.]{fig_data_gpr}. 
    \caption{\label{fig:main:fig_data_gpr} In (a)-(d), prediction points $\mu_{\state^*|y}$ [purple] are plotted against time steps, $n$. We plot the true phase sequence,  $f$, [black] and  $f$ at the begining of the run [red dotted]. Predictions are generated in a single run by a trained GPR model with a periodic kernel corressponding to a Fourier domain basis comb spacing, $\omega_0^B$. Data collection of $N_T$ measurements [not shown] ceases at $n=0$. For simplicity, the true $f$ is a deterministic sine with frequency, $\omega_0$. (a) Perfection projection is possible $\omega_0 / \omega_0^B \in Z$ natural numbers, $\omega_0 = 3$ Hz. Kernel resolution is exactly the longest time domain correlation in dataset, $2 \pi / \omega_0^B \equiv \Delta t N_T \implies \kappa = 0$.   (b) Imperfect projection, with $\omega_0 / \omega_0^B \notin Z$, $\omega_0 / 2 \pi = 3 \frac{1}{3}$ Hz, $\kappa=0$. (c) We increase kernel resolution to be arbitrarily high, $\kappa >> 0 $, such that $\omega_0 / \omega_0^B >> 0 \notin Z $ for original $ \omega_0 / 2 \pi = 3$ Hz. (d) We test (b) and (c) for $\kappa >>0$, $ \omega_0 / \omega_0^B \notin Z$, $\omega_0 / 2 \pi = 3 \frac{1}{3}$ Hz. For all (a)-(d), $N_T = 2000, N_P = 150$ steps, $\Delta t = 0.001s$ and applied measurement noise level $1\%$.} 
\end{figure}

\clearpage \newpage

\begin{figure}
    \includegraphics[scale=1.0]{fig_data_all}
    \caption{\label{fig:main:fig_data_all} We plot state predictions against time steps $n > -50$ obtained from trained AKF, LKFFB and LSF algorithms. We plot true $f$ [black] and measurement data [grey dots], where measurements for $n \in [-N_T, -50]$ are omitted [(left)]. A single run contributes to Bayes prediction risk over an ensemble of 50 runs normalised against predicting the mean, $\mu_\state$, of dephasing noise [right]. A normalised risk $<1$ for $n > 0$ defines a desirable forward prediction horizon. A single run phase sequence $f$ is drawn from a flat top spectrum with $J$ true Fourier components spaced $\omega_0$ apart and uniformly randomised phases $\in [0, 2\pi]$. A trained LKFFB is implemented with comb spacing $\omega_0^B / 2\pi = 0.5$ Hz and $J^B =100$ oscillators; while trained AKF / LSF models corresspond to high $q \approx 100$. Relative to LKFFB,  (a) and (b) corresspond to perfect projection $\omega_0 / \omega_0^B  \in Z $ for $J= 40, \omega_0 / 2\pi = 0.5$ Hz. In (c) and (d), we simulate realistic noise with $\omega_0 / \omega_0^B  \notin Z$, $J = 45000$, $\omega_0 / 2\pi = \frac{8}{9} \times 10^{-3}$ Hz such that $>500$ number of true components fall between adjacent LKFFB oscillators. For (a)-(d), $N_T = 2000, N_P = 100$ steps, $\Delta t = 0.001s$ such that we fulfill $r_{Nqy} >> 2$, $N_T / \Delta t < \omega_0/2\pi$. Measurement noise level is $10\%$.}
\end{figure} 

\begin{figure}
    \includegraphics[scale=1.0]{fig_data_specrecon}
    \caption{\label{fig:main:fig_data_specrecon} We compare the true power spectrum for $f$ with derived spectral estimates from LKFFB and AKF. From (a)-(d), we vary true $f$ cutoff relative to an apriori noise bandwidth assumption $f_B$ such that $\omega_0 / 2\pi = 0.5$ Hz, $J = 20, 40, 80, 200$. For LKFFB, we use learned amplitude information from a single run ($\propto ||x^j_n||^2 $) with $\omega_0^B / 2\pi = 0.497$ Hz for $j \in J^B = 100$ oscillators. For AKF, we plot [EQN REF]\cref{eqn:main:ap_ssp_ar_spectden} using optimally trained $\{\phi_{q' \leq q}\}$ and $\sigma^2$, with order $q \approx 100$. The zeroth Fourier component and its estimates are omitted to allow for log scaling; and $N_T = 2000, N_P = 50$ steps, $\Delta t = 0.001s, r_{Nqy}=20$ and measurement noise level is $1\%$.} 
\end{figure} 

\clearpage \newpage

\begin{widetext}
    \begin{figure*} 
    \includegraphics[scale=1.0]{figure_lkffb_path}
    \caption{\label{fig:main:figure_lkffb_path} 
    We compare LKFFB and AKF performance when a true phase sequence $f$ is generated from a flat top spectrum in (a)-(d) by varying  $\omega_0 / 2\pi = 0.5, 0.499, \frac{8}{9} \times 10^{-3}, \frac{8}{9} \times 10^{-3}$ Hz and $J = 80, 80, 45000, 80000$ respectively. For (a)-(d), we depict normalised Bayes prediction risk for LKFFB, AKF, and LSF against time steps $n>0$. For LKFFB, these regimes corresspond to perfect learning in (a); imperfect projection on basis in (b); finite computational Fourier resolution in (c); and a relaxed bandwidth assumption ($f_B < \omega_0 / 2\pi$) in (d). In the panels (e)-(l), we depict optimisation of Kalman noise parameters ($\sigma^2, R$) for LKFFB [top row] and AKF [bottom row] for the four regimes in (a)-(d). Low loss regions represent risk values $< 10\%$ of the median risk incurred during Kalman hyperparameter optimisation for 75 trials of of randomised ($\sigma^2, R$) pairs. Optimal ($\sigma^*, R^*$) minimise state estimation risk. For each trial, a risk point is an expectation over 50 runs of true $f$ and noisy datasets during state estimation ($n \in  [-N_{SE}, 0]$) or prediction ($n \in  [0, N_{PR}]$). We choose $ N_{PR}=N_{SE}=50$ such that the shape of total loss over time steps form sensible optimsation problems over the range of numerical experiments in this paper. A scan of $N_{PR}, N_{SE}$ values do not appear to simplify our Kalman optimisation problem. We plot optimisation results for LKFFB in (e)-(h) and AKF in (i)-(l). A KF filter is `tuned' if optimal ($\sigma^*, R^*$) lies in the overlap of low loss regions for state estimation and prediction. This condition is violated in (h). KF algorithms are set up with $q = 100$ for AKF; $J^B = 100, \omega_0^B / 2\pi = 0.5$ Hz for LKFFB, with $N_T = 2000, N_P = 100$ steps, $\Delta t = 0.001s, r_{Nqy}=20$ and applied measurement noise level $1\%$.}
    \end{figure*} 
\end{widetext}

\clearpage \newpage

\begin{figure}
    \includegraphics[scale=1.]{fig_data_akfvlsf}
    \caption{\label{fig:main:fig_data_akfvlsf} (a) We plot the ratio of normalised Bayes prediction risk from AKF to LSF against time steps $n>0$.  AKF and LSF share identical $\{ \phi_q \}$ and  a value below $<1$ indicates AKF outperforms LSF. In (i)-(iv), applied measurement noise level is increased from $0.1 - 25 \%$, where the noise level is defined as standard deviation of additive Gaussian measurement noise relative to the sample standard deviation of random variables one realisation of true $f$. (b) We plot normalised Bayes Risk against time steps $n>0$ for AKF and LKFFB corressponding to cases (i) -(iv) and confirm a desirable forward prediction horizon underpins ratios in (a). True $f$ is drawn from a flat top spectrum with $\omega_0 / 2\pi = \frac{8}{9} \times 10^{-3}$ Hz, $J = 45000$, $N_T = 2000, N_P = 100$ steps, $\Delta t = 0.001s, r_{Nqy}=20$ such that \cref{fig:main:figure_lkffb_path}(c) corressponds to case (ii) in this figure. }
\end{figure}

\begin{figure}[h!]
    \includegraphics[scale=1.]{fig_data_qkf}
    \caption{\label{fig:main:fig_data_qkf2} We plot Bayes prediction risk for QKF against time steps $n>0$. In (a)-(b), we vary true $f$ cutoff relative to an apriori noise bandwidth assumption such that $J f_0 / f_B = 0.2, 0.4, 0.6, 0.8$ for an initially generated true $f$ in \cref{fig:main:fig_data_specrecon} with $\omega_0/ 2\pi = 0.497 $ Hz, $J = 20, 40, 60, 80$. Measurement noise is incurred on $f$ at $1 \%$ level for the linear measurement record and on $z$ at $1\%$ level corressponding to the non linear measurement record. In (a), we obtain $\{\phi_{q' \leq q}\}, q=100$ coefficients from AKF/LSF acting on a linear measurement record generated from true $f$. We re-generate a new truth, $f'$, from an autoregressive process by setting $\{\phi_{q'\leq q}\}, q=100$ as true coefficents and by defining a known, true $\sigma$. We generate quantised measurements from $f'$ and data is corrupted by measurement noise of a true, known strength $R$. Hence, QKF in (a) incorporates true dynamics and noise parameters $\{\{\phi_{q' \leq q} \}, \sigma, R\}$ but acts on single shot qubit measurements. In (b), we use $\{\phi_{q' \leq q} \}, q=100$ coefficients from (a) but we generate quantised measurements from the original, true $f$. We auto-tune QKF noise design parameters in a focused region ($\sigma_{AKF}^* \leq \sigma_{QKF}$, $R_{AKF}^* \leq R_{QKF}$) . For (a)-(b), forward prediction horizons are shown with $N_T = 2000, N_P = 50$ steps, $\Delta t = 0.001s, r_{Nqy}>> 2$.}
\end{figure}

\clearpage \newpage 



\clearpage \newpage
cite something for latex build \cite{mavadia2017} 
% \section{Introduction} 

Machine learning frameworks are advancing predictive estimation capabilities in diverse fields such as engineering, finance, econometrics, meteorology, seismology, and physics. In predictive estimation, a dynamically evolving system is observed and any temporal correlations encoded in the observations are used to predict the future state of the system. For sophisticated predictive estimation problems, machine learning tools offer unique opportunities to develop better predictors under alternative theoretical frameworks; and /or to leverage computational resources to optimise predictors using large datasets. For classical systems, machine learning techniques have enabled state tracking, control, and forecasting for highly non-linear and noisy dynamical trajectories or complex measurement protocols (e.g. \cite{garcia2016optimal, bach2004learning, tatinati2013hybrid, hall2011reinforcement, hamilton2016ensemble}). Particle-based Bayesian frameworks (e.g. particle filtering, unscented or sigma-point filtering) yield significant advantages in accommodating non-linear models arising from a physical system, system dynamics, or measurement protocols \cite{candy2016bayesian}. Recently, an ensemble of unscented Kalman filters demonstrated state estimation and forward predictions for chaotic, non-linear systems in the absence of a prescribed model and instead used nearest neighbour strategies between `particles' to model dynamics  \cite{hamilton2016ensemble}. For non-chaotic, multi-component stationary random signals, other algorithmic approaches have been particularly useful for tracking instantaneous frequency and phase information (e.g. \cite{boashash1992estimating2, ji2016gradient}) to enable short-run forecasting.  
\\
\\ 
However, it is not straightforward to extend machine learning predictive estimation techniques to non-classical systems with observations consisting of quantum projective measurements. A projective measurement forces a freely evolving quantum system to assume a particular quantum state. In contrast, observing a classical system does not influence its inherent dynamical evolution. Hence, predictive estimation problems using projective measurement records for quantum systems are fundamentally different to their purely classical counterparts. The analysis of projective measurement records is often conceptualised as pattern recognition or image reconstruction problems in machine learning, as examples, in characterising the initial or final state of quantum system (e.g. \cite{struchalin2016experimental, sergeevich2011characterization, mahler2013adaptive}) or reconstructing the historical evolution of a quantum system based on large measurement records (e.g. \cite{stenberg2016characterization, shabani2011efficient, shen2014reconstructing, de2016estimation, tan2015prediction, huang2017neural}). In adaptive or sequential Bayesian learning applications, it is often the case that a projective measurement protocol is designed or adaptively manipulated to efficiently yield noise filtered information about a quantum system (e.g. \cite{bonato2016optimized, wiebe2015bayesian}). Typically, the object of interest is either static or dynamically uncorrelated in time (white) as  measurement protocols are reapplied. Hence, the canonical real-time tracking and prediction problem in classical applications - where a non-linear, stochastic trajectory of a system is tracked using noisy measurements and short-run forecasts are made - is under-explored for quantum systems with projective measurements.
\\ 
\\
%We anticipate that accurate short-run predictions will reduce the number of projective measurements required for qubit state tracking; enable the implementation of control strategies, and improve the fraction of time the qubit is available for experimentation rather tracking or control.  
% \\
% \\
% In analysing projective measurement records to infer information about a quantum system, we are typically led to a general, non-linear Bayesian inference problem. A general Bayesian inference problem typically uses machine learning techniques to develop numerical solutions, particularly when analytics are intractable.
% Machine learning techniques for projective measurement records have found diverse applications,as examples, in developing optimal or adaptive projective measurement protocols to learn parameters governing dynamics of quantum system (e.g. \cite{bonato2016optimized, wiebe2015bayesian}; characterising the initial or final state of quantum system (e.g. \cite{struchalin2016experimental, sergeevich2011characterization, mahler2013adaptive}); reconstructing the evolution of a quantum system based on large measurement records (e.g. \cite{stenberg2016characterization, shabani2010efficient, shen2014reconstructing, de2016estimation, tan2015prediction, huang2017neural}). 
In this manuscript, we develop numerical approaches to track the stochastic evolution of the state of a single two level system (qubit) using a sequence of projective measurements and we forecast the qubit state once the measurement record ceases. A sequence of binary data is obtained by projectively measuring - and resetting - the qubit, where  a 0 or 1 outcome represents the qubit state in a single measurement. Under a slowly drifting, non-Markovian environmental dephasing noise, the probability of measuring a qubit state inherits properties of dephasing noise. Consequently, our binary, time-series data encodes that the probability of getting a 0 or 1 qubit state is slowly drifting. If each qubit outcome is effectively a biased coin flip, then in our application, the bias on the coin changes stochastically under dephasing noise. Further, a coin-flip (projective measurement) discretises a continous time random dephasing process into a sequence of \textit{stochastic but temporally correlated} qubit phases. These phases govern the superposition of 0 and 1 qubit states prior to measurement and hence they affect the probability of a seeing a 0 or 1 qubit state. We use machine learning algorithms to extract temporal correlations from measurements and forecast the qubit state. The performance benchmark of an algorithm is to maximise the forecast period where its qubit state predictions are better than predicting the mean behaviour of the qubit under dephasing noise. We test our algorithms in numerical experiments representing realistic operating environments encountered in a laboratory. Our analysis considers two important types of measurement records, such that (a) single shot qubit outcomes are pre-processed before being given to an algorithm and (b) an algorithm acts directly on 0 or 1 qubit outcomes. The latter measurement record necessitates a complex, \textit{non-linear} measurement model for predictive estimation algorithms.
\\
\\
We seek appropriate machine learning frameworks  firstly, to track and predict non-Markovian qubit state evolution, and secondly, to accommodate non-linear measurement models. In standard Bayesian learning protocols, the output of Bayesian analysis at one instant of time is a probability distribution describing best possible knowledge of a true current state of a system. Typically, this knowledge is propagated in time using a theoretically known transition probability distribution. This transition probability distribution encodes, in signal processing language, time-domain (stochastic) `dynamics' for the true state. For Markov (temporally uncorrelated) processes, the form of the transition probability is simple and the `one step ahead' true state is conditioned only on the current state of the system and not on the system's history. An analytically simple transition probability distribution is widely used for resampling procedures in particle-based methods and used for marginalisation procedures in sequential Bayesian methods \cite{candy2016bayesian}. In our application, the Markov condition is immediately violated. One may design an appropriate transition probability distribution such that increasing non-Markovianity would contribute the increasing dimensionality of the problem as the `one-step ahead' state is conditioned on a increasing set of past states. Developing a theoretical, non-Markovian transition probability distribution in the context of qubit tracking in our application is beyond the scope of this manuscript, though it is the subject of recent research for classical applications (e.g  \cite{jacob2017bayesian}).   
\\ 
\\ 
As an alternative approach, we design a deterministic way to correlate Markovian processes such that a certain general class of non-Markovian dynamics can be approximately tracked without violating assumptions of a machine learning protocol. While this approach applies to a range of machine learning frameworks, we choose theoretically accessible and computationally efficient frameworks for encoding non-Markovian stochastic dynamics  - namely, Kalman Filtering and Gaussian Process Regression. Kalman Filtering (KF) and Gaussian Process Regression (GPR) are both examples of well established Bayesian approaches for tracking stochastic, non-linear true trajectories with Gaussian Markov noise inputs. Both KF and GPR represent mechanisms by which temporal correlations (equally, dynamics) are encoded into an algorithm's structure such that projection of datasets onto this structure enables meaningful learning, white noise filtering and forward prediction.  KF represents a recursive learning technique that easily lends itself to real time, adaptive filtering and control protocols. The wide-scale success of KF frameworks is based on well established extensions to non-linear, non-Gaussian regimes \cite{grewal2001theory}. GPR represents a batch learning algorithm with immense flexibility for tracking non-linear stochastic state dynamics, but tolerates only a linear measurement action \cite{rasmussen2005gaussian}. Both  KF and GPR allow us to approximately track non-Markovian stochastic dynamics, and additionally, a Kalman framework allows us to incorporate non-linear measurement models.
% KF and GPR typically differ in both theoretical structure and computational requirements to other methodologies designed to solve pattern recognition, image reconstruction or classification problems.
\\
\\ 
We simulate the use of suitably modified KF and GPR algorithms for real-time qubit state tracking and short-run predictions in realistic operating environments. It can be shown that for a certain class of true stochastic qubit state trajectories, a general representation using a collection of oscillators or using so-called `autoregressive' processes of finite order are guaranteed to converge to the true, unknown trajectory \cite{karlin2012first}. Using pre-processed measurements, we design a KF algorithm and a GPR algorithm to track general qubit dynamics using an oscillator approach. We contrast oscillator approaches against a Kalman filter with autoregressive dynamics. We use machine learning protocols and simulated measurements to train algorithms such that we can track and predict the qubit state trajectory for a \textit{single realisation} of non-Markovian dephasing noise engineered from arbitrary power spectral densities. These numerical experiments investigate whether theoretical convergence occurs on timescales which enable meaningful qubit state predictions, where timescales are relative to sampling rates and properties of `true' (engineered) dephasing noise. We find that an autoregressive Kalman framework yields model-robust forward prediction horizons compared to other approaches. We extend autoregressive Kalman framework to incorporate a non-linear, coin-flip measurement model and we observe that a Kalman filter can use binary qubit outcomes to conduct qubit state prediction.
\\
\\  
In what follows, we describe our physical setting in \cref{sec:main:PhysicalSetting}. We provide an overview of GPR and KF frameworks in  \cref{sec:main:OverviewofPredictive Methodologies}, and we specify algorithms under consideration in this paper. For pre-processed measurement records, we consider four algorithmic approaches: a Least Squares Filter (LSF) from \cite{mavadia2017}; an Autoregressive Kalman Filter (AKF); a so-called Liska Kalman Filter from \cite{livska2007} adapted for a Fixed oscillator Basis (LKFFB); and a suitably designed GPR learning protocol. For binary qubit outcomes, we extend AKF to a Quantised Kalman Filter (QKF). In \cref{sec:main:Optimisation}, we present optimisation procedures for tuning all algorithms. Results from numerical investigations are presented in \cref{sec:main:Performance} and predictive performance of all algorithms is discussed in \cref{sec:main:discussion}. 
% An alternative approach is to choose machine learning frameworks where non-Markovian dynamics are approximately tracked under a Markov framework and choose machine learning frameworks 
% An alternative approach is to design a deterministic way to correlate Markovian processes such that non-Markovian dynamics can be approximately tracked using a Markov framework. We follow this approach and we choose computationally efficient Bayesian frameworks for encoding non-Markovian stochastic dynamics  - namely, Kalman Filtering and Gaussian Process Regression. Kalman Filtering (KF) and Gaussian Process Regression (GPR) are both examples of well established Bayesian frameworks for tracking stochastic, non-linear true trajectories with Gaussian white noise inputs. These learning frameworks typically differ in both theoretical structure and computational requirements to other machine learning methodologies designed to solve pattern recognition, image reconstruction or classification problems. Both frameworks represent mechanisms by which time domain correlations (equally, dynamics) are encoded into an algorithm's structure such that projection of datasets onto this structure enables meaningful learning, white noise filtering and forward prediction.  KF represents a sequential learning technique that easily lends itself to real time, adaptive filtering and control protocols. 

% % shitty phrasing
% A classical Kalman Filter is provably optimal for linear state dynamics and measurement models . However, . 

% The extensions produce good results as long as errors during the filtering process remain small.  %ends on the wrong challenge

% % differentiate the two challenges 
% % dynamical model - we choose KF and GPR from machine learning universe
% lack of analytic for qubits under any sort of dephasing
% % msmt model - KF over GPR
% \\
% \\

% \\
% \\
% In this manuscript, we track the stochastic evolution of the state of a single two level system (qubit) using a sequence of projective measurements, and we predict the qubit state in forward time once data collection ceases.  Stochastic evolution of a qubit state arises when a qubit is coupled to a slowly drifting environmental dephasing process. A sequence of binary data are obtained by projectively measuring - and resetting - the qubit. The resulting time-series data reflects that the probability of getting a 0 or 1 qubit outcome is drifting, namely, each qubit outcome is a biased coin flip, where the bias on the coin changes stochastically under environmental dephasing. Additionally, the coin-flip measurement encodes a continuous time random dephasing process into a discrete time sequence of stochastic qubit phases, and these phases govern the superposition of 0 and 1 qubit states prior to a projective measurement. We use GPR and KF frameworks to design learning algorithms which extract temporal correlations in our measurement record and maximise the forward time horizon for which we can predict the qubit state better than the mean beahviour under dephasing. We consider first, linear measurements, where single shot qubit outcomes are pre-processed before being given to an algorithm, and secondly, a non-linear measurement record, where an algorithm acts directly on 0 or 1 qubit outcomes. 
%  In all cases, we quantify the prediction horizon with respect to realistic operating scenarios, for example, against measurement noise levels or the ratio between true noise cut-off and the sampling rate. 
% We extend the linear autoregressive Kalman Filter to accomodate a non-linear, bit-flip measurement model such that our filter tracks single shot qubit measurements. 
% \\
% \\
%  (e.g. \cite{garcia2016optimal, bach2004learning, tatinati2013hybrid, hall2011reinforcement, hamilton2016ensemble}); multi-component signal frequency estimation and phase retrival problems (e.g. \cite{boashash1992estimating2, ji2016gradient}), and system characterisation with potentially adaptive measurements (e.g. \cite{muller2001introduction, stenberg2016characterization,struchalin2016experimental, sergeevich2011characterization, mahler2013adaptive}). In quantum physics, stochastic dynamical evolution of complex systems has been analysed using pattern recognition and classification learning algorithms (e.g. \cite{shabani2010efficient, shen2014reconstructing, de2016estimation, tan2015prediction}); with extensions to many body dynamics (e.g. \cite{huang2017neural}).


%  Bayesian learning techiques have found a natural home in the analysis of these large datasets and in enbaling adaptive learning procedures for quantum predictive estimation.
% \\
% \\
% In recent years, predictive estimation has been studied for quantum systems, in particular, using projective measurements, where each measurement resets the state of the quantum system. Accurate short-run predictions enhance state tracking and enable the implementation of control strategies
%  Machine learning techniques have found a natural application in the analysis of these measurement records 

%  It needs to contextualize and set up the problem, why it's interesting, and why it's not just "textbook."  None of that happens in the intro as constructed - see the notes for specific comments.

% The first two paragraphs should be background on context and explanation of fundamental concepts (like forward prediction in time), forecasting, etc.  You should contextualize where else this is useful - like finance, robotic control, etc. You should not talk at all about what you do until you reach the third “in this manuscript” paragraph. 

% The general classes of algorithms for state estimation and predictive control (KF and GPR) need to be introduced in this background material and contextualized.
% This paragraph makes little sense as it describes some of the “in this manuscript” material.
% But that paragraph comes next? Also, this sentence is very misleading - it suggests something experimental. You should be clear, when discussing your work, what you have done.

% This paragraph needs to summarize what has been done.  That means a complete summary of approaches (including reference to numeric studies), algorithms, and main findings. 

% Slowly drifting environmental decoherence in the laboratory gives rise to stochastic qubit evolution. The probability of measuring a particular qubit state drifts stochastically with time. A sequence of measurements performed on a qubit over timescales comparable to the slow drift is seen to encode information about environmental decoherence. This information consists of completely classical correlations between two measurements separated in time.  It provides a unique opportunity to harness a rich collection of control engineering and machine learning techniques to learn noise information and enable qubit state tracking and state predictions. In both engineering and physics, machine learning have found diverse applications in non-linear dynamical control and stochastic time series forecasting (e.g. \cite{garcia2016optimal, bach2004learning, tatinati2013hybrid, hall2011reinforcement, hamilton2016ensemble}); pattern recognition and system characterisation with potentially adaptive measurements (e.g. \cite{muller2001introduction, stenberg2016characterization,struchalin2016experimental, sergeevich2011characterization, mahler2013adaptive}), and multi-component signal frequency estimation and phase retrival problems (e.g. \cite{boashash1992estimating2, ji2016gradient}). Stochastic dynamical evolution of complex systems has been analysed using machine learning tools (e.g. \cite{shabani2010efficient, shen2014reconstructing, de2016estimation, tan2015prediction}); with extensions to many body dynamics (e.g. \cite{huang2017neural}).
% \\
% \\
% We investigate classical techniques to learn noise correlations encoded in qubit measurements to enable robust predictive control and stablisation of qubit dynamics in real time. We stabilise qubit dynamics against environmental decoherence in realistic laboratory settings. We learn noise correlations in a measurement record and we predict the qubit state for as long as possible in forward time once data collection has ceased i.e. we maximise the forward prediction horizon.  We analyse a sequence of projective measurements under Bayesian frameworks where each projective measurement resets the qubit state. The objective of algorithm design under these frameworks is to yield model robust predictions which maximise the forward prediction horizon in a variety of realistic operational scenarios. A maximal forward prediction horizon beyond the measurement record enables future control interventions while reducing the need for projective measurements, for example, by interleaving periods of data collection with periods of unsupervised control. The first demonstration of predictive control using `batch' machine learning algorithms operating on a sequence of projective measurements was conducted in \cite{mavadia2017}.
% \\
% \\
% In this manuscript, we seek model robust techniques to track a qubit state evolving under arbitrary environmental dephasing using projective measurements. We assess algorithmic predictive performance for maximising the forward prediction horizon relative to predicting the mean of the environmental dephasing. In adapting classical predictive estimation techniques for our application, our key challenge is that there is no theoretical model for capturing stochastic qubit evolution. Further, single qubit measurements represent a non-linear, quantised measurement action.  Many classical techniques are optimal for linear filtering and significant complexity is introduced in the regime of non-linear filtering with quantised measurement outcomes. 
% \\
% \\
% In what follows, we describe our physical setting in \cref{sec:main:PhysicalSetting}. We provide an overview of predictive methodologies in  \cref{sec:main:OverviewofPredictive Methodologies}, and we specify algorithms under consideration in this paper. In \cref{sec:main:Optimisation}, we present optimisation procedures for tuning algorithms. Predictive performance of algorithms is compared through results from numerical investigations in \cref{sec:main:Performance}. 

\section{Physical Setting \label{sec:main:PhysicalSetting}}  
\label{sec:main:1} 

Our physical set-up considers a sequence of projective measurements performed on a qubit. Each projective measurement yields a 0 or 1 outcome representing the state of the qubit. The qubit is then reset, and the exact procedure is repeated. If no dephasing is present, then the probability of obtaining a binary outcome does not change as more qubit measurements are performed. If slowly drifting environmental dephasing is present, then the probability of obtaining a given binary outcome also drifts stochastically. In essence, we are using a qubit to probe dephasing noise and our procedure encodes a continuous time dephasing process into time-stamped, discrete binary samples.  The correlation between any two time-separated qubit measurements arises entirely from a classical, non-Markov dephasing noise field. 
\\
\\
Formally, an arbitrary environmental dephasing process manifests as time-dependent stochastic detuning, $\delta \omega (t)$, between the qubit frequency and the master clock. This detuning is an experimentally measurable quantity in a Ramsey protocol, as in \cref{fig:main:Predive_control_Fig_overview_17_one} (a). A non-zero detuning induces a relative stochastic phase accumulation between the two possible $0$ and $1$ states of a qubit, thereby affecting the statistical likelihood of measuring a particular qubit outcome. 
\\
\\
In a sequence of $n$ Ramsey measurements spaced $\Delta t$ apart with a \textit{fixed Ramsey wait time}, $\tau$, the change in the statistics of measured outcomes over this measurement record depends solely on the dephasing  $\delta \omega(t)$.   We assume that the measurement action over $\tau$ timescales is much faster than the slow time dependence of dephasing, and $\Delta t \gg \tau$. The resulting measurement record is a set of binary outcomes,  $\{d_n\}$,  where qubit state dynamics were governed by $n$ true stochastic phases, $\state := \{\state_n\}$. We define the statistical likelihood for observing a single shot, $d_n$, using Born's rule \cite{ferrie2013}:

\begin{align}
P(d_n | \state_n, \tau, n \Delta t) &= \begin{cases} \cos(\frac{\state(n \Delta t, n\Delta t + \tau)}{2})^2 \quad \text{for $d=1$} \\   \sin(\frac{\state(n \Delta t, n\Delta t + \tau)}{2})^2  \quad \text{for $ d=0$} \end{cases} \label{eqn:main:likelihood}
\end{align}
where  $ \state(n \Delta t, n\Delta t + \tau) \equiv \int_{n \Delta t}^{n \Delta t +\tau} \delta \omega(t') dt'$ and we use the shorthand $\state(n \Delta t , n\Delta t + \tau) \equiv \state_n$. The notation $P(d_n | \state_n, \tau, n \Delta t)$ refers to the conditional probability of seeing a measurement $d_n$ given that a stochastic phase, $\state_n$, accumulated over the qubit at $t = n \Delta t$. In the noiseless case, $P(d=1| f, \tau) = 1 \quad \forall n $, such that a qubit can be manipulated perfectly in the absence of net phase accumulation due to environmental dephasing. This procedure discretises $\delta \omega(t)$ into a random process, $f$, governing qubit dynamics. 
\\
\\
\begin{figure}[h!]
    \includegraphics[scale=1]{Predive_control_Fig_overview_17_one} 
    \caption{ \label{fig:main:Predive_control_Fig_overview_17_one} Physical Setting: In (a), we define the Hamiltonian for stochastic qubit dynamics under arbitrary environmental dephasing using a covariance stationary, non-Markovian detuning $\delta \omega(t)$ with an arbitrary power spectral density. A sequence of Ramsey experiments with fixed wait time $\tau$ yield single shot outcomes $\{ d_n \}$, with likelihood $P(d_n|\state_n, \tau,n)$,  conditioned on a mean-square ergodic sequence of true phases, $\{ f_n\}$, with $n \in [-N_T, 0]$ indexing time during data collection. Our objective is to maximise forward time $n \in [0, N_P]$ for which an algorithm uses measurement data to predict a future qubit state and incurs a lower Bayes prediction risk relative to predicting the mean value of the dephasing noise [dark gray shaded]. In (b), single shot outcomes are processed to yield noisy accumulated phase estimates, $\{ y_n\}$, corrupted by measurement noise $\{v_n\}$. The choice of $\{d_n\}$ or $\{y_n\}$ as datasets for predictive estimation corresponds to non-linear or linear measurement records in (b) and (c).}
\end{figure} 
\cref{fig:main:Predive_control_Fig_overview_17_one}(b) depicts a non-linear measurement record, $\{ d_n\}$. Each $d_n$ [black dots] corresponds to a single projective measurement on a qubit yielding a 0 or 1 outcome. The sequence $\{ d_n\}$ can be considered as a sequence of biased coin flips, where the underlying bias of the coin is a non-Markovian, discrete time process. The value of the bias is given by \cref{eqn:main:likelihood} at each $n$. Subsequently, the qubit state is reset, but the dephasing noise correlations manifest again through Born's rule to yield another random value for the bias at $n+1$. The non-linearity of the measurement model is defined with respect to $f$ where \cref{eqn:main:likelihood} is interpreted as a non-linear measurement action on $f$ for Bayesian learning frameworks. 
\\
\\
% \cref{fig:main:Predive_control_Fig_overview_17_one}(b) depicts a non-linear measurement record, $\{ d_n\}$. Each $d_n$ [black dots] corresponds to a single projective measurement on a qubit yielding a 0 or 1 outcome. The true probability of getting a particular $d_n$ drifts stochastically with time. The sequence $\{ d_n\}$ can be considered as a sequence of biased coin flips, where the underlying bias of the coin is a non-Markovian, discrete time process. The value of the bias is given by \cref{eqn:main:likelihood} at each $n$. Subsequently, the qubit state is reset, but the dephasing noise correlations manifest again through Born's rule to yield another random value for the bias at $n+1$. As a final comment, the non-linearity of the measurement model is defined with respect to the true stochastic qubit phase sequence, $f$, and \cref{eqn:main:likelihood} is interpreted as a non-linear measurement action on $f$ for Bayesian learning frameworks. 
\cref{fig:main:Predive_control_Fig_overview_17_one}(c) depicts a linear measurement record, $\{ y_n\}$.  Each $y_n$ is the sum of a true qubit phase, $\state_n$, and Gaussian white measurement noise, $v_n$.  The sequence $\{ y_n\}$ is generated by pre-processing binary measurements, $\{ d_n\}$. Pre-processing refers to averaging procedures over $\tau$-like timescales much faster than drift of $\delta \omega (t)$ such that $\{ y_n\}$ is a measurement record fed to learning algorithms. Pre-processing summarises a range of experimental techniques to extract $\{ y_n\}$ from $\{ d_n\}$. Firstly, one may low-pass (or decimation) filtered a sequence of $\{ d_n\}$ binary outcomes to yield $\hat{P}(d_t | \state_t, \tau, t)$ from which accumulated phase corrupted by measurement noise, $\{ y_n\}$, can be obtained from \cref{eqn:main:likelihood} (see Appendices). Secondly, one may perform $M$ runs of the experiment over which $\delta \omega (t)$ is approximately constant under the slow drift assumption. For $M\tau << \Delta t$, we obtain an estimate of  $\state_n$ at $t = n \Delta t $ using a Bayesian scheme or Fourier analysis. 
\\
\\
In the case of a qubit evolving under stochastic dephasing, we have no apriori dynamical model for a qubit state evolution under dephasing. Our task is to build a dynamical model to approximately track $\state$ and enable qubit state predictions. We impose properties on environmental dephasing such that our theoretical design in GPR and KF enable meaningful predictions. We assume dephasing is non-Markovian, covariance stationary and mean square ergodic, that is, a single realisation of the process $\state$ is drawn from a power spectral density of arbitrary but non-Markovian shape. We further assume that $\state$  is a Gaussian process. In the subsequent section, we define the theoretical structure of KF and GPR algorithms. 

\section{Overview of Predictive Methodologies \label{sec:main:OverviewofPredictive Methodologies}}
% \begin{widetext}
\begin{figure*}
    \includegraphics[scale=1.]{Predive_control_Fig_overview_17_two} 
    \caption{ \label{fig:main:Predive_control_Fig_overview_17_two} Predictive Methodologies: (a) In GPR, a prior  distribution over true phase sequences $P(\state), \state \equiv \{ \state_n \}$ is constrained by a linear Bayesian likelihood of observed data, $\{ y_n\}$. The prior encodes dephasing noise correlations by defining covariance relations for the $i, j$-th time points using  $\Sigma_\state^{i, j}$ and optimising over its free parameters during training. The moments of the resulting predictive distribution $P(\state^*|y)$ are interpreted as pointwise predictions and their pointwise uncertainties when evaluated for $n>0$.  (b) In KF, the Kalman state and its variance correspond to moments of a Gaussian distribution propagated in time via $\Phi$, and filtered via the Kalman gain, $\gamma$ at timestep $n$. The design of $\Phi$ deterministically colors a white noise process $\{w_n \}$ and `encodes' an apriori structure for learning dephasing noise correlations. Prediction proceeds by propagating forwards with $\gamma_n=0, n>0$. Additive white Gaussian measurement noise $v_n$ corrupts all measurement records.}
\end{figure*}
% \end{widetext}
We introduce algorithmic learning under KF and GPR frameworks in \cref{fig:main:Predive_control_Fig_overview_17_two}. Stochastic qubit evolution is depicted for one realisation of $\state$ [red] given noisy observations [black dots] corrupted by Gaussian white measurement noise $v_n$.  Both frameworks start with a prior Gaussian distribution over qubit states that is constrained by the measurement record to yield a posterior Gaussian distribution of the qubit state. The prior captures assumptions about the qubit state before any data is seen and the posterior captures our best knowledge of qubit state under a Bayesian framework.  Both KF and GPR yield a posterior distribution which is used to generate qubit state estimates for $n<0$ and predictions for $n>0$ [black solid]. For linear measurement records, $h(x) \mapsto Hx$ and $f \equiv Hx$ linking GPR and KF notation. 
\\
\\
The key feature of a Kalman filter is a recursive learning procedure shown in \cref{fig:main:Predive_control_Fig_overview_17_two} (a) and we comment on the physical interpretation of Kalman notation. Our knowledge of the qubit state is summarised by the prior and a posteriori Gaussian probability distributions and these are created and collapsed at each time step. The mean of these distributions is the true Kalman state, $x$, and the covariance of these distributions, $P$, is the uncertainty in our knowledge of $x$. The Kalman gain, $\gamma_n$, updates our knowledge of $(x_n, P_n)$ within each time step $n$. The dynamical model  $\Phi_n$ propagates the $(x_n, P_n)$ to the next time step, such that the posterior moments at $n$ define the prior at $n+1$.  Predictions at $n=0$ occur when moments are propagated using $\Phi_n, n>0 $ but for zero gain. 
\\
\\
In order to encode stochastic qubit dynamics in KF algorithms, we deviate from standard implementations such that our true Kalman state and its uncertainty, $(x, P)$, do not have a direct physical interpretation. In standard KF implementations, the sequence $\{x_n\}$, defines a hidden signal that cannot be observed with incurring measurement noise and $\Phi_n$ is known. Often, Kalman $x$ can represent a multi-component signal and while $x$ is driven by white noise, typical filtering implementations specify a deterministic component to the evolution of $x$ or provide a desired reference trajectory for the filter to follow. In our application, we define the Kalman state, $x$, the dynamical model $\Phi$, and a measurement action $h(x)$ such that the Kalman Filtering framework can track a non-Markovian qubit state trajectory due to an arbitrary realisation of $\state$.  Kalman $x$ has no apriori deterministic component and corresponds to arbitrary power spectral densities associated with $f$. As an illustrative example, in a linear regime, a true qubit phase sequence $\state$ has a physical interpretation, but $x$ and $\Phi$ are abstract entities designed to yield a sequence $\{x_n\}$  such that upon noiseless measurement, we recover $\state \equiv Hx$. Hence, the role of the Kalman $x$ is to represented a correlated process that, upon measurement, yields physically relevant quantities governing qubit dynamics.
\\
\\
A GPR learning protocol in \cref{fig:main:Predive_control_Fig_overview_17_two} (b)  chooses \textit{a random process} to best describe overall dynamical behaviour of the qubit state under one realisation of $f$. The key point is that sampling the prior or a posterior distribution in GPR yields random realisations of discrete time \textit{sequences}, not  individual random variables, and GPR considers the entire measurement record at once. The output of a GPR protocol is predictive distribution which we can evaluate at arbitrarily chosen collection of time labels, $n^*$, where we interpret the result as state estimation if $n^* < n =0$ and predictions if $n^* > n =0$. 
\\
\\
We encode stochastic qubit dynamics in GPR using a so-called `periodic kernel'. In standard GPR implementations, if any two observations are correlated such that the correlation strength depends only on the separation distance of the index of these observations, then these correlations are given by the covariance matrix, $\Sigma_\state$. Each $\Sigma_f^{i, j}$ element describe how any two observations at time step $i$ and $j$ separated by a distance $|i-j|$ must be correlated. For us, the non-Markovian dynamics of $f$ are not specified explicitly but are encoded in a general way through the choice of a `kernel' or covariance function, prescribing how $\Sigma_\state^{i,j}$ should be calculated. The Fourier transform of the kernel represents a power spectral density in Fourier space. A general design of $\Sigma_f^{i, j}$ allows one to probe arbitrary stochastic dynamics and equivalently, explore arbitrary regions in the Fourier domain. For example, Gaussian kernels (RBF) and mixtures of Gaussian kernels (RB) capture the continuity assumption that correlations die out as separation distances increase. We choose an infinite basis of oscillators summarised by the periodic kernel to enable us to probe arbitrary power spectral densities for $f$.
\\
\\
Irrespective of the choice of KF or GPR framework, an algorithm must maximise the forward prediction horizon. The forward prediction horizon is the number of time steps beyond the measurement record for which predictions of the qubit state are better than predicting the average behaviour of a qubit under dephasing. The fidelity of our algorithm during state estimation and prediction relative to the true state is expressed by the mathematical quantity known as a Bayes Risk, where a zero risk value corresponds to perfect predictive estimation. At each timestep, $n$, the Bayes risk is a mean square distance between truth, $\state$ , and prediction, $\hat{\state}$ calculated over an ensemble of $M$ different realisations of truth $\state$ and noisy datasets $\mathcal{D}$:
\begin{align}
L_{BR}(n | I) & \equiv \langle(\state_n - \hat{\state}_n)^2 \rangle_{\state,\mathcal{D}} \label{eqn:main:sec:ap_opt_LossBR}
\end{align}
The notation $L_{BR}(n | I)$ expresses that the Bayes Risk value at $n$ is conditioned on, $I$, a placeholder for free parameters in the design of the predictor,  $\hat{\state}_n$. State estimation risk is Bayes Risk incurred during $n \in [-N_T, 0]$; prediction risk is the Bayes Risk incurred during $n \in [0, N_P]$. State estimation and prediction risk regions for one realisation of dephasing noise are shaded in \cref{fig:main:Predive_control_Fig_overview_17_one,fig:main:Predive_control_Fig_overview_17_two,Predive_control_Fig_overview_17_three}.  The forward prediction horizon is the number of time steps for $ n \in [0, N_P]$ during which a predictive algorithm incurs a lower Bayes prediction risk than predicting $\hat{\state}_n \equiv \mu_f = 0 \quad \forall n$, namely, the mean qubit behaviour under zero mean dephasing noise. %The measurement record - $\{ d_n\}$ or $\{ y_n\}$ -  ceases at $n=0$  and we desire qubit state predictions over $ n \in [0, N_P]$. 
\\
\\
To follow, we introduce Kalman Filtering (KF) algorithms acting on linear and non-linear measurement records, and Gaussian Process Regression (GPR) on linear measurement records. 
% \cref{fig:main:Predive_control_Fig_overview_17_two} (a) depicts the distinguishing recursive learning protocol for Kalman filtering - namely that a prior and a posteriori Gaussian distribution is created and collapsed at each time step, $n$.  The Kalman gain, $\gamma_n$, updates the prior for a qubit state estimate within each time step, $n$, yielding a posterior distribution over the qubit state estimate at $n$.  The mean of posterior distribution is the true Kalman state, $x$, and the covariance of the posterior, $P$,is the uncertainty in state estimate $x$. These moments are propagated to the next time step using $\Phi_n$ and define the prior for $n+1$ time step. Predictions at $n=0$ occur when moments are propagated using $\Phi_n, n>0 $ but for zero gain. In standard implementations, the sequence $\{x_n\}$, defines a hidden signal that cannot be observed with incurring measurement noise. Further, the physical dynamical model for the true state $x$ is known. Our key challenge is that we must define the Kalman state, $x$, the dynamical model $\Phi$, and a measurement action $h(x)$ such that the Kalman Filtering framework can track a non-Markovian qubit state trajectory due to an arbitrary realisation of $\state$. For example, in a linear regime, $\state$ has a physical interpretation, but $x$ and $\Phi$ are abstract entities designed to yield a sequence $\{x_n\}$  such that upon noiseless measurement, we recover $\state \equiv Hx$. 
% \\
% \\
% \cref{fig:main:Predive_control_Fig_overview_17_two} (b) shows that a GPR framework encompasses the entire measurement record at once. The prior probability distribution is a distribution over stochastic processes - for us, sampling $P(\state)$ yields random realisations of discrete time sequences, not random variables. This prior distribution is constrained by the measurement record. The output of a GPR protocol is predictive distribution which we can evaluate at arbitrarily chosen time labels, $n^*$, where we interpret the result as state estimation if $n^* < n =0$ and predictions if $n^* > n =0$. In standard GPR implementations, if any two observations are correlated such that the correlation strength depends only on the separation distance of the index of these observations, then these correlations are given by the covariance matrix, $\Sigma_\state$. Each $\Sigma_f^{i, j}$ element describe how any two observations at time step $i$ and $j$ separated by a distance $|i-j|$ must be correlated. For us, the non-Markovian dynamics of $f$ are not specified explicitly but are encoded in a general way through the choice of a `kernel' or covariance function, prescribing how $\Sigma_\state^{i,j}$ should be calculated. The Fourier transform of the kernel represents a power spectral density in Fourier space. A general design of $\Sigma_f^{i, j}$ allows one to probe arbitrary stochastic dynamics and equivalently, explore arbitrary regions in the Fourier domain. We choose an infinite basis of osccillators summarised by the periodic kernel to enable us to probe arbitrary power spectral densities for $f$. Other popular kernel choices are Gaussian kernels (RBF) and mixtures of Gaussian kernels (RB) and these capture the continuity assumption that  correlations die out as separation distances increase. 
% \\
% \\

% Further, $h(x)$ must reconstruct the arbitrary power spectral density of $f$ for time domain predictions to make sense. 
% \cref{fig:main:Predive_control_Fig_overview_17_two}(a) depicts a KF framework where we recast the problem of tracking non-Markovian dynamics under $\state$ as a filter design problem. In particular, we design a deterministic dynamical model, $\Phi$, to `color' white noise such that the true Kalman state evolution is non-Markovian and  mimics true stochastic qubit dynamics. The mean ($x_n$) and variance ($P(x_{n|n})$) of Gaussian distribution representing the true Kalman state are updated at each time-step accoring to the Kalman gain, $\gamma$. Kalman state estimates are propagated in time according to $\Phi$. Prediction ensues when the true state is propagated in time with zero Kalman gain. A KF framework can incorporate both linear and non-linear measurement records depicted in \cref{fig:main:Predive_control_Fig_overview_17_one}.
% \\
% \\
% \cref{fig:main:Predive_control_Fig_overview_17_two} (b) depicts that a GPR framework is not recursive and instead considers the entire measurement record at once. In GPR, we define apriori a collection of stochastic time domain processes  $\{\state\}$, not random variables. A prior probability distribution over the time domain processes is constrained by the entire measurement record. 
% We define `test points' in time, $n^*$, at which we want to evaluate the moments of predictive aposteriori distribution, where the word predictive communicates that predictions are conditioned on posterior over the entire dataset. 
% \\
% \\
% \cref{fig:main:Predive_control_Fig_overview_17_two}(b) conceptually summarises a GPR approach. We define a prior distribution of a family of Gaussian random processes $P(\state)$. Sampling $P(\state)$ yields random realisations of time domain sequences where the correlations between any two time steps are given by the covariance matrix, $\Sigma_\state$. The non-Markovian dynamics of $f$ are not specified explicitly but are encoded in a general way through the choice of a kernel prescribing how $\Sigma_\state^{i,j}$ should be determined. A general design of $\Sigma_f^{i, j}$ allows one to probe arbitrary stochastic dynamics and free parameters in the kernel are discovered via tuning the algorithm using measurement datasets. For a single realisation of dephasing noise, state predictions, $f^*$, are conditioned on measurement data $y$, yielding a predictive probability distribution $P(f^*|y)$. The mean of $P(f^*|y)$ is evaluated over the prediction horizon and the result is interpreted as `state predictions'. For Gaussian Process Regression, the measurement model must be linear.
% \\
% \\
% a priori distribution is a probability distribution defined over a collection of time domain processes $\{\state\}$, not random variables. In our application, this prior probability distribution is completely specified by a mean $\mu_\state$ and a covariance matrix $\Sigma_\state$. 

% We consider our measurement record,  $\{d_n\}$ or $\{y_n\}$, under a Bayesian framework and enable qubit state predictions.We use KF and GPR  approximate stochastic dynamics by propagating and constraining the underlying probability distributions describing the most likely trajectory for a qubit state conditioned on measurement data. 
% % \\
% - one step ahead prediction
% - non-markovian
% - transision probabilities
% - what's new? what's hard? what's different? - what's not textbook?
% - comparison to existing approaches - unscent kf

% \begin{itemize}
% \item Introduce algoritshsm, what they do, and how they will be used for predictive estimation. Justify why these algorithms can be used. 
% \item Overview figure on all approaches 
% \item Define the forward prediction horizon. Explain the key concept of how, using different classes of algorithm, you can do prediction and maximise the prediction horizon.
% \item provide a narrative for each algorithm and their equations and explain their relationship to experimentally relevant quantities 
% \end{itemize}

\subsection{ Kalman Filtering (KF)}

A Kalman Filter recursively tracks the stochastic evolution of a hidden true state. An incoming stream of unreliable (noisy) observations are fed to a Kalman Filter, and the objective of the Kalman Filter is to recursively improve its estimate of the true state at any time, $n\Delta t$, given the past $n$ measurements. In order for a Kalman Filter to track a stochastically evolving qubit state in our application, the hidden true Kalman state $x_n$ must mimic stochastic dynamics  of a qubit under environmental dephasing. We propagate the hidden state $x_n$ according to a dynamical model $\Phi_n$ corrupted by Gaussian white process  noise, $w_n$.  
\begin{align}
x_n & = \Phi_n x_{n-1} + \Gamma_n w_n \label{eqn:KF:dynamics} \\
w_n & \sim \mathcal{N}(0, \sigma^2) \quad \forall n 
\end{align}
Process noise has no physical meaning in our application - $w_n$ is shaped by $\Gamma_n$ and deterministically colored by the dynamical model $\Phi_n$ to yield a non-Markovian $x_n$ representing qubit dynamics under generalised environmental dephasing. 
\\
\\
We measure $x_n$ using an ideal measurement protocol $h(x_n)$ and incur additional Gaussian white measurement noise $v_n$ with scalar covariance strength $R$, yielding scalar noisy observations $y_n$:
\begin{align}
y_n &= z_n + v_n \\
z_n & \equiv  h(x_n) \\
v_n & \sim \mathcal{N}(0, R) \quad \forall n
\end{align}
The measurement procedure, $h(x_n)$, can be linear or non-linear, allowing us to explore both regimes in our physical application.
\\
\\
% State estimation in Kalman filtering proceeds as follows. At time step $n$, the state estimates $x_{n-1}, P_{n-1}$  are propagated into the current time step via $\Phi_n$ and they define moments of a prior Gaussian distribution at $n$. Within each time step, a Bayesian update occurs via the Kalman gain, $\gamma_n$. This yields the aposteriori Gaussian distribution at $n$ with moments, $x_{n}, P_{n}$. For $n>0$, the filter propagates moments using $\Phi_n$ but with $\gamma_n \equiv 0 \forall n>0$ and these yield qubit state predictions in forward time. 
% \\
% \\
 Since we do not have a known dynamical model $\Phi$ for describing stochastic qubit dynamics under $\state$, we will need to make design choices for  $\{ x, \Phi, h(x), \Gamma \}$  such that $\state$ can be approximately tracked. These design choices will completely specify algorithms in this manuscript. For a linear measurement record,   $h(x) \mapsto Hx$ and we compare predictive performance if $\Phi$ models stochastic dynamics either via so-called `autoregressive' processes in AKF, or via a collection of oscillators in LKKFB. Second, we use dynamics of AKF to define a Quantised Kalman filter (QKF) with a non-linear, quantised measurement model such that the filter can act directly on binary qubit outcomes. We provide details in sub-sections below. 
 
\begin{figure} [h]
    \includegraphics[scale=1]{Predive_control_Fig_overview_17_three}
    \caption{\label{Predive_control_Fig_overview_17_three} Apriori Structure for $\Phi$: All Kalman dynamical models, $\Phi$, are mean square approximations to qubit dynamics under arbitrary covariance stationary, non-Markovian, mean square ergodic $\state$. AKF/QKF: Kalman $\Phi$, implements a weighted sum of $q$ past measurements driven by process noise, $w$. We represent $\Phi$ using a lag operator, $L^r: \state_n \mapsto \state_{n-r}$, and coefficients, $ \{ \phi_{q' \leq q} \}$ learned from LSF in \cite{mavadia2017}. This defines an autoregressive process of order $q$ and we use a high $q$ model to approximate any covariance stationary $\state$ [top]. LKFFB: Kalman $\Phi$ represents a collection of $J$ oscillators driven by process noise, $w$, where frequency of oscillators must span dephasing noise bandwidth. The instantaneous amplitude and phase of each basis oscillator can be derived from the Kalman state estimate $x_{j, n}$ at any $n$. Predictions combine learned amplitudes and phases for each basis oscillator and sum contributions over all $J$ [bottom].}
\end{figure}

\subsubsection{Autoregressive Kalman Filter (AKF)}

An AKF probes arbitrary, covariance stationary qubit dynamics such that the dynamic model is a weighted sum of $q$ past values driven by white noise i.e. an autoregressive process of order $q$, AR($q$). By Wold's decomposition, any zero mean covariance stationary process representing qubit dynamics has a representation in the mean square limit by an autoregressive process of finite order $q_c$, AR($q_c$), where the ability to approximate arbitrary power spectral density for a covariance stationary process typically falls with model complexity, $c$ \cite{west1996bayesian}. We design the Kalman dynamical model, $\Phi$, such that the true Kalman state is AR($q$) process that approximately tracks the qubit state. The study of AR($q$) processes falls under the study of a general class of techniques based on autoregressive moving average (ARMA) models in classical control engineering. For high $q$ models in a typical time-series  analysis, it is possible to decompose an AR($q$) into an ARMA model with fewer parameters \cite{brockwell1996introduction, salzmann1991detection}. However, we retain high $q$ model to probe arbitrary power spectral densities. Further, literature suggests a high $q$ approach is relatively easier than a full ARMA estimation problem and enables lower prediction errors \cite{wahlberg1989estimation,brockwell1996introduction}.
\\
\\
We write the Kalman dynamical operator $\Phi$ in terms of a lag operator, $L$, where each application of the lag operator delays a true state by one time step:
\begin{align}
L^r: f_n &\mapsto f_{n-r} \quad \forall r \leq n \\
\Phi(L) & \equiv  1 - \phi_1 L - \phi_2 L^2 - ... - \phi_q L^q 
\end{align}
Here, the set of $q$ coefficients $\{ \phi_q \}$ are the set of autoregressive coefficients which specify the dynamical model. Hence, the true stochastic Kalman state dynamics are:
\begin{align}
\Phi(L) \state_n & = w_n \\ 
\implies \state_n & = \phi_1 L \state_n + \phi_2 L^2 \state_n + ... + \phi_q L^q \state_n + w_n \\
 \state_n &= \phi_1 \state_{n-1} + \phi_2 \state_{n-2} + ... + \phi_q \state_{n-q} + w_n \label{eqn:main:ARprocess}
\end{align}
Formally, \cref{eqn:main:ARprocess} is an AR($q$) process. For small $q < 3$, it is possible to extract simple conditions on the coefficients, $\{ \phi_q \}$, that guarantee properties of $f$, for example, that $f$ is covariance stationary and mean square ergodic. In our application, we freely run arbitrary $q$ models via machine learning in order to improve our approximation of an arbitrary $f$. Any AR($q$) process can be recast (non-uniquely) into state space form, and we define the AKF by the following substitutions into Kalman equations:
\begin{align}
x_n & \equiv  \begin{bmatrix} f_{n} \hdots f_{n-q+1} \end{bmatrix}^T \\
\Gamma_n w_n & \equiv \begin{bmatrix} w_{n} 0 \hdots 0 \end{bmatrix}^T \\
\Phi_{AKF} & \equiv 
\begin{bmatrix}
\phi_1 & \phi_2 & \hdots & \phi_{q-1} & \phi_q \\ 
1 & 0 & \hdots & 0 & 0 \\  
0 & 1 & \ddots & \vdots & \vdots \\ 
0 & 0 & \ddots & 0 & 0 \\ 
0 & 0 & \hdots & 1 & 0 
\end{bmatrix} \quad \forall n \label{eqn:akf_Phi} \\
H & \equiv \begin{bmatrix} 1 0 \hdots 0 \end{bmatrix} \quad \forall n  
\end{align}
The matrix $\Phi_{AKF}$ is the dynamical model used to recursively propagate the unknown state during state estimation in the AKF. In general, the ${\phi_i}$ in $\Phi_{AKF}$ must be learned through an optimisation procedure where the total number of parameters to be optimised are $\{\phi_1, \hdots, \phi_q, \sigma^2, R \}$. This procedure yields the optimal configuration of the autoregressive Kalman filter, but at the computational cost of a $q+2$ dimensional optimisation problem for arbitrarily large $q$.
\\
\\
The Least Squares Filter (LSF) in \cite{mavadia2017} considers a weighted sum of past measurements to predict the $m$-th step ahead measurement outcome. A gradient descent algorithm learns the weights, $\{\phi_{q' \leq q}\}, q' = 1, ... , q $ for the previous $q$ past measurements, and a constant offset value for non-zero mean processes, to calculate the $m$ step ahead prediction, $m \in [0, N_P]$. The set of $m$ LSF models, collectively, define the set of predicted qubit states under an LSF.
\\
\\
For $m=1$, we assert that learned $\{\phi_{q' \leq q}\}$ in LSF effectively implements an AR($q$) process and we test via numerical experiments comparing LSF and AKF. For $m=1$, and for zero mean $w_n$, LSF in \cite{mavadia2017} by definition searches for coefficients for the weighted linear sum of past $q$ measurements, as in \cref{eqn:main:ARprocess}. We test our assertion by using an LSF to reduce the computational tractability of $q+2$ optimisation problem for an AKF for high order $q$. Namely, we use  $\{\phi_{q' \leq q}\}$ from LSF to define $\Phi_{AKF}$. Since Kalman noise parameters ($\sigma^2, R$) are subsequently auto-tuned using a Bayes Risk optimisation procedure, we optimise over potentially remaining model errors and measurement noise.
\\
\\
The choice $q$ for AKF and LSF is set by LSF while training LSF models. In general, LSF performance improves as $q$ increases and a full characterisation of model selection decisions for LSF are given in \cite{mavadia2017}. An absolute value of $q$ is somewhat arbitrary as it is relative to the extent to which a true $f$ is oversampled. For all analysis, we fix the ratio $q \Delta t = 0.1 [a.u.]$, where the experimental sampling rate is $1/\Delta t$ and $\{\phi_{q' \leq q}\}$ are identical in AKF and LSF. For simplicity, we fix a high $q$ model for all numerical experiments considered in this manuscript such that they exhibit numerical convergence behaviour during LSF training. In particular, numerical convergence for LSF means an analysis of errors generated as a gradient descent optimiser is used to learn autoregressive coefficients. Our choice of high $q$ is such that (a) state estimation errors gradually reduce with the number of iterations during a gradient descent optimisation in LSF, and (b), we operate in regimes where net state estimation error at the \textit{end} of a gradient descent optimisation exhibits diminishing returns as $q$ is increased in the underlying LSF model. Implementation details of gradient descent optimisation for LSF are relegated to \cite{mavadia2017}. We numerically confirm that gains for finely tuning $q$ for both LSF and AKF are insignificant for comparisons made in this manuscript.
\\
\\
The structure of AKF above is well-studied in classical engineering and control applications (e.g. a recursive least squares algorithm for adaptive feedforward control \cite{moon2006real} ) and presents opportunities to leverage existing knowledge for quantum control strategies.

% While AKF recasts the dynamics of LSF in recursive form, the predictive performance from LSF in \cite{mavadia2017} is expected to be equivalent to AKF in low measurement noise regimes as they share a common dynamical model. In high measurement noise regimes, a Kalman framework should enable additional measurement noise filtering through the regularising effect of $R$. 

\subsubsection{Liska Kalman Filter with Fixed Basis (LKFFB)}
In LKFFB, we probe stochastic qubit dynamics using a collection of oscillators.  We project our measurement record on $J^B$ oscillators with fixed frequency $\{ \omega_0^B j, j = 1, \hdots, J^B\}$. The structure of this Kalman filter, referred to as the Liska Kalman Filter (LKF), was developed in \cite{livska2007}. We incorporate a fixed basis to enable our application, yielding a Liska Kalman Filter with a Fixed Basis (LKFFB).
\\
\\
We track instantaneous amplitudes and phases explicitly for each basis oscillator With explicit phase and amplitude tracking, we enable state predictions by combining learned amplitudes and phases and projecting forwards in time. The superscript $ ^B$ indicates Fourier domain information about an algorithmic basis, as opposed to information about the true (unknown) dephasing process, and we drop this superscript for convenience.
\\
\\ 
For our application, the true hidden Kalman state, $x$, is a collection of sub-states, $x^j$, for each $j^{th}$ oscillator. Each sub-state is labeled by a real and imaginary component:
\begin{align}
x_n & \equiv \begin{bmatrix} x^{1}_{n} \hdots x^{j}_{n} \hdots x^{J}_{n} \end{bmatrix} \\
x^{j,1}_{n} & \equiv \text{estimates real $f$ component for $\omega_j$} \\
x^{j,2}_{n} & \equiv \text{estimates imaginary $f$ component for $\omega_j$} \\
x^j_n &\equiv \begin{bmatrix} x^{j,1}_{n} \\ x^{j,2}_{n} \\ \end{bmatrix} \equiv \begin{bmatrix} A^j_{n} \\ B^j_{n}  \end{bmatrix}
\end{align} 
We track the real and imaginary parts of the Kalman sub-state  simultaneously in order calculate the instantaneous amplitudes ($\norm{x^j_n}$) and phases ($\theta_{x^j_n}$)  for each Fourier component:
\begin{align}
\norm{x^j_n} & \equiv \sqrt{(A^j_{n})^2 + (B^j_{n})^2} \\
\theta_{x^j_n} & \equiv \tan{\frac{B^j_{n}}{A^j_{n}}}
\end{align}
We may probe dephasing noise to an arbitrarily high resolution for tracking qubit dynamics by choosing an arbitrarily high value for the ratio $J/\omega_0$ when defining the computational basis.
\\
\\
The dynamical model for LKFFB is a stacked collection of independent oscillators. The sub-state dynamics match the formalism of a Markovian stochastic process defined on a circle for each basis frequency, $\omega_j$, as in \cite{karlin2012first}. We stack $\Phi(j \omega_0 \Delta t) $ for all $\omega_j$ along the diagonal to obtain the full dynamical matrix for $\Phi_n$:
\begin{align}
\Phi_{n} & \equiv \begin{bmatrix} 
\Phi(\omega_0 \Delta t)\hdots 0  \\ 
 \hdots \Phi(j\omega_0 \Delta t) \hdots \\
0 \hdots \Phi(J \omega_0 \Delta t)  \end{bmatrix}\\ 
\Phi(j \omega_0 \Delta t) &\equiv \begin{bmatrix} \cos(j \omega_0 \Delta t) & -\sin(j \omega_0 \Delta t) \\ \sin(j \omega_0 \Delta t) & \cos(j \omega_0 \Delta t) \\ \end{bmatrix} \label{eqn:ap_approxSP:LKFFB_Phi} 
\end{align}
% In standard Kalman filters, the recursion for the state $x$ can be decoupled for the recursion required to estimate $P$, where the former depends on measurement observations but the latter can be propagated in the absence of data via the so-called Ricatti equation [REFS]. One advantage of a decoupled state estimation and state variance estimation procedure is that one can precalculate Kalman gains and assess filter performance through the recursion of $P$ alone in the absence of measurement data. The AKF is an example where Kalman gains can be pre-calculated, for example, to aid an FPGA implementation in the laboratory. 
% \\
We observe numerically that instantaneous amplitude and phase information for different basis components are resolved at different timescales while the filter is receiving an incoming stream of measurements (see Appendices). In \cite{livska2007}, a state dependent process noise shaping matrix is introduced to enable potentially non-stationary instantaneous amplitude tracking in LKKFB for each individual oscillator: % In simulated Kalman filtering runs, amplitude and phase estimates for long correlation lengths appear non-stationary over most of the run, and display time-stationarity only near the end of the run. [NOT SURE]
\begin{align}
\Gamma_{n-1} &\equiv \Phi_{n-1}\frac{x_{n-1}}{\norm{x_{n-1}}}
\end{align}
For the scope of this manuscript, we retain the form of $\Gamma_{n}$ in our application even if true qubit dynamics are covariance stationary. As such, $\Gamma_{n}$ depends on state estimates $x$. For this choice of $\Gamma_{n}$, we deviate from classical Kalman filters because recursive equations for $P$ cannot be propagated in the absence of measurement data. Consequently, Kalman gains cannot be pre-computed prior to experimental data collection. Details of gain pre-computation in classical Kalman filtering can be found in standard textbooks (e.g. \cite{grewal2001theory}).
\\
\\
We obtain a single estimate of the true hidden state by defining the measurement model, $H$, by concatenating $J$ copies of the row vector $[1 0]$ :
\begin{align}
H & \equiv \begin{bmatrix} 1 0 \hdots 1 0 \hdots 1 0 \end{bmatrix}
\end{align}
Here, the unity values of $H$ pick out and sum the Kalman estimate for the real components of $\state$ while ignoring the imaginary components, namely, we sum $x^{j,1}_{n}$ for all $J$ basis oscillators.
\\
\\
There are two ways to conduct forward prediction for LKFFB and both are numerically equivalent for the choice of basis outlined in Appendices: namely, we set the Kalman gain to zero and recursively propagate using $\Phi$. Alternatively, we define a harmonic sum using the basis frequencies and learned $\{\norm{x^j_n}, \theta_{x^j_n} \}$.  This harmonic sum can be evaluated for all future time to yield forward predictions a single calculation. 

\subsubsection{QKF}

In QKF, we implement a Kalman filter that acts directly on `0' or `1' outcomes. To reiterate the discussion of  \cref{fig:main:Predive_control_Fig_overview_17_one}(a), this means that the measurement action in QKF must be (a) non-linear and (b) receive quantised measurement data. This holds true irrespective of our dynamical model, $\Phi$.  Since we wish to test the performance of a complex measurement action in QKF, we freeze the dynamical model in QKF to be identical to AKF. With unified notation across AKF and QKF, we define a measurement model $h(x)$ and its Jacobian, $H$ as:
\begin{align}
z_n &  \equiv h(f_n) \equiv 0.5\cos(\state_{n}) \\
& \equiv h(x_n[0]) \\
\implies H_n &\equiv \frac{d h(\state_n)}{d\state_n} =  -0.5\sin(\state_{n})
\end{align}
During filtering, QKF applies $h(x)$ to compute the residuals when updating the true Kalman state, $x$. The Jacobian of $h(x)$, $H_n$, is used to propagate the state variance estimate and to compute the Kalman gain. The linearisation of $h(x)$ by $H_n$ holds if errors during the filtering process, including model errors in dynamical propagation, remain small. 
\\
\\
The entity $z$ is associated with an abstract `signal' - a likelihood function for a single qubit measurement in \cref{eqn:main:likelihood}. We note that the bias of a coin flip, namely, $ P(d_n|f_n, \tau, t) \propto z_n$, cannot be measured directly but only inferred in the frequentist sense for a large number of parallel runs, or in the Bayesian sense, by deconstructing the problem further using Bayes rule. In our application, we track  the correlated phase sequence $f$ as our Kalman hidden state, $x$. Subsequently, we extract an estimate of the true bias, $z$, as an unnatural application of the Kalman measurement model.  
\\
\\
The sequence $z$ is not observable, but can only be inferred over a large number of experimental runs. To complete the measurement action, we implement a biased coin flip within the QKF filter given $y$.   While the qubit is naturally quantised, we require a theoretical model, $\mathcal{Q}$, to generate quantised measurement outcomes with statistics that are consistent with Born's rule. If $z$ was a real signal, one could use \cite{karlsson2005,widrow1996} to encode $z$ into a binary sequence. This is a classical linear transformation where one discretises the amplitude of any signal by discretising the probability distribution of the underlying Gaussian errors generated from quantisation of a continuous amplitude value to its nearest allowed level. We modify the procedure in \cite{karlsson2005} to encode $z$ using biased coin flips. Notationally, we represent a black-box quantiser, $\mathcal{Q}$, that gives only a $0$ or a $1$ outcome based on $y_n$:
\begin{align}
d_n &= \mathcal{Q}(y_n)\\
&=  \mathcal{Q}(h(\state_n) + v_n)
\end{align}
\\
\\ 
Our model for projective measurements are biased coin flips where the bias of the coin is stochastically drifting due to $\{ y_n\}$:
\begin{align}
P(d_n | y_n, \state_{n}, \tau) & \equiv \mathcal{B}(d_n=n=1;p= y_n + 0.5 ) \label{eqn:main:qkf:binomial}
\end{align}
We saturate values of $|y_n| \leq 0.5$ and \cref{eqn:main:qkf:binomial} defines a biased coin flip used in QKF. We comment briefly on the statistical description of the action of the coin-flip quantiser below such that the machinery outlined in \cite{karlsson2005} can be applied to the QKF for future analysis. In particular, a binomial distribution parameterised by a random variable $y_n$ means that $\mathcal{Q}$ defines the likelihood of getting a $0$ or a $1$ after marginalising over all possible values of $y_n$.  
\begin{align}
\mathcal{Q}: & P(d_n | \state_{n}, \tau), \quad |y_n| \leq b = 0.5\\
& \equiv  \int (P(y_n | \state_{n}, \tau) * \mathcal{U}(b) ) P(d_n | y_n, \state_{n}, \tau) dy_n \\
\mathcal{U}(b) & \equiv \mathcal{U}(-b, b)
\end{align}
%  The rationale for enabling this quantisation procedure comes from the statistical study for amplitude quantisation of analogue signals using an $m$-bit quantiser, where the continous amplitudes of a analogue signal are discreted into $2^m$ levels. In our application, the continuous amplitude trace, while not a real signal, is the likelihood function in \cref{eqn:main:likelihood}. This function is discretised in allowed values of $ [0,1]$. Such a discretisation procedure corresponds to amplitude quantisation using a single bit ($m=1$) quantiser, in classical engineering, but further modified to allowed for biased coin flips described by $\mathcal{Q}$. Details of this  scheme are provided in Supplementary Information. 
The convolution with a uniform distribution arises from the need to saturate a Gaussian distributed  $y_n$ between allowed values $|y_n| \leq b = 0.5$ for our application such that the resulting probability distribution of $y_n$ retains positivity (see Appendices for details). 
\\
\\
The definitions of $\{ \mathcal{Q}, h(x_n), H_n \}$ in this subsection, and $\{x, \Phi, \Gamma\}$ from AKF completely specify the QKF algorithm for single shot measurement record depicted in \cref{fig:main:Predive_control_Fig_overview_17_one} (a).  
\\
\\
\subsection{Gaussian Process Regression (GPR)}

% We outline how a GPR framework learns dephasing noise correlations and uses learned information for forward prediction of the qubit state. Stochastic qubit dynamics are governed by dephasing noise correlations relations.
In GPR, dephasing noise correlations in the measurement record can be learned if one projects data on a distribution of Gaussian processes, $P(\state)$ with an appropriate encoding of their covariance relations via a kernel, $\Sigma_\state^{i,j}$. In a linear measurement regime, let $\state_n$ be the true random phase belonging to the process $\state$ at time step $n$. Our measurement record is corrupted by additive zero mean white Gaussian noise, $v_n$ with scalar covariance strength $R$, yielding scalar noisy observations $y_n$:
\begin{align}
y_n &= \state_n + v_n \\
v_n & \sim \mathcal{N}(0, R) \quad \forall n
\end{align}
Under linear operations, the distribution of measured outcomes, $y$, is also a Gaussian. The  mean and variance of $P(y)$  depends on the mean $\mu_\state$ and variance $\Sigma_\state$ of the prior $P(\state)$, and the mean $\mu_v \equiv 0$ and variance $R$ of the measurement noise, $v_n$: 
\begin{align}
\state & \sim P_\state(\mu_\state,\Sigma_\state ) \\
y & \sim P_y(\mu_\state,\Sigma_\state + R ) 
\end{align}
For covariance stationary $\state$, correlation relationships depend solely on the time lag, $v \equiv \Delta t|n_i - n_j|$ between any two random variables at $t_i, t_j$.  An element of the covariance matrix, $\Sigma_\state^{i,j}$, corresponds to one value of lag, $v$, and the correlation for any given $v$  is specified by the covariance function, $R(v)$:
\begin{align}
\Sigma_\state^{i,j} & \equiv R(v_{i,j}) 
\end{align}
Any unknown parameters in the encoding of correlation relations via $R(v)$ are learned by solving the optimisation problem in \cref{sec:main:Optimisation}. The optimised GPR model is then applied to new datasets corresponding to new realisations of the dephasing process. Let indices $n,m \in N_T \equiv [-N_T, 0]$ denote training points, and $n^*,m^* \in N^* \equiv [-N_T, N_P]$ denote testing (including prediction) points in machine learning language. We now define the joint distribution $P(y,\state^*)$, where $\state^*$ is our prediction for the true process at test points: 
\begin{align}
\begin{bmatrix} \state^* \\y \end{bmatrix} & \sim \mathcal{N} (\begin{bmatrix} \mu_{\state^*} \\ \mu_y
\end{bmatrix} , \begin{bmatrix}   K(N^*,N^*)&K(N_T,N^*) \\ K(N^*,N_T) & K(N_T,N_T) + R \end{bmatrix} )
\end{align}
The additional `kernel' notation $\Sigma_\state  \equiv K(N_T, N_T)$ is ubitiquous in GPR. $K(N_T, N_T)$ depicts an $N_T$ by $N_T$ matrix where the diagonals correspond to $v=0$ and $i, j$-th off-diagonal element correspond to $|i-j|$ lag values. The kernel is calculated for each value of $v$ in the matrix.  we include it to help provide visibility of the time domain set of points over which the covariance function is being calculated. Following \cite{rasmussen2005gaussian}, the moments of the conditional predictive distribution $P(\state^*|y)$ can be derived from the joint distribution $P(y,\state^*)$ via standard Gaussian identities:
\begin{align}
\mu_{\state^*|y} &= \mu_\state + K(N^*,N_T)(K(N_T,N_T) + R )^{-1} (y - \mu_y) \\
\Sigma_{\state^*|y} &= K(N^*,N^*) \nonumber \\
& - K(N^*,N_T)(K(N_T, N_T) + R)^{-1}K(N_T,N^*) 
\end{align}
The above prediction procedure holds true for any choice kernel, $R(v)$. In any GPR implementation, the dataset, $y$, constrains the prior model yielding an aposteriori predictive distribution. The mean of this predictive distribution, $\mu_{\state^*|y}$, are the state predictions for the qubit under dephasing at test points $\in N^*$.
\\
\\
Our choice of a `periodic kernel' in this manuscript encodes a covariance function which is theoretically guaranteed to approximate any zero mean covariance stationary process, $f$, in the mean square limit, namely, by having the same structure as a covariance function for trigonometric polynomials with infinite harmonic terms \cite{solin2014explicit, karlin2012first}. The sine squared exponential kernel represents an infinite basis of oscillators and can be summarised as:
\begin{align}
R(v) &\equiv \sigma^2 \exp (- \frac{2\sin^2(\frac{\omega_0 v}{2})}{l^2}) 
% R(v) &=  \sigma^2 \exp (- \frac{1}{l^2}) \sum_{n = 0}^{\infty} \frac{1}{n!} \frac{\cos^n(\omega_0 v)}{l^{2n}} \\
\end{align} A derivation is provided in Appendices using commentary in \cite{solin2014explicit}.
% In the Supplementary Information, we follow \cite{solin2014explicit} to show that the periodic kernel reduces to the covariance function describing trigonometric polynomials if spectral components are truncated to a finite number, $J$:
% \begin{align}
% R(v) &- \sigma^2 p_{0,J}  = \sigma^2 \sum_{j=0}^{J} p_{j,J} \cos(j\omega_0 v)\\
% p_{0,J} & \equiv \frac{1}{2} \exp (- \frac{1}{l^2}) \sum_{\alpha = 0}^{\alpha = \lfloor\frac{J}{2}\rfloor} \frac{1}{(2l^2)^{(2\alpha)}} \frac{1}{(2\alpha)!} \binom{2\alpha}{\alpha} \label{eqn:p0J}\\
% p_{j,J} & \equiv \exp (- \frac{1}{l^2}) \sum_{\beta = 0}^{\beta = \lfloor\frac{J-j}{2}\rfloor} \frac{2}{(2l^2)^{(j + 2\beta)}} \frac{1}{(j + 2\beta)!} \binom{j + 2\beta}{\beta} \label{eqn:pjJ} \\
% \omega_0 &\equiv \frac{\omega_j}{j}, j \in \{0, 1,..., J\} 
% \end{align}
The sine-squared kernel is summarised by two key hyper-parameters: the frequency comb spacing for our infinite basis of oscillators, $\omega_0$, and a dimensionless length scale, $l$. We use physical sampling considerations to approximate their initial conditions prior to an optimisation procedure, namely, that the longest correlation length encoded in the data, $N \Delta t $, sets the frequency resolution of the comb, and the scale at which changes in $f$ are resolved is of order  $\Delta t$:
\begin{align}
\frac{\omega_0}{2\pi} & \sim  \frac{1}{\Delta t N} \\
l & \sim \Delta t
\end{align} 
We exclude popular kernel choices from analysis. They include the Gaussian kernel (RBF); a scale mixture of Gaussian kernels (RQ); the Matern family of kernels; and a spectral mixture of Gaussian kernels \cite{rasmussen2005gaussian, tobar2015learning}. These exclusions are based on kernel properties as follows. An arbitrary scale mixture of zero mean Gaussian kernels will probe an arbitrary area around zero in the Fourier domain, as schematically depicted in \cref{fig:main:Predive_control_Fig_overview_17_two}(a). While such kernels capture the continuity assumption ubitiquous in machine learning, they are structurally inappropriate in probing a dephasing noise process of an arbitrary power spectral density (e.g. ohmic noise).  Matern kernels of order $q + 1/2$ correspond to a certain class of random process, known as autoregressive processes of order $q$, which are naturally considered under AKF in this manuscript. We do not duplicate our investigations under GPR. A class of GPR methods, namely, spectral mixture kernels and sparse spectrum approximation using GPR have been explored in \cite{wilson2013, quia2010}. However, these techniques require efficient optimisation procedures to learn many unknown kernel parameters, whereas the sine-squared exponential is parameterised only by two hyper-parameters.  Some literature suggests the use of kernel which is the product of the periodic kernel with kernels representing white noise \cite{klenske2016gaussian}. A detailed investigation of the application of spectral mixture and kernel product methods for forward prediction beyond pattern recognition and with limited computational resources, is beyond the scope of this manuscript. 

\section{Noise Engineering and Simulated Measurements\label{sec:main:NoiseEngineering}}

We engineer true environmental dephasing through the procedure described in \cite{soare2014} to enable future experimental verification of simulations reported in this manuscript. We generate $ N = N_T + N_P$ number of points in one sequence of true dephasing noise spaced $\Delta t $ apart in time. We define the discretised process, $\state$, as:
\begin{align}
\state_n &= \alpha \omega_0 \sum_{j=1}^{J} j F(j)\cos(\omega_j n \Delta t + \psi_j) \\
F(j) & = j^{\frac{p}{2}-1} 
\end{align}

Using the notation of \cite{soare2014}, $\alpha$ is an arbitrary scaling factor, $\omega_0$ is the fundamental spacing between true adjacent discrete frequencies, such that $\omega_j = 2 \pi f_0 j =\omega_0 j, j = 1, 2, ...J$. For each frequency component, there exists a uniformly distributed random phase, $\psi_j \in [0, \pi]$. The free parameter $p$ allows one to specify an arbitrary shape of the true power spectral density of $\state$. In particular, the free parameters $\alpha, J, \omega_0, p$ are true dephasing noise parameters which any prediction algorithm cannot know beforehand. With uniformly distributed phase information, it is straightforward to show that $f$ is mean square ergodic and covariance stationary \cite{gelb1974applied}. However, each $n^{th}$ member of the sequence $f$ is Gaussian distributed only by the central limit theorum for large $J$ (in simulations, $J \approx 20$ satisfies Gaussianity, see Appendices). For results in this manuscript, we choose $p=0$ flat top spectrum - this choice of a power spectral density theoretically favors no particular choice of algorithm. For linear regimes, we choose $\alpha$ arbitrarily high relative to machine precision for recursive Kalman calculations, and in non-linear regimes, we use $\alpha$ to rescale $f \in [0, \pi]$ before taking projective measurements. 
\\
\\
While $\{ \alpha, J, \omega_0, p \}$ represents true, unknown environmental dephasing, the choice of $\{N, \Delta t\} $ represents a sampling rate and Fourier resolution set by the experimental protocol. We choose regimes where Nyquist $r \gg 2$.
\\
\\
In generating noisy simulated datasets, we corrupt a noiseless measurement by additive Gaussian white noise. Since $\state$ is Gaussian, the measurement noise level, $NL$ is defined as a ratio between the standard deviation of additive Gaussian measurement noise, $\sqrt{R}$ and the maximal spread of random variables in any realisation $\state$. We approximate this computationally as three sample standard deviations, $\hat{\sigma}_\state$ of one realisation of true $\state$:
\begin{align}
NL = \frac{\sqrt{R}}{3\hat{\sigma}_\state}
\end{align}
To establish a link to GPR notation, $\hat{\sigma}_\state \equiv \sqrt{\hat{\Sigma}_f^{i,i}}$, where the superscript $\hat{}$ denotes sample statistics. This computational procedure enables a consistent application of measurement noise for $f$ from arbitrary, non-Markovian power spectral densities. For the case where binary outcomes are required, we apply a biased coin flip using \cref{eqn:main:qkf:binomial}.

\section{Algorithmic Optimisation \label{sec:main:Optimisation}}

All algorithms in this manuscript employ machine learning principles to auto-tune unknown design parameters. The physical intuition associated with optimising our filters is that we are cycling through a large class of general models for environmental dephasing. This allows each filter to track stochastic qubit dynamics under arbitrary covariance stationary, non-Markovian dephasing. 
\\
\\
In borrowing machine learning principles, we design and solve an optimisation problem to discover unknown KF and GPR design parameters using simulated training datasets. Our approach represents the simplest form of tuning algorithm parameters based on data. Sophisticated, data driven model selection schemes are described for both KF and kernel learning machines (such as GPR) in literature (e.g. \cite{arlot2009data, vu2015understanding}). We chose minimal computational complexity to enable nimble deployment of KF and GPR algorithms in realistic laboratory settings, particularly since LSF optimisation is extremely rapid for our application \cite{mavadia2017}.
\\
\\
For arbitrary power spectral densities of dephasing noise, an optimisation problem posed for Kalman Filters is extremely difficult to solve with standard local optimisers. There are no theoretical bounds on the values of ($\sigma, R$) and consequently, large, flat regions are generated by the Bayes Risk function. Further, the recursive structure of the Kalman filter means that no analytical gradients are accessible for optimising a choice of cost function and a large computational burden is incurred for any optimisation procedure. Beyond standard local gradient and simplex optimisers, we consider coordinate ascent \cite{abbeel2005} and particle swarm optimisation techniques \cite{robertson2017particle} as promising, nascent candidates and their application remains an open research question. % This generates large, flat regions of Bayes Risk which are difficult for many of the standard local optimisers to traverse. Such a failure can be diagnostically reproduced by engineering narrow dips on large, flat plains and documenting the performance of standard gradient and simplex algorithms in finding these features. An additional issue is that 
\\  
\\
For optimisation of KF filters in this manuscript, we randomly distribute $\{(\sigma_{k}, R_{k}), k=1, \hdots K \}$ pairs over several orders of magnitudes in two dimensions:
\begin{align}
\sigma_k, R_k &\equiv \alpha_0 10^{\alpha_1} \\
\alpha_0 & \sim U[0, 1]\\
\alpha_1 & \sim U[\{ -p_{max}, -p_{max} + 1,  \hdots,  p_{min}\}]
\end{align}
Scale magnitudes are set by $\alpha_1$, a random integer chosen with uniform probability over $\{ -p_{max}, \hdots, p_{min} \}$ where we set $p_{min} = 3, p_{max} = 8$  such that $p_{max}$ is higher than machine floating point precision $\approx 10^{-11}$. Uniformly distributed floating points for $\sigma_k, R_k $ in each order of magnitude is set by $\alpha_0$. 
\\
\\
We generate a sequence of loss values $\{L(\sigma_k, R_k), k = 1, \hdots K\}$:
\begin{align}
L(\sigma_k, R_k) \equiv  \sum_{n=1}^{N'} L_{BR}(n | I= \{\sigma_k, R_k \})
\end{align}
$L_{BR}(n | I= \{\sigma_k, R_k \})$ is given by \cref{eqn:main:sec:ap_opt_LossBR}. Time horizons for state estimation ($N' = |N_{SE}| , N_{SE} \in  [-N_{T}, 0]$) or prediction ($N' = N_{PR}, N_{PR}  \in [0, N_{P}]$) are chosen such that the sequence $\{L(\sigma_k, R_k) \}$ defines sensible shapes of the loss function over parameter space and the numerical experiments in this manuscript. As illustrative guidelines, a choice of small $N_{SE}$ values ensures that we assess state estimates only once Kalman Filters are approaching convergence. Meanwhile, large $N_{PR}$ values will flatten the true prediction loss function as long term prediction errors will dominate and obscure low loss values for short term prediction horizons of interest, namely, small $n>0$. One may incorporate a different prior over $I= \{\sigma_k, R_k \}$ and/or construct optimisation problem over different choices of cost function (e.g. maximum likelihood). Our approach is computationally efficient given the recursive nature of the Kalman filter and a quantitative review of Bayesian hyper-parameter optimisation procedures is beyond the scope of this manuscript. 
\\
\\
We accept an optimal candidate ($\sigma^*, R^*$) which minimises the Bayes state estimation risk over $K$ trials by comparing the low loss regions for state estimation and prediction. We define a low loss regions for state estimation and prediction as being the set $ \{ (\sigma_k, R_k) : L(\sigma_k, R_k) < 0.1 L_0 \}$ where the $10 \%$ low loss threshold is defined relative to the median risk, $L_0 \equiv \text{median}\{  L(\sigma_k, R_k) \}$, incurred during the optimisation procedure over $K$ trials. If the low loss region in state estimation has an overlap with low loss regions during prediction in parameter space, and optimal ($\sigma^*, R^*$) candidate falls within this overlap region, then we accept that the KF filter is sensibly tuned. If an overlap of low loss regions for state estimation and prediction does not exist, or if the optimal candidate does not reside in the overlap region, then the optimisation problem is deemed `broken' as training is uncorrelated with prediction performance. 
\\
\\
The GPR optimisation procedure is not the same as for Kalman Filtering. In GPR, no recursion exists and analytic gradients are accessible to simplify the overall optimisation problem. Instead of minimising Bayes state estimation risk, we follow a popular practice of maximising the Bayesian likelihood. The set of parameters in GPR, $I = \{\sigma, R, p, l \}$ require optimisation. We use physical arguments in \cref{sec:main:OverviewofPredictive Methodologies} to provide initial conditions and/or to constrain the optimisation of $\{ p, l\}$. 

\section{Algorithm Performance Characterisation \label{sec:main:Performance}}

In the results to follow, our metric for characterising performance of optimally tuned algorithms will be the normalised Bayes prediction risk:
\begin{align}
\normpr \equiv \frac{L_{BR}(n|I)}{\langle \state_n^2 \rangle_{f, \mathcal{D}}} 
\end{align}
A desirable forward prediction horizon corresponds to maximal $n \in [0, N_P]$ for which normalised Bayes prediction risk at all time steps $n' \leq n$ is less than unity. We compare the difference in maximal forward prediction horizons between algorithms in context of realistic operating scenarios. 
% In the results to follow, we compare the difference in maximal forward prediction horizons between algorithms for the same numerical experiment. We consider these differences in context of realistic operating scenarios, namely: engineering true dephasing noise that appears `close to continuous' relative to an algorithm's computation resolution in the Fourier domain; enabling imperfect projection of data on the algorithmic basis (GPR, LKFFB); reducing the over-sampling ratio; and increasing measurement noise strength.
% In the non-linear regime, we simulate measurements by applying a biased coin flip to get a single shot outcome from a binomial distribution with $n=k=1$ where the bias of the coin is parameterised using $\{ y_n \}$. For the results presented in the non-linear regime, we use an AR(2) process to define the true qubit state. The change in noise generation from linear to non-linear regimes reflects a desire to isolate true noise dynamics and focus on the non-linear quantised measurement action in the latter case.
% Our numerical approach is to 
% - in the linear regime, we simulate stochastic detunings according to a PSD which is flat top in this manuscript
% - in the non-linear regime, generate stochastic detunings and apply a biased coin flip to get a 0 or a 1 outcome
% - we add additive Gaussian white noise - define noise strength
% 5 things: (a) no figure to justify oversampling, and (b) the low power of RBF and RQ kernels about zero should help in extracting long time correlations. (c) Low loss regions and noise level as defined in numerics. (d) REDO so inset is a separate figure at [1,0] for prediction risk only. (e) QKF average the gain squared. (f) add labels on LKFFB graphs; remoe purple background in Time domain . (g) font type and size to be standardised. 
\subsection{KF (Linear Measurement)}
\begin{figure}
    \includegraphics[scale=1.0]{fig_data_all}
    \caption{\label{fig:main:fig_data_all} We plot state predictions against time steps $n > -50$ obtained from optimised AKF, LKFFB and LSF algorithms for $K=75$ trials. We plot true $f$ [black] and measurement data [grey dots], where measurements for $n \in [-N_T, -50]$ are omitted [(left)]. A single run contributes to Bayes prediction risk over an ensemble of $M=50$ runs normalised against predicting the mean, $\mu_\state$, of dephasing noise [right]. A normalised risk $<1$ for $n > 0$ defines a desirable forward prediction horizon. A single run phase sequence $f$ is drawn from a flat top spectrum with $J$ true Fourier components spaced $\omega_0$ apart and uniformly randomised phases $\in [0, 2\pi]$. A trained LKFFB is implemented with comb spacing $\omega_0^B / 2\pi = 0.5$ Hz and $J^B =100$ oscillators; while trained AKF / LSF models correspond to high $q = 100$. Relative to LKFFB,  (a) and (b) correspond to perfect projection $\omega_0 / \omega_0^B  \in Z $ for $J= 40, \omega_0 / 2\pi = 0.5$ Hz. In (c) and (d), we simulate realistic noise with $\omega_0 / \omega_0^B  \notin Z$, $J = 45000$, $\omega_0 / 2\pi = \frac{8}{9} \times 10^{-3}$ Hz such that $>500$ number of true components fall between adjacent LKFFB oscillators. For (a)-(d), $N_T = 2000, N_P = 100$ steps, $\Delta t = 0.001s$ such that we fulfill $r_{Nqy} \gg 2$, $N_T / \Delta t < \omega_0/2\pi$. Measurement noise level $ NL= 10\%$.}
\end{figure} 

\subsubsection{General predictive performance}

\cref{fig:main:fig_data_all} depicts predictive performance of LSF, AKF and LKFFB algorithms using a linear measurement record. In (a), we depict a single time domain run. We engineer a case when perfect projection of the true state on the LKFFB basis is \textit{theoretically} achievable. In (b), we plot $\normpr$ prediction risk such that one run in the ensemble is depicted in (a).  In (a) and (b), we find that LKFFB learns all information about the dephasing noise and qubit state dynamics are nearly perfectly predictable when perfect projection is theoretically enabled. Meanwhile, AKF and LSF share autoregressive coefficients and therefore, both algorithms have nearly identical $\normpr$ prediction risk trajectories. Both AKF and LSF cannot extract all information from dephasing noise.
\\
\\
We relax perfect projection relative to LKFFB basis in \cref{fig:main:fig_data_all}  (c) and (d). We engineer $f$ such any true spectral component in $f$ can never be perfectly projected on an LKFFB basis oscillator. Further, we enforce that  computational resolution for any practical application will be limited, namely, $J \gg J^B$ number of LKFFB basis oscillators. Meanwhile, no explicit basis considerations apply to AKF/LSF. A single run is plotted in (c), and $\normpr$ prediction risk is plotted in (d). We compare results from LKFFB and AKF and observe that autoregressive dynamics of LSF / AKF enable a larger forward prediction horizon than LKFFB. 
\\
\\
\begin{figure}
    \includegraphics[scale=1.0]{fig_data_maxfwdpred}
    \caption{\label{fig:main:fig_data_maxfwdpred} From (a)-(c), we plot $\normpr$ against forward time $n \in [0, N_P]$ for LSF, AKF and LKFFB. In each panel, we vary true $f$ cutoff relative to an apriori noise bandwidth assumption $f_B$ such that $\omega_0 / 2\pi = 0.5$ Hz, $J = 20, 40, 60, 80, 200$. We depict the maximal forward prediction horizon for each case using vertical lines at approximately $ n_{max} \mid  \normpr \lesssim 0.8 < 1$, where a threshold less than unity is chosen to reduce artifacts arising from Bayes risk oscillations around mean behaviour. For LKFFB, $\omega_0^B / 2\pi = 0.497$ Hz for $j \in J^B = 100$ oscillators. For LSF and AKF, $q = 100$. In all cases,  $N_T = 2000, N_P = 50$ steps, $\Delta t = 0.001s, r_{Nqy}=20$, with optimisation performed for $M=50$ runs, $K=75$ trials and measurement noise level $NL = 1\%$.} 
\end{figure} 
In \cref{fig:main:fig_data_maxfwdpred}, we plot $\normpr$ prediction risk against an increasing ratio of $Jf_0 / f_B$ where $f_B$ is the true dephasing noise bandwidth assumption in specifying the basis for LKFFB. We engineer $f$ with a power spectral density comparable to \cref{fig:main:fig_data_all} (a)-(b) but mildly detune the LKFFB basis to engineer imperfect projection. We confirm that as oversampling is reduced, the absolute forward prediction horizon shrinks, namely, $\normpr > 1 $ for increasing small $n>0$. The forward prediction horizon is approximately quantified using vertical lines. We confirm that absolute prediction horizons for any algorithm are arbitrary in the sense that they can be increased via increased oversampling. We restrict our analysis to comparative statements between algorithms for future results. 

\subsubsection{Power spectral density extraction}

We plot learned Fourier domain information associated with \cref{fig:main:fig_data_maxfwdpred} in \cref{fig:main:fig_data_specrecon}. For LKFFB, we plot the learned instantaneous amplitudes from a single run [blue dots] against the true dephasing noise power spectral density [black]. 
\\
\\
For AKF/LSF, we extract optimised algorithm parameters to calculate the spectrum using \cref{eqn:main:ap_ssp_ar_spectden} [red dots]. Under the assertion that LSF implements an AR($q$) process, the set of trained parameters, $\{  \{\phi_{q' \leq q}\}, \sigma^2\}$ from LSF and AKF allows us to derive experimentally measurable quantities, including the power spectral density of the dephasing process \cite{brockwell1996introduction}:
\begin{align}
S(\omega) & = \frac{\sigma^2}{2 \pi }\frac{1}{|\Phi(e^{-i\omega})|^2} \label{eqn:main:ap_ssp_ar_spectden} 
\end{align}
Here, we use the same summarised notation, $\Phi(L)$, but  $L$ is no longer the time domain lag operator and has been redefined as $L \equiv e^{-i\omega}$ in the Fourier domain. In all cases except LKFFB in \cref{fig:main:fig_data_specrecon} (d), all algorithms correctly discern the cut-off frequency of true dephasing. The reconstruction enabled by \cref{eqn:main:ap_ssp_ar_spectden} provides additional numerical evidence to validate our assertion. 
\\
\\
\begin{figure}
    \includegraphics[scale=1.0]{fig_data_specrecon}
    \caption{\label{fig:main:fig_data_specrecon} We compare the true power spectrum for $f$ with derived spectral estimates from LKFFB and AKF. From (a)-(d), we vary true $f$ cutoff relative to an apriori noise bandwidth assumption $f_B$ such that $\omega_0 / 2\pi = 0.5$ Hz, $J = 20, 40, 80, 200$. For LKFFB, we use learned amplitude information from a single run ($\propto ||x^j_n||^2 $) with $\omega_0^B / 2\pi = 0.497$ Hz for $j \in J^B = 100$ oscillators. For AKF, we plot \cref{eqn:main:ap_ssp_ar_spectden} using optimally trained $\{\phi_{q' \leq q}\}$ and $\sigma^2$, with order $q = 100$. The zeroth Fourier component and its estimates are omitted to allow for log scaling; and $N_T = 2000, N_P = 50$ steps, $\Delta t = 0.001s, r_{Nqy}=20$, with optimisation performed for $M=50$ runs, $K=75$ trials and measurement noise level $NL = 1\%$.} 
\end{figure} 
Further, we compare LKFFB and AKF/LSF in extracting power spectral density information for true dephasing. We find that spectrum reconstruction from LKFFB is of higher fidelity by several orders of magnitude compared to AKF/LSF, even when learning environments are imperfect. This is true for \cref{fig:main:fig_data_specrecon} (a)-(c) but fails if the true noise bandwidth assumption underpinning all analysis is relaxed, as in \cref{fig:main:fig_data_specrecon} (d). The discrepancy between AKF/LSF spectrum reconstruction and the truth depends on the accuracy of scaling factor given by optimally tuned $\sigma$ in AKF and the spectral content learned via $\{\phi_{q' \leq q}\}$ in LSF in \cref{eqn:main:ap_ssp_ar_spectden}. In contrast, instantaneous amplitudes are tracked in one run of LKFFB and are less susceptible to optimisation over model parameters.

% \begin{widetext}
    \begin{figure*} 
    \includegraphics[scale=1.0]{figure_lkffb_path}
    \caption{\label{fig:main:figure_lkffb_path} 
    We compare LKFFB and AKF performance when a true phase sequence $f$ is generated from a flat top spectrum in (a)-(d) by varying  $\omega_0 / 2\pi = 0.5, 0.499, \frac{8}{9} \times 10^{-3}, \frac{8}{9} \times 10^{-3}$ Hz and $J = 80, 80, 45000, 80000$ respectively. For (a)-(d), we depict normalised Bayes prediction risk for LKFFB, AKF, and LSF against time steps $n>0$. For LKFFB, these regimes correspond to perfect learning in (a); imperfect projection on basis in (b); finite computational Fourier resolution in (c); and a relaxed bandwidth assumption ($f_B < \omega_0 / 2\pi$) in (d). In the panels (e)-(l), we depict optimisation of Kalman noise parameters ($\sigma^2, R$) for LKFFB [top row] and AKF [bottom row] for the four regimes in (a)-(d). Low loss regions represent risk values $< 10\%$ of $L_0$, the median risk incurred during Kalman hyperparameter optimisation for $K=75$ trials of of randomised ($\sigma^2, R$) pairs. Optimal ($\sigma^*, R^*$) minimise state estimation risk. For each trial, a risk point is an expectation over $M=50$ runs of true $f$ and noisy datasets during state estimation ($n \in  [-N_{SE}, 0]$) or prediction ($n \in  [0, N_{PR}]$). We choose $ N_{PR}=N_{SE}=50$ such that the shape of total loss over time steps form sensible optimsation problems and a scan of $N_{PR}, N_{SE}$ values do not appear to simplify our Kalman optimisation problem. We plot optimisation results for LKFFB in (e)-(h) and AKF in (i)-(l). A KF filter is `tuned' if optimal ($\sigma^*, R^*$) lies in the overlap of low loss regions for state estimation and prediction. This condition is violated in (h). KF algorithms are set up with $q = 100$ for AKF; $J^B = 100, \omega_0^B / 2\pi = 0.5$ Hz for LKFFB, with $N_T = 2000, N_P = 100$ steps, $\Delta t = 0.001s, r_{Nqy}=20$ and applied measurement noise level $ NL = 1\%$.}  
    \end{figure*} 
% \end{widetext}

\subsubsection{Model robustness}
We compare the model robustness of LKFFB and AKF in realistic operating environments. In \cref{fig:main:figure_lkffb_path}, these experiments correspond to (a) perfect learning in LKFFB; (b) imperfect projection relative to LKFFB basis; (c) imperfect projection in (b) combined with finite algorithm resolution; and (d), where case (c)  is extended to an ill-specified basis relative to true noise bandwidth. We find that LKFFB performance deteriorates relative to AKF / LSF as pathologies are introduced in \cref{fig:main:figure_lkffb_path} (a)-(d). 
\\
\\
We expose the underlying optimisation results for choosing an optimal $(\sigma^*, R^*)$ for LKFFB in \cref{fig:main:figure_lkffb_path} (e)-(h) and for AKF in \cref{fig:main:figure_lkffb_path} (i)-(l). The overlap area of low loss choices between state estimation (blue) and prediction (purple) Bayes Risk shrinks for LKFFB in \cref{fig:main:figure_lkffb_path} (e)-(g), and regions are disjoint in (h), indicating that training has diminishing returns for LKFFB predictive performance as the algorithm breaks. In contrast, overlap of low loss Bayes Risk regions do not change for AKF across \cref{fig:main:figure_lkffb_path} (i)-(l).

\begin{figure}
    \includegraphics[scale=1.]{fig_data_akfvlsf}
    \caption{\label{fig:main:fig_data_akfvlsf} (a) We plot the ratio of $\normpr$ prediction risk from AKF to LSF against time steps $n>0$.  AKF and LSF share identical $\{ \phi_q \}$ and  a value below $<1$ indicates AKF outperforms LSF. In (i)-(iv), applied measurement noise level is increased from $0.1 - 25 \%$. (b) We plot normalised Bayes Risk against time steps $n>0$ for AKF and LKFFB corresponding to cases (i) -(iv) and confirm a desirable forward prediction horizon underpins ratios in (a). True $f$ is drawn from a flat top spectrum with $\omega_0 / 2\pi = \frac{8}{9} \times 10^{-3}$ Hz, $J = 45000$, $N_T = 2000, N_P = 100$ steps, $\Delta t = 0.001s, r_{Nqy}=20$ such that \cref{fig:main:figure_lkffb_path}(c) corresponds to case (ii) in this figure. Optimisation is performed for $M=50$ runs, $K=75$ trials.}
\end{figure}

\subsubsection{Measurement noise filtering}
Since the AKF algorithm recasts an AR($q$) process from LSF into Kalman form, we confirm that the KF framework enables additional measurement noise filtering for qubit dynamics than LSF alone. In \cref{fig:main:fig_data_akfvlsf} (a), we plot $\normpr$ prediction risk for AKF and LSF as a ratio such that a value greater than unity implies LSF outperforms AKF:
\begin{align}
AKF / LSF \equiv \frac{\normpr ^{AKF}}{\normpr^{LSF}}, \quad n \in [0, N_P]
\end{align}
In cases (i)-(iv), we increase the applied measurement noise level to our noisy datasets $\{ y_n \}$. For the low measurement noise $NL = 0.1\%$ in (i), the ratio $AKF/LSF > 1$ and LSF outperforms AKF. For applied measurement noise level $NL > 1\%$ in (ii)-(iv), we find that $AKF/LSF <1 $ and AKF outperforms LSF in numerical simulations. In \cref{fig:main:fig_data_akfvlsf} (b), we plot $\normpr$ prediction risk for each measurement noise level (i)-(iv). The plots in (b) confirm that all ratios reported in (a) correspond to a desirable forward prediction horizon where both AKF and LSF outperform predicting the mean value of dephasing. 

\subsection{KF (Non Linear, Quantised Measurements)}
We use QKF to test if a Kalman Framework can incorporate single shot qubit outcomes for predictive estimation. To re-iterate, QKF estimates and tracks hidden phase information, $f$, using the Kalman true state $x$, and the associated probability for a projective qubit measurement outcome, $\propto z$ is not inferred or measured directly but given deterministically by Born's rule encoded in the non-linear measurement model, $z = h(x)$. The measurement action is completed by performing a biased coin flip, where $z$ determines the bias of the coin.  The $\text{N.} \langle (z_n - \hat{z}_n)^2 \rangle_{f, \mathcal{D}} $ risk in  \cref{fig:main:fig_data_qkf2} is calculated with respect to  $z$, instead of the stochastic phase sequence $f$, as the relevant quantity parametering qubit state evolution. We investigate if $\text{N.} \langle (z_n - \hat{z}_n)^2 \rangle_{f, \mathcal{D}} < 1, n\in [0, N_P] $ can be achieved for numerical experiments considered previously in the linear regime. In particular, we generate true $f$ defined in numerical experiments in \cref{fig:main:fig_data_maxfwdpred,fig:main:fig_data_specrecon} for $q=100$ such that $f_0 J / f^B = 0.2, 0.4, 0.6, 0.8$.  
\\
\\
\begin{figure}[h!]
    \includegraphics[scale=1.]{fig_data_qkf}
    \caption{\label{fig:main:fig_data_qkf2} We plot Bayes prediction risk for QKF against time steps $n>0$. In (a)-(b), we vary true $f$ cutoff relative to an apriori noise bandwidth assumption such that $J f_0 / f_B = 0.2, 0.4, 0.6, 0.8$ for an initially generated true $f$ in \cref{fig:main:fig_data_specrecon} with $\omega_0/ 2\pi = 0.497 $ Hz, $J = 20, 40, 60, 80$. Measurement noise is incurred on $f$ at $NL = 1 \%$ for the linear measurement record and on $z$ at $NL = 1\%$ level corresponding to the non-linear measurement record. In (a), we obtain $\{\phi_{q' \leq q}\}, q=100$ coefficients from AKF/LSF acting on a linear measurement record generated from true $f$. We re-generate a new truth, $f'$, from an autoregressive process by setting $\{\phi_{q'\leq q}\}, q=100$ as true coefficients and by defining a known, true $\sigma$. We generate quantised measurements from $f'$ and data is corrupted by measurement noise of a true, known strength $R$. Hence, QKF in (a) incorporates true dynamics and noise parameters $\{\{\phi_{q' \leq q} \}, \sigma, R\}$ but acts on single shot qubit measurements. In (b), we use $\{\phi_{q' \leq q} \}, q=100$ coefficients from (a) but we generate quantised measurements from the original, true $f$. We auto-tune QKF noise design parameters in a focused region ($\sigma_{AKF}^* \leq \sigma_{QKF}$, $R_{AKF}^* \leq R_{QKF}$) with with $M=50$ runs, $K=75$ trials. For (a)-(b), forward prediction horizons are shown with $N_T = 2000, N_P = 50$ steps, $\Delta t = 0.001s, r_{Nqy}\gg 2$.}
\end{figure}
\cref{fig:main:fig_data_qkf2} (a) isolates the performance of the measurement action by specifying a true dynamical model and true Kalman noise parameters. To specify true dynamics, we approximate $f$ by $f'$, where $f'$ is generated from a sequence of $\{ \phi_{q'\leq q}\}$ obtained from LSF acting on $f$ in the linear regime considered previously. By using $f'$, QKF incorporates a high $q$ true autoregressive dynamical model $\{ \phi_{q'\leq q}\}$. We generate single shot qubit measurements based on $f'$ and we input true noise parameters $(\sigma, R)$. \cref{fig:main:fig_data_qkf2} (a) depicts that a desirable forward prediction horizon $\text{N.} \langle (z_n - \hat{z}_n)^2 \rangle_{f, \mathcal{D}} < 1, n\in [0, N_P] $ is achieved for sufficiently oversampled regime, and the forward prediction horizon shrinks in $n$ when oversampling is reduced. As in the linear case, the absolute forward prediction horizon is arbitrary relative to $f_0 J / f^B$ and implicitly, an optimisation over the choice of $q$ in our application. % Hence, we confirm that the non-linear, quantised measurement model in QKF enables tracking and forward prediction of qubit state using single shot measurements. 
\\
\\
\cref{fig:main:fig_data_qkf2} (b) depicts QKF performance for a realistic learning procedure. We generate single shot qubit measurements based on the true dephasing $f$. QKF incorporates a learned dynamical model from AKF in the linear regime and we tune $(\sigma, R)$ for QKF. In particular, we explore $\sigma \geq \sigma_{AKF}^*$ to incorporate model errors as $\{\phi_{q' \leq q}\}$ were learned in the linear regime.  We explore $R \geq R_{AKF}^*$ to incorporate increased measurement noise as QKF receives raw data that has not been pre-processed or low pass filtered. The underlying optimisation problems are well behaved for all cases in \cref{fig:main:fig_data_qkf2}(b) [not shown]. As oversampling is reduced, the QKF forward prediction horizon disappears rapidly i.e $\text{N.} \langle (z_n - \hat{z}_n)^2 \rangle_{f, \mathcal{D}} > 1 $ prediction risk for all $n>0$.  However, we confirm that in a highly oversampled regime, it is possible for the QKF to achieve a forward prediction horizon that is slightly better than predicting the mean value for dephasing noise. 

\subsection{GPR} 
In this section, we use a trained GPR model to track a deterministic sine curve with a single Fourier component as our `true' state trajectory, instead of a stochastic qubit dynamics described earlier. % carry a limited interpretation for predictions beyond the measurement record.
\begin{figure}
    \includegraphics[scale=1.]{fig_data_gpr}. 
    \caption{\label{fig:main:fig_data_gpr} In (a)-(d), prediction points $\mu_{\state^*|y}$ [purple] are plotted against time steps, $n$. We plot the true phase sequence,  $f$, [black] and  $f$ at the beginning of the run [red dotted]. Predictions are generated in a single run by a trained GPR model with a periodic kernel corresponding to a Fourier domain basis comb spacing, $\omega_0^B$. Data collection of $N_T$ measurements [not shown] ceases at $n=0$. For simplicity, the true $f$ is a deterministic sine with frequency, $\omega_0$. (a) Perfection projection is possible $\omega_0 / \omega_0^B \in Z$ natural numbers, $\omega_0 = 3$ Hz. Kernel resolution is exactly the longest time domain correlation in dataset, $2 \pi / \omega_0^B \equiv \Delta t N_T \implies \kappa = 0$.   (b) Imperfect projection, with $\omega_0 / \omega_0^B \notin Z$, $\omega_0 / 2 \pi = 3 \frac{1}{3}$ Hz, $\kappa=0$. (c) We increase kernel resolution to be arbitrarily high, $\kappa \gg 0 $, such that $\omega_0 / \omega_0^B \gg 0 \notin Z $ for original $ \omega_0 / 2 \pi = 3$ Hz. (d) We test (b) and (c) for $\kappa \gg0$, $ \omega_0 / \omega_0^B \notin Z$, $\omega_0 / 2 \pi = 3 \frac{1}{3}$ Hz. For all (a)-(d), $N_T = 2000, N_P = 150$ steps, $\Delta t = 0.001s$ and applied measurement noise level $1\%$.} 
\end{figure}
For this simple example, the periodic kernel learns Fourier information in the measurement record enabling interpolation using test-points $n^* \in [-N_T, 0]$ for all cases (a)-(d) in \cref{fig:main:fig_data_gpr}. Time domain predictions $n^* >0$  in (a) appear sensible when perfect learning is possible given the theoretical structure of the simulation. When learning is imperfect in (b)-(d), GPR predictions for the region  $n^* \in [0, N_P]$ show a pronounced discontinuity at a deterministic quantity, $\kappa$.  We increase the kernel resolution in \cref{fig:main:fig_data_gpr} (c) and (d) and test whether prediction artifacts in (b) can be reduced by using a fine Fourier comb in the periodic kernel. This is not the case in \cref{fig:main:fig_data_gpr} (c) and (d) and the algorithm sinks to zero for long regions before reviving discontinuously at $\kappa$. For all cases, we compare GPR predictions for $n^*>0$ with the true state dynamics at the start of the training run [red dotted] and we find good agreement. 

\section{Discussion} \label{sec:main:discussion}


Our studies in  \cref{fig:main:fig_data_all,fig:main:fig_data_specrecon,fig:main:figure_lkffb_path} revealed that autoregressive approaches to modeling stochastic dynamics via joint LSF / AKF implementation led to model robust forward predictions of the qubit state under dephasing. In contrast, oscillator based approaches in LKFFB and GPR (with a periodic kernel) were outperformed for yielding time domain predictions in realistic (imperfect) learning scenarios. We discuss the loss in performance for LKFFB and GPR below. 
%  In constrast,  but fidelity of time domain predictions are outperformed by autoregressive approaches in AKF and LSF for realistic learning scenarios in \cref{fig:main:figure_lkffb_path} (b)-(d). 
\\
\\
In investigating the loss of performance for LKFFB, we find that the efficacy of our LKFFB approach depends on a careful choice of a \textit{probe} (i.e. a fixed computational basis) for the dephasing noise. \cref{fig:main:figure_lkffb_path} reveals that the largest loss in time domain predictive performance for LKFFB arises from imperfect projection onto its basis of oscillators in \cref{fig:main:figure_lkffb_path}(b). In the imperfect projection regime of \cref{fig:main:fig_data_maxfwdpred} and identically, \cref{fig:main:fig_data_specrecon}, LKFFB reconstructs Fourier domain information to a high fidelity across a range of sampling regimes in \cref{fig:main:fig_data_specrecon} (a)-(c) but LKFFB is outperformed by AKF in the time domain in \cref{fig:main:fig_data_maxfwdpred}. Since LKFFB tracks instantaneous amplitude and phase information explicitly for each basis oscillator, the loss of LKFFB time domain predictive performance must accrue from difficulty in tracking instantaneous phase, not amplitude, information. 
\\
\\
One could anticipate improved LKFFB performance on real dephasing noise in a laboratory, as opposed to simulations, where some projection on the LKFFB basis is guaranteed as a real noise power spectrum is continuous in the Fourier domain.  In simulated noise traces, dephasing noise is necessarily discrete in the Fourier domain and the imperfect projection regime is unnecessarily severe.
\\
\\
While difficulty of instantaneous phase estimation is likely to disadvantage time domain predictive performance of LKFFB, we note that an oscillator approach yielded high fidelity reconstructions of true noise power spectral density. These reconstructions are robust against imperfect projection on the LKFFB oscillator basis even as oversampling is reduced. This suggests that an application of LKFFB outside of predictive estimation could be tested against standard spectral estimation techniques in future work.
\\
\\
In GPR, we wanted to exploit the equivalence of amplitude and phase information in an infinite collection of oscillators summarised by a \textit{periodic} kernel to enable qubit prediction. Our investigations reveal that predictions with a periodic kernel are useful for interpolation but have limited meaning for forward predictions for time steps $n >0$.  We find that a fundamental period is set by the comb spacing in the kernel and we expect that learned Fourier information will repeat in the time domain deterministically at the fundamental period, namely, $\kappa$ in all cases depicted in \cref{fig:main:fig_data_gpr}. When learning is perfect, a repeated pattern can be interpreted as qubit state predictions and no discontinuities are seen in forward predictions. When learning is imperfect, GPR  with a periodic kernel is able to learn Fourier amplitudes to provide good state estimates for $n<0$ but one cannot interpret state predictions for $n>0$ without a formal procedure for actively tracking and correcting phase information for each individual basis oscillator at $n= \kappa$. Since phase information can be recast as amplitude information for any oscillator, one expects that forward predictions can be improved by reducing the comb spacing for an infinite basis of oscillators in the periodic kernel.  We find that this is not the case - an increase in kernel resolution means that we are probing time domain correlations longer than the physical time spanned by the measurement record. As such, the GPR algorithm predicts zero for $n \in [0, \kappa], \kappa > 0$, before reviving at $\kappa$.  In fact, if prediction test points were not specified beyond $\kappa$, then a flat region in (c) or (d) may be misinterpreted as predicting zero mean noise rather than a numerical artifact.
\\
\\
% We test whether there is a formal procedure to isolate (and track) phase information from a periodic kernel. There is a theoretical procedure to reduce an infinite basis in the periodic kernel to a finite number of oscillators described in a linear Kalman filter (see Appendices or \cite{solin2014explicit}). However, the theoretical procedure requires an arbitrarily chosen point at which to truncate the infinite basis of oscillators. The choice of the truncation point affects the properties of the finite collection of state space oscillators. It is difficult to see how phase corrections can be introduced for GPR with a periodic kernel.
% \\
% \\
Having discussed oscillator based approaches, we return to autoregressive models and examine the performance of AKF and LSF in high measurement noise regimes. We expect that a Kalman framework enables increased measurement noise filtering compared to LSF alone if dynamics are identically specified.  Our study reveals that the Kalman procedure outperforms LSF alone for applied measurement noise levels $NL \geq 1\%$. There are two theoretical reasons for why this is the case: measurement noise filtering is enabled in the Kalman framework through the optimisation procedure for $R$ and has a regularising (smoothening) effect. Secondly, an imperfectly learned dynamical model $\Phi$ is optimised through the tuning of $\sigma$. The joint optimisation procedure over $(\sigma, R)$ ensures that the relative strength of noise parameters is also optimised.
\\
\\
In QKF, we wish to use single shot qubit data while enabling model-robust qubit state tracking and increased measurement noise filtering via AKF. \cref{fig:main:fig_data_qkf2} reveals that the QKF is vulnerable to the build of errors for arbitrary applications and we provide three explanatory remarks from a theoretical perspective. Firstly, the Kalman gains are recursively calculated using a set of \text{linear} equations of motion which incorporate the Jacobian $H_n$ of $h(x_n)$ at each $n$. All non-linear Kalman filters perform well if errors during filtering remain small such that the linearisation assumption holds at all time steps. Secondly, measurements are quantised and hence residuals must be one of $\{-1, 0, 1 \}$ rather than floating point numbers.  In our case, the Kalman update to $x_n$ at $n$, mediated by the Kalman gain cannot benefit from a gradual reduction in residuals. A third effect incorporates consequences of both quantised residuals and a non-linear measurement action. In simple, linear Kalman filtering, Kalman gains can be pre-calculated in advance of any measurement data. Namely, the recursion of Kalman state variances $P$, can be decoupled from the recursion of Kalman state means, $x$ \cite{grewal2001theory}. The former governs computation of the gains, and the latter depends residuals computed from measurement records. In our application, quantised residuals affect the Kalman update of $x$, and further, they affect the recursion for the Kalman gain via the state dependent Jacobian, $H_n$. These three effects means that QKF is extremely sensitive to a rapid build of errors during the filtering process such that meaningful predictive estimation is no longer possible.
\\
\\
In this context, we demonstrate numerically that the QKF achieves a desirable forward prediction horizon when build of errors during filtering are minimised, for example, by specifying Kalman state dynamics and noise strengths perfectly; and/or by severely oversampling relative to the true $f$.   In \cref{fig:main:fig_data_qkf2}(a) we demonstrate that the forward prediction horizon can be tuned relative to the oversampling ratio $f_0J / f^B$ if true state dynamics and noise strengths are incorporated into QKF, suggesting robustness of the measurement model design.  If dynamics and noise strengths are unknown but learned by machine learning procedures, then QKF achieves a forward prediction horizon only in a highly oversampled regime. It is possible that QKF forward prediction horizons in realistic learning environments can be improved by solving the full $q+2$ optimisation problem for $\{\{ \phi_{q' \leq q}\}, \sigma, R\}$, rather than the approach taken in this manuscript. However, a solution to the full optimisation problem is beyond the scope of our analysis. At present, we interpret QKF as demonstration that one may track stochastic qubit dynamics using single shot measurements under a Kalman framework.

% [PLACEHOLDER]
% In our studies, we have employed Bayesian frameworks to track stochastic qubit dynamics under covariance stationary, non-Markovian dephasing and we predict the qubit state beyond the measurement record. By measurement record, we considered a linear regime where observations are sequence of simulated Ramsey phase measurements, $\state$, and we briefly explored a non-linear regime, where observations are single shot binary qubit outcomes. We considered predictive estimation under GPR and KF frameworks, in particular, we compared autoregressive approaches and a collection of oscillators to approximately represent stochastic dynamics in the mean square limit. We tested our tracking mechanisms under a range of realistic, imperfect learning conditions. 
% \\
% \\
% We compared the forward prediction horizon for qubit states under arbitrary dephasing, for algorithms under GPR and KF frameworks. In the absence of active phase tracking and correction, we conclude GPR state predictions  with a periodic kernel have a limited interpretation in forward time $n>0$ but retain their usefulness in data interpolation applications. For LSF and variants of KF algorithms, we find that the absolute forward prediction horizon can be arbitrarily increased by increasing the oversampling ratio.
% We find that model robust predictive performance is best enabled by a joint AKF/LSF framework in our studies. A joint AKF/LSF implementation enables additional measurement noise filtering and model optimisation than a LSF framework alone. These improvements are granted by optimising over model errors and smoothening over measurement noise while tuning Kalman design parameters, $(\sigma, R)$. Our studies indicate that autoregressive representations are model robust in realistic operating environments compared to alternative theoretical representations where a collection of oscillators is used to learn a covariance stationary random process in the mean square limit. 
% \\
% \\
% In GPR, stochastic dynamics were encoded as correlations specified by a choice of a kernel that maximally probes the Fourier domain by summarising an infinite basis of oscillators. We find that without active phase tracking, learned Fourier amplitudes correspond to a time domain pattern that is repeated at the fundamental period of the kernel, $\kappa$, in \cref{fig:main:fig_data_gpr}. Such a procedure enables high fidelity interpolation but carries limited meaning for forward prediction outside the zone of the measurement data.
% \\
% \\
% In KF using linear measurement records, we compared the efficacy of autoregressive and oscillator approaches to model stochastic dynamics and yield model robust predictions. We find that autoregressive approaches yield better predictions than probing stochastic dynamics with a basis of oscillators for realistic, imperfect learning scenarios. In particular, a basis of oscillators is a discrete probe relative to a continuous dephasing spectrum. While it is always possible to outperform predicting the mean value of dephasing, an oscillator approach in LKFFB has a shorter forward prediction horizon relative to AKF if the LKFFB oscillator basis is too coarse or if true Fourier components do not match basis frequencies.
% \\
% \\
% We extended the Kalman model to act on single shot projective measurements from a qubit, namely, by defining a non-linear, coin-flip measurement action in QKF. State tracking behaviour is seen if dynamics, in principle, could be perfectly specified. In this idealised case, we reduced the oversampling ratio and observed an anticipated gradual reduction in the forward prediction horizon for QKF.  This provides evidence that our non-linear, coin-flip measurement action does not prohibit predictive estimation if the build of errors remain small during filtering, namely, that the linearisation of $h(x)$ by $H$ holds during propagation of Kalman moments $(x, P)$. The linearisation condition is violated in realistic learning scenarios when we introduce a learned dynamical model and imperfectly tuned Kalman noise parameters. In our procedure, predictive estimation becomes prohibitively difficult with QKF. It remains an open research question whether an alternative optimisation procedure could improve QKF predictive performance. 

\section{Conclusion \label{sec:main:Conclusion}}

We considered a sequence of projective measurements obtained from a single qubit under non-Markovian environmental dephasing. Predictive estimation algorithms in this manuscript learn noise correlations in the data and forecast the qubit state beyond the measurement record. To address the absence of a theoretical model describing stochastic qubit dynamics, we use high order autoregressive processes and a collection of oscillators to learn qubit state dynamics under arbitrary dephasing. 
\\
\\
To accommodate stochastic dynamics under arbitrary dephasing, we choose two Bayesian learning protocols - Gaussian Process Regression (GPR) and Kalman Filtering (KF).  All Kalman algorithms predict the qubit state in forward time better than predicting mean qubit behaviour under dephasing.  Forward prediction horizons can be arbitrarily increased for all Kalman algorithms by oversampling true dephasing noise.  In contrast, under GPR, we encode dynamics using an infinite basis of oscillators and we find numerical evidence that this approach enables interpolation but not forward predictions beyond the measurement record.  
\\
\\
We analyse two measurement models under the Kalman framework - a linear regime, where qubit outcomes and pre-processed, and a non-linear regime, where algorithms act on raw projective qubit measurements. In the linear regime, a key insight is that Kalman-based autoregressive approaches exhibit model-robust dynamical tracking of qubits compared to Kalman-based oscillator approaches. We confirm autoregressive Kalman filters enable increased measurement noise filtering compared to autoregressive, least squares procedures in \cite{mavadia2017} alone. Subsequently, we implement a non-linear, coin-flip measurement model in an autoregressive Kalman filter to demonstrate qubit state tracking and prediction can occur using only 0 or 1 qubit measurements. 
\\
\\
% Under arbitrary dephasing power spectral densities, robust optimisation procedures to tune algorithms is a key enabler of predictive performance. In particular, optimising over a recursive filter, such as Kalman algorithms, remains challenging when a large number of Kalman design elements are unknown.  Efficient optimisation procedures will bolster uptake of Kalman algorithms for real time prediction and control applications and simplify filter tuning in realistic laboratory environments.
% \\
% \\
There are exciting opportunities for machine learning algorithms to increase our understanding of dynamically evolving quantum systems in real time using  projective measurement records. Quantum systems coupled to classical spatially or temporally varying fields provide opportunities for classical algorithms to analyse correlation information and enable predictive control of qubits. Moving beyond a single qubit, we anticipate that measurement records will grow in complexity allowing us to exploit the natural scalability offered by machine learning for mining large datasets. In realistic laboratory environments, the success of algorithmic approaches will be contingent on robust and computationally efficient algorithmic optimisation procedures. The pursuit of these opportunities is a subject of ongoing research.

\section{Acknowledgments}
 The LSF filter is written by V. Frey and S. Mavadia \cite{mavadia2017}. The GPR framework is implemented and optimised using standard protocols in GPy \cite{gpy2014}. Authors thank C. Grenade, K. Das, V. Frey, S. Mavadia, H. Ball, C. Ferrie and T. Scholten for useful comments. 
 
%  Simulations for all numerical experiments were designed and analysed as part of the research effort for this paper. To support this, we developed Python programs to engineer dephasing noise; simulate measurement records; and analyse results. We developed Python programs for Kalman Filters (AKF, LKFFB, and QKF) and the optimisation procedure for Kalman filters.

\section{Version Control \label{sec:main:versioncontrol}}
Notes: NOTES-v0-1 | Notes-2017-Main-v7
\\
Data: Fig v5 (resized, with maxpredfwd)
% \\
% \\
% Code: Git Hub Branch quantised-kf last updated 19 Sept 2017
% \\
% \\


%  In context of time-series predictions, 
% \subsection{Bayesian Framework Beyond GPR and KF}
% [PLACEHOLDER]
% In a general Bayesian learning framework, the Bayes update for prior distribution $P(\state_n | \mathcal{D}_{n-1}, \tau)$ based on the the likelihood, $P(d_n | \state_n, \tau)$ for an incoming measurement at $n$: 
% \begin{align}
% P(\state_n| \mathcal{D}_n, \tau)  \propto P(d_n | \state_n, \tau) P(\state_n | \mathcal{D}_{n-1}, \tau)
% \end{align}
% The output (aposteriori) distribution, $P(\state_n| \mathcal{D}_n, \tau)$, is the solution to the general, non-linear Bayesian inference problem. It is often numerically estimated and subsequently, the mean and the variance of the aposteriori distribution is interepreted as the state estimate and the state variance estimate at $n$. However, we have not yet propagated the state estimates forward in time from $n$ to $n+1$. If $\state$ was Markovian, then it would be straightforward to write a `dynamical model' to enable resampling according to a transition probability, $P(\state_{n+1} | \state_n)$:
% \begin{align}
% P(\state_{n+1} | \mathcal{D}_{n}, \tau) = \int P(\state_{n+1} | \state_n) P(\state_n | \mathcal{D}_{n}, \tau) d\state_n
% \end{align}
% Particle filtering and sequential Bayesian adaptive learning protocols for Markov $\state$ have been applied to our problem as $ P(\state_{n+1} | \state_n)$ exists, namely, a global likelihood for all time steps is written as a product of likelihoods at each timestep under a Markov model \cite{ferrie2013, wiebe2015}. Relaxing this Markov condition in particle filtering techniques has been the subject of recent research(\cite{wiebe2015bayesian, jacob2016}). An alternative way to track simple non-Markovian time series processes was considered for a switching problem in \cite{rogers2017}. A generalisation of \cite{rogers2017} to arbitrary non-Markovian processes would increase the dimensionality of the underlying Bayesian inverse problem and lends this methdology for classification rather than time series regression applications [CHECK]. Developing a theoretical, non-Markovian transition probability distribution for arbitrary dephasing processes in the context of time series tracking in our application is beyond the scope of this paper.  
\section{Introduction} 

In predictive estimation, a dynamically evolving system is observed and any temporal correlations encoded in the observations are used to predict the future state of the system.  This generic problem is well studied in diverse fields such as engineering, econometrics, meteorology, and seismology~\cite{groen2013real,dong2009unscented,ko2009gp,harvey1990forecasting,cheng2015time}, and is addressed in the control-theoretic literature as a form of filtering.  Applying these approaches to state estimation on qubits is complicated by a variety of factors; dominant among these is the violation of the assumption of linearity inherent in most filtering applications as qubit states are formally bilinear. The case of an idling, or freely evolving qubit subject to dephasing is more complicated still, as an a priori model of system evolution suitable for implementation within standard filtering algorithms will not in general be available.

Fortunately there are many lessons to learn from classical control, even in the presence of such complications.  For classical systems, machine learning techniques have enabled state tracking, control, and forecasting for highly non-linear and noisy dynamical trajectories or complex measurement protocols (e.g. \cite{garcia2016optimal, bach2004learning, tatinati2013hybrid, hall2011reinforcement, hamilton2016ensemble}). These demonstrations move far beyond the simplified assumptions underlying many basic filtering tasks such as linear dynamics and white (uncorrelated) noise processes. For instance, so-called particle-based Bayesian frameworks (e.g. particle filtering, unscented or sigma-point filtering) allow state estimation and tracking in the presence of non-linearities in system dynamics or measurement protocols~\cite{candy2016bayesian}.  Further extensions approach the needs of a stochastically evolving system; recently, an ensemble of so-called unscented Kalman filters, named after the underlying mathematical transformation, demonstrated state estimation and forward predictions for chaotic, non-linear systems in the absence of a prescribed model~\cite{hamilton2016ensemble}. For non-chaotic, multi-component stationary random signals, other algorithmic approaches have been particularly useful for tracking instantaneous frequency and phase information, ~\cite{boashash1992estimating2, ji2016gradient}, enabling short-run forecasting.  

In the field of quantum control, work has begun to incorporate the additional challenges faced when considering state estimation on qubits, notably quantum-state collapse under projective measurement.  Under such circumstances, in which the measurement backaction strongly influences the quantum state (in contrast with the classical case), it is not straightforward to extend machine learning predictive estimation techniques.  Work to date has approached the analysis of projective measurement records on qubits as pattern recognition or image reconstruction problems, for example, in characterising the initial or final state of quantum system (e.g. \cite{struchalin2016experimental, sergeevich2011characterization, mahler2013adaptive}) or reconstructing the historical evolution of a quantum system based on large measurement records (e.g. \cite{stenberg2016characterization, shabani2011efficient, shen2014reconstructing, de2016estimation, tan2015prediction, huang2017neural}). In adaptive or sequential Bayesian learning applications, a projective measurement protocol may be designed or adaptively manipulated to efficiently yield noise-filtered information about a quantum system (e.g. \cite{bonato2016optimized, wiebe2015bayesian}). 

The demonstrations above typically assume the object of interest is either static, or evolves in a manner which is dynamically uncorrelated in time (white) as measurement protocols are repeated. This simplifying assumption falls well short of typical laboratory based experiments where noise processes are frequently correlated in time, and evolution may also occur rapidly relative to a measurement protocol. In such a circumstance, further complexity is introduced as the Markov condition commonly assumed in Bayesian learning frameworks~\cite{candy2016bayesian} is immediately violated.  Even in the classical case, the problem of designing an appropriate representation of non-Markovian dynamics in Bayesian learning frameworks is an active area of research (e.g  \cite{jacob2017bayesian}).  Hence, the canonical real-time tracking and prediction problem - where a non-linear, stochastic trajectory of a system is tracked using noisy measurements and short-run forecasts are made - is under-explored for quantum systems with projective measurements.

In this manuscript, we develop and explore a broad class of predictive estimation algorithms allowing us to track a qubit state undergoing \emph{stochastic but temporally correlated} evolution using a record of projective measurements, and forecast its future evolution. Our approaches employ machine learning algorithms to extract temporal correlations from the measurement record and use this information to build an effective dynamical model of the system's evolution.  We design a deterministic protocol to correlate Markovian processes such that a certain general class of non-Markovian dynamics can be approximately tracked without violating the assumptions of a machine learning protocol, based on the theoretically accessible and computationally efficient frameworks of Kalman Filtering (KF) and Gaussian Process Regression (GPR).  Both frameworks provide a mechanism by which temporal correlations (equally, dynamics) are encoded into an algorithm's structure such that projection of data-sets onto this structure enables meaningful learning, white-noise filtering, and effective forward prediction.  We perform numerical simulations to test the effectiveness of these algorithms in maximizing the prediction horizon under various conditions, and quantify the role of the measurement sampling rate relative to the noise dynamics in defining the prediction horizon.  Simulations incorporate a variety of measurement models, including pre-processed data yielding a continuous measurement outcome and discretised outcomes commonly associated with single-shot projective qubit measurements.   We find that in most circumstances an autoregressive Kalman framework yields the best performance, providing model-robust forward prediction horizons and effective filtering of measurement noise.  Finally, we demonstrate that standard GPR-based protocols employing a variety of kernels, while effective for the problem of filtering (fitting) a measurement record, are not suitable for real-time forecasting beyond the measurement record.  

In what follows, we describe in detail the physical setting for our problem in~\cref{sec:main:PhysicalSetting} and explain how this leads to a specific choice of algorithm which may be deployed for the task of tracking non-Markovian state dynamics in the absence of a dynamical model for system evolution.  We provide an overview of the central GPR and KF frameworks in \cref{sec:main:OverviewofPredictive Methodologies}, and we specify a series of algorithms under consideration in this paper tailored to different measurement processes. For pre-processed measurement records, we consider four algorithmic approaches: a Least Squares Filter (LSF) from \cite{mavadia2017}; an Autoregressive Kalman Filter (AKF); a so-called Liska Kalman Filter from \cite{livska2007} adapted for a Fixed oscillator Basis (LKFFB); and a suitably designed GPR learning protocol. For binary measurement outcomes, we extend the AKF to a Quantised Kalman Filter (QKF). In \cref{sec:main:Optimisation}, we present optimisation procedures for tuning all algorithms. Numerical investigations of algorithmic performance are presented in \cref{sec:main:Performance} and a comparative analysis of all algorithms is provided in \cref{sec:main:discussion}. 

\section{Physical Setting \label{sec:main:PhysicalSetting}}  
\label{sec:main:1} 

Our physical setting considers a sequence of projective measurements performed on a qubit. Each projective measurement yields a 0 or 1 outcome representing the state of the qubit. The qubit is then reset, and the exact procedure is repeated. By considering a qubit state initialized in a superposition of the measurement basis (for us, Pauli $\p{z}$ eigenstates), we gain access to a direct probe of qubit phase evolution.  If, for instance, no dephasing is present, then the probability of obtaining a binary outcome remains static in time as sequential qubit measurements are performed. If slowly drifting environmental dephasing is present, then the probability of obtaining a given binary outcome also drifts stochastically. In essence, the qubit probes dephasing noise and our procedure encodes a continuous-time non-Markovian dephasing process into time-stamped, discrete binary samples through the nonlinear projective measurement, carrying the underlying correlations in the noise.  It is this series of measurements which we seek to process in our algorithmic approaches to qubit state tracking and prediction. 

Formally, an arbitrary environmental dephasing process manifests as time-dependent stochastic detuning, $\delta \omega (t)$, between the qubit frequency and the system master clock. This detuning is an experimentally measurable quantity in a Ramsey protocol, as shown schematically in \cref{fig:main:Predive_control_Fig_overview_17_one} (a). A non-zero detuning over measurement period $\tau$ (starting from $t=0$) induces a stochastic relative phase accumulation (in the rotating frame) for a qubit superposition state as $\left|0\right\rangle+e^{-i\state(0, \tau)}\left|1\right\rangle$ between qubit basis states.  The accumulated $\state(0, \tau)$ at the end of a single Ramsey experiment is mapped to a probability of obtaining a particular outcome in the measurement basis via the form of the Ramsey sequence.  

\begin{figure}[h!]
    \includegraphics[scale=1]{Predive_control_Fig_overview_17_one_reduced} 
    \caption{ \label{fig:main:Predive_control_Fig_overview_17_one} (a) A Ramsey experiment at $t=n\Delta t$ with fixed wait time $\tau$ and time-steps, $n$, spaced $\Delta t > \tau$ apart. A $\pi/2$ pulse rotates qubit state to super-position of $\ket{d}$ states, $d\in \{0,1\}$; qubit evolves via $\op{\mathcal{H}}_N(t)$ accumulating relative stochastic $\state_n$, for non-zero environmental dephasing $\delta \omega (t)$. Jittering arrows depict potential qubit state vectors permitted for (unknown) random $f_n$. Qubit state is measured as $d_n=d$ in $\p{z}$ basis after a second $\pi/2$ rotation. (b) Black dots depict $\{d_n\}$ against time steps, $n$; data collection stops at $n=0$ separating past state estimation from future prediction [blue region].  Black solid line shows true qubit state likelihood $ \propto h(f_n)$; and  red solid line shows state estimate (prediction) for $n<0$ ($n>0$). A prediction horizon is $n < n^* \in [0,N_P]$ for which dark-grey region between red and black lines is minimised (Bayes prediction risk) relative to predicting the mean of dephasing noise; algorithmic tuning occurs by minimising light-grey region (Bayes state estimation risk). $\mathcal{Q}$ quantises black line into noisy qubit measurements, $d_n$, under Gaussian uncertainty $v_n$. (c) Single shot outcomes in (b) are pre-processed to yield noisy measurements $\{ y_n\}$ [black dots]; $y_n$ is linear in $\state_n$ and $v_n$ represents additive white Gaussian measurement noise.}
\end{figure} 

In a sequence of $n$ Ramsey measurements spaced $\Delta t$ apart with a fixed duration, $\tau$, the change in the statistics of measured outcomes over this measurement record depends solely on the dephasing  $\delta \omega(t)$.   We assume that the measurement action over $\tau$ is much faster than the temporal dynamics of the dephasing process, and $\Delta t \gtrsim \tau$. The resulting measurement record is a set of binary outcomes,  $\{d_n\}$, determined probabilistically from $n$ true stochastic qubit phases, $\state := \{\state_n\}$. Here the accumulated phase in each Ramsey experiment, $ \state(n \Delta t, n\Delta t + \tau) \equiv \int_{n \Delta t}^{n \Delta t +\tau} \delta \omega(t') dt'$ and we use the shorthand $\state(n \Delta t , n\Delta t + \tau) \equiv \state_n$.  We define the statistical likelihood for observing a single shot, $d_n$, using Born's rule \cite{ferrie2013}:

\begin{align}
Pr(d_n=d | \state_n, \tau, n \Delta t) &= \begin{cases} \cos^2(\frac{\state_{n}}{2}) \quad \text{for $d=1$} \\   \sin^2(\frac{\state_{n}}{2})  \quad \text{for $ d=0$}  \end{cases} \label{eqn:main:likelihood} 
\end{align}
The notation $Pr(d_n | \state_n, \tau, n \Delta t)$ refers to the conditional probability of obtaining measurement outcome $d_n$ given a true stochastic phase, $\state_n$, accumulated over $\tau$, beginning at time $t = n \Delta t$. In the noiseless case, $Pr(d_n=1|\state_n, \tau, n \Delta t) = 1, \quad \forall n $, such that a qubit exhibits no additional phase accumulation due to environmental dephasing. Following a single measurement the qubit state is reset, but the dephasing noise correlations manifest again via Born's rule for another random value of the bias at time-step $n+1$. A detailed discussion of \cref{eqn:main:likelihood} can be found in \cref{sec:app:setup_1}.

The action of measurement, expressed as $h(\state_n)$, is given by $Pr(d_n=d| \state_n, \tau, n \Delta t) \equiv \frac{1}{2} - (-1)^d h(\state_n) $ and is depicted in \cref{fig:main:Predive_control_Fig_overview_17_one}(b) as a probability of seeing the qubit in the $d=1$ state.  We begin by describing here a `raw' non-linear measurement record, $\{ d_n\}$ where each $d_n$ [black dots] corresponds to a binary outcome derived from a single projective measurement on a qubit. The sequence $\{ d_n\}$ can be treated as a sequence of biased coin flips, where the underlying bias of the coin is a non-Markovian, discrete-time process and the value of the bias is given by \cref{eqn:main:likelihood} at each $n$. The non-linearity of the measurement, $h(\state_n)$, is defined with respect to $\state_n$ where \cref{eqn:main:likelihood} is interpreted as a non-linear measurement action for Bayesian learning frameworks.

This data series is contrasted with a linear measurement record, $\{ y_n\}$, depicted in \cref{fig:main:Predive_control_Fig_overview_17_one}(c).  Each value $y_n$ is derived from the sum of a true qubit phase, $\state_n$, and Gaussian white measurement noise, $v_n$.  The sequence $\{ y_n\}$ is generated by pre-processing raw binary measurements, $\{ d_n\}$ via a range of experimental techniques subject to a separation of timescales such that $\sim\tau$ is much faster than drift of $\delta \omega (t)$.  In the most common case, one performs $M$ runs of the experiment over which $\delta \omega (t)$ is approximately constant, giving an estimate of  $\state_n$ at $t = n \Delta t $ using averaging, a Bayesian scheme, or Fourier analysis. A more complex linearization protocol involves the use of low-pass or decimation filtering on a sequence $\{ d_n\}$  to yield $\hat{Pr}(d_n | \state_n, \tau, n\Delta t)$, from which accumulated phase corrupted by measurement noise, $\{ y_n\}$, can be obtained from \cref{eqn:main:likelihood}. 

We impose properties on environmental dephasing such that our theoretical designs can enable meaningful predictions. We assume dephasing is non-Markovian, covariance stationary and mean-square ergodic.  That is, a single realisation of the process $\state$ is drawn from a power spectral density of arbitrary, but non-Markovian form. We further assume that $\state$ is a Gaussian process and the separation of timescales between measurement protocols and dephasing dynamics articulated above are met.

Given these conditions, our task is to build a dynamical model to approximately track $\state$ over past measurements ($n<0$), and enable qubit state predictions in future times ($n>0$).  This prediction is represented by the red line in \cref{fig:main:Predive_control_Fig_overview_17_one}(b-c), and differs from the truth by the so-called estimation (prediction) risk for past (future) times as indicated by shading.  We represent our estimate of $\state$ for all times using a hat in both the linear and nonlinear measurement models.  The major challenge we face in developing this estimate, $\hat{\state}$ (equivalently $\hat{Pr}(d_n | \state_n, \tau, n\Delta t)$), is that for a qubit evolving under stochastic dephasing (true state given by black solid line in \cref{fig:main:Predive_control_Fig_overview_17_one}(b) and (c)), we have no a prior dynamical model for the underlying evolution of $\state$.  In the next section, we define the theoretical structure of KF and GPR algorithms which allow us to build that dynamical model directly from the historical measurement record. 




\section{Overview of Predictive Methodologies \label{sec:main:OverviewofPredictive Methodologies}}

\begin{figure*}[htp]
        \includegraphics[scale=1.]{Predive_control_Fig_overview_17_two_reduced} 
        \caption{ \label{fig:main:Predive_control_Fig_overview_17_two} Comparison of the algorithmic structure between the KF and GPR by superposing lower panels of \cref{fig:main:Predive_control_Fig_overview_17_one} with KF and GPR predictive frameworks. (a) KF: Purple distribution represents a prior, with mean $x_n$, and covariance $P_n$; propagated in time-steps, $n$, using Kalman dynamics $\Phi_n$, and updated within each $n$ by the Kalman gain $\gamma_n$ to yield  posterior distribution (red) at $n$. The posterior at $n$ is the prior at $n+1$. The mean of a posterior distribution at each $n$ is used to derive predictions given by the red line using $h(x_n)$. In blue region, the red posterior predictive distribution is propagated using $\Phi_n$ but $\gamma_n \equiv 0$. Gaussian white Kalman `process' noise, $w_n$, is coloured by $\Phi_n$ to yield dynamics for $x_n$. (b) Purple prior distribution defined over sequences, $\state$, with mean, $\mu_\state$ and variance $\Sigma_\state$ is constrained by the entire measurement record. The resulting posteriori predictive distribution (red) is evaluated at test-points in time, $n^\ddagger \in [-N_T, N_P]$; state estimates (predictions) is the mean, $\mu_{\state^\ddagger}$ at $n^\ddagger < 0$ ($n^\ddagger > 0$). A choice of kernel defines each element in $\Sigma_{\state}, \Sigma_{\state^\dagger}$. In both (a)-(b), the purple shadow represents posterior state variance (diagonal $P_n$ or $\Sigma_{\state^\ddagger}$ elements) constrained by data and filtered measurement noise $v_n$.}
\end{figure*}

Our objective is to implement an algorithm permitting learning of underlying qubit dynamics in such a way as to maximize the forward prediction horizon for a given qubit data record.  We first quantify the quality of our state estimation procedure.  The fidelity of any underlying algorithm during state estimation and prediction, relative to the true state, is expressed by the mathematical quantity known as a Bayes Risk, where zero risk corresponds to perfect estimation. At each time-step, $n$, the Bayes risk is a mean square distance between truth, $\state$, and prediction, $\hat{\state}$, calculated over an ensemble of $M$ different realisations of true $\state$ and noisy data-sets $\mathcal{D}$:
\begin{align}
L_{BR}(n | I) & \equiv \langle(\state_n - \hat{\state}_n)^2 \rangle_{\state,\mathcal{D}} \label{eqn:main:sec:ap_opt_LossBR}
\end{align}
The notation $L_{BR}(n | I)$ expresses that the Bayes Risk value at $n$ is conditioned on $I$, a placeholder for free parameters in the design of the predictor, $\hat{\state}_n$. State estimation risk is Bayes Risk incurred during $n \in [-N_T, 0]$; prediction risk is the Bayes Risk incurred during $n \in [0, N_P]$. State estimation and prediction risk regions for one realisation of dephasing noise are shaded in \cref{fig:main:Predive_control_Fig_overview_17_one}-\ref{Predive_control_Fig_overview_17_three}.  We therefore define the forward prediction horizon as the number of time-steps for $ n^{*} \in [0, N_P]$ during which a predictive algorithm incurs a lower Bayes prediction risk than naively predicting $\hat{\state}_n \equiv \mu_f = 0 \quad \forall n$, the mean qubit behaviour under zero-mean dephasing noise. 

With this concept in mind, we introduce two general approaches for algorithmic learning relevant to the strictures of the problem we have introduced.  Our general approach is shared between all algorithms employed and is represented schematically for the KF and GPR in \cref{fig:main:Predive_control_Fig_overview_17_two}. Stochastic qubit evolution is depicted for one realisation of $\state$ [black solid line] given noisy linear measurements [black dots] corrupted by Gaussian white measurement noise $v_n$.  Our overall task is to produce an estimate, given by the red line, which minimizes risk for the prediction period.  Ideally both estimation risk and prediction risk are minimized simultaneously for well performing implementations.

Examining the insets in both panels of \cref{fig:main:Predive_control_Fig_overview_17_two}, both frameworks start with a prior Gaussian distribution over qubit states [purple] that is constrained by the measurement record to yield a posterior Gaussian distribution of the qubit state [red]. The prior captures assumptions about the qubit state before any data is seen and the posterior expresses our best knowledge of the qubit state under a Bayesian framework.  The posterior distribution in both KF and GPR is used to generate qubit state estimates ($n<0$) and predictions ($n>0$) [red solid line].  However the computational process by which this posterior is inferred differs significantly between the two methods; we provide an overview of the central features of these algorithms below. 

The key feature of a Kalman filter is the recursive learning procedure shown in the inset to \cref{fig:main:Predive_control_Fig_overview_17_two}(a). Our knowledge of the qubit state is summarised by the prior and a posterior Gaussian probability distributions and these are created and collapsed recursively \emph{at each time step}. The mean of these distributions is the true Kalman state, $x_n$, and the covariance of these distributions, $P_n$, captures the uncertainty in our knowledge of $x_n$; together both  define the Gaussian distribution. The Kalman filter produces an \emph{estimate} of the state, $\hat{x}_{n}$ at each step through this recursive procedure taking into account two factors. First, the Kalman gain, $\gamma_n$, updates our knowledge of $(x_n, P_n)$ within each time step $n$ and serves as a weighting factor for the difference between  incoming data, and our best estimate for an observation based on $\hat{x}_n$, suitably transformed via the measurement action, $h(\hat{x}_{n})$. Next, the dynamical model $\Phi_n$ propagates the state and covariance, $(x_n, P_n)$, to the next time step, such that the posterior moments at $n$ define the prior at $n+1$.  This process occurs for each time step and an estimate of a true $x_n$ state is built up recursively based on all of our existing knowledge, namely, a linear combination of all past measurements; and all previously generated state estimates.  Beyond $n=0$ we perform predictions in the absence of further measurement data by simply propagating the dynamic model with the Kalman gain set to zero.  Full details of the KF algorithm appear below in \cref{Subsec:KF}.

In our application, we define the Kalman state, $x_n$, the dynamical model $\Phi_n$, and a measurement action $h(x_n)$ such that the Kalman Filtering framework can track a non-Markovian qubit state trajectory due to an arbitrary realisation of $\state$. In standard KF implementations, the discrete-time sequence $\{x_n\}$, defines a ``hidden'' signal that cannot be observed, and the dynamic model $\Phi_n$ is known.  We deviate from this standard construction such that our true Kalman state and its uncertainty, $(x_n, P_n)$, do not have a direct physical interpretation.  Kalman $x_n$ has no a priori deterministic component and corresponds to arbitrary power spectral densities describing $\state$. Hence, the role of the Kalman $x_n$ is to represent an abstract correlated process that, upon measurement, yields physically relevant quantities governing qubit dynamics.  Moreover a key challenge described in detail below is to construct an effective $\Phi_{n}$ from the measurement record.   

In contrast to the recursive approach taken in the KF, a GPR learning protocol illustrated schematically in \cref{fig:main:Predive_control_Fig_overview_17_two}(b) selects \textit{a random process} to best describe overall dynamical behaviour of the qubit state under one realisation of $\state$. The key point is that sampling the prior or posterior distribution in GPR yields random realisations of discrete time \textit{sequences}, rather than individual random variables, and GPR considers the entire measurement record at once.  In a sense, it corresponds to a form of fitting over the entire data set.  The output of a GPR protocol is a predictive distribution which we can evaluate at an arbitrarily chosen sequence of test-points, where the test points can exist  for $n<0$ ($n>0$) such that we extract state estimates (forward predictions) from the predictive distribution. Due to the nature of this procedure, we wish to distinguish the set of test points (in units of time-steps)  using a ${}^\ddagger$, namely, that we are evaluating the predictive posterior distribution of a GPR protocol at desired time labels. In this notation, $\{ n^{\ddagger} \}, \quad n^{\ddagger} \in [-N_T, N_P]$ are test-points; $N^{\ddagger}$ is the total length of an array of test points; where state estimation occurs if $n^{\ddagger} \leq 0$ and predictions occur if $n^{\ddagger}>0$. 

The process of building the posterior distribution is implemented using a kernel, or basis, from which to construct the effective fit.  In standard GPR implementations, the correlation between any two observations depends only on the separation distance of the index of these observations, and correlations are captured in the covariance matrix, $\Sigma_\state$. Each element, $\Sigma_\state^{n_1, n_2}$, describes this correlation for observations at arbitrary time-steps indexed by $n_1$ and $n_2$: this quantity is given in a form set by the selected kernel. 

In our application, the non-Markovian dynamics of $\state$ are not specified explicitly but are encoded in a general way through the choice of kernel, prescribing how $\Sigma_\state^{n_1, n_2}$ should be calculated. The Fourier transform of the kernel represents a power spectral density in Fourier space. A general design of $\Sigma_\state^{n_1, n_2}$ allows one to probe arbitrary stochastic dynamics and equivalently, explore arbitrary regions in the Fourier domain. For example, Gaussian kernels (RBF) and mixtures of Gaussian kernels (RQ) capture the continuity assumption that correlations die out as separation in time increases. We choose to employ an infinite basis of oscillators implemented by the so-called periodic kernel to enable us to represent arbitrary power spectral densities for $\state$.  Prediction occurs simply by extending the GPR fit by choosing test-points $n^{\ddagger}>0$.

%We define this as the number of time-steps beyond the measurement record for which predictions of the qubit state are better than naively predicting the average behaviour of the qubit under dephasing.
In the following subsections we provide details of the specific classes of learning algorithm employed here with an eye towards evaluating their predictive performance on qubit-measurement records.  We introduce a series of KF algorithms capable of handling both linear and non-linear measurement records, and restrict our analysis of GPR to linear measurement records. 



%%%%%%%%%%%%%%%%%%%%%%%%%%%%%%%%%%%%%%%%%%%%%%%%%%%%%%%%%%%%%%%%%%%%%%%%%%%%%%%%
\subsection{ Kalman Filtering (KF)}\label{Subsec:KF}

%A Kalman Filter recursively tracks the stochastic evolution of a hidden true state. An incoming stream of unreliable (noisy) observations are fed to a Kalman Filter, and the objective of the Kalman Filter is to recursively improve its estimate of the true state at any time, $n\Delta t$, given the past $n$ measurements. 
In order for a Kalman Filter to track a stochastically evolving qubit state in our application, the hidden true Kalman state at time-step $n$, $x_n$, must mimic stochastic dynamics of a qubit under environmental dephasing. We propagate the hidden state $x_n$ according to a dynamical model $\Phi_n$ corrupted by Gaussian white process noise, $w_n$.  
\begin{align}
x_n & = \Phi_n x_{n-1} + \Gamma_n w_n \label{eqn:KF:dynamics} \\
w_n & \sim \mathcal{N}(0, \sigma^2) \quad \forall n 
\end{align}
Process noise has no physical meaning in our application - $w_n$ is shaped by $\Gamma_n$ and deterministically colored by the dynamical model $\Phi_n$ to yield a non-Markovian $x_n$ representing qubit dynamics under generalised environmental dephasing. In addition to coloring via the dynamical model, the process noise covariance matrix, $Q_n \equiv \Gamma_n\Gamma_n^T $, offers an additional mechanism to shape input white noise by designing $\Gamma_n$.

We measure $x_n$ using an ideal measurement protocol, $h(x_n)$, and incur additional Gaussian white measurement noise $v_n$ with scalar covariance strength $R$, yielding scalar noisy observations $y_n$:
\begin{align}
y_n &= z_n + v_n \\
z_n & \equiv  h(x_n) \\
v_n & \sim \mathcal{N}(0, R) \quad \forall n
\end{align}
The measurement procedure, $h(x_n)$, can be linear or non-linear, allowing us to explore both regimes in our physical application.

With appropriate definitions, the Kalman equations below specify all Kalman algorithms in this paper. At each time step, $n$, we denote estimates of the moments of the prior and posterior distributions (equivalently, estimates of the true Kalman state) with $(\amx{n}, \amp{n})$ and $(\apx{n}, \app{n})$ respectively. The Kalman update equations take a generic form (c.f.~\cite{grewal2001theory}) :

\begin{align}
\amx{n} & = \Phi_{n-1} \apx{n-1} \label{eqn:main:KF:dynamic_x}\\ 
Q_{n-1} & = \sigma^2 \Gamma_{n-1}\Gamma_{n-1}^T  \label{eqn:main:KF:Q}\\
\amp{n}&= \Phi_{n-1} \app{n-1} \Phi_{n-1}^T + Q_{n-1} \label{eqn:main:KF:dynamic_P}\\
\gamma_n &= \amp{n} H_n^T(H_n\amp{n}H_n^T + R_n)^{-1} \label{eqn:main:KF:gain}\\
\hat{y}_n(-) & = h(\amx{n}) \label{eqn:main:KF:step_ahead}\\
\apx{n} &= \amx{n} + \gamma_n (y_n - \hat{y}_n(-)) \label{eqn:main:KF:bayesian_x}\\
\app{n} &= \left[1  - \gamma_n H_n \right] \amp{n} \label{eqn:main:KF:bayesian_p}
\end{align}
To reiterate, \cref{eqn:main:KF:dynamic_x} and \cref{eqn:main:KF:dynamic_P} bring the best state of knowledge from the previous time step into the current time step, $n$, as a prior distribution. Dynamical evolution is modified by features of process noise, as encoded in \cref{eqn:main:KF:Q}, and propagated in \cref{eqn:main:KF:dynamic_P}. The propagation of the moments of the a priori distribution, as outlined thus far, does not depend on the incoming measurement, $y_n$, but is determined entirely by the a priori (known) dynamical model, in our case $\Phi \equiv \Phi_n, \forall n$. 

The Kalman gain in \cref{eqn:main:KF:gain} depends on the uncertainty in the true state, $\amp{n}$ and is modified by features of the measurement model, $H_n$, and measurement noise, $R_n \equiv R,\; \forall n$. It serves as an effective weighting function for each incoming observation.  Before seeing any new measurement data, the filter predicts an observation $\hat{y}_n(-)$ corresponding to the best available knowledge at $n$ in \cref{eqn:main:KF:step_ahead}. This value is compared to the actual noisy measurement $y_n$ received at $n$, and the difference is used to update our knowledge of the true state via \cref{eqn:main:KF:bayesian_x}. If measurement data is noisy and unreliable (high $R$), then $\gamma$ has a small value, and the algorithm propagates Kalman state estimates according to the dynamical model and effectively ignores data. In particular, only the second terms in both \cref{eqn:main:KF:bayesian_x} and \cref{eqn:main:KF:bayesian_p} represent the Bayesian update of the moments of a prior distribution ($(-)$ terms) to the posterior distribution ($(+)$ terms) at $n$. If $\gamma_n \equiv 0$, then the prior and posterior moments at any time step are exactly identical by \cref{eqn:main:KF:bayesian_x,eqn:main:KF:bayesian_p}, and only dynamical evolution occurs using \cref{eqn:main:KF:dynamic_x,eqn:main:KF:Q,eqn:main:KF:dynamic_P}.  This is the condition we employ when we seek to make forward predictions beyond a single time-step, and hence we set $\gamma \equiv 0$ during future prediction.

Since we do not have a known dynamical model $\Phi$ for describing stochastic qubit dynamics under $\state$, we will need to make design choices for  $\{ x, \Phi, h(x), \Gamma \}$  such that $\state$ can be approximately tracked. These design choices will completely specify algorithms introduced below and represent key findings with respect to our work in this manuscript. For a linear measurement record, $h(x) \mapsto Hx$ and we compare predictive performance for $\Phi$ modeling stochastic dynamics either via so-called `autoregressive' processes in the AKF, or via projection onto a collection of oscillators in the LKKFB.  In addition, we use the dynamics of AKF to define a Quantised Kalman filter (QKF) with a non-linear, quantised measurement model such that the filter can act directly on binary qubit outcomes. We provide the relevant details in sub-sections below. 
 


\subsubsection{Autoregressive Kalman Filter (AKF)}

Recursive autoregressive methods are well-studied in classical control applications  (\emph{c.f.}~\cite{moon2006real}) presenting opportunities to leverage existing engineering knowledge in developing quantum control strategies.  In our application, we use an autoregressive Kalman filter to probe arbitrary, covariance-stationary qubit dynamics such that the dynamic model is constructed as a weighted sum of $q$ past values driven by white noise {\em i.e.} an autoregressive process of order $q$, AR($q$). Using Wold's decomposition, it can be shown that any zero mean covariance stationary process representing qubit dynamics has a representation in the mean-square limit by an autoregressive process of finite order, as in \cref{sec:app:AKF}.

The study of AR($q$) processes falls under a general class of techniques based on autoregressive moving average (ARMA) models in adaptive control engineering and econometrics (e.g.  \cite{landau1998adaptive,hamilton1994time} respectively). For high-$q$ models in a typical time-series analysis, it is possible to decompose an AR($q$) into an ARMA model with a small number of parameters \cite{brockwell1996introduction, salzmann1991detection}. However, we retain a high-$q$ model to probe arbitrary power spectral densities. Further, literature suggests employing a high-$q$ model is relatively easier than a full ARMA estimation problem and enables lower prediction errors \cite{wahlberg1989estimation,brockwell1996introduction}. 

To construct the Kalman dynamical operator $\Phi$ for the AKF, we introduce a
set of $q$ coefficients $\{\phi_{q' \leq q}\}, q' = 1, ... , q $ to specify the dynamical model:
\begin{align}
 \state_n &= \phi_1 \state_{n-1} + \phi_2 \state_{n-2} + ... + \phi_q \state_{n-q} + w_n \label{eqn:main:ARprocess}
\end{align}

\noindent We thus see that the dynamical model is constructed as a weighted sum of time-retarded samples of $\state$, with weighting factors given by the autoregressive coefficients up to order (and hence time lag) $q$. For small $q < 3$, it is possible to extract simple conditions on the coefficients, $\{ \phi_{q' \leq q} \}$, that guarantee properties of $\state$: for example, that $\state$ is covariance stationary and mean square ergodic. In our application, we freely employ arbitrary-$q$ models via machine learning in order to improve our approximation of an arbitrary $\state$. Any AR($q$) process can be recast (non-uniquely) into state space form (\cite{harvey1990forecasting}), and we define the AKF by the following substitutions into Kalman equations:
\begin{align}
x_n & \equiv  \begin{bmatrix} f_{n} \hdots f_{n-q+1} \end{bmatrix}^T \\
\Gamma_n w_n & \equiv \begin{bmatrix} w_{n} 0 \hdots 0 \end{bmatrix}^T \\
\Phi_{AKF} & \equiv 
\begin{bmatrix}
\phi_1 & \phi_2 & \hdots & \phi_{q-1} & \phi_q \\ 
1 & 0 & \hdots & 0 & 0 \\  
0 & 1 & \ddots & \vdots & \vdots \\ 
0 & 0 & \ddots & 0 & 0 \\ 
0 & 0 & \hdots & 1 & 0 
\end{bmatrix} \quad \forall n \label{eqn:akf_Phi} \\
H & \equiv \begin{bmatrix} 1\;\;0\;\;0\;\;0\hdots0 \end{bmatrix} \quad \forall n  
\end{align}
The matrix $\Phi_{AKF}$ is the dynamical model used to recursively propagate the unknown state during state estimation in the AKF, as represented schematically in the upper half of \cref{Predive_control_Fig_overview_17_three}. In general, the $\{\phi_{q' \leq q}\}$ employed in $\Phi_{AKF}$ must be learned through an optimisation procedure using the measurement record, where the set of parameters to be optimised is $\{\phi_1, \hdots, \phi_q, \sigma^2, R \}$. This procedure yields the optimal configuration of the autoregressive Kalman filter, but at the computational cost of a $q+2$-dimensional Bayesian learning problem for arbitrarily large $q$.

The Least Squares Filter (LSF) in \cite{mavadia2017} considers a weighted sum of past measurements to predict the $i$-th step ahead measurement outcome, $i \in [0, N_P]$. A gradient descent algorithm learns the weights, $\{\phi_{q' \leq q}\}$ for the previous $q$ past measurements, and a constant offset value for non-zero mean processes, to calculate the $i$-th step ahead prediction. The set of $N_P$ LSF models, collectively, define the set of predicted qubit states under an LSF acting on a measurement record.  For $i=1$, equivalent to the single-step update employed in the Kalman filter, we assert that learned $\{\phi_{q' \leq q}\}$ in LSF effectively implements an AR($q$) process (we validate numerically in \cref{sec:main:Performance}). Under this condition, and for zero-mean $w_n$, the LSF in \cite{mavadia2017} by definition searches for coefficients for the weighted linear sum of past $q$ measurements, as described in in \cref{eqn:main:ARprocess}. 

We use the parameters $\{\phi_{q' \leq q}\}$ learned in the LSF to define $\Phi$ in \cref{eqn:akf_Phi}, therefore reducing the computational complexity of the remaining optimisation from ($(q+2)\to 2$)-dimensional for an AKF of order $q$. Since Kalman noise parameters ($\sigma^2, R$) are subsequently auto-tuned using a Bayes Risk optimisation procedure (see \cref{sec:main:Optimisation}), we optimise over potential remaining model errors and measurement noise.  

\begin{figure} [tp]
    \includegraphics[scale=1]{Predive_control_Fig_overview_17_three_v2_reduced}
    \caption{\label{Predive_control_Fig_overview_17_three} Approaches to construction of the KF dynamical model.  Panel (a) from \cref{fig:main:Predive_control_Fig_overview_17_two} superimposed with Kalman dynamical models, $\Phi \equiv \Phi_n, \forall n$.  (a) AKF/QKF: A set of autoregressive coefficients, $\{\phi_{q'\leq q}\}$, define $\Phi$ to yield $f_n$ as a weight sum of $q$ past measurements. (b) LKFFB: Red arrows with heights $\norm{x^j_n}$ depict set of basis oscillators for $j = 1, \hdots, J^{(B)}$ probe true purple spectrum of $\state_n$ and yields time domain dynamics of $\state_n$ as a stacked system of resonators, $\Theta_j$. Black L-shaped arrows depict a single instance of $\state_n$ at $n=0$ based on historical $\{f_{n-1}, f_{n-2}, \hdots\} $.} 	
  	 %{All Kalman dynamical models, $\Phi \equiv \Phi_n, \quad \forall n$, are mean square approximations to qubit dynamics under dephasing. [Top] AKF/QKF: Kalman $\Phi$ implements a weighted sum of $q$ past measurements driven by process noise, $w_n$. The model $\Phi$ incorporates coefficients, $ \{ \phi_{q' \leq q} \}$, learned from LSF in \cite{mavadia2017}, implementing an autoregressive process of order $q$ in state space form. [Bottom] LKFFB: Kalman $\Phi$ represents a collection of $j = 1, \hdots, J^{(B)}$ oscillators driven by process noise, $w_n$, where the total oscillator basis [red arrows] probes dephasing noise spectrum [purple shaded]. One obtains instantaneous amplitude and phase for each oscillator from $x^j_n$ at any $n$; combining over all $J^{(B)}$ oscillators yields an approximation to the true state.}
\end{figure}

In general, LSF performance improves as $q$ increases and a full characterisation of model-selection decisions for LSF are given in \cite{mavadia2017}. Defining an absolute value for the optimal $q$ is somewhat arbitrary as it is defined relative to the extent to which a true $\state$ is oversampled in the measurement routine and the finite size of the data. For all analyses presented here, we fix the ratio $q \Delta t = 0.1 $ [a.u.] and $q / N_T = 0.05$ [a.u.], where the experimental sampling rate is $1/\Delta t$, $N_T$ and $\{\phi_{q' \leq q}\}$ are identical in the AKF and LSF.   In practice this ensures numerical convergence of the LSF during training.
%For simplicity, we fix a high $q$ model for all numerical experiments considered in this manuscript such that they exhibit numerical convergence behaviour during LSF training. In particular, numerical convergence for LSF means an analysis of errors generated as a gradient descent optimiser is used to learn autoregressive coefficients. Our choice of high $q$ is such that (a) state estimation errors gradually reduce with the number of iterations during a gradient descent optimisation in LSF, and (b), we operate in regimes where net state estimation error at the \textit{end} of a gradient descent optimisation exhibits diminishing returns as $q$ is increased in the underlying LSF model. Implementation details of gradient descent optimisation for LSF are relegated to \cite{mavadia2017}. We numerically confirm that gains for finely tuning $q$ for both LSF and AKF are insignificant for comparisons made in this manuscript.

\subsubsection{Liska Kalman Filter with Fixed Basis (LKFFB)}
In LKFFB, we effectively perform a Fourier decomposition of the underlying $\state$ in order to build the dynamic model, $\Phi$, for the Kalman filter.   Here, we project our measurement record on $J^{(B)}$ oscillators with fixed frequency $\omega_{j}\equiv j\omega_0^{(B)}$ with $j$ an integer as $j = 1, \hdots, J^{(B)}$. The temporal resolution of the state tracking procedure is set by the maximum frequency in the selected basis and properties of the spacing between adjacent basis frequencies. The superscript $ ^{(B)}$ indicates Fourier domain information about an algorithmic basis, as opposed to information about the true (unknown) dephasing process.  The LKFFB allows instantaneous amplitude and phase tracking for each basis oscillator, directly enabling forward prediction from the learned dynamics.  The structure of this Kalman filter, referred to as the Liska Kalman Filter (LKF), was developed in \cite{livska2007}; adding a fixed basis in this application yields the Liska Kalman Filter with a Fixed Basis (LKFFB). 

For our application, the true hidden Kalman state, $x$, is encoded as a collection of sub-states, $x^j$, for the $j^{th}$ oscillator. For clarity we remind that the superscript is used as an index rather than a power.  Each sub-state is labeled by a real and imaginary component which we represent in vector notation: 
\begin{align}
x_n & \equiv \begin{bmatrix} x^{1}_{n} \hdots x^{j}_{n} \hdots x^{J^{(B)}}_{n} \end{bmatrix} \\
A^j_{n} & \equiv \textrm{Re}(x^{j}_{n}) \\
B^j_{n} & \equiv \textrm{Im}(x^{j}_{n}) \\
x^j_n & \equiv \begin{bmatrix} A^j_{n} \\ B^j_{n}  \end{bmatrix}
\end{align} 
The algorithm tracks the real and imaginary parts of the Kalman sub-state simultaneously in order calculate the instantaneous amplitudes ($\norm{x^j_n}$) and phases ($\theta^{j}_{n}$)  for each Fourier component:
\begin{align}
\norm{x^j_n} & \equiv \sqrt{(A^j_{n})^2 + (B^j_{n})^2} \\
\theta^{j}_{n} & \equiv \tan{\frac{B^j_{n}}{A^j_{n}}}
\end{align}

The dynamical model for LKFFB is now constructed as a stacked collection of these independent oscillators. The sub-state dynamics match the formalism of a Markovian stochastic process defined on a circle for each basis frequency, $\omega_j$, as in Ref.~\cite{karlin1975first}. We stack $\Theta(j \omega_0^{(B)}\Delta t) $ for all $\omega_j$ along the diagonal to obtain the full dynamical matrix for $\Phi_n$:
\begin{align}
\Phi_{n} & \equiv \begin{bmatrix} 
\Theta(\omega_0^{(B)}\Delta t)\hdots 0  \\ 
 \hdots \Theta(j\omega_0^{(B)}\Delta t) \hdots \\
0 \hdots \Theta(J^{(B)} \omega_0^{(B)}\Delta t)  \end{bmatrix}\\ 
\Theta(j \omega_0^{(B)}\Delta t) &\equiv \begin{bmatrix} \cos(j \omega_0^{(B)}\Delta t) & -\sin(j \omega_0^{(B)}\Delta t) \\ \sin(j \omega_0^{(B)}\Delta t) & \cos(j \omega_0^{(B)}\Delta t) \\ \end{bmatrix} \label{eqn:ap_approxSP:LKFFB_Phi} 
\end{align}

We obtain a single estimate of the true hidden state by defining the measurement model, $H$, by concatenating $J^{(B)}$ copies of the row vector $[1\;\;0]$ :
\begin{align}
H & \equiv \begin{bmatrix} 1\;\;0 \hdots 1\;\;0 \hdots 1\;\;0 \end{bmatrix}
\end{align}
Here, the unity values of $H$ pick out and sum the Kalman estimate for the real components of $\state$ while ignoring the imaginary components, namely, we sum $A^{j}_{n}$ for all $J^{(B)}$ basis oscillators.

%We observe numerically that instantaneous amplitude and phase information for different basis components are resolved at different timescales while the filter is receiving an incoming stream of measurements (see Appendices). 
In \cite{livska2007}, a state-dependent process-noise-shaping matrix is introduced to enable potentially non-stationary instantaneous amplitude tracking in LKKFB for each individual oscillator: 
\begin{align}
\Gamma_{n-1} &\equiv \Phi_{n-1}\frac{x_{n-1}}{\norm{x_{n-1}}}
\end{align}
For the scope of this manuscript, we retain the form of $\Gamma_{n}$ in our application even if true qubit dynamics are covariance stationary. As such, $\Gamma_{n}$ depends on the state estimates $x$. For this choice of $\Gamma_{n}$, we deviate from classical Kalman filters because recursive equations for $P$ cannot be propagated in the absence of measurement data. Consequently, Kalman gains cannot be pre-computed prior to experimental data collection. Details of gain pre-computation in classical Kalman filtering can be found in standard textbooks (e.g. \cite{grewal2001theory}).

There are two ways to conduct forward prediction for LKFFB and both are numerically equivalent for an appropriate choice of basis: (i) we set the Kalman gain to zero and recursively propagate using $\Phi$; (ii) we define a harmonic sum using the basis frequencies and learned $\{\norm{x^j_n}, \theta^{j}_{n} \}$.  This harmonic sum can be evaluated for all future time to yield forward predictions in a single calculation. The choice of basis for an LKFFB and its implications for optimal predictive performance are discussed in \cref{sec:app:subsec:LKFFB}.





\subsubsection{Quantised Kalman Filter (QKF)}

In QKF, we implement a Kalman filter that acts directly on discretised measurement outcomes, $d \in \{0,1\}$. To reiterate the discussion of  \cref{fig:main:Predive_control_Fig_overview_17_one}(a), this means that the measurement action in QKF must be non-linear and take as input quantised measurement data. This holds true irrespective of our dynamical model, $\Phi$.  In our application we set the dynamical model to be identical to that employed in the AKF, allowing isolation of the effect of the nonlinear, quantised measurement action.

With unified notation across AKF and QKF, we define a non-linear measurement model $h(x)$ and its Jacobian, $H$ as:
\begin{align}
z_n &  \equiv h(x_n[0]) \equiv \frac{1}{2}\cos(\state_{n}) \\
\implies H_n &\equiv \frac{d h(\state_n)}{d\state_n} =  -\frac{1}{2}\sin(\state_{n})
\end{align}
During filtering, $z_n = h(x_n[0])$ is used to compute measurement residuals when updating the true Kalman state, $x_n$, whereas the state variance estimate, $P_n$, is propagated using the Jacobian, $H_n$. Further, the Jacobian  is used to compute the Kalman gain. Hence the filter can quickly destabilise if the linearisation of $h(\cdot)$ by $H_n$ doesn't hold during dynamical propagation, resulting in a rapid build up of errors. 

In this construction, the entity $z_n$ is associated with an abstract `signal': a sequence formed by repeated applications of the likelihood function for a single qubit measurements in \cref{eqn:main:likelihood}.  The true stochastic qubit phase, $\state_n$, is our Kalman hidden state, $x_n$. Subsequently, we extract an estimate of the true bias, $z_n$, as an unnatural association of the Kalman measurement model with Born's rule. The sequence $\{z_n\}$ is not observable, but can only be inferred over a large number of experimental runs. 

To complete the measurement action, we implement a biased coin flip within the QKF filter given $\tilde{y}_n$.  While the qubit provides measurement outcomes which are naturally quantised, we require a theoretical model, $\mathcal{Q}$, to generate quantised measurement outcomes with statistics that are consistent with Born's rule in order to propagate the dynamic Kalman filtering equations appropriately. In order to build this machinery we modify the procedure in \cite{karlsson2005} to quantise $z_n$ using biased coin flips. In our notation, we represent a black-box quantiser, $\mathcal{Q}$, that gives only a $0$ or a $1$ outcome based on $\tilde{y}_n$:
\begin{align}
d_n &= \mathcal{Q}(\tilde{y}_n)\\
&=  \mathcal{Q}(h(\state_n) + v_n)
\end{align}
The use of the notation $\tilde{y}_n$ is meant to indicate a correspondence with $y_{n}$ introduced earlier, while the physical meaning differs due to the discretised nature of the QKF.  Therefore, the stochastic changes in $\{ \tilde{y}_n\}$ are represented in the bias of a coin flip, subject to proper normalisation constraints which maintains $|\tilde{y}_n| \leq 0.5$:
\begin{align}
Pr(d_n| \tilde{y}_n, \state_{n}, \tau) & \equiv \mathcal{B}(n_{\mathcal{B}}=1;p_{\mathcal{B}}= \tilde{y}_n + 0.5 ) \label{eqn:main:qkf:binomial}
\end{align}
QKF uses \cref{eqn:main:qkf:binomial} to define a biased coin-flip during filtering, where $n_{\mathcal{B}}$ represents a single coin flip, $p_{\mathcal{B}}$ represents the stochastically drifting bias on the coin. Kalman filtering with the coin-flip quantisation defined by \cref{eqn:main:qkf:binomial} presents a departure from classical amplitude quantisation procedures in \cite{widrow1996, karlsson2005}.

From a computational perspective, we modify the process noise features definition from AKF to QKF. We set $Q \equiv \sigma^2\Gamma \Gamma^T \to \sigma^2 \mathcal{I} \quad \forall n $, $\mathcal{I}$ is $q\times q$ identity matrix, from AKF to QKF. This rationale for this modification is that it smears out the effect of white process noise in a way that stabilizes inversions in the gain calculation in the Kalman filter, but does not correlate any two Kalman states in time (diagonal matrix). In practice, this modification only yields mild improvements over the original AKF process noise features matrix.

The definitions of $\{ \mathcal{Q}, h(x_n), H_n, Q \}$ in this subsection, and dynamics $\{x_n, \Phi\}$ from the AKF now completely specify the QKF algorithm for application to a discrete, single-shot measurement record as depicted in \cref{fig:main:Predive_control_Fig_overview_17_one} (a).  

\subsection{Gaussian Process Regression (GPR)}

In GPR, correlations in the measurement record can be learned if one projects data on a distribution of Gaussian processes, $Pr(\state)$ with an appropriate encoding of their covariance relations via a kernel, $\Sigma_\state^{n_1, n_2}$. We return to the linear measurement record and the definition of scalar noisy observations $y_{n}$ corrupted by Gaussian measurement noise, $v_n$, as considered previously for AKF, LSF, and LKFFB.  
Under linear operations, the distribution of measured outcomes, $y_n$, is also a Gaussian. The  mean and variance of $Pr(y)$  depends on the mean $\mu_\state$ and variance $\Sigma_\state$ of the prior $Pr(\state)$, and the mean $\mu_v \equiv 0$ and variance $R$ of the measurement noise: 
\begin{align}
\state & \sim Pr_\state(\mu_\state,\Sigma_\state ) \\
y & \sim Pr_y(\mu_\state,\Sigma_\state + R ) 
\end{align}
For covariance stationary $\state$, correlation relationships depend solely on the time lag, $\nu \equiv \Delta t|n_1 - n_2|$ between any two time points  $n_1, n_2 \in [-N_T, N_P]$.  An element of the covariance matrix, $\Sigma_\state^{n_1,n_2}$, corresponds to one value of lag, $\nu$, and the correlation for any given $\nu$  is specified by the covariance function, $R(\nu)$:
\begin{align}
\Sigma_\state^{n_1,n_2} & \equiv R(\nu) 
\end{align}
Any unknown parameters in the encoding of correlation relations via $R(\nu)$ are learned by solving the optimisation problem outlined in \cref{sec:main:Optimisation}. The optimised GPR model is then applied to datasets corresponding to new realisations of $\state$. Let indices $n \in N_T \equiv [-N_T, 0]$ denote training points, and let a length $N^{\ddagger} $ vector contain arbitrary testing points $n^{\ddagger} \in [-N_T, N_P]$. These testing points in machine learning language encompass both state estimation and prediction points in our notation. We now define the joint distribution $Pr(y,\state^{\ddagger})$, where $\state^{\ddagger}$ represents the true process evaluated by GPR at desired test points: 
\begin{align}
\begin{bmatrix} \state^{\ddagger} \\y \end{bmatrix} & \sim \mathcal{N} (\begin{bmatrix} \mu_{\state^{\ddagger}} \\ \mu_y
\end{bmatrix} , \begin{bmatrix}   K(N^{\ddagger},N^{\ddagger})&K(N_T,N^{\ddagger}) \\ K(N^{\ddagger},N_T) & K(N_T,N_T) + R \end{bmatrix} )
\end{align}
The additional `kernel' notation $\Sigma_\state  \equiv K(N_T, N_T)$ is ubitiquous in GPR. Time domain correlations specified by $R(\nu)$ populate each element of a matrix $K(\cdot, \cdot \cdot)$, where the dimensions of the matrix depend on the vector length of each argument. For example, for $K(N_T,N_T)$, the notation defines a square matrix where diagonals correspond to $\nu=0$ and off-diagonal elements correspond to separation of two arbitrary points in time i.e. $\nu \neq 0 $. 
 
Following \cite{rasmussen2005gaussian}, the moments of the conditional predictive distribution $Pr(\state^{\ddagger}|y)$ can be derived from the joint distribution $Pr(y,\state^{\ddagger})$ via standard Gaussian identities:
\begin{align}
\mu_{\state^{\ddagger}|y} &= \mu_\state + K(N^{\ddagger},N_T)(K(N_T,N_T) + R )^{-1} (y - \mu_y) \\
\Sigma_{\state^{\ddagger}|y} &= K(N^{\ddagger},N^{\ddagger}) \nonumber \\
& - K(N^{\ddagger},N_T)(K(N_T, N_T) + R)^{-1}K(N_T,N^{\ddagger}) 
\end{align}
The prediction procedure outlined above holds true for any choice of kernel, $R(\nu)$. In any GPR implementation, the dataset, $y$, constrains the prior model yielding an a posteriori predictive distribution. The mean values of this predictive distribution, $\mu_{\state^{\ddagger}|y}$, are the state predictions for the qubit under dephasing at test points in $N^{\ddagger}$.

In our work we focus on a `periodic kernel' to encode a covariance function which is theoretically guaranteed to approximate any zero-mean covariance stationary process, $\state$, in the mean square limit, by having the same structure as a covariance function for trigonometric polynomials with infinite harmonic terms \cite{solin2014explicit, karlin1975first}. The sine squared exponential kernel represents an infinite basis of oscillators and is defined as:
\begin{align}
R(\nu) &\equiv \sigma^2 \exp (- \frac{2\sin^2(\frac{\omega_0^{(B)}\nu}{2})}{l^2}) 
% R(v) &=  \sigma^2 \exp (- \frac{1}{l^2}) \sum_{n = 0}^{\infty} \frac{1}{n!} \frac{\cos^n(\omega_0^{(B)}\nu)}{l^{2n}} \\
\end{align} 
This kernel is described using just two key hyper-parameters: the frequency-comb spacing for our infinite basis of oscillators, $\omega_0$, and a dimensionless length scale, $l$. We use physical sampling considerations to approximate their initial conditions prior to an optimisation procedure, namely, that the longest correlation length encoded in the data sets the frequency resolution of the comb, and the scale at which changes in $\state$ are resolved is limited physically by the minimum time taken between sequential Ramsey measurements:
\begin{align}
\frac{\omega_0^{(B)}}{2\pi} & \sim  \frac{1}{\Delta t N} \\
l & \sim \Delta t
\end{align} 
Because the periodic kernel can be shown to be formally equivalent to the basis of oscillators employed in the LKFFB algorithm in a limiting case (see \cref{sec:app:spec_methods} for a discussion using results in \cite{solin2014explicit}), the inclusion of GPR using this kernel permits a comparison of the underlying algorithmic structures for the task of predictive estimation using spectral methods. %  A derivation is provided in Appendices using commentary in \cite{solin2014explicit}.

For the analysis of covariance stationary time series under a GPR framework, we de-emphasise popular kernel choices such as: a Gaussian kernel (RBF),  a scale mixture of Gaussian kernels (RQ), and Matern kernels (e.g. MAT32) \cite{rasmussen2005gaussian,tobar2015learning}. An arbitrary-scale mixture of zero-mean Gaussian kernels will probe an arbitrary area around zero in the Fourier domain, as schematically depicted in \cref{fig:main:Predive_control_Fig_overview_17_two}(a). While such kernels capture the continuity assumption ubitiquous in machine learning, they are structurally inappropriate for probing a process characterized by an arbitrary power spectral density (e.g. ohmic noise). Another common kernel for time-series analysis is a quasi periodic kernel (QPER) defined by a product of an RBF with a periodic kernel \cite{roberts2013gaussian}. This corresponds to a convolution in the Fourier domain giving rise to a comb of Gaussians at the expense of an increase in the number of parameters required for kernel tuning. One can also consider specific types of AR($q$) processes using Matern kernels of order $q+1/2$ but with increased restrictions on the form of coefficients \cite{rasmussen2005gaussian,stein2012interpolation}. A simple consideration of autoregressive approaches suggest that a Matern kernel for $q=1$ (MAT32) can be briefly trialed under GPR, whereas high-$q$ autoregressive processes are naturally and generally treated under a KF framework.  Further discussion of kernel choice appears in Sec.~\ref{sec:main:discussion}.

\section{Algorithm Performance Characterisation \label{sec:main:Performance}}

In the results to follow, our metric for characterising performance of optimally tuned algorithms will be the normalised Bayes prediction risk:
\begin{align}
\normpr \equiv \frac{L_{BR}(n|I)}{\langle \left(\state_n - \mu_\state \right)^2 \rangle_{\state, \mathcal{D}}}, \quad \mu_\state \equiv 0
\end{align}
A desirable forward prediction horizon corresponds to maximal $n^* \in [0, N_P]$ for which normalised Bayes prediction risk at all time-steps $n \leq n^*$ is less than unity. We compare the difference in maximal forward prediction horizons between algorithms in the context of realistic operating scenarios.  We begin here by introducing the numerical methods employed for generating data-sets on which predictive estimation is performed.

We simulate environmental dephasing through a Fourier-domain procedure described in \cref{sec:app:truenoise} \cite{soare2014} in order to simulate an $\state$ which is mean-square ergodic and covariance stationary.  For the results in this manuscript, we choose a flat top spectrum with a sharp high-frequency cutoff for simplicity as this choice of a power spectral density theoretically favors no particular choice of algorithm but shows strong non-Markovianity. 

In our simulations we also must mimic a measurement process which samples the underlying ``true'' dephasing process.  The algorithmic parameters $\{N_T, \Delta t\} $ represent a sampling rate and Fourier resolution set by the simulated measurement protocol; we choose regimes where the Nyquist rate, $r \gg 2$. In generating noisy simulated measurement records, we corrupt a noiseless measurement by additive Gaussian white noise. Since $\state$ is Gaussian, the measurement noise level, $N.L.$, is defined as a ratio between the standard deviation of additive Gaussian measurement noise, $\sqrt{R}$ and the maximal spread of random variables in any realisation $\state$. We approximate the maximal spread of $\state$ as three sample standard deviations of one realisation of true $\state$, $N.L. = \sqrt{R}/3\sqrt{\hat{\Sigma}_\state^{n,n}}$. The use of a hat in this notation denotes sample statistics. This computational procedure enables a consistent application of measurement noise for $\state$ from arbitrary, non-Markovian power spectral densities. For the case where binary outcomes are required, we apply a biased coin flip using \cref{eqn:main:qkf:binomial}.

\subsection{Algorithmic Optimisation \label{sec:main:Optimisation}}

All algorithms in this manuscript employ machine learning principles to tune unknown design parameters based on training data-sets. The physical intuition associated with optimising our filters is that we are cycling through a large class of general models for environmental dephasing and seeking the model(s) which best fit the data subject to various constraints. This allows each filter to track stochastic qubit dynamics under arbitrary covariance-stationary, non-Markovian dephasing.  We elected to deploy an optimisation routine with minimal computational complexity to enable nimble deployment of KF and GPR algorithms in realistic laboratory settings, particularly since LSF optimisation is extremely rapid for our application \cite{mavadia2017}. 

Kalman filtering in our setting poses a significant challenge for general optimisers as the lack of theoretical bounds on the values of ($\sigma, R$) result in large, flat regions of the Bayes Risk function. Further, the recursive structure of the Kalman filter means that no analytical gradients are accessible for optimising a choice of cost function and a large computational burden is incurred for any optimisation procedure. We randomly distribute $(\sigma_{k}, R_{k})$ pairs for $k=1, \hdots K $ over ten orders of magnitude in two dimensions in order to sample the optimisation space.

We then generate a sequence of loss values $L(\sigma_k, R_k)$ for each $k$ by considering a small region around $n=0$, where the size of the region is $n_L$ number of time steps we look forward or backwards from $n=0$:
\begin{align}
L(\sigma_k, R_k) \equiv  \sum_{n=1}^{n_L} L_{BR}(n | I= \{\sigma_k, R_k \}) \label{eqn:main:risk:optriskvalue}.
\end{align}
Here, $L_{BR}(n | I= \{\sigma_k, R_k \})$ is given by \cref{eqn:main:sec:ap_opt_LossBR} and it is summed over $0\leq n_L\leq |N_{T}|$  ($0\leq n_L\leq |N_{P}|$) backwards (forwards) time-steps for state estimation (prediction). In the notation for $I$ above, we omit Kalman dynamical model design parameters for ease of reading.  Typically $I$ would include, for instance, the set of autoregressive coefficients in AKF and the set of fixed basis frequencies in LKFFB. Values of $n_L$ are chosen such that the sequence $\{L(\sigma_k, R_k) \}$ defines sensible shapes of the total loss function over parameter space and the numerical experiments in this manuscript. A choice of small $n_L$ in state-estimation ensures that data near the prediction horizon are employed - a region where the Kalman filter is most likely to have converged.  Similarly, in state prediction, large $n_L$ will flatten the true prediction loss function as long-term prediction errors dominate smaller loss values occurring during the short term prediction period. In addition, one can weight state estimation and state prediction loss functions differently by choosing different values of $n_L$ for state estimation and prediction, though we set $n_L$ to be the same in both regions. While simple and by no means optimal, our tuning approach is computationally tractable and efficient compared to the application of standard optimisation routines where each loss value calculation requires a recursive filter to act on a long measurement record. Further, our approach ensures tuning procedures are performed off-line such that a tuned algorithm is simple in its recursive structure and performs rapid calculations at each time-step.

An ideal parameter pair ($\sigma^*, R^*$) minimises Bayes risk over $K$ trials for both state estimation and prediction.  We define acceptable low loss regions for state estimation and prediction as being the set which returns loss less than 10\% of the median risk over $K$ trials.  In the event that low risk regions do not exist for both state estimation and prediction for a given parameter pair, we deem the optimisation to have failed as backwards state estimation performance is uncorrelated with forward prediction (for illustration, see panel (h) of \cref{fig:main:fig_data_specrecon}).

In GPR the set of parameters $I = \{\sigma, R, \omega_0^{(B)}, l \}$ requires optimisation.  However, in contrast to the KF, no recursion exists and analytic gradients are accessible to simplify the overall optimisation problem. Instead of minimising Bayes state-estimation risk, we follow a popular practice of maximising the Bayesian likelihood. Initial conditions and optimisation constraints are derived from physical arguments as described in \cref{sec:main:OverviewofPredictive Methodologies}.

\subsection{Performance of the KF using linear measurement}
\begin{figure}
    \includegraphics[scale=1.0]{fig_data_state_pred_reduced}
    \caption{\label{fig:main:fig_data_state_pred} (a) Solid dots depict $y_n$ against time-steps $n$ and data collection ceases at $n=0$. Optimised LSF, AKF and LKFFB yield predictions $n>0$ in the blue region plotted as open, coloured markers. A black solid line shows one realisation of true $\state_n$, drawn from a flat top spectrum with $J$ true Fourier components spaced $\omega_0$ apart and uniformly randomised phases. Other parameters: $\omega_0 / \omega_0^{(B)}  \notin \mathcal{Z}$ (natural numbers), $J = 45000$, $\omega_0 / 2\pi = \frac{8}{9} \times 10^{-3}$ Hz such that $>500$ number of true components fall between adjacent LKFFB oscillators; $ N.L.= 10\%$. (b)-(d) Procedure in (a) is repeated for ensemble $M$ different realisations of $\state$ and noisy datasets to compute $\normpr$ for LSF, AKF, and LKFFB. $\normpr$ v. $n \in [0, N_P]$ is plotted; black horizontal line marks $\normpr \equiv 1$ for predicting the mean $\mu_\state \equiv 0$. Vertical dashed lines mark the forward prediction horizon, $ n^* $, where $  \normpr \lesssim 0.8 < 1$ for all prediction time steps  $0< n \leq n^*$ in out-performing predicting the noise mean. Marker color (dark indigo to pink) depicts true $\state$ cutoff, $J\omega_0$ varied relative to $\omega^{(B)} \equiv \omega_0^{(B)}J^{(B)} \approx r \omega_{(S)}$, with fixed Nyquist $r\gg2$; $\omega_0 / 2\pi = 0.497$ Hz, $J = 20, 40, 60, 80, 200$; $N.L. = 1\%$.  For all (a)-(d), a trained LKFFB is implemented with $\omega_0^{(B)} / 2\pi = 0.5$ Hz and $J^{(B)} =100$ oscillators; trained AKF / LSF models are $q = 100$; with $N_T = 2000, N_P = 50$ steps, $\Delta t = 0.001$s, $M=50$ runs, $K=75$ optimisation trials.} 
\end{figure}  


The general performance of the various KF algorithms discussed above is illustrated in Fig.~\ref{fig:main:fig_data_state_pred} which compares the AKF and LKFFB algorithms using a linear measurement record.  Here the solid black line represents the underlying true $\state$ and solid markers indicate noisy simulated linear measurement data.  Future predictions using the various KF formalisms and the (non-recursive) LSF filter~\cite{mavadia2017} are shown as coloured open markers, based on these data.  The selected single realization of the prediction process demonstrated in (a) is representative of a broad ensemble of simulated data sets and demonstrates the ability of all algorithms to perform future prediction with varying degrees of success.  


In general, our objective is to maximise the forward prediction horizon, $n^{*}$, in any algorithmic setting.  In Fig.~\ref{fig:main:fig_data_state_pred}(b)-(d), we explore the key determining factors setting the value of the prediction horizon under the three main Kalman filtering algorithms treated here.  We plot the ensemble-averaged $\normpr$ as a function of forward prediction time when adjusting the ratio of the cutoff frequency in the noise, $J\omega_{0}$, to the sample rate in the measurement routine ($\omega_{(S)}=2\pi/\Delta t$) without physical aliasing such that Nyquist $r \gg2$ and $\omega_{(S)} \approx \omega^{(B)} / r$, where $\omega^{(B)}$ incorporates a (potentially incorrect) bandwidth assumption about dephasing noise for LKFFB. Here again, we have a forward prediction horizon for time-steps $0 < n < n^*$ if $\normpr\lesssim 1$ for all time-steps in this region and an algorithm seeks to maximise $n^*$. In this region, each algorithm predicts future dynamics better than naively predicting the mean behaviour of $\state$ ($\mu_f \equiv 0$), indicated by a gray horizontal line.

The prediction horizon, indicated approximately by dashed vertical lines, for all algorithms increases as the measurement becomes sufficiently fast to sample the highest frequency dynamics of $\state$.  We confirm numerically that absolute prediction horizons for any algorithm are arbitrary and adjustable through the sample rate, allowing us to restrict our analysis to comparative statements between algorithms for future results.  While differences between protocols appear reasonably small we note that in most cases examined the AKF demonstrates superior performance to the LKFFB subject to the realistic constraint that the true dynamics of $\state$ cannot be perfectly projected onto the basis used in LKFFB (the latter situation corresponding to substantial a priori knowledge of the dynamics of $\state$).  The role of undersampling in the LKFFB becomes pronounced as predictive estimates lead to unstable behavior relative to the naive prediction of $\mu_f = 0$ in the case $J\omega_{0}/\omega^{(B)}=2$ in \cref{fig:main:fig_data_state_pred}(d).  The AKF and LSF share autoregressive coefficients and therefore both algorithms demonstrate comparable $\normpr$ prediction risk in the ensemble average.

\begin{figure}[b]
    \includegraphics[scale=1.]{fig_data_akfvlsf_reduced}
    \caption{\label{fig:main:fig_data_akfvlsf} Measurement noise filtering in AKF v. LSF. (a) Dashed-lines with markers depict the ratio of $\normpr$ for AKF to LSF against time-steps $n>0$; for cases (i)-(iv) with $N.L. = 0.1, 1.0, 10.0, 25.0 \%$. Green trajectory shows LSF outperforms AKF with ratio $>1$ for $n\leq n^*$; crimson trajectories show AKF outperforms LSF with ratio $<1$ for $n\leq n^*$. (b) $\normpr$ against $n$ is plotted for cases (i)-(iv) confirms a maximal forward prediction horizon marked by $n^*$, exists for all ratios in (a) for both LSF and AKF. In (a) and (b), AKF and LSF share identical $\{ \phi_q \}$. True $\state$ is drawn from a flat top spectrum with $\omega_0 / 2\pi = \frac{8}{9} \times 10^{-3}$ Hz, $J = 45000$, $N_T = 2000, N_P = 100$ steps, $\Delta t = 0.001s, r=20$ such that \cref{fig:main:figure_lkffb_path}(c) corresponds to case (ii) in this figure. AKF is optimised with $M=50$ runs, $K=75$ trials.}
\end{figure}

\begin{figure*} [htp]
\includegraphics[scale=1.0]{figure_lkffb_path_reduced}
\caption{\label{fig:main:figure_lkffb_path} 
Comparison of KF performance under various imperfect learning scenarios. (a)-(d) True noise properties are varied to introduce pathological learning with respect to fixed algorithmic configuration: $\omega_0 / 2\pi = 0.5, 0.499, \frac{8}{9} \times 10^{-3}, \frac{8}{9} \times 10^{-3}$ Hz and $J = 80, 80, 45000, 80000$ respectively. The relationship between LKFFB basis and true noise spectrum is shown schematically above columns: (a) perfect learning; (b) imperfect projection on LKFFB basis; (c) finite computational Fourier resolution; (d) relaxed basis bandwidth assumption. (a)-(d) $\normpr$  against time-steps $n>0$ is shown for LKFFB, AKF, and LSF. (e)-(l) Optimisation results for LKFFB [top row] and AKF [bottom row] in each of the four regimes in (a)-(d). Grey dots depict $K$ random ($\sigma^2, R$) pairs; where $M$ realisations of $\state, \mathcal{D}$ are used to calculate $\normpr$ for each pair. Purple (crimson) circles represent low loss regions where risk value in \cref{eqn:main:risk:optriskvalue},for  ($\sigma^2, R$) is $< 10\%$ of the median  risk value during state estimation (prediction) for $-n_L < n <0$ ($n_L > n >0$),  with $n_L=50$. Black star, ($\sigma^*, R^*$), minimises risk values over purple circles during state estimation.  A KF filter is `tuned' if optimal ($\sigma^*, R^*$) lies in the overlap of low loss regions for state estimation [purple] and prediction [crimson]; disjoint regions in (h) show LKFFB tuning failure. KF algorithms set up with $q = 100$ for AKF; $J^{(B)} = 100, \omega_0^{(B)} / 2\pi = 0.5$ Hz for LKFFB; with $N_T = 2000, N_P = 100$ steps, $\Delta t = 0.001$s, $r=20$; $ N.L. = 1\%$.}  
\end{figure*} 


A key implied benefit of the use of Kalman filtering vs the LSF with high-order autoregressive dynamics alone is the addition of robustness against measurement noise.  In order to probe this numerically, we perform direct comparisons of filter performance under varying measurement-noise strength for both the AKF and LSF.  Since autoregressive coefficients learned in (noisy) environments are re-cast in Kalman form, we test measurement-noise filtering in Kalman frameworks enabled by the design parameter $R$. In \cref{fig:main:fig_data_akfvlsf} (a), we plot $\normpr$ prediction risk for AKF and LSF as a ratio such that a value greater than unity implies LSF outperforms AKF. In cases (i)-(iv), we increase the applied noise level to our data-sets $\{ y_n \}$ representing simulated measurements on $\state$. For applied measurement noise level $N.L. > 1\%$ in (ii)-(iv), we find that $AKF/LSF <1 $ and AKF outperforms LSF for the conditions studied here, with a general trend towards increasing benefits as noise increases until the noise becomes so large (iv) that the benefits fluctuate as a function of $n$. Calculations of the ensemble-averaged $\normpr$ in \cref{fig:main:fig_data_akfvlsf} (b) demonstrate that all ratios reported in (a) correspond to a useful forward prediction horizon. 


In machine learning or optimal control settings, the robustness of the learning procedure to small changes in the underlying system is an essential characteristic of the algorithm.  In our case, we have already seen that the quality of projection of the true dynamics of $\state$ onto the LKFFB basis can have a significant impact on the quality of learning and predictive estimation.  We explore this initial finding in more detail.  

In \cref{fig:main:figure_lkffb_path}, we simulate various learning conditions including (a) perfect learning in LKFFB; (b) imperfect projection relative to the LKFFB basis; (c) imperfect projection combined with finite algorithm resolution; and (d) imperfect learning and undersampling relative to true noise bandwidth. The ordering of figure presentation highlights the degree of impact of the introduced pathologies on LKFFB.  By contrast we find reasonable model robustness in AKF/LSF at the expense of performance in the somewhat unrealistic perfect learning case.  

We expose the underlying optimisation results for choosing an optimal $(\sigma^*, R^*)$ for LKFFB in \cref{fig:main:figure_lkffb_path} (e)-(h) and for AKF in \cref{fig:main:figure_lkffb_path} (i)-(l). Individual sample points are highlighted as solid dots while low-loss pairs in this 2D space are highlighted for giving low state-estimation [purple] or prediction [crimson] risk via shaded circles.  As the model pathologies indicated above increase, these data demonstrate a divergence between regions of the optimisation space which permit low-loss state estimation and forward prediction for LKFFB.  In contrast, overlap of low loss Bayes Risk regions do not change for AKF across \cref{fig:main:figure_lkffb_path} (i)-(l).

Kalman filtering algorithms employed here combine recursive state estimation with the establishment of a dynamical model in the Fourier domain.  Therefore, one way to explore algorithmic performance is to look directly at the efficacy of spectral estimation relative to the true (here numerically engineered) hidden dynamics of $\state$.  For both the LKFFB and AKF we plot the extracted power spectral density, $S(\omega)$, as a function of angular frequency $\omega$, for different measurement sampling conditions in \cref{fig:main:fig_data_specrecon} against the true spectrum used to define $\state$.  These simulated experimental conditions match those introduced in \cref{fig:main:fig_data_state_pred} (b). 

In the case of LKFFB, we plot the learned instantaneous amplitudes from a single run [blue markers] and for AKF we extract optimised algorithm parameters as described above [red markers]. Under the assertion that the LSF implements an AR($q$) process, the set of trained parameters, $\{  \{\phi_{q' \leq q}\}, \sigma^2\}$ from AKF allows us to derive experimentally measurable quantities, including the power spectral density of the dephasing process: $S(\omega) = \sigma^2 \left(2 \pi |1 - \sum_{q'=1}^q \phi_{q'} e^{-i\omega q'}|^2)\right)^{-1} $ \cite{brockwell1996introduction}. %\label{eqn:main:ap_ssp_ar_spectden}  

\begin{figure}[tp]
    \includegraphics[scale=1.0]{fig_data_specrecon_reduced}
     \caption{\label{fig:main:fig_data_specrecon} (a)-(d) Blue (red) open markers plot LKFFB (AKF) spectrum estimates; true spectrum (flat top) of $\state$ plotted in black solid line. Dashed black vertical line marks true noise cutoff, $J\omega_0$, and this is varied relative to a measurement sampling rate, $\omega_{(S)}$, and $\omega^{(B)}\equiv \omega_0^{(B)} J^{(B)} \approx \omega_{(S)}/r $ in LKFFB; such that $\omega_0 / 2\pi = 0.497$ Hz, $J = 20, 40, 80, 200$. For LKFFB, blue open markers are $\propto ||\hat{x}^j_n||^2 $ in a single run with $\omega_0^{(B)} / 2\pi = 0.5$ Hz for $j \in J^{(B)} = 100$ oscillators; dashed blue vertical line marks edge of LKFFB basis. For AKF,  red markers are $\hat{S}(\omega)$ computed using learned $\{\phi_{q' \leq q}\}$ and optimised $\sigma^*$, with order $q = 100$.  In all plots, the zeroth Fourier component is omitted on the log scale; and $N_T = 2000, N_P = 50$ steps, $\Delta t = 0.001s, r=20$, with $M=50$ runs, $K=75$ trials; $N.L. = 1\%$.} 
    %\caption{\label{fig:main:fig_data_specrecon} Comparison of the true power spectrum for $\state$ with derived spectral estimates from LKFFB and AKF. (a)-(d) True $\state$ cutoff, $J\omega_0$, is varied relative to a measurement sampling rate, $\omega_{(S)}$, and (equivalently) a priori noise bandwidth assumption $\omega^{(B)}\equiv \omega_0^{(B)} J^{(B)} \approx \omega_{(S)}/r $ in LKFFB; such that $\omega_0 / 2\pi = 0.497$ Hz, $J = 20, 40, 80, 200$. For LKFFB, we use learned amplitude information from a single run ($\propto ||x^j_n||^2 $) with $\omega_0^{(B)} / 2\pi = 0.5$ Hz for $j \in J^{(B)} = 100$ oscillators. For AKF, we plot $S(\omega)$ using optimally trained $\{\phi_{q' \leq q}\}$ and $\sigma^2$, with order $q = 100$. The zeroth Fourier component and its estimates are omitted to allow for plotting on a log scale; and $N_T = 2000, N_P = 50$ steps, $\Delta t = 0.001s, r=20$, with optimisation performed for $M=50$ runs, $K=75$ trials and measurement noise level $N.L. = 1\%$.} 
\end{figure} 

The critical feature in these data-sets is the existence of a flat-top spectrum possessing a sharp high frequency cutoff.  Both classes of Kalman filtering algorithm successfully identify this structure and locate this high-frequency cutoff.  In general, however, the LKFFB provides superior spectral estimation relative to the AKF, and enables better estimation of the signal strength in the Fourier domain even in the presence of imperfect projection of $\state$ onto the basis used in LKFFB.  The only case in which the LKFFB fails is in \cref{fig:main:fig_data_specrecon}(d), where the LKFFB basis is ill-specified relative to the true noise bandwidth. The observed behavior is somewhat surprising given the generally superior performance of the AKF in predictive estimation, but does highlight the practical difference between Fourier-domain spectral estimation and time-domain prediction.  %The performance of the AKF in this procedure is determined by the accuracy of the scaling factor given by optimally tuned parameter $\sigma$, pointing to the importance of optimisation for algorithmic performance. In contrast, instantaneous amplitudes are tracked in one run of LKFFB and are less susceptible to optimisation over model parameters.


%\newpage
\subsection{Performance of the quantised Kalman filter}
The discrete nature of projective measurement outcomes in quantum systems poses a potential challenge for Kalman filters in the event that measurement pre-processing as in \cref{fig:main:Predive_control_Fig_overview_17_one}(b) is not performed.  We test filter performance for predictive estimation when only binary measurement outcomes are available via the QKF.  To re-iterate, QKF estimates and tracks hidden information, $\state_n$, using the Kalman true state $x_n$.  In our construction the associated probability for a projective qubit measurement outcome, $\propto z_n$ is not inferred or measured directly but given deterministically by Born's rule encoded in the non-linear measurement model, $z_n = h(f_n)$. The measurement action is completed by performing a biased coin flip, where $z_n$ determines the bias of the coin.  

For QKF, the normalised ensemble-averaged prediction risk, $ \langle (z_n - \hat{z}_n)^2 \rangle_{f, \mathcal{D}} / \langle (z_n - \mu_z)^2 \rangle_{f, \mathcal{D}}$, is calculated with respect to $z$ as the relevant quantity parameterising qubit-state evolution, instead of the stochastic underlying $\state$. This quantity is labeled as Norm. Risk in \cref{fig:main:fig_data_qkf2} and we see if $ \langle (z_n - \hat{z}_n)^2 \rangle_{f, \mathcal{D}} / \langle (z_n - \mu_z)^2 \rangle_{f, \mathcal{D}} < 1$ for $0< n < n^*$ can be achieved for numerical experiments considered previously in the linear regime. In particular, we generate true $\state$ defined in numerical experiments in \cref{fig:main:fig_data_state_pred}(b) (and \cref{fig:main:fig_data_specrecon}) for $q=100$ and varying sample rates.

We isolate the role of the measurement action by first inputting into the QKF a true dynamical model rather than a dynamical model learned as in the standard AKF.  To specify true dynamics, we begin with a set of $\{ \phi_{q'\leq q}\}$ and exactly derive a new $f'$.  As a result the full set of parameters relevant to the filter, $\{\{\phi_{q' \leq q} \}, \sigma, R\}$, are perfectly defined and known, and the filter simply acts on single shot qubit measurements.  These simulations reveal that subject to generic measurement oversampling conditions introduced above the QKF is able to successfully enable predictive estimation.  As in the linear case, the absolute forward prediction horizon is arbitrary relative to $\omega_0 J / \omega^{(B)}$ and implicitly, an optimisation over the choice of $q$ for a finite data size, $N_T$,  in our application. 

\begin{figure}[h!]
    \includegraphics[scale=1.]{fig_data_qkf_reduced}
    \caption{\label{fig:main:fig_data_qkf2}Norm. Risk against $n>0$ plotted for QKF in open markers; black line at $\mu_\state \equiv 0$ depicts performance under predicting the noise mean. QKF outperforms predicting the mean if open markers lie in green regions. Marker colour (dark indigo to pink) depicts true noise cutoff varied $J \omega_0 / \omega_{(B)} = 0.2, 0.4, 0.6, 0.8$ for $\state$ defined identically in  \cref{fig:main:fig_data_specrecon} with $\omega_0/ 2\pi = 0.497 $ Hz, $J = 20, 40, 60, 80$;  $N.L. = 1 \%$. (a) We obtain $\{\phi_{q' \leq q}\}, q=100$ coefficients from AKF/LSF acting on a linear measurement record generated from true $\state$. A new truth, $\state'$, is generated from an AR($q$) process using $\{\phi_{q'\leq q}\}, q=100$ as true coefficients and by defining a known, true $\sigma$. Quantised measurements from $f'$ are obtained; data is corrupted by measurement noise of a true, known strength $R$. (b) We use $\{\phi_{q' \leq q} \}, q=100$ coefficients from (a) but we generate quantised measurements from the original, true $\state$. QKF noise design parameters are optimised for ($\sigma_{AKF}^* \leq \sigma_{QKF}$, $R_{AKF}^* \leq R_{QKF}$) with $M=50$ runs, $K=75$ trials. For (a)-(b), $N_T = 2000, N_P = 50$ steps, $\Delta t = 0.001$s, $r\gg 2$.}
    %\caption{\label{fig:main:fig_data_qkf2} Bayes prediction risk for QKF against time-steps $n>0$. True $\state$ cutoff varied relative to an a priori noise bandwidth assumption such that $J f_0 / f_{(B)} = 0.2, 0.4, 0.6, 0.8$ for an initially generated true $\state$ in \cref{fig:main:fig_data_specrecon} with $\omega_0/ 2\pi = 0.497 $ Hz, $J = 20, 40, 60, 80$. Measurement noise is incurred on $\state$ at $N.L. = 1 \%$ for the linear measurement record and on $z$ at $N.L. = 1\%$ level corresponding to the non-linear measurement record. (a) We obtain $\{\phi_{q' \leq q}\}, q=100$ coefficients from AKF/LSF acting on a linear measurement record generated from true $\state$. A new truth, $\state'$, is generated from an AR($q$) process using $\{\phi_{q'\leq q}\}, q=100$ as true coefficients and by defining a known, true $\sigma$. Quantised measurements from $f'$ are obtained; data is corrupted by measurement noise of a true, known strength $R$. (b) We use $\{\phi_{q' \leq q} \}, q=100$ coefficients from (a) but we generate quantised measurements from the original, true $\state$. QKF noise design parameters are optimised for ($\sigma_{AKF}^* \leq \sigma_{QKF}$, $R_{AKF}^* \leq R_{QKF}$) with with $M=50$ runs, $K=75$ trials. For (a)-(b), forward prediction horizons are shown with $N_T = 2000, N_P = 50$ steps, $\Delta t = 0.001$s, $r\gg 2$.}
\end{figure}


Our simulations reveal that the QKF is considerably more sensitive to measurement noise, model errors, and the degree of undersampling than the linear model as shown in \cref{fig:main:fig_data_qkf2} (b). Here the QKF incorporates a learned dynamical model from AKF in the linear regime and we tune $(\sigma, R)$ for use in the QKF.  In particular, we explore $\sigma \geq \sigma_{AKF}^*$ to incorporate model errors as $\{\phi_{q' \leq q}\}$ were learned in the linear regime.  We also incorporate increased measurement noise via $R \geq R_{AKF}^*$ as QKF receives raw data that has not been pre-processed or low-pass filtered. The underlying optimisation problems are well behaved for all cases in \cref{fig:main:fig_data_qkf2}(b) [not shown].  As the sampling rate is reduced, the QKF forward prediction horizon collapse rapidly i.e $\langle (z_n - \hat{z}_n)^2 \rangle_{f, \mathcal{D}} / \langle (z_n - \mu_z)^2 \rangle_{f, \mathcal{D}} > 1 $ prediction risk for all $n>0$.  




\subsection{Failure of GPR in predictive estimation} 
Under a GPR framework, we test whether predictive performance can be improved by considering the entire measurement record (at once) and projecting this record on an infinite basis of oscillators summarised by a periodic kernel. We investigate several different types of GPR models for $M=50$ realisations of $\state$ in the top panel of \cref{fig:main:fig_data_gpr}. For the results shown, we use a popular choice of a maximum-likelihood optimisation procedure implemented via L-BFGS in GPy \cite{gpy2014}.

We find that the underlying optimisation procedure for training on our measurement records remains difficult despite having access to an analytical calculation for the cost function. For all results in \cref{fig:main:fig_data_gpr}(a) and (b), we use significant manual tuning prior to deploying the automated procedures in GPy. Hence, we focus on using numerical results under GPR to illuminate structural implications of the choice of kernels in our application, rather than making comparative statements about kernel performance.
 
 The results we have assembled demonstrate that the implementation of GPR with a periodic kernel critically depends on the frequency basis comb spacing, $\omega_0^{(B)}$, or equivalently, a deterministic quantity, $\kappa$:
\begin{align}
\kappa & \equiv \frac{2\pi}{\Delta t \omega_0^{(B)}} - N_T 
\end{align}
The term $ 2\pi /  \Delta t \omega_0^{(B)}$ is the theoretical number of measurements that, in principle, would be required to \emph{physically} achieve the Fourier resolution set by the kernel hyper-parameter, $\omega_0^{(B)}$, and the fundamentally discrete nature of a sequential Ramsey measurement record, expressed by $\Delta t$. Hence, if $\kappa = 0$, the physical Fourier resolution determined by the data set matches the comb spacing in the periodic kernel. For $\kappa > 0$, the comb spacing in the periodic kernel is less than the Fourier spacing defined by the experimental data collection protocol, with total measurements $N_T$. 

In \cref{fig:main:fig_data_gpr}(a), we see that GPR predictive performance for the periodic kernel improves as the Kernel's comb spacing is reduced. For each value of $\kappa$ we plot $\normpr$ against time-steps forward, $n^\ddagger$, where the $\ddagger$ corresponds to the evaluation of a predictive GPR distribution on arbitrarily chosen test points, $n^\ddagger = -N_T, \hdots, -1, 0, 1 \hdots, N_P$. Here, the optimiser is constrained to a region in $2\pi/ \omega_0^{(B)}$ parameter space that corresponds to the order of magnitude for $\kappa$. Grey markers correspond to $\kappa \leq 0$, where the algorithm operates above (or at) the Fourier resolution.  In this physically motivated parameter regime, prediction fully fails.  It is not until we set $\kappa\sim10^{3}$ -- a nominally unphysical operating regime where the algorithm's frequency-comb spacing is smaller than the Fourier resolution -- that prediction succeeds [red traces]. This latter case is physically difficult to interpret given that in this regime we find the best ensemble-averaged predictive performance only by providing unphysical freedom to the algorithm.  We note that the optimised length scale for the periodic kernel remains on of order $ \Delta t \sim 10 \Delta t$, such that for all red trajectories in panel (a), we are operating in a high $2\pi/ \omega_0^{(B)}$, low $l$ limit. 

We contextualise the predictive performance of the GPR periodic kernel (PER) [red solid line] in the high-$\kappa$, low-$l$ limit by comparing against predictions derived using other standard kernels [dotted lines] in the inset to ~\cref{fig:main:fig_data_gpr}(a). In such circumstances the predictive performance of the periodic kernel predictive is on par with an application of a Gaussian kernel (RBF) and a scale mixture of zero mean Gaussians with different decay lengths (RQ).  A Matern kernel (MAT32) and a quasi periodic kernel (QPER) yield lower-than-anticipated performance. Further discussion of the choice of kernel appears in Sec.~\ref{sec:main:discussion}. For each individual time-trace contributing to the ensemble averages appearing here, we observe that all kernels (PER, RBF, RQ, MAT32, QPER) yield good state estimation and the state estimate at $n^\ddagger=-1$ agrees well with the truth. For GPR with a PER, RBF, and RQ kernels, the state estimate at $n^\ddagger=-1$ smoothly decays to the mean value (zero) for $n^\ddagger \geq 0$ and this effect yields a favourable normalised Bayes prediction risk immediately after $n^\ddagger>0$ depicted by the solid lines in inset of \cref{fig:main:fig_data_gpr}(a).

\begin{figure}
    \includegraphics[scale=1.]{fig_data_gpr_2_reduced}. 
    \caption{\label{fig:main:fig_data_gpr} (a) $\normpr$ v. $n^\ddagger$ (in units of number of time steps) are plotted for GPR with a periodic kernel. Black horizontal line at unity for $\mu_\state \equiv 0$ marks $\normpr$ under predicting the mean; GPR outperforms predicting the mean if data falls below this line. Grey-black markers correspond to optimisation within physical bounds for $\kappa \leq 0 $ (kernel resolution at or above Fourier resolution); crimson markers and lines depict optimisation within unphysical regimes, $\kappa >0$; with solid lines in high $\kappa \gg 0$ regime. Remaining  $\{R, \sigma, l\}$ optimised for non-negative values. Inset (a) $\normpr$ v. $n^\ddagger$ of periodic kernel (PER) with $\kappa \approx 10^3$ is plotted against results from naively trained Gaussian kernels (RBF, RQ); a Matern kernel (MAT32) and a quasi-periodic kernel (QPER). (b)-(d) True state $\state_n$ v. $n$ [black solid line] and GPR predictions $\hat{\mu}_{\state^\ddagger}$ v. $n^\ddagger$ [open markers]  plotted for periodic kernel for tracking a sinusoid with frequency, $\omega_0$; noisy data record [not shown] ceases at $n=0$. We fix $\kappa = 0, 70$; triangles plot predictions for manually tuned $\{R, \sigma, l\}$; circles plot predictions for optimised $\{R, \sigma, l\}$. Vertical dashed lines mark $n=\kappa$, where we overlay true $\state$ at the beginning of the data record as a red dashed line. (b) Perfection projection is possible $\omega_0 / \omega_0^{(B)} \in \mathcal{Z}$ (natural numbers), $\omega_0/2\pi = 3$ Hz. (c) Imperfect projection, with $\omega_0 / \omega_0^{(B)} \notin \mathcal{Z}$, $\omega_0 / 2 \pi = 3 \frac{1}{3}$ Hz, $\kappa=0$. (d) Moderately raise $\kappa > 0 $, such that $\omega_0 / \omega_0^{(B)} \gg 0 \notin \mathcal{Z}$ for original $ \omega_0 / 2 \pi = 3$ Hz. (e) Test (c) and (d) for $\kappa > 0$, $ \omega_0 / \omega_0^{(B)} \notin \mathcal{Z}$, $\omega_0 / 2 \pi = 3 \frac{1}{3}$ Hz. For (b)-(e), $N_T = 2000, N_P = 150$ steps, $\Delta t = 0.001$s; $N.L.= 1\%$.}    	
    	%(a) $\normpr$ v. $n^\ddagger$ is plotted for GPR models with $M=50$ realisations of true noise with properties identically defined in \cref{fig:main:fig_data_akfvlsf}(b)(ii). A periodic kernel comb spacing undergoes constrained optimisation such that $\kappa \leq 0, =0, \gg 0$.  Inset (a) $\normpr$ v. $n^\ddagger$ of periodic kernel (PER) is plotted against results from naively trained Gaussian kernels (RBF, RQ); a Matern kernel (MAT32) and a quasi periodic kernel (QPER). In (b)-(e), GPR predictions   $(\mu_{\state^*|y})$  are plotted against time-steps, $n^\ddagger$, for a single run;  with fixed $\kappa$ and manually tuning [triangle markers] or optimisation [circles] of all other free parameters. The true $\state$ [black solid line] and  $\state$ at the beginning of the run [red dashed line] are shown.  Data collection of $N_T$ measurements [not shown] ceases at $n=n^\ddagger=0$. For simplicity, the true $\state$ is a deterministic sine with frequency, $\omega_0$. (b) Perfection projection is possible $\omega_0 / \omega_0^{(B)} \in \mathcal{Z}$ (natural numbers), $\omega_0/2\pi = 3$ Hz. Kernel resolution is exactly the longest time domain correlation in dataset at $\kappa = 0$. (c) Imperfect projection, with $\omega_0 / \omega_0^{(B)} \notin \mathcal{Z}$, $\omega_0 / 2 \pi = 3 \frac{1}{3}$ Hz, $\kappa=0$. (d) We moderately raise $\kappa > 0 $, such that $\omega_0 / \omega_0^{(B)} \gg 0 \notin \mathcal{Z}$ for original $ \omega_0 / 2 \pi = 3$ Hz. (e) We test (c) and (d) for $\kappa > 0$, $ \omega_0 / \omega_0^{(B)} \notin \mathcal{Z}$, $\omega_0 / 2 \pi = 3 \frac{1}{3}$ Hz. For all (b)-(e), $N_T = 2000, N_P = 150$ steps, $\Delta t = 0.001$s and applied measurement noise level $1\%$.} 
\end{figure}

In order to illustrate the operating mechanism for the periodic kernel, we dramatically simplify the model used for $\state$  in  \cref{fig:main:fig_data_gpr} (a) and replace it with a single-frequency sine curve.  \cref{fig:main:fig_data_gpr} (b)-(e) demonstrates the prediction routine for GPR using a periodic kernel on a simplified version of $\state$, and as before, prediction is always conducted from time-step zero. For this simple example, the periodic kernel learns Fourier information in the measurement record enabling interpolation using test-points $n^{\ddagger} \in [-N_T, 0]$ for all cases (b)-(e) in \cref{fig:main:fig_data_gpr}, and atypical features are seen only for test-points in the prediction region [blue shaded region]. We consider predictions from a manually tuned model [triangles] and an optimised GPR model where remaining free $\{\sigma, R, l \}$ parameters are tuned using GPy [circles]. 

An examination of different cases for imperfect learning reveal that this discontinuity exhibits deterministic behavior linked to the underlying structure of the algorithm, namely, to the value of $\kappa$. In our numerical experiments, we find that in all cases of imperfect learning under GPR with a periodic kernel, a discontinuity in the prediction sequence arises at  $n^\ddagger = \kappa$. This is marked by the vertical dashed lines in all panels of \cref{fig:main:fig_data_gpr}(b)-(e).  However, another feature appears which we identify as being linked to oversampling of the underlying process determining $\state$.  In such cases, the algorithm simply predicts zero out to $n^\ddagger=\kappa$ before discontinuously predicting future evolution which does not appear similar to the true value of $\state$.  By contrast an optimised model gives smoothly varying predictions, which still adhere to the underlying behaviour set by $\kappa$ for $n^\ddagger>0$. 

In \cref{fig:main:fig_data_gpr}(b)-(e), we also plot the value of $\state$ as given from $n=-N_{T}$, the start of the data set, on top of the prediction from $n^\ddagger=\kappa$.  Here we see that the prediction provided by GPR matches the earliest stages of the underlying data set well.  Through various numeric experiments we find that the action of GPR in such parameter regimes (moderately positive $\kappa >0$) appears to be to simply repeat the learned values of $\state$ from $n=-N_{T}$ beginning at $n^\ddagger=\kappa$.  Accordingly these predictions rarely describe the underlying forward dynamics of $\state$ well. 

As we enter the high $\kappa$ regime,  $\kappa  \gg 0$, the features in \cref{fig:main:fig_data_gpr}(b)-(e) disappear, and GPR predictions begin to track the (slow moving) `truth' for $n^\ddagger \gg 0$. Analogously to inset (a), we see the performance of PER approach that of standard Gaussian kernels in this simplified case.



\section{Discussion} \label{sec:main:discussion}
The numeric simulations we have performed probe a wide variety of operating conditions in order to explore the algorithmic pathologies of leading forecasting techniques drawn from engineering, econometrics, and machine learning communities.  Our central finding is that overall the autoregressive Kalman filter provides an effective path to perform both state estimation and forward prediction for non-Markovian qubit dynamics. Recasting dynamics into an AKF filter, importantly, provides model robustness against details of the underlying dynamics as well as filtering of noise that allows it to outperform the simpler LSF in \cite{mavadia2017}.  Measurement noise filtering is enabled in the Kalman framework through the optimisation procedure for $R$ and has a regularising (smoothing) effect. Additionally optimisation of the imperfectly learned dynamical model is provided through the tuning of $\sigma$. The joint optimisation procedure over $(\sigma, R)$ ensures that the relative strength of noise parameters is also optimised.  

AKF has also been demonstrated to work well with discretised projective measurement models via what we refer to as the QKF.  In QKF, we employ single-shot, discretised qubit data while enabling model-robust qubit state tracking and increased measurement noise filtering via the underlying AKF algorithm.  However we find that the QKF is vulnerable to the buildup of errors for arbitrary applications and we provide three explanatory remarks from a theoretical perspective. First, the Kalman gains are recursively calculated using a set of \text{linear} equations of motion which incorporate the Jacobian $H_n$ of $h(x_n)$ at each $n$. All non-linear Kalman filters perform well if errors during filtering remain small such that the linearisation assumption holds at all time-steps. Second, measurements are quantised and hence residuals must be $\{-1, 0, 1 \}$ rather than continuously represented floating-point numbers.  In our case, the Kalman update to $x_n$ at $n$, mediated by the Kalman gain cannot benefit from a gradual reduction in residuals.  A third effect incorporates consequences of both quantised residuals and a non-linear measurement action. In linear Kalman filtering, Kalman gains can be pre-calculated in advance of the acquisition of any measurement data: the recursion of Kalman state-variances $P_{n}$, can be decoupled from the recursion of Kalman state-means, $x_{n}$ \cite{grewal2001theory}.  In our application, quantised residuals affect the Kalman update of $x_{n}$, and further, they affect the recursion for the Kalman gain via the state dependent Jacobian, $H_n$. 

In this context, we demonstrate numerically that the QKF achieves a desirable forward prediction horizon when the build of errors during filtering is minimised, for example, by specifying Kalman state dynamics and noise strengths perfectly, and/or by severely oversampling relative to the true dynamics of $\state$.   At present, we simply interpret our results on the QKF as demonstration that one may in principle track stochastic qubit dynamics using single shot measurements under a Kalman framework.  The QKF also has the benefit, as constructed, of reverting to the AKF if suitable pre-processing of data is performed prior to execution of the iterative state-estimation algorithm.  In common laboratory settings the measurement protocol may be effectively linearised through simple averaging of multiple single-shot measurements, application of Bayesian estimation protocols, or other pre-processing as identified above.  So long as the pre-processing takes place on timescales fast relative to the underlying qubit dynamics, the measurement linearization has no impact other than to change the effective sample rate of the measurements.  Thus it is our view that full implementation of the QKF is not essential if improved optimization routines are not accessible.

It is possible that QKF forward prediction horizons in realistic learning environments can be improved by solving the full $q+2$ optimisation problem for $\{\{ \phi_{q' \leq q}\}, \sigma, R\}$, rather than employing the approach taken in this manuscript. However, this poses its own challenges given the observations we make about the optimisation landscape even for the 2D optimisation problem faced in the AKF.  More sophisticated, data-driven model selection schemes are described for both KF and kernel learning machines (such as GPR) in literature (e.g. \cite{arlot2009data, vu2015understanding}). Beyond standard local-gradient and simplex optimisers, we consider coordinate ascent \cite{abbeel2005} and particle swarm optimisation techniques \cite{robertson2017particle} as promising, nascent candidates and their application remains an open research question. One may also consider switching from a high order AR($q$) to an ARMA model with a smaller number of optimisation parameters. Typically, this is accomplished by incorporating either greater a priori information about the underlying dynamic process in the design of the ARMA model and/or using model-less particle-based / unscented filtering techniques to overcome non-linearities in an ARMA representation (e.g. \cite{dong2009unscented}). The latter set of techniques are well adapted for non-linear models but are likely to require a modification to allow for non-Markovian dynamics (e.g. by designing an appropriate transition probability for otherwise Markov re-sampling procedures); in contrast, a typical recursive ARMA formulation for our application may track temporal correlations but be ill-equipped for non-linear, coin-flip measurements. One expects that a straightforward application of such procedures will be complicated.
 
Our general results on the use of autoregressive models for building Kalman dynamical models stand in contrast to Fourier-domain approaches in LKFFB and GPR using a periodic kernel;  both show significant performance degradation in cases when learning of state dynamics was imperfect.  In investigating the loss of performance for LKFFB, we find that the efficacy of this approach depends on a careful choice of a \textit{probe} (i.e. a fixed computational basis) for the dynamics of $\state$ capturing the effect of dephasing noise on the qubit.  In the imperfect learning regime of \cref{fig:main:fig_data_state_pred} and identically, \cref{fig:main:fig_data_specrecon}, LKFFB reconstructs Fourier domain information to a high fidelity across a range of sampling regimes but is outperformed by AKF in the time domain (\cref{fig:main:fig_data_state_pred}). Since LKFFB tracks instantaneous amplitude and phase information explicitly for each basis frequency, the loss of LKFFB time-domain predictive performance must accrue from difficulty in tracking instantaneous phase, rather than amplitude, information. 

While difficulty of instantaneous phase estimation is likely to disadvantage the time-domain predictive performance of LKFFB, our results show that a Fourier-domain approach yields high fidelity reconstructions of power spectral density describing $\state$. These reconstructions appear robust against imperfect projection on the LKFFB oscillator basis even as oversampling is reduced. This suggests that an application of LKFFB outside of predictive estimation could be tested against standard spectral estimation techniques in future work.

The challenge in adapting GPR for the task of time-domain predictive estimation has proved more striking.  In our numerical simulations, under conditions comparable to those tested in the AKF, the values of normalised Bayes prediction risk for all GPR models are at least an order of magnitude greater than the comparable performance of the AKF or LKFFB (refer to panel \cref{fig:main:fig_data_akfvlsf}(b- ii), equivalently, \cref{fig:main:figure_lkffb_path}(c)). This difference is somewhat surprising because in the limit that $\Gamma_n$ is set to the identity in LKFFB and an infinite basis of oscillators in the periodic kernel is truncated at the finite value, $J^{(B)}$, both LKFFB and the GPR-PER are formally equivalent to classical Kalman filtering for a collection of $J^{(B)}$ independent state-space resonators \cite{solin2014explicit}. In this limit, the true $\state$ is described by theoretically identical covariance functions in both KF and GPR frameworks. While we do not operate in this regime, one would expect predictive capabilities of these two algorithms to be comparable. 

In contrast to our observations for the various flavors of KF tested here, we observe that GPR predictions with a periodic kernel are useful for filtering/retrodiction but appear to have limited meaning for forward predictions for time-steps $n= n^\ddagger >0$.  In our application, predictive performance of GPR with a periodic kernel for $\kappa=0$ is shown to yield poor predictive performance over the ensemble average (\cref{fig:main:fig_data_gpr}(a)). For the unexpected regime of $\kappa \gg 0$ and relatively small fixed $l$, predictive performance improves and the periodic kernel performs similarly to RBF and RQ. In this a high $\kappa$ and a low $l$ regime, the $\sin$ term of the periodic kernel is slowly moving ($\sin(x) \approx x$) and hence the argument of the exponential in the periodic kernel approximates a Gaussian, reducing to an RBF kernel. Our numerical investigations show that an optimised RQ kernel consistently chooses parameter regimes where an RQ also converges to an RBF.  For the operating regimes pertinent to our application, it appears that the choice of the periodic, RBF, and RQ kernels will produce theoretically equivalent results for forward predictions of the qubit state. In our analysis, these `forward predictions' simply arise from a smoothed decay of state estimates starting from test-point $n^\ddagger=-1$ to the noise mean for test-points $n^\ddagger>0$; and are difficult to interpret compared to their Kalman counterparts.

Our numerical characterisation of the periodic kernel for a simple, noiseless $\state$ demonstrates that this kernel learns Fourier domain amplitude information in a way that is better suited for pattern fitting than forward prediction. The predictive time domain sequence of state estimates is repetitive at $ n=n^\ddagger= \kappa$, and can be interpreted as successful qubit-state predictions only when $\state$ is perfectly learned (no discontinuities appear). When learning is imperfect, however, GPR with a periodic kernel is able to learn Fourier amplitudes to provide good retrodictive state estimates for $n^\ddagger<0$, but forward predictions for $n^\ddagger>0$ typically fail.  Unlike LKFFB, we believe the periodic kernel does not permit actively extracting and updating phase information for each individual basis oscillators at $n^\ddagger= \kappa$.  Since phase information can be recast as amplitude information for any fixed-frequency oscillator, one would naively expect that forward predictions can be improved by increasing $\kappa$ moderately, such that the higher order terms in a series expansion of the $\sin$ term are non trivial and $\sin(x)\approx x$ cannot apply. However, any positive value of $\kappa$ means that we are probing dynamics at frequencies lower than appearing in the data-set. As such, a GPR-PER model predicts zero for $n^\ddagger \in [0, \kappa], \kappa > 0$, before reviving at $\kappa$.  The use of a procedure optimising kernel noise parameters $\{\sigma, R\}$ does not change the behavior as $ n^\ddagger \to\kappa$, but does smooth the discontinuities, as illustrated in \cref{fig:main:fig_data_gpr}(f). In letting $\kappa \gg 0$ (extremely large), we lose the uniqueness of the periodic kernel in summarising an infinite basis of oscillators, and standard Gaussian kernels (e.g. RBF, RQ) are likely to apply. 

It is possible that the choice of more complex kernels could enhance forward time series predictions via GPR, but they bring additional complications which thus far remain unresolved in relation to the current application. As one example, our ability to use numerical investigations to inform kernel design is further distorted by the need for a robust optimisation procedure, as illustrated by lower-than anticipated predictive performance observed for QPER.  Another class of GPR methods, namely, spectral mixture kernels and sparse spectrum approximation using GPR have been explored in \cite{wilson2013, quia2010}. However, these techniques also require efficient optimisation procedures to learn many unknown kernel parameters, whereas the sine-squared exponential in the periodic kernel is parameterised only by two hyper-parameters. Aside from spectral methods, the generalisation of MAT32 to higher $q + 1/2$ models probes only a subset of all possible AR($q$) processes, as the restrictions on autoregressive coefficients in Matern kernels are greater than the general case considered under an AKF in this manuscript. A detailed investigation of the application of such methods for forward prediction beyond pattern recognition and with limited computational resources, remains an area of future investigation.


\section{Conclusion \label{sec:main:Conclusion}}

In this manuscript, we provided a detailed survey of machine learning and filtering techniques applied to the problem of tracking the state of a qubit undergoing non-Markovian dephasing via a record of projective measurements.  We specifically considered the task of performing predictive estimation: learning dynamics of the system from the measurement record and then predicting evolution forward in time. To accommodate stochastic dynamics under arbitrary dephasing, and without an a priori dynamical model, we chose two Bayesian learning protocols - Gaussian Process Regression (GPR) and Kalman Filtering (KF).  All Kalman algorithms predicted the qubit state forward in time better than predicting mean qubit behaviour, indicating successful prediction, though an autoregressive approach to building the Kalman dynamical model demonstrated enhanced robustness relative to Fourier-domain approaches.  Forward prediction horizons could be arbitrarily increased for all Kalman algorithms by oversampling the underlying dephasing noise.  Our investigations included studies of both linear and non-linear measurement routines and validate the utility of the Kalman filtering framework for both.  In contrast, under GPR, we found numerical evidence that this approach enables retrodiction but not forward predictions beyond the measurement record.  

There are exciting opportunities for machine learning algorithms to increase our understanding of dynamically evolving quantum systems in real time using projective measurements. Quantum systems coupled to classical spatially or temporally varying fields may benefit from classical algorithms to analyse correlation information and enable predictive control of qubits for applications in quantum information, sensing, and the like. Moving beyond a single qubit, we anticipate that measurement records will grow in complexity allowing us to exploit the natural scalability offered by machine learning for mining large datasets. In realistic laboratory environments, the success of algorithmic approaches will be contingent on robust and computationally efficient algorithmic optimisation procedures as well as the extensions beyond Markovian dynamics studied here. The pursuit of these opportunities is the subject of ongoing research.

\section{Acknowledgments}
 The LSF filter is written by V. Frey and S. Mavadia \cite{mavadia2017}. The GPR framework is implemented and optimised using standard protocols in GPy \cite{gpy2014}. Authors thank C. Granade, K. Das, V. Frey, S. Mavadia, H. Ball, C. Ferrie and T. Scholten for useful comments. This work partially supported by the ARC Centre of Excellence for Engineered Quantum Systems CE110001013, the US Army Research Office under Contract W911NF-12-R-0012, and a private grant from H. \& A. Harley.
 



%When the periodic kernel begins sampling frequencies corresponding to timescales longer than the entire measurement record this results in an unphysical interpretation from the algorithms, and may indicate ``over-fitting'' by an algorithmic demand for more freedom in its estimation process than a user would nominally provide. 
%[XXX ADD CONTEXT AND CLARIFY TO MAKE THE FOLLOWING COMMENTARY FIT HERE OR ANOTHER SUITABLE LOCATION] %Even if one considers a measurement record such that spacing between observations is non-constant or randomly distributed in time, in our application, we still experimentally discretise continuous time dephasing processes, at minimum, on the timescale set by the Ramsey protocol, $\tau$. This means GPR can be implemented for the linear measurement record in Kalman filtering, or, by randomly picking $N_T$ pairs of time labels and measurements and both do not change the characterization of the periodic kernel by $\kappa$.
% \end{widetext}  
% \section{Body}

Objective
\begin{itemize}
\item track a slowly drifting stochastic phase of a classical control field interacting with a qubit by learning noise correlations encoded within a sequence of projective measurements 
\item  predict noise evolution beyond the measurment record
\item maximise prediction horizon through choice of machine learning algorithm to enable future control strategies, e.g. interleaving periods of measurement with periods of unsupervised control. 
\end{itemize}

Unique challenge 
\begin{itemize}
\item non linear, quantised measurement model
\item absence of any apriori information about true noise dynamics 
\item true stochastic process being tracked is non-Markovian
\end{itemize}

Approach
\begin{itemize}
\item develop non linear, quantised measurement model but implement analysis in a simpler regime where measurement model is linear and quantised measurment outcomes are pre-processed
\item reframe the absence of noise dynamics as a design problem where a deterministic transformation `colors' an initial input of white noise
\item additionally corrupt experimental data with measurement noise; and seek model robust prediction tools in realistic noise regimes with complex power spectral densities 
\item test Kalman filtering with a fully nonlinear, quantised measurement model for a true state with simple dynamics
\end{itemize}

Structure of this document....

\subsection{Measurement Model}

Ramsey Measurement: phase noise jitter in the intitial Ramsey control pulse is assumed to manifest as a constant detuning $ \beta_t \equiv  \dot{\phi}_N\dd |_t$ at time $t$ for short wait time $\tau$. The procedure is repeated to obtain the next measurement, at $t + \Delta t$ for $\Delta t >> \tau$, thereby discretising the continous time process, $\dot{\phi}_N\dd$. 
\\
\\
The probability of obtaining a measurement outcome, $d \in [0,1]$ is:

\begin{align}
P(d | \beta_t, \tau, t) = \begin{cases} \cos(\frac{\beta_t \tau}{2})^2 \quad \text{for $d=1$} \\   \sin(\frac{\beta_t \tau}{2})^2  \quad \text{for $ d=0$} \end{cases}
\end{align}

where the state vector is delivered to the equatorial plane of the Bloch sphere evolves freely for short Ramsey wait times, $\tau$, before it is rotated via a second control pulse and a projective measurement is taken with respect to the $\p{z}$ axis.

\subsubsection{Non Linear, Quantised Measurement Model}

Bayesian analysis optimal for non linear measurement models but cannot be adopted 
\begin{itemize}
\item $P(d | \beta_t, \tau, t)$ defines the Bayes likelihood for single shot
\item $P(\beta_t| d, \tau, t)  \propto P(d | \beta_t, \tau, t) P(\beta_t| \tau, t)$ holds for each time step $t$, but cannot assume Markovianity for propagating to $t+1$;  
\end{itemize}

Consider $P(\beta_t| d, \tau, t)$ an abstract discrete time signal in state space. Then non linear measurement is captured in $ h(\beta_t)$:
\begin{align}
\beta_t &= \Phi_{t-1} \beta_{t-1} + \Gamma_{t-1} w_{t-1} \\
y_t &= h(\beta_t) + v_t 
\end{align}

Just as sampling holds in discrete time, we quantise the amplitude as:

\begin{align}
z_t &= \mathcal{Q}(y_t) = \mathcal{Q}(h(\beta_t) + v_t)
\end{align}

In simulations, the quantiser $\mathcal{Q}$ is drawing from a binomial distribution where the bias of coin flip at $t$ is given by a true (engineered) $P(d | \beta_t, \tau, t)$. In experiment, $\mathcal{Q}$ is a naturally quantised physical sensor, namely, a qubit.

\subsubsection{Approximately Linear Measurement Model}
Assume that experimental data is pre-processed such that the measurement record is the a set of estimates of $\{ \hat{\beta}_t \}$, not single shot outcomes. Achieved in one of two ways:

\begin{itemize}
\item Standard technqiue:  average single shots over many parallel runs for different wait times to obtain $\hat{P}(d | \beta_t, \tau, t)$ vs. $\tau$. One may deduce $\beta_t$ from a Fourier transform of $P(d | \beta_t, \tau, t) $ vs. $ \tau$; or condense all pre-processing into Bayesian treatment of single shots as in CITE, with the assumption that $\beta_t$ is constant over $M\tau$ measurements performed, $M\tau << \Delta t$. 
\item Alternative: $M$ single shots are performed with the same wait time, $\tau$, such that drifts in probability are due to $\beta_t$. This binary signal can be decimation filtered to yield $P(d | \beta_t, \tau, t) $ vs. $t/\tau$, where $ \tau \equiv \Delta t$, and the inversion $\beta_t = \frac{1}{\tau} (1- 2\hat{P}(d | \beta_t, \tau, t))$ holds as long as $\beta_t \tau < \pi $. 
\end{itemize}

FIG: Single Shot Outcomes to Decimation Filtering under Linear Measurement Models

\subsection{Stochastic Dynamics with Linear Measurement}
In the absence of a theoretical dynamical model for the evolution of $\beta_t$:
\begin{itemize} 
\item impose properties on the phase noise field and its derivative
\item under these properties, reconstruction of phase noise is enabled in the mean square limit by spectral decomposition using harmonic sums; and by Wold's decomposition, via autoregressive processes of finite order.
\item purpose of numerical analysis is to test whether our mean square approximate reconstructions enable experimentally sensible state tracking and prediction in the time domain
\end{itemize}

The approximate representations of covariance stationary processes informs the structure of learning algorithms. Hence, we cover each reconstruction and algorithmic performance in the sections below:

\subsubsection{Least Squares and Autoregressive Kalman Filter}

\begin{itemize}
\item Autoregressive (AR) processes of finite order: define time domain structure and power spectral density
\item LS Filter: for one step ahead, this is an AR process and AR coefficients solved via with gradient descent
\item AKF: recast AR coefficients as Kalman dynamical model to increase measurement noise filtering capabilities 
\begin{itemize}
\item List AKF state space dynamical and measurement model 
\end{itemize}
\end{itemize}

FIG: Predictive performance with increased measurement noise: LSF vs AKF

\subsubsection{Gaussian Process Regression}
\begin{itemize}
\item Periodic kernel represents trignometric polynomial with infinite terms to reconstruct phase noise process
\item If trigometric polynomial is truncated at finite $J$, then GPR with periodic kernel represents $J$-th order state space model in a classical Kalman filter
\begin{itemize}
\item List GPR predictive equations
\item List GPR Periodic Kernel
\end{itemize}
\item Covariance matrix in predictive equations (above) is $N$ periodic for a  measurement record with $N$ terms: pattern reconstruction is enabled by learning Fourier domain amplitudes via a Periodic Kernel, but time domain predictions do not make sense without active tracking of phase information.
\end{itemize}
FIG: GPR-P Predictions outside the zone of measurement data repeat the intial measurement record.
\begin{itemize}
\item Other kernels: excluded on the basis that their shape in the Fourier domain is narrowband; or difficulty of optimising hyper-parameters for our application 
\end{itemize}
\subsubsection{Liska Kalman Filter with Fixed Basis}
\begin{itemize}
\item Tracks amplitudes and phases of a fixed collection of resonators used to probe the noise 
\item Enables full knowledge of phase noise if perfect projection is possible; but predictive power deteriorates rapidly for realistic noise scenarios relative to the Autoregressive Kalman Filter.
\end{itemize} 
FIG: Predictive performance of AKF vs LKFFB for realistic noise scenarios

\subsection{Optimisation}
\begin{itemize}
\item Bayes Risk calculated during state estimation and prediction relative to true engineered noise
\item Optimisation problem breaks down for LKFFB as realistic noise scenarios are considered; but remains robust for AKF
\item Cost function contains pathologies for standard gradient and simplex algorithms - a simple random sampling is used here; but diagnostics suggest coordinate ascent and particle swarm techniques could be promising candidates for future research
\end{itemize}
\subsection{Stochastic Dynamics with Non Linear, Quantised Measurements}

We update the Kalman filtering framwork for quantised sensor information, where a one bit classical qunatiser is considered. Inputs into the Kalman filter are 0 or 1  outcomes from a binomial distribution with a stochastically drifitng bias. The interpretation of measurement noise (previously) is now additive white quantisation noise. We track a simple sum of sinusoids with random phases where perfect projection is possible for LKFFB, and compare RMS performance from single shots with equivalent performance from tracking pre-processed data.

\subsubsection{Autoregressive Kalman Filter}
NOT DONE 
\subsubsection{Liska Kalman Filter with Fixed Basis}
NOT DONE
\\
\\
FIG: NOT DONE. However, early demos suggest that errors from non linear measurement model make it sub-optimal to use Kalman filtering directly on single shot outcomes; and that  LKFFB fails more than AKF. 
\\
\\
Particle filtering may work better; but an implementation of particle fitlering for non Markovian stochastic dynamics (and non linear, quantised  measurments) are out of scope of this paper. 

\section{Appendices}
\subsection{Physical Set Up and Measurement Model}
\subsection{Stationary Stochastic Processes }
\subsection{Linear Predictors for Covariance Stationary Processes}
\subsection{Kernel Selection in Gaussian Process Regression}
\subsection{Optimisation Procedure for Tuning Filter Parameters}

\section{References}
\appendix 
% \begin{widetext} 
  
\mbox{}

% \section{Time Series Random Processes, Noise \& Probability Distributions}
\nomenclature{$\delta \omega(t)$}{Environmental dephasing} 
\nomenclature{$\state$}{True state i.e. a true sequence of stochastic phases governing qubit dynamics under environmental dephasing noise}
\nomenclature{$\state'$}{An approximation to the true covariance stationary $\state$ using autoregressive methods (rather than a periodic random signal) to generate realisations of stochastic phases under dephasing}
\nomenclature{$Pr(x), Pr_x(\mu_x, \Sigma_x)$}{Probability distribution of $x$, with mean $\mu_x$ and variance $\Sigma_x$} 
\nomenclature{$\hat{Pr}(d_n | \state_n, \tau, n\Delta t)$}{Likelihood of a single shot qubit outcome given by Born's rule }
\nomenclature{$w$}{Zero mean additive Gaussian white process noise in a Kalman framework}
\nomenclature{$\sigma^2$ $(\sigma^*)$}{True (optimised) process noise covariance strength in Kalman filtering; process variance strength in GPR}
\nomenclature{$v$}{Zero mean additive Gaussian white measurement noise}
\nomenclature{$R$ $(R^*)$ }{True (optimised) measurement noise covariance strength in GPR and Kalman Filtering}
\nomenclature{$N.L.$}{Applied measurement noise level for simulated datasets, defined as the ratio of true measurement noise strength $R$ and variance of the true $\state$ under dephasing}
\nomenclature{$\mathcal{B}$}{Binomial distribution}
\nomenclature{$p_{\mathcal{B}}$}{Binomial distribution parameter for the bias of a weighted coin toss}
\nomenclature{$n_{\mathcal{B}}$}{Binomial distribution parameter for number of tosses in a coin toss experiment}
\nomenclature{$\mathcal{U}$}{Uniform distribution}
\nomenclature{$b$}{Saturation for probability distributions associated with quantised measurements }
\nomenclature{$\mu$}{Generic mean of a Gaussian distribution}
\nomenclature{$\Sigma$}{Generic covariance matrix of a Gaussian distribution}
\nomenclature{$(+)$}{A posteriori statistical quantity (output of a Bayesian estimator)}
\nomenclature{$(-)$}{A priori statistical quantity (input of a Bayesian estimator)}

% \section{Classical and Quantum Operators}
\nomenclature{$\hat{\cdot}$}{Quantum mechanical operator; or denotes estimators based on finite sample data for classical random variables (evident from context)}
\nomenclature{$\mathcal{L}$}{Lag operator for an autoregressive process}
\nomenclature{$\mathcal{Q}$}{Quantiser defined for a Kalman state space framework such that the output is either $0$ or $1$. This corresponds to a statistical description of outcomes given by a likelihood $Pr(d_n | \state_{n}, \tau)$}
\nomenclature{$\op{U}$}{Unitary time evolution quantum operator}
\nomenclature{$\p{x}, \p{y}, \p{z}$ $(\op{x}, \op{y}, \op{x})$}{Pauli operator (Pauli basis)}


% \section{Experimental Sampling and Data Collection}
\nomenclature{$t$}{Wall time}
\nomenclature{$\tau$}{Ramsey wait time}
\nomenclature{$\Delta t$}{Time step between measurements}
\nomenclature{$N_T$}{Number of training points}
\nomenclature{$N_P$}{Number of prediction points beyond the measurement record}
\nomenclature{$N$}{Total number of time steps ($N_T + N_P$)}
\nomenclature{$r_{Nqy}$}{Nyquist multiplier}
\nomenclature{$f_{(B)}$}{Bandwidth assumption such that $f_{(S)} \equiv r_{Nqy} f_{(B)}$}
\nomenclature{$f_{(S)}$}{Experimentally controlled sampling rate $f_{(S)} \equiv 1/\Delta t$}
\nomenclature{$\omega_{(S)}$}{Experimentally controlled sampling rate $2 \pif_{(S)}$}
\nomenclature{$S(\omega)$ ($\hat{S}(\omega)$)}{True (estimated) power spectral density for true $\state$}


% \section{Algorithm Design}
\nomenclature{$f_0^{(B)}$}{Computational basis  frequency comb spacing (Hz) in LKFFB or GPR (Periodic Kernel)}
\nomenclature{$\omega_0^{(B)}$}{Computational basis  frequency comb spacing (rad) in LKFFB or GPR (Periodic Kernel)}
\nomenclature{$J^{(B)}$}{Computational basis - total number of basis oscillators in LKFFB}
\nomenclature{$h(\cdot)$}{A non linear state space measurement model }
\nomenclature{$H_n$}{Linear state space measurement model or Jacobian of a non-linear measurement model}
\nomenclature{$x$}{State space unobserved true state in Kalman Framework}
\nomenclature{$P$}{True uncertainty of the true Kalman state in the Kalman Framework}
\nomenclature{$\gamma_{n}$}{Kalman gain}
\nomenclature{$\Gamma_{n}$}{Kalman process noise features}
\nomenclature{$\Phi$}{State space dynamical model for the true state in the Kalman Framework}
\nomenclature{$\state^{(\star)}$}{State estimation and/or predictions from GPR about a true state $\state$, for test points collected in a length $N^{(\star)}$ vector.}
\nomenclature{$K({}\cdot{}, {}\cdot \cdot{})$}{A Gram Schmidt ($\cdot$) by ($\cdot \cdot$) matrix whose elements are specified by $R(\nu)$  in GPR}
\nomenclature{$\nu$}{The separation distance between any two time steps, $n_1, n_2$. }
\nomenclature{$l$}{Length scale for the periodic kernel}
\nomenclature{$\kappa$}{A deterministic time-step at which discontinuities in GPR can be predicted.}
\nomenclature{$n^*_C$}{[Appendix only] LKFFB: optimal  number of points for training before commencing predictions}

\nomenclature{$A^j_{n} $}{LKFFB - real part of true Kalman sub-state $x^j_n$ }
\nomenclature{$B^j_{n}$}{LKFFB - imag part of true Kalman sub-state $x^j_n$ }
\nomenclature{$ \norm{x^j_n}$}{LKFFB - norm of true Kalman sub-state $x^j_n$}
\nomenclature{$\theta^{j}_{n} $}{LKFFB - phase of true Kalman sub-state $x^j_n$}
 
% \section{Bayes Risk and Performance Analysis}
\nomenclature{$n^*$}{Time step denoting the maximal forward prediction horizon $n^* \in [0, N_P]$ for which an algorithm predicts better relative to predicting the mean behaviour of the qubit under dephasing}
\nomenclature{$I$}{A set of model design parameters (known a priori or optimised during algorithmic tuning)}
\nomenclature{$L_{BR}(n | I)$}{Bayes Risk value at $n$ defined as an expectation value of datasets $\mathcal{D}$ and conditioned on a set of model parameters $I$}
\nomenclature{$\normpr $}{Normalised Bayes Risk at $n$: Risk calculated over datasets $\mathcal{D}$ and normalised against qubit behaviour under mean dephasing}
\nomenclature{$L(I_k)$}{Loss value defined as the sum of Bayes Risk $L_{BR}(n | I_k)$ over a fixed number of time steps ($N'$) during state estimation ($N' \equiv N_{SE}$) or prediction ($N' \equiv N_{PR}$) for $k$-th choice of a set of model parameters, $I_k$.}
\nomenclature{$N'$}{Loss function parameter for algorithmic tuning, namely, $N' \in \{ N_{SE}, N_{PR}\}$). This defines the number of time steps which contributes to the total prediction risk value, for a given choice of model parameters, $I$.}
\nomenclature{$N_{PR}$}{Loss function parameter for algorithmic tuning, namely, the number of time steps after $n=0$ which contributes to the total prediction risk value, for a given choice of model parameters, $I$.}
\nomenclature{$N_{SE}$}{Loss function parameter for algorithmic tuning, namely, the number of time steps before $n=0$ which contributes to the total state estimation risk value, for a given choice of model parameters, $I$.}
\nomenclature{$L_0$}{Loss threshold below which a `low loss region' $L(I_k)$ is defined in the space spanned by model parameters in $I$, using an ensemble of experiments with data $\mathcal{D}$}


% \section{Measurement Records} 
\nomenclature{$\{ y_n\}$}{Observations of a true state corrupted by measurement noise (not quantised) }
\nomenclature{$\{ d_n\}$}{Quantised zero or one measurement records}
\nomenclature{$z_n$}{A noiseless (ideal) non linear measurement of the true Kalman state, $h(x)$}
\nomenclature{$\mathcal{D}$}{A collection of noisy datasets corressponding to $M$ total runs of an experiment under different realisations of $\state$}


% \section{Indicies}
\nomenclature{$X$}{[Nomenclature only] Dummy index variable to define sets of \emph{index variables}}
\nomenclature{$\{n: -N_T, \hdots, N_P \}$}{Index representing the number of discrete time steps for wall time $t_n = n\Delta t$}
\nomenclature{$\{n^{(\star)}_X: X = 1, 2, \hdots, N^{(\star)} \}$}{Index of test points in GPR, namely, a label for points in time where a GPR state estimate or forward prediction is desired}
\nomenclature{$N^{(\star)}$}{The length of a vector containing test points in GPR}
\nomenclature{$\{m: 1, 2, \hdots M\}$}{Index representing the number of trials of the same experiment }
\nomenclature{$\{k: 1, 2, \hdots K\}$}{Index representing the number of trials of a choice of model (hyper-parameters) for a given ensemble of experiments }
\nomenclature{$\{j: 1, 2, \hdots J \text{ or } J^{(B)}\}$}{Index representing the number of Fourier components in true dephasing noise ($J$) or in a computational Fourier protocol ($J^{(B)}$)}
\nomenclature{$\{i: 1, 2, \hdots N_P\}$}{Index representing the $i$-th step ahead prediction from $n=0$, where $i=1$ is the step ahead prediction in the Kalman filter}
\nomenclature{$\{q': 1, 2, \hdots q\}$}{Index representing the $q'$-th autoregressive term in an autoregressive model of order $q$}
\nomenclature{$\iota_X$}{A set of quantities generated for the purpose of randomly sampling parameter space associated with Kalman design parameters $(\sigma, R)$, for $\{ X: 0, 1, min, max\}$}
\nomenclature{$\phi_{q'}$}{Autoregressive coefficient for the $q'$ term in an AR($q$) process, $q' \leq q$}
\nomenclature{$\mathcal{Z}$}{The set of natural numbers}


% Physical Setting
\nomenclature{$\eta$}{[Appendix only] Complex scalar representing system-field coupling strength for an atom-field Hamiltonian in the rotating wave approximation}
\nomenclature{$f_0$}{True noise frequency comb spacing (Hz)}
\nomenclature{$\omega_0$}{True noise frequency comb spacing (rad)}
\nomenclature{$J$}{True noise - total number of Fourier components}
\nomenclature{$p$}{[Appendix only] true $\state$ -  power spectral density shape, such that $F(j) = j^{\frac{p}{2}-1} $ }
\nomenclature{$F(j)$}{[Appendix only] true $\state$ -  Fourier amplitude for the $j$-th oscillator }
\nomenclature{$\psi_j$}{[Appendix only] true $\state$ - uniformly distributed random phases }
\nomenclature{$\alpha$}{[Appendix only] true $\state$ - arbitrary real, constant, scaling factor}



% \nomenclature{sym}{def}
% \nomenclature{sym}{def}
% \nomenclature{sym}{def}

\printnomenclature 
 



 
  \clearpage     
  % % ##############################################################################
\section{Physical Setting \label{sec:app:setup}}
% ##############################################################################


We consider a qubit under environmental dephasing.  For any two level system, a quantum mechanical description of physical quantities of interest can be provided in terms of the Pauli spin operators $\{ \p{x}, \p{y}, \p{z}\}$. If $\hbar \omega_A$ corresponds to an energy difference separating these two qubit states, then the Hamiltonian for a single qubit in free evolution can be written in the Pauli representation. We consider a qubit states in the $\p{z}$ basis, $\ket{0}$ or $\ket{1}$ with energies $E_0, E_1$ in our notation, corresponding to a 0 or 1 outcome upon measurement. This yields a Hamiltonian for a single qubit as:

\begin{align}
%\op{\mathcal{H}}_0 &= E_0\ket{0}\bra{0} + E_1\ket{1}\bra{1} \\
\p{z} &\equiv \ket{1}\bra{1} - \ket{0}\bra{0} \\
\op{\mathcal{I}} & \equiv \ket{0}\bra{0} + \ket{1}\bra{1} \\
\op{\mathcal{H}}_0 & = \frac{1}{2} (E_0\ket{0}\bra{0}+ E_1\ket{1}\bra{1}) \\
& + \frac{1}{2} [(E_1 - E_0)\p{z} + E_0 \ket{1}\bra{1} + E_1 \ket{1}\bra{1}]\\
& = \op{\mathcal{I}} \left( \frac{E_0 + E_1}{2} \right) + \p{z}(\frac{E_1 - E_0}{2}) \\
\quad E_{0,1} &\equiv \mp \frac{1}{2} \hbar \omega_A \\
\op{\mathcal{H}}_0 &= \frac{1}{2} \hbar \omega_A \p{z}
\end{align}

In this representation, the effect of dephasing noise on a free qubit system is that any initially prepared qubit superposition of $\ket{0}$ and $\ket{1}$ states will decohere over time in the presence of dephasing noise. This physical effect is modelled as a stochastically fluctuating process $\delta\omega(t)$ that couples with the $\p{z}$ operator. The noise Hamiltonian is described as:
\begin{align} 
\op{\mathcal{H}}_{N}(t) & \equiv \frac{\hbar}{2}\delta\omega(t)\p{z}
\end{align}
In the formula above, $\delta\omega(t)$ is a classical, stochastically fluctuating parameter that models environmental dephasing and $\hbar/2$ appears as a convenient scaling factor. The total Hamiltonian for a single qubit under dephasing is:
\begin{align} 
\op{\mathcal{H}}(t) &\equiv \op{\mathcal{H}}_0 + \op{\mathcal{H}}_{N}(t)
\end{align}

Since $\op{\mathcal{H}}_{N}(t)$ commutes with $\op{\mathcal{H}}_0$, we can transform away $\op{\mathcal{H}}_0$ by moving to a rotating frame with respect to $H_0$. Let $\ket{\psi (t)}$ be a state in the lab frame, let $\op{U}$ define a transformation to a rotating frame, and let $\ket{\tilde{\psi} (t)}$ be the state in the rotating frame. The notation, $\tilde{}$, indicates operators and states in the transformed frame. In this simple case, the transformed Hamiltonian governing the evolution of $\ket{\tilde{\psi} (t)}$ will just be $\op{\mathcal{H}}_{N}(t)$:

\begin{align}
\op{U} &\equiv e^{-i\op{\mathcal{H}}_0 t / \hbar } \\
\ket{\tilde{\psi} (t)} &\equiv \op{U}^\dagger \ket{\psi (t)} \\
i \hbar \frac{d}{dt} \ket{\tilde{\psi} (t)} & \equiv i \hbar \frac{d}{dt} \op{U}^\dagger \ket{\psi (t)} \\
&= -\op{\mathcal{H}}_0 \op{U}^\dagger \ket{\psi (t)} + i\hbar \op{U}^\dagger \frac{d}{dt} \ket{\psi (t)} \\
&= -\op{\mathcal{H}}_0 \op{U}^\dagger \ket{\psi (t)} + \op{U}^\dagger \mathcal{H}  \ket{\psi (t)} \\
&= -\op{\mathcal{H}}_0 \op{U}^\dagger \ket{\psi (t)} + \op{U}^\dagger \mathcal{H} \op{U} \op{U}^\dagger \ket{\psi (t)}, \quad \op{U}\op{U}^\dagger \equiv 1 \\
&= -\op{\mathcal{H}}_0 \op{U}^\dagger \ket{\psi (t)} + \op{U}^\dagger \mathcal{H} \op{U} \op{U}^\dagger \ket{\psi (t)}, \quad \op{U}\op{U}^\dagger \equiv 1 \\
 &=  ( \op{U}^\dagger \mathcal{H} \op{U} -\op{\mathcal{H}}_0  ) \op{U}^\dagger \ket{\psi (t)} \\
& =  ( \op{U}^\dagger \mathcal{H} \op{U} -\op{\mathcal{H}}_0  ) \ket{\tilde{\psi} (t)}
\end{align}

\begin{align}
\implies \op{\tilde{\mathcal{H}}} &\equiv \op{U}^\dagger \mathcal{H} \op{U} -\op{\mathcal{H}}_0 \\
& = \op{U}^\dagger \op{\mathcal{H}}_0\op{U}  + \op{U}^\dagger \op{\mathcal{H}}_{N}(t) \op{U} -\op{\mathcal{H}}_0 \\
& = \op{U}^\dagger \op{U} \op{\mathcal{H}}_0  + \op{U}^\dagger \op{U} \op{\mathcal{H}}_{N}(t)  -\op{\mathcal{H}}_0, \quad [\op{U}, \op{\mathcal{H}}_0 ] = [\op{U}, \op{\mathcal{H}}_{N}(t) ] = 0 \\
& = \op{\mathcal{H}}_{N}(t)
\end{align}
In the semiclassical approximation,  $\op{\mathcal{H}}_{N}(t)$ commutes with itself at different $t$, and hence we can write a unitary time evolution operator in the rotating frame as:
\begin{align}
\op{\tilde{U}}(t, t + \tau) &\equiv  e^{-\frac{i}{\hbar}  \int_{t}^{t + \tau} \op{\mathcal{H}}_{N}(t') dt'  } \\
& = e^{-\frac{i}{2} \state(t, t + \tau) \p{z} } \\
\state(t, t + \tau) & \equiv  \int_{t}^{t + \tau} \delta \omega (t') dt' \label{eqn:app:phases}
\end{align}
In the rotating frame, we prepare an initial state that is a superposition of $\ket{0}$ and $\ket{1}$ states. This state evolves under $\op{\mathcal{H}}_{N} (t)$ during a Ramsey experiment for duration $\tau$.  Subsequently, the qubit state is rotated before a projective measurement is performed with respect to the $\p{z}$ axis i.e. the measurement action resets the qubit. 

Without loss of generality, define the initial state as  $\ket{\tilde{\psi} (0)} \equiv \frac{1}{\sqrt{2}} \ket{0} + \frac{1}{\sqrt{2}} \ket{1}$ in the rotating frame. Then, the probability of measuring the same state after time $\tau$ in a single shot measurement, $d \in \{0, 1\}$, is:
\begin{align} 
Pr(d | f(0, \tau), \tau) & = |\bra{\tilde{\psi} (0)} \op{\tilde{U}}(0, \tau) \ket{\tilde{\psi} (0)}|^2 \\
& = |\frac{1}{2} \bra{\tilde{\psi} (0)} e^{-\frac{i}{2} \state(0, \tau) \p{z} } \left(  \ket{0} + \ket{1} \right)|^2 \\
& = |\frac{1}{2}  \bra{\tilde{\psi} (0)} \left( e^{\frac{i}{2} \state(0, \tau) }\ket{0} + e^{-\frac{i}{2} \state(0, \tau)}\ket{1} \right)|^2\\
& = |\frac{1}{2}  ( e^{\frac{i}{2} \state(0, \tau)} + e^{-\frac{i}{2} \state(0, \tau)} )|^2\\
& = \cos(\frac{\state(0, \tau)}{2})^2 \label{eqn:app:likelihood}
\end{align}
The second $\pi/2$ control pulse rotates the state vector such that a measurement in $\p{z}$ basis is possible, and the probabilities correspond to observing the qubit in the   $\ket{1}$ state. Hence, \cref{eqn:app:likelihood} defines the likelihood for single shot qubit measurement. Further, \cref{eqn:app:likelihood} defines the non linear measurement action on phase noise jitter, $\state(0, \tau)$.  We impose a condition that $\state(0, \tau)/2 \leq \pi$  such that accumulated phase over $\tau$ can be inferred from a projective measurement on the $\p{z}$ axis. 



\newpage
% ##############################################################################
\section{Experimentally Controlled Discretisation of Dephasing Noise \label{sec:app:exptres}} 
% ##############################################################################
 In this section, we consider a sequence of Ramsey measurements. At time $t$, \cref{sec:app:setup} describes the qubit measurement likelihood at one instant under dephasing noise. We assume that the dephasing noise is slowly drifting with respect to a fast measurement action on timescales of order $\tau$. In this regime, \cref{eqn:app:phases} discretises the continuous time process $\delta\omega(t)$, at time $t$, for a number of $n= 0, 1, ..., N$ equally spaced measurements with $t = n \Delta t$. Performing the integral for $\tau \ll \Delta t$ and slowly drifting noise such that we substitute the following terms in \cref{eqn:app:phases}:
\begin{align}
\delta\bar{\omega}_n &\equiv \delta\omega(t')|_{t'=n \Delta t } \\
\state_n &\equiv \state(n\Delta t, n\Delta t + \tau) \\
& \equiv \frac{\hbar}{2}  \int_{n\Delta t}^{n\Delta t + \tau} \delta\bar{\omega}_n dt' \\
& \equiv \frac{\hbar}{2}  \delta\bar{\omega}_n \int_{n\Delta t}^{n\Delta t + \tau}  dt' \\
& = \frac{\hbar}{2}\p{z}\delta\bar{\omega}_n \tau \label{eqn:app:phases_constantdetuning}
\end{align}
In this notation, $\delta\bar{\omega}_n $ is a random variable realised at time, $t = n \Delta t$, and it remains constant over short duration of the measurement action, $\tau$.  We use the shorthand $\state_n \equiv \state(n\Delta t, n\Delta t + \tau)$ to label a sequence of stochastic, temporally correlated qubit phases $ \state \equiv \{\state_n \}$. 

Since the qubit is reset by each projective measurement at $n$, the unitary operator governing qubit evolution is also reset such that $\op{\tilde{U}}_n \equiv \op{\tilde{U}}(n\Delta t, n\Delta t + \tau)$ are a collection of $N$ unitary operators describing qubit evolution for each new Ramsey experiment. They are not to be interpreted, for example, as describing qubit free evolution without re-initialising the system. Hence, for each stochastic qubit phase $\state_n$, the true probability for observing the $\ket{1}$ in a single shot is given by substituting $\state_n $ for $ \state(0,1)$ in \cref{eqn:app:likelihood}.
\begin{align}
Pr(d_n | \state_n, \tau, n \Delta t) &= \begin{cases} \cos(\frac{\state_{n}}{2})^2 \quad \text{for $d=1$} \\   \sin(\frac{\state_{n}}{2})^2  \quad \text{for $ d=0$} \end{cases} 
\end{align}
The last line follows from the fact that total probability of the qubit occupying either state must add to unity. This yields \cref{eqn:main:likelihood} in the main text.

\iffalse
 \clearpage
 The discretisation of $\delta\omega(t)$ into  a discrete time stochastic phase sequence, $\state$, governs the physical Fourier resolution at which dephasing noise is sampled.
 If we take a sequence, $\state$, to be $N$ samples long, then we define the resulting physical Fourier domain resolution :
 \begin{align}
 \Delta f_{EXPT} & \equiv \frac{1}{\Delta t N} \\
 N & \equiv N_T + N_P
 \end{align} By specifying  $\Delta t$, training points, $ N_T$, and forward prediction time steps, $ N_P$, an experiment fully defines the physical Fourier resolution. In particular, an implicit assumption about the bandwidth of dephasing noise, $f_B$ is captured in $\Delta t$, namely:
\begin{align}
\Delta t \equiv \frac{1}{f_s} \equiv \frac{1}{r_{Nqy}f_B}
 \end{align}
In numerical simulations, we choose  Nyquist multipler $r_{Nqy} \gg 2$ and fix $f_B, \Delta t, N_T, N_P$. For true noise engineering, we compare true noise spacing $\omega_0/ 2\pi$ relative to the physical resolution of the entire experimental run, i.e. $\Delta f_{EXPT}$. Further, if the oversampling regime for any algorithm is varied, it can interpreted as relaxing $f_B$ assumption or equivalently, reducing $r_{Nqy}$, however, we ensure there is no physical aliasing and $r_{Nqy}>2$ in all cases.
\\
\\
The computational resolution in algorithms has a natural interpretation such that  $\omega_0^B / 2\pi \geq \frac{1}{\Delta t N_T} > \Delta f_{EXPT}$. This is necessarily the case since an algorithm ceases to receive measurement data but we continue to `sample' true noise to confirm accuracy of predictions in our study. Numerically, this is significant in optimising spacing between adjacent oscillators for  LKFFB and GPR algorithms and linking their performance to a physical interpretation of sampling rates.  
\fi




\newpage
% ##############################################################################
\section{True Dephasing Noise Engineering \label{sec:app:truenoise}} 
% ##############################################################################
In the absence of an apriori model for describing qubit dynamics under dephasing noise, we impose the following properties on a sequence of stochastic phases, $\state  \equiv \{ \state_n \}$ such that we can design meaningful predictors of qubit state dynamics. We assert that a stochastic process, $\state_n$, indexed by a set of values, $ n = 0, 1, \hdots N $ satisfies: 

\begin{align}
\ex{\state_n} &= \mu \quad \forall n \label{eqn:app:f_mean} \\
\ex{\state_n^2} & < \infty \quad \forall n \label{eqn:app:f_var} \\
\ex{(\state_n - \mu)(\state_m - \mu)} &= R(\nu), \quad  \nu = |n-m|, \quad \forall n, m \in N  \label{eqn:app:f_covar} \\
R(\nu) & \neq \sigma^2  \delta(\nu) \label{eqn:app:f_Markovian} 
\end{align}
Covariance stationarity of $\state$ is established by satisfying \cref{eqn:app:f_mean,eqn:app:f_var,eqn:app:f_covar}, namely that the mean is independent of $n$, the second moments are finite, and the covariance of any two stochastic phases at arbitrary time-steps, $n, m$, do not depend on time steps but only on the separation distance, $\nu$. The $\delta(\nu)$ in the last condition,  $ \cref{eqn:app:f_Markovian}$, is the Dirac-delta function and establishes that $\state$ is not delta-correlated (white). This condition captures the slowly drifting assumption for environmental dephasing noise. 


We also require that correlations in $\state$ eventually die off as $\nu \to \infty$ otherwise any sample statistics inferred from noise-corrupted measurements are not theoretically guaranteed to converge to the true moments. The mean square ergodicity is defined below:
\begin{align}
 \lim_{K \to \infty} \frac{1}{K} \sum_{v=0}^{K-1} R(\nu) & = 0  \iff  \lim_{K \to \infty} \ex{(\bar{\state_K} - \mu)^2} = 0 \nonumber \\
\text{for} \quad v &= |n_k - n_j|, \quad \forall k,j \in K, n_k, n_j \in N  \nonumber \\
\text{with} \quad \bar{f_K} &= \frac{1}{K} \sum_{k=0}^K f_{n_k} \quad \text{(sample mean of $\state$)}  \label{eqn:app:f_msergodic}  
\end{align}
The statement above means that a true $R(\nu)$ associated with $\state$ is bandlimited for sufficiently large (but unknown) $K$. If correlations never `die out', then any designed predictors for one realisation of dephasing noise will fail for a different realisation of the same true dephasing. For the purposes of experimental noise engineering, we satisfy the assumptions above by engineering discretised process, $\state$, as:
\begin{align}
\state_n &= \alpha \omega_0 \sum_{j=1}^{J} j F(j)\cos(\omega_j n \Delta t + \psi_j) \label{eqn:app:noiseengineering} \\
F(j) & = j^{\frac{p}{2}-1}  
\end{align}

Using the notation of \cite{soare2014}, $\alpha$ is an arbitrary scaling factor, $\omega_0$ is the fundamental spacing between true adjacent discrete frequencies, such that $\omega_j = 2 \pi f_0 j =\omega_0 j, j = 1, 2, ...J$. For each frequency component, there exists a uniformly distributed random phase, $\psi_j \in [0, \pi]$. The free parameter $p$ allows one to specify an arbitrary shape of the true power spectral density of $\state$. In particular, the free parameters $\alpha, J, \omega_0, p$ are true dephasing noise parameters which any prediction algorithm cannot know beforehand.

It is straightforward to show that $\state$ is covariance stationary. To show mean square ergodicity of $\state$, one requires phases are randomly uniformly distributed over one cycle for each harmonic component of $\state$ \cite{gelb1974applied}. Subsequently, one shows that an ensemble average and a long time average of multi-component engineered $\state$ are equal. 

\subsection{Proof  $\state$ is mean square ergodic and Gaussian}

Consider one term in the harmonic sum $f^j_n(\psi_j)$, $j$ denoting the $j$-th true spectral component (not a power) of $f_n$, such that $A_j \equiv \alpha \omega_0 j F(j), \omega \equiv \omega_j$. Note that $\{ \psi_j \}$ are randomly uniformly distributed over one cycle, and $f^j_n(\psi_j)$ is a function of random variable $\psi_j$ defining a random process over the time index, $n$. 

\begin{align}
f^j_n(\psi_j) & \equiv A_j \cos(\omega_j \Delta t n + \psi_j ) 
\end{align}

Let two arbitrarily chosen time indices, $n_1, n_2$ be spaced $\nu = |n_1 - n_2|$ apart. We consider an ensemble of realisations of $\state$, such that the random phase $\psi$ is different in each realisation, the distribution of these phases is uniform over one cycle. Consider the joint probability density of incurring any two random phases (see, e.g. \cite{gelb1974applied}):
\begin{align}
g_2(\psi_1, n_1; \psi_2, n_2) & \equiv \frac{1}{2\pi}, \psi \in [0, 2\pi] \label{eqn:SS_ensble_prob_density} \\
\end{align}

An expectation over an ensemble, $\mathcal{D}$, of sequences of $ \{ f^j_n(\psi_j) \}$ yields the covariance function:
\begin{align}
R(n_1, n_2) &\equiv \ex{(f^j_{n_1}(\psi_1) - \mu_\state )(f^j_{n_2}(\psi_2) - \mu_\state )}_\mathcal{D} \\
& = \ex{f^j_{n_1}(\psi_1) f^j_{n_2}(\psi_2)},  \quad \mu_\state \equiv 0 \\
& = \int  d\psi_1  \int  d\psi_1  \quad f^j_{n_1}(\psi_1) f^j_{n_2}(\psi_2) g_2(\psi_1, n_1; \psi_2, n_2) \\
& = \int_0^{2\pi} d \psi \frac{1}{2\pi} \quad f^j_{n_1}(\psi) f^j_{n_2}(\psi) \\
& = \frac{A_j^2}{4\pi} \int_0^{2\pi} d \psi \quad 2\cos(\omega_j \Delta t n_1 + \psi) \cos(\omega_j \Delta t n_2 + \psi) \\
& = \frac{A_j^2}{4\pi} \int_0^{2\pi} d \psi \quad \cos(\omega_j \Delta t (n_1 -n_2))  + \cos(\omega_j \Delta t (n_1 + n_2) + 2\psi) \\
& = \frac{A_j^2}{2} \cos(\omega_j \Delta t (n_1 -n_2))  + \frac{A_j^2}{4 \pi}\int_0^{2\pi} d \psi  \cos(\omega_j \Delta t (n_1 + n_2) + 2\psi) \\
& = \frac{A_j^2}{2} \cos(\omega_j \Delta t \nu), \nu = |n_1 -n_2|
\end{align}

A long time average taken over one $f^j_n(\psi_j)$ yields:
\begin{align}
R(n, n + \nu)  & \equiv \lim_{N \to \infty} \frac{1}{\Delta t N} \int_{-\Delta t N/2}^{\Delta t N/2} dn \quad  f^j_n(\psi_j) f^j_{n + \nu}(\psi_j)   \\
& = \lim_{N \to \infty} \frac{A_j^2}{2 \Delta t N} \int_{-\Delta t N/2}^{\Delta t N/2} dn \quad 2 \cos(\omega_j \Delta t n + \psi) \cos(\omega_j \Delta t n + \omega_j \Delta t \nu + \psi) \\
& = \lim_{N \to \infty} \frac{A_j^2}{4 \Delta t N} [ 2\Delta t N\cos(-\omega_j \Delta t \nu)   + \int_{-\Delta t N/2}^{\Delta t N/2} dn \quad \cos(2\omega_j \Delta t n + \omega_j \Delta t v + 2\psi) ]\\
& =  \frac{A_j^2}{2} \cos(\omega_j \Delta t \nu) \\
\end{align}
Typically, $f^j_n(\psi_j)$ so defined is only ergodic for uniformly distributed phases. To compute mean square ergodicity for the full $\state$, we reintroduce the sum over $j = 1, 2, ... , J$:
\begin{align}
f_n & = \sum_j^J f^j_n(\psi_j) \\
\ex{f_n} &= \sum_j^J \ex{f^j_n(\psi_j)} = 0 \\
\ex{f_{n_1}f_{n_2}} &=  \ex{\sum_j^J f^j_{n_1}(\psi_j) \sum_{j'}^{J} f^{j'}_{n_2}(\psi_{j'})}  \\
 & = \ex{\sum_j^J f^j_{n_1}(\psi_j) f^j_{n_2} (\psi_j)} + \ex{\sum_{j'}\sum_{j\neq j'}^J f^j_{n_1}(\psi_j) f^{j'}_{n_2}(\psi_{j'})} \\
  & = \sum_j^J \ex{f^j_{n_1}(\psi_j) f^j_{n_2}(\psi_j)} + \sum_{j'}\sum_{j\neq j'}^J \ex{f^j_{n_1}(\psi_j) f^{j'}_{n_2}(\psi_{j'})} \\
 & = \sum_j^J \frac{A_j^2}{2} \cos(\omega_j \Delta t \nu) + \sum_{j'}\sum_{j\neq j'}^J \ex{f^j_{n_1}(\psi_j) f^{j'}_{n_2}(\psi_{j'})}, \nu = |n_1 - n_2 |  \label{eqn:SS_fn_crossterm}\\
 & = \sum_j^J \frac{A_j^2}{2} \cos(\omega_j \Delta t \nu), \nu  = |n_1 - n_2 |, j = j'
\end{align}
 
 I argue that the second term in \ref{eqn:SS_fn_crossterm} is zero because $\psi_j, \psi_{j'}$ are uniformly distributed phases for cycles with different angular frequencies.
 I suggest that the joint probability density function of $ g_2(\psi_j, n_1; \psi_{j'}, n_2) = g(\psi_j, n_1) g(\psi_{j'}, n_2), j \neq j'$:

\begin{align}
\sum_{j'}\sum_{j\neq j'}^J \ex{f^j_{n_1}(\psi_j) f^{j'}_{n_2}(\psi_{j'})} & = A_j A_{j'} \int d \psi_j \quad g(\psi_j, n_1) \cos(\omega_j \Delta t n_1 + \psi_j) \int d \psi_{j'} \quad g(\psi_{j'}, n_2) \cos(\omega_{j'} \Delta t n_2 + \psi_{j'}) \\
& = A_j A_{j'} \int_0^{2\pi} d \psi_j \quad \frac{1}{2\pi} \cos(\omega_j \Delta t n_1 + \psi_j) \int_0^{2\pi} d \psi_{j'} \quad \frac{1}{2\pi}  \cos(\omega_{j'} \Delta t n_2 + \psi_{j'}) \\
& = 0 
\end{align}
 \\
 \\
  A long term time average yields:
\begin{align}
\ex{f_n f_{n+\nu}} & \equiv \lim_{N \to \infty} \frac{1}{\Delta t N} \int_{-\Delta t N/2}^{\Delta t N/2} dn \quad  \sum_j^J f^j_n(\psi_j) \sum_{j'}^{J} f^{j'}_{n + \nu}(\psi_{j'}) \\
&= \sum_j^J  \sum_{j'}^{J} \lim_{N \to \infty} \frac{1}{\Delta t N} \int_{-\Delta t N/2}^{\Delta t N/2} dn \quad   A_j A_{j'} \cos (\Delta t n(\omega_j - \omega_{j'}) - \Delta t \omega_{j'} \nu)  \cos (\Delta t n(\omega_j + \omega_{j'}) + \Delta t \omega_{j'} \nu + 2\psi)\\
&= \begin{cases}
& \sum_j^J \frac{A_j^2}{2} \cos(\omega_j \Delta t \nu),  j = j' \\
& 0, j \neq j'
\end{cases}
\end{align}

Hence, $f(n)$ is a covariance stationary erogdic process. For the evaluation of the long time average, we use product-to-sum formulae and observe that the case $j\neq j'$ has a zero contribution as any finite contribution from cosine terms over a symmetric integral are reduced to zero as $N \rightarrow \infty $.  For $j = j'$, only a single cosine term survives. The surviving term depends on $\nu$ and not $n$ - yielding a cancellation of $N$ and a finite contribution that matches the ensemble average.

We briefly comment that $\state$ is Gaussian by the central limit theorum in the regimes considered in this manuscript. The probability density function of a sum of random variables is a convolution of the individual probability density functions. The central limit theorum grants that each element of $\state_n$ at $n$ appears Gaussian distributed for large $J$, irrespective of the underlying properties of $x_{j,n}$, or the distribution of the phases $\psi$. Numerical analysis shows that $J > 15$ results in each $\state_n$ appearing approximately Gaussian distributed. 




\newpage
% ##############################################################################
\section{ AKF \label{sec:app:AKF}}
% ##############################################################################
% We consider two theoretic frameworks to represent any covariance stationary processes. The first is motivated by Wold's decomposition and gives rise to autogressive high order $p$ approaches in this paper. The second is given by the spectral decomposition theorum, and motivates the use of oscillator approaches in this paper. 

% \subsection{Autoregressive (AR($q$)) and Moving Averages (MA($p$)) Processes}
In this section, we justify the representation of $\state$ assumed by the AKF. In particular, we justify any $\state$ of arbitrary power spectral density satisfying the properties in \cref{sec:app:truenoise} can be approximated by a high order autoregressive process.   In particular, we will consider autoregressive (AR) processes of order $q$, (AR($q$)), and  moving average processes of order, $p$ (MA($p$)). A model incorporating both types of processes is known as an ARMA($q,p$) model in our notation. 

First, we define the lag operator, $\mathcal{L}$. This operator defines a map between time series sequences and enables a compact description of ARMA processes. For an infinite time series $\{ f_n \}_{n = -\infty}^{\infty}$ and a constant scalar, $c$, the lag operator is defined by the following properties:
\begin{align}
\mathcal{L} f_n & = f_{n-1} \\
\mathcal{L}^q f_n & = f_{n-q} \\
\mathcal{L}(cf_n) & = c\mathcal{L}f_n = cf_{n-1}  \\
\mathcal{L}f_n & = c, \quad \forall n, \implies \mathcal{L}^q f_n  = c
\end{align}
Next, we define a Gaussian white noise sequence, $\xi$, under the strong condition than what is stated simply in \cref{eqn:app:ARMA:xi_indep}, that $\xi_n, \xi_m$ are independent $\forall n, m $:
\begin{align}
\ex{\xi} &\equiv 0  \\
\ex{\xi_n \xi_m} &\equiv \sigma^2 \delta(n-m)\label{eqn:app:ARMA:xi_indep}   
\end{align}

With these definitions, we can define an autoregressive process and a moving average process of unity order.  \cref{eqn:app:ARMA:AR_1} defines an AR($q=1$) process and dynamics of $\state_n$ are given as lagged values of the variable $\state$. The second definition in \cref{eqn:app:ARMA:MA_1} depicts a MA($p = 1$) process where dynamics are given by lagged values of Gaussian white noise $\xi$. 
\begin{align}
(1 - \phi_1 \mathcal{L}) f_n  & = c + \xi_n  \label{eqn:app:ARMA:AR_1} \\
f_n & = c' + (\Psi_1 \mathcal{L} + 1)\xi_n  \label{eqn:app:ARMA:MA_1} 
\end{align}
Here, $\Psi_1, \phi_1$ are known scalars defining dynamics of $\state_n$; $w_n$ is a white noise Gaussian process, and $c, c'$ are fixed scalars. It is well known that an MA($\infty$) representation is equivalently an AR($1$) process, and the reverse relationship also applies. For example, we can re-write \cref{eqn:app:ARMA:AR_1} as:
\begin{align}
f_n & = c + \xi_n + \phi_1 f_{n-1} \\
& = w_n + \phi_1 f_{n-1} \\
& = w_n + \phi_1 (w_{n-1}+ \phi_1 f_{n-2} ) \\
& \vdots \\
& = \phi_1^{n+1} F_0 + \phi_1^{n} w_{0} + \phi_1^{n-1} w_{1} + \hdots w_n \\
& = \phi_1^{n+1} F_0 + \phi_1^{n} (c + \xi_{0}) + \hdots + (c + \xi_{n}) \\
& = \phi_1^{n+1} F_0 +  c (\phi_1^{n} + \phi_1^{n-1} + \hdots + 1) + \sum_{m=0}^{n} \phi_1^m \xi_{n-m} \\
w_n & \equiv c + \xi_n \\
F_0 & \equiv f_{n=-1} 
\end{align} We restrict $|\phi_1| < 1$ such that $\state$ is covariance stationary \cite{hamilton1994time}. % An MA process of any order is covariance stationary for any choice of coefficients - instead, the restrictions on the MA representation arise in the form of invertibility of an MA process \cite{hamilton1994time}.
Under these conditions, we take the limit of $f$ capturing an infinite past, or namely, as $n$ indexes an infinite number of terms. The initial state $F_0$ is eventually forgotten, $\phi_1^{n+1} F_0 \approx 0$ if $n$ is large and $|\phi_1| < 1$. Similarly, the terms $c (\phi_1^{n} + \phi_1^{n-1} + \hdots + 1)$  can be summarised as a geometric series in $\phi_1$. The remaining terms satisfy the definition of an MA($\infty$) process:
\begin{align}
f_n &= c \frac{1}{1 - |\phi_1|}  +  \sum_{m=0}^{\infty} \phi_1^m \xi_{n-m}, \quad |\phi_1| < 1
% & = c' + \sum_{m=0}^{\infty} \Psi_m \xi_{n-m}\\
% \Psi_m  & \equiv \Phi^m, \quad \sum_{m=0}^{\infty}  |\Psi_m| = \sum_{m=0}^{\infty}  |\phi|^m < \infty 
\end{align}
It is straightforward to show that the reverse is true, namely, an MR($1$) is equivalent to an AR($\infty$) representation \cite{hamilton1994time}.

The consideration of an MA($\infty$) process leads us directly to Wold's decomposition for arbitrary covariance stationary processes, namely, that any covariance stationary $\state$ can be represented as:
\begin{align}
\state_n & \equiv  c' + \sum_{k=0}^{\infty} \Psi_k \mathcal{L}^k \xi_{n}  \label{eqn:app:ARMA:MAinf} \\
% & =  \tilde{f}_n + \sum_{k=0}^{\infty} \Psi_k \mathcal{L}^k \xi_{n}  \\
c' & \equiv \ex{\state_n | \state_{n-1}, \state_{n-2}, \hdots} \\
% \xi_n & \equiv   \state_n - \ex{\state_n | \state_{n-1}, \state_{n-2}, \hdots} \\
\Psi_0 & \equiv 1 \\
\sum_{k=0}^{\infty} \Psi_k^2 & < \infty
% & = \sum_{k=0}^{\infty} \Psi_k \xi_{n-k} 
\end{align}
\cref{eqn:app:ARMA:MAinf} defines an MA($\infty$) process derived previously as an AR($1$) process. This process is ergodic for Gaussian $\xi$. However, such a representation of $\state$ requires fitting data to an infinite number of parameters $\{\Psi_1, \Psi_2, \hdots \}$  and approximations must be made. 

% A standard approximation is to leverage the dual behaviour of MA($p$) and AR($q$) process and consider finite order polynomials in an ARMA($q,p$) model with MA coefficients given by $\theta_{p \leq p'}$ and AR coefficients given by $\phi_{p \leq p'}$:
% \begin{align}
% \sum_{k=0}^{\infty} \Psi_k \xi_{n-k} & \to \frac{ 1 + \theta_{1}\mathcal{L}^{1} + \hdots + \theta_{p}\mathcal{L}^{p}}{1 - \phi_{1}\mathcal{L}^{1} - \hdots - \phi_{1}\mathcal{L}^{q}}\\
% & \quad |\phi_i|  < 1, i = 1, \hdots, q
% \end{align}
% Instead of using an infinite past or an ARMA approximation,

We approximate an arbitrary covariance stationary $\state$ using finite but high order AR($q$) processes. Below we show that any finite order AR($q$) process has an MA($\infty$) representation satisfying Wold's theorum.

We define an arbitrary AR($q$) process as:
\begin{align}
\xi_n & \equiv (1 - \phi_1 \mathcal{L}  - \phi_2 \mathcal{L} ^2 - \hdots -\phi_q \mathcal{L} ^q) (\state_n - c)
% & = (1 - \lambda_1 \mathcal{L}) \hdots (1 - \lambda_q \mathcal{L}) (\state_n - c) \label{eqn:app:ARMA:ar_p_1}
% \state_n - c & \equiv \frac{1}{(1 - \lambda_1 \mathcal{L}) \hdots (1 - \lambda_q \mathcal{L})}
\end{align}
In particular, we define $\lambda_i, i = 1, \hdots, q$ as eiqenvalues of the dynamical model, $\Phi$:
\begin{align}
\Phi &\equiv \begin{bmatrix} \phi_1 & \phi_2 & \phi_3 & \hdots & \phi_{q-1}  &\phi_q \\
1 & 0 & 0 & \hdots & 0 & 0 \\
0 & 1 & 0 & \hdots & 0 & 0 \\
0 & 0 & 1 & \hdots & 0 & 0 \\
\vdots & \vdots & \vdots & \hdots & \vdots & \vdots \\
0 & 0 & 0 & \hdots & 1 & 0 \\
 \end{bmatrix} \\
\bf{\lambda} & \equiv \begin{bmatrix} \lambda_1 \dots \lambda_q \end{bmatrix} \quad \text{s.t.} |\Phi - \bf{\lambda}\mathcal{I}_q|  = 0 \\
\end{align}
We use the following result from \cite{hamilton1994time} without proof that the above implies:
\begin{align}
1 & - \phi_1 \mathcal{L}  - \phi_2 \mathcal{L}^2 - \hdots -\phi_q \mathcal{L}^q \\
&\equiv (1 - \lambda_1\mathcal{L}) \hdots (1 - \lambda_q \mathcal{L}) 
\end{align}
This yields:
\begin{align}
\xi_n & = (1 - \lambda_1 \mathcal{L}) \hdots (1 - \lambda_q \mathcal{L}) (\state_n - c) \label{eqn:app:ARMA:ar_p_1}
\end{align}
For us to invert this problem and recover an MA process, we need to show that the inverse for each $(1 - \lambda_{q'} \mathcal{L})$ term exists for $q' = 1, \hdots, q$. To do this, we start by defining the operator $\Lambda_q(\mathcal{L}) $ :
\begin{align}
\Lambda_q(\mathcal{L}) & \equiv \lim_{k\to \infty} (1 + \lambda_q \mathcal{L} + \hdots + \lambda_q^k\mathcal{L})
\end{align}
% \begin{align}
% \xi_n & = (1 - \lambda_q \mathcal{L}) (\state_n - c) \\
% \Lambda_q(\mathcal{L}) \xi_n & = \Lambda_q(\mathcal{L}) (1 - \lambda_q \mathcal{L}) (\state_n - c) \\
%  & = \lim_{k\to \infty}(1 + \lambda_q^{k+1}\mathcal{L}^{k+1})(\state_n - c)
% \end{align}
We consider an arbitrary $q'$-th eigenvalue term in process and we multiply $\Lambda_q(\mathcal{L})$ :
\begin{align}
\Lambda_0(\mathcal{L}) \hdots \Lambda_q(\mathcal{L}) \hdots \xi_n &= 
\Lambda_0(\mathcal{L}) \hdots \Lambda_{q'}(\mathcal{L}) \hdots  (1 - \lambda_{0} \mathcal{L}) \hdots (1 - \lambda_{q'} \mathcal{L}) \hdots(\state_n - c) \\
 & = \lim_{k\to \infty}(1 + \lambda_{0}^{k+1}\mathcal{L}^{k+1}) \hdots \lim_{k'\to \infty}(1 + \lambda_{q'}^{k'+1}\mathcal{L}^{k'+1}) \hdots (\state_n - c)
\end{align}
Each of the residual terms,  $\lambda_{q'}^{k'+1}\mathcal{L}^{k'+1} \to 0 $ if $|\lambda_{q'}| < 1$  for large $k'$, and this case $\Lambda_{q'}(\mathcal{L})$ defines the inverse $(1 - \lambda_{q'} \mathcal{L})^{-1}$. This procedure is repeated for all $q'$ eigenvalues to invert \cref{eqn:app:ARMA:ar_p_1} and subsequently perform a partial fraction expansion as follows:
\begin{align}
\state_n - c & = \frac{1}{(1 - \lambda_1 \mathcal{L}) \hdots (1 - \lambda_q \mathcal{L})} \xi_n\\
& = \sum_{q'=1}^{q}\frac{a_{q'}}{1- \lambda_{q'} \mathcal{L}} \xi_n \\
a_{q'} & \equiv \frac{\lambda_{q'}^{q-1}}{\prod_{q''=1, q''\neq q'}^{q} (\lambda_{q'} - \lambda_{q''})}
\end{align} The coefficients are $a_{q'}$ as obtained via the partial fraction expansion method during which $\mathcal{L}$ is treated as an ordinary polynomial. At present, we have a represent $\state$ via a finite $q$ weighted average of values of $\xi$. However, in substituting the definition of $ \Lambda_{q'} \equiv (1- \lambda_{q'} \mathcal{L})^{-1}$, we recover the form of an MA representation (setting $c \equiv \tilde{\state}_n  = 0, \quad  \forall n$ for simplicity): 
\begin{align}
\state_n & = \left[ \sum_{q'=1}^{q} a_{q'} \mathcal{L}^0 +  \lim_{k \to \infty}  \sum_{k'=1}^{k} \left( \sum_{q'=1}^{q} a_{q'}  \lambda_{q'}^{k'} \right) \mathcal{L}^{k'}\right] \xi_n \\
& = \Psi_0 + \sum_{k=1}^{\infty} \Psi_k \mathcal{L}^k \xi_{n}  \\
\Psi_0 & \equiv \sum_{q'=1}^{q} a_{q'} \mathcal{L}^0  \\
\Psi_k & \equiv \sum_{q'=1}^{q} a_{q'}  \lambda_{q'}^{k'}
\end{align}
By examining the properties of $\Phi$ raised to arbitrary powers, it can be shown that $\sum_{q'=1}^{q} a_{q'} \equiv 1$ and $\Psi_k$ is the first element of $\Phi$ raised to the $k$-the power \cite{hamilton1994time}, yielding absolute summability of $\Psi_k$ if $|\phi_{q'<q}| < 1$. This ensures that Wold's theorum is fully satisfied and an AR($p$) process has an MA($\infty$) representation. In moving to an arbitrarily high $q$, we enable the approximation of any covariance stationary $\state$.

The proofs that high $q$ AR approximations for covariance stationary $f$ improve with $q$ for example, in \cite{wahlberg1989estimation}. The key correspondence is that the number of finite lag terms $q$ in an AR($q)$) model contribute to the first $q$ values of the covariance function. This approximation improves with $q$ even if $\state$ is not a true AR process \cite{wahlberg1989estimation,west1996bayesian}. Asymptotically efficient coefficient estimates for any $MA(\infty)$ representation of $\state$ are obtained by letting the order of a purely AR($q$) process tend to infinity and increasing total data size, $N$ \cite{wahlberg1989estimation}. 

When data is fixed at $N$, we expect a high $q$ model to gradually saturate in predictive estimation performance. One can arbitrarily increase performance by increasing both $q, N$ \cite{wahlberg1989estimation}.  In our application with finite data $N$, we increase $q$ to settle on a high order AR model while training LSF to track arbitrary covariance stationary power spectral densities \cite{brockwell1996introduction}. A high $q$ AR model is often the first step for developing models with smaller number of parameters, for example, considering a mixture of finite order AR($q$) and MA($p$) models and estimating $p+q$ number of coefficients using a range of standard protocols \cite{brockwell1996introduction,west1996bayesian}. The design of potential ARMA models for our application requires further investigation beyond the scope of this manuscript.




\newpage
% ##############################################################################
\section{Spectral Methods for LKFFB and GPR (Periodic Kernel)} \label{sec:app:spec_methods}
% ##############################################################################

 The well-known spectral representation theorum  guarantees that any covariance stationary random process (real or complex) can be represented in a generalised harmonic basis.  We defer a detailed treatment of spectral analysis of covariance stationary processes in standard textbooks, for example, \cite{hamilton1994time,karlin1975first} and present background and key results to provide insights into the choice of LKFFB and GPR (periodic kernel).

In this Appendix, the term $\state_n $ is granted to us by the spectral representation theorum and it defines the `true model' for an algorithm. However, $\state_n $ (from the spectral representation theorum) only approximates the true covariance stochastic phases of \cref{sec:app:truenoise} in the limit where total size of available sample data increases to infinity. The subtle difference between $\state_n$ defined in \cref{sec:app:truenoise} and $\state_n $ via the spectral representation theorum (below) arises from the fact that we have no apriori true model of describing stochastic qubit phases, and must rely on mean square approximations for tracking qubit phases. Henceforth, we retain $\state_n $ to be the true model for an algorithm with an understanding that this refers to an approximate representation of an arbitrary, covariance stationary sequence of stochastic qubit phases. We reserve the use of the $\hat{f_n}$ for the state estimates and predictions that an algorithm makes having considered a single measurement record. 

 The spectral representation theorum states that any covariance stationary random process has a representation given by $\state_n$, and correspondingly,  a probability distribution, $F(\omega)$ over $[-\pi, \pi]$ in the dual domain such that:
 
\begin{align}
\state_n & = \mu_\state + \int_{0}^{\pi} [ a(\omega) \cos(\omega n) +  b(\omega) \sin(\omega n) ] d\omega \\
R(\nu) & = \int_{-\pi}^{\pi} e^{-i\omega \nu } dF(\omega)
\end{align}
Here, $\mu_\state $ is the true mean of the process $\state$.  The processes $a(\omega) $ and $b(\omega)$ are zero mean and serially and mutually uncorrelated, namely, $\int_{\omega_1}^{\omega_{2}} a(\omega) d\omega$ is uncorrelated with $\int_{\omega_3}^{\omega_{4}} a(\omega) d\omega$ and $\int_{\omega_j}^{\omega_{j'}} b(\omega) d\omega$ for any $\omega_1 < \omega_2 < \omega_3 < \omega_4$ and any choice of $j, j'$ within the half cycle  $[0, \pi]$.

The distribution $F(\omega)$ exists as a limiting case of considering cumulative probability density functions for $\state_n$ at each $n$ and letting $n \to \infty$ such that a sequence of these density functions approach $F(\omega)$ \cite{karlin1975first}.  If $F(\omega)$ is differentiable with respect to $\omega$, then we see the power spectral density $S(\omega)$ and $R(\nu)$ are Fourier duals \cite{karlin1975first}:
\begin{align}
R(\nu) & = \int_{-\pi}^{\pi} e^{-i\omega \nu } S(\omega)d\omega \\
S(\omega) & \equiv \frac{dF(\omega)}{d\omega} 
\end{align}
The duality of the covariance function and the spectral density is formally expressed  in literature by the  Wiener Khinchin theorum.

We consider the finite sample analogue of the spectral representaiton theorum considered above by following \cite{hamilton1994time}. To proceed, we define mean square convergence as a distance metric for determining when a sequence of random variables $\{ \hat{f}_n\}$ converges to a random variable, $f_n$ in the mean square limit if:
\begin{align}
\ex{\hat{f}_n^2} & < \infty \quad \forall n \\
\lim_{n \to \infty}\ex{\hat{f}_n - f_n} & = \lim_{n \to \infty} ||\hat{f}_n- f_n || = 0
\end{align} 
The statement $||\hat{f}_n- f_n || = 0$ measures the closeness between random variables $\hat{f}_n$ and $f_n$ even though the mean square limit is defined for terms of a sequence of random variables, $\{ \hat{f}_n\}$, where convergence improves with $n \to \infty$. In context of this study, we define $\hat{f}_n$ as a linear predictor of $f_n$ belonging to a covariance stationary $\state$. Hence, each $\hat{f}_n$ for large $n$ is a linear combination of the set of random variables belonging to $\state$ (priori to $n$) and, in Kalman filtering, all past predictors. Mean square convergence of $||\hat{f}_n- f_n || = 0$ in our context is a statement of the quality of the predictor, $\hat{f}_n$ , in predicting $f_n$ as the total measurement data grows.

Next, we account for finite data and define the finite sample analogue for the spectral representation theorum. We suppose there exists a set of arbitrary, fixed frequencies  $\{\omega_j\}$  for $j = 1, \hdots , J$. We let $n$ denote finite time steps for observing $\state_n$ at $n= 1, \hdots, N$. Further, we define a set of zero mean, mutually and serially uncorrelated random process  $\{a_j \}$ and $\{b_j\}$ as finite sample analogues of the true  $a(\omega)$ and $b(\omega)$ for the $j$-th spectral component. In particular, these processes are constant over $n$ by covariance stationarity of $\state$. Then, the finite sample analogue for the spectral representation theorum becomes \cite{hamilton1994time}:
\begin{align} 
\state_n &= \mu_f  + \sum_{j=1}^{J}  [ a_j \cos(\omega_j n) +  b_j \sin(\omega_j n) ] \\
\ex{a_j} & = \ex{b_j} = 0\\
\ex{a_ja_{j'}} &= \ex{b_jb_{j'}} = \sigma^2 \delta(j - j') \\
\ex{a_jb_{j'}} &= 0 \quad \forall j, j' 
\end{align}

The first two moments are of the form:
\begin{align}
\ex{f_n} &=  \mu_\state +  \sum_{j=0}^{J} E[a_j] \cos(\omega_j n) + E[b_j] \sin(\omega_j n)  = 0\\
R(\nu) &= \sum_j^{J} \sum_{j'}^{J} \sigma_j^2\delta{j,j'} [\cos(\omega_j n)\cos(\omega_j' (n+\nu)) + \sin(\omega_j n)\sin(\omega_j' (n+\nu)) ]\\
% &= \sum_j^{J} \sigma_j^2 \cos(\omega_j\nu), \quad \text{$(\cos(a)\cos(b) + \sin(a)\sin(b) = \cos(a-b))$} \\
&= \sigma^2 \sum_j^{J}  p_j \cos(\omega_j\nu) \\
p_j & \equiv \frac{\sigma_j^2}{\sigma^2} \equiv \frac{\sigma_j^2}{\sum_j \sigma_j^2} 
\end{align}

We introduce process noise, $w_n$, into the formula for true $f_n$, and this establishes the link with Kalman filtering:
\begin{align} 
\state_n &= \mu_f  + \sum_{j=1}^{J}  [ a_j \cos(\omega_j (n-1)) +  b_j \sin(\omega_j (n-1)) ] + w_n 
\end{align}

In the absence of measurement noise and operating in the oversampling regime, an ordinary least squares (OLS) regression can be constructed by providing a collection of $J^{(B)}$ basis frequencies $\{\omega_j^{(B)}\}$, as in \cite{hamilton1994time}. The OLS problem is constructed by separating the set of coefficients $\{\hat{\mu}_\state, \hat{a}_1, \hat{b}_1, \hdots \hat{a}_J, \hat{b}_J\}$ and regressors $\{1,\cos(\omega_1 (n-1)), \sin(\omega_1 (n-1)), \hdots, \cos(\omega_J^{(B)} (n-1)), \sin(\omega_J^{(B)} (n-1)) \}$. For the specific particular choice of basis,  $J^{(B)} = (N-1)/2$, (odd $N$) and $\omega_j^{(B)} \equiv 2\pi j / N$, we state the key result from \cite{hamilton1994time} that the coefficient estimates are obtained as:

\begin{align}
\hat{\state}_n &= \hat{\mu}_f  + \sum_{j=1}^{J^{(B)}}  [\hat{a}_j \cos(\omega_j^{(B)} (n-1)) +  \hat{b}_j \sin(\omega_j^{(B)} (n-1)) ] \\
\hat{a}_j &\equiv \frac{2}{N} \sum_{n'=1}^{N} \hat{\state}_{n'} \cos(\omega_j^{(B)}(n'-1)) \\
\hat{b}_j &\equiv \frac{2}{N} \sum_{n'=1}^{N} \hat{\state}_{n'} \sin(\omega_j^{(B)}(n'-1))
\end{align}
This choice of basis results in the number of regressors being the same as the length of the measurement record. Further, the term $(\hat{a}_j^2 + \hat{b}_j^2)$ is proprotional to the total contribution of the $j$-th spectral component to the total sample variance of $\state$, or in other words, the amplitude estimate for the power spectral density of true $\state$.

Next, we depart from the OLS problem above by in several ways, firstly, by introducing measurement noise and secondly, by changing basis oscillators considered in the problem above. As in the main text, the linear measurement record is defined as:
\begin{align}
y_n &\equiv  f_n + v_n 
\end{align}
The link in GPR (periodic kernel) is direct and the link with LKFFB is made by setting $f_n \equiv H_nx_n$. In both frameworks, we incorporate the effect of measurement noise through the measurement noise variance, $R$, which has the effect of regularising the least squares estimation process discussed above {\color{red} [XX REF, e.g. \cite{west1996bayesian}]}

\subsection{GPR (Periodic Kernel)}\label{sec:ap_approxSP:GPRPKernel}

The departure from simple OLS to GPR (periodic kernel) arises from the fact that data is projected on an infinite basis of oscillators, namely, $J^{(B)} \to \infty$, and we follow high level remarks in \cite{solin2014explicit} to illustrate this below.

We use high level remarks in \cite{solin2014explicit} to explicitly work out that a sine squared exponential (Periodic Kernel) used in Gaussian Process Regression satisfies covariance function of trigometric polynomials. Here, the indice $j$ labels an infinite comb of oscillators and $m$ represents the higher order terms in the power reduction forumulae in the last line of the definition below:
\begin{align}
\omega_0^{(B)}  &\equiv \frac{\omega_j^{(B)} }{j}, j \in \{0, 1,..., J^{(B)}\} \\
R(\nu) &\equiv \sigma^2 \exp (- \frac{2\sin^2(\frac{\omega_0^{(B)}  \nu}{2})}{l^2}) \\
&=  \sigma^2 \exp (- \frac{1}{l^2}) \exp (\frac{\cos(\omega_0^{(B)}  \nu)}{l^2}) \label{eqn:periodic_0}\\
&=  \sigma^2 \exp (- \frac{1}{l^2}) \sum_{m = 0}^{M  \to\infty} \frac{1}{m!} \frac{\cos^m(\omega_0^{(B)}  \nu)}{l^{2m}} \label{eqn:periodic_1}
\end{align}
Next, we expand each cosine using power reduction formulae for odd and even powers respectively, and we re-group terms. For example, we expande the terms for  $m = 0,1,2,3,4,5...$ as:
\begin{align}
R(\nu) &= \sigma^2 \exp (- \frac{1}{l^2}) \cos(\omega_0^{(B)}  \nu) \left[ \frac{2}{(2l^2)}\binom{1}{0} + \frac{2}{(2l^2)^3} \frac{1}{3!} \binom{3}{1} +  \frac{2}{(2l^2)^5} \frac{1}{5!}\binom{5}{2} \dots \right] \label{eqn:cosine1}\\
& + \sigma^2 \exp (- \frac{1}{l^2}) \cos(2\omega_0^{(B)}  \nu) \left[ \frac{2}{(2l^2)^2} \frac{1}{2!} \binom{2}{0} + \frac{2}{(2l^2)^4} \frac{1}{4!} \binom{4}{1} + \dots \right] \\
& + \sigma^2 \exp (- \frac{1}{l^2}) \cos(3\omega_0^{(B)}  \nu) \left[ \frac{2}{(2l^2)^3} \frac{1}{3!} \binom{3}{0} + \frac{2}{(2l^2)^5} \frac{1}{5!}\binom{5}{1} \dots \right] \\
& + \sigma^2 \exp (- \frac{1}{l^2}) \cos(4\omega_0^{(B)}  \nu) \left[ \frac{2}{(2l^2)^4} \frac{1}{4!} \binom{4}{0} + \dots \right] \\
& + \sigma^2 \exp (- \frac{1}{l^2}) \cos(5\omega_0^{(B)}  \nu) \left[ \frac{2}{(2l^2)^5} \frac{1}{5!}\binom{5}{0} + \dots \right] \label{eqn:cosine5}\\
& \vdots \nonumber \\
& + \sigma^2 \exp (- \frac{1}{l^2}) \left[ \frac{1}{(2l^2)^2} \frac{1}{2!} \binom{2}{1} + \frac{1}{(2l)^4} \frac{1}{4!} \binom{4}{2} + \dots \right] + \sigma^2 \exp (- \frac{1}{l^2}) \label{eqn:eventerms}
\end{align}
In the expansion above, the vertical and horizontal dots represent contributions from $m>5$ terms. The key message is that truncating $m$ to a finite number of terms $M$ will forecably truncate $j$ to represent a finite number of oscillators. For the example above, if the power reduction expansion indexed by $m$ above was trucated to $M=5$ terms, then the  number of basis oscillators (number of rows) would also be truncated.  We now summarise the amplitudes \cref{eqn:cosine1} to  \cref{eqn:cosine5} in second term of $R(\nu)$ and  \cref{eqn:eventerms} corresponds to $p_{0,M}$ term below:
\begin{align}
R(\nu) &= \sigma^2 (p_{0,M} + \sum_{j=0}^{\infty} p_{j,M} \cos(j\omega_0^{(B)}  \nu))\\
p_{j,M} & \equiv \sigma^2 \exp (- \frac{1}{l^2}) \sum_{\beta = 0}^{\beta = \beta_{j,m}^{MAX}} \frac{2}{(2l^2)^{(j + 2\beta)}} \frac{1}{(j + 2\beta)!} \binom{j + 2\beta}{\beta} \label{eqn:beta_series2} \\
\beta &\equiv  0,1,..., \beta_{j,m}^{MAX}  \\
p_{0,M} &= \exp (- \frac{1}{l^2}) \sum_{\alpha = 0}^{\alpha = \alpha_{m}^{MAX}} \frac{1}{(2l^2)^{(2\alpha)}} \frac{1}{(2\alpha)!} \binom{2\alpha}{\alpha} \label{eqn:alpha_series}\\
\alpha &\equiv  0,1,..., \alpha_{m}^{MAX} 
\end{align}
By examining the cosine expansion, one sees that a truncation at $(M, J^{(B)} )$ means our summarised formulae will require $\beta_{j,M}^{MAX} = \lfloor\frac{M-j}{2}\rfloor$ and $\alpha_{M}^{MAX} = \lfloor\frac{M}{2}\rfloor$  where $\lfloor \rfloor$ denotes the ceiling floor. If we truncate with $M \equiv J^{(B)} $ such that $\alpha_{M}^{MAX} = \lfloor\frac{J^{(B)} }{2}\rfloor, \beta_{j,M}^{MAX} =  \lfloor\frac{J-j}{2}\rfloor $ and re-adjust the kernel for the zero-th frequency term, then we agree with results in \cite{solin2014explicit}.

In a truncated form, it is easier to see the correspondence with a covariance function for $f_n$ appoximating a covariance stationary process under the spectral representation theorum. We note, however, that the covariances $p_{j,M}$ are specified exactly by \cref{eqn:beta_series2} and this is not identical to those under the spectral representation theorum. Further, the periodicity of the kernel set by $\omega_0^{(B)} $ means that it may define non-stationary processes given the choices of hyperpameters, $\omega_0^{(B)}, l$ for a particular time series application.  

\subsection{LKFFB}
 In LKFFB, we depart from the OLS problem considered earlier by specifying a fixed basis of oscillators at the physical Fourier resolution established by the measurement record, and incorporating apriori assumptions about the extent to which a fast measurement action oversamples slowly drifting non-Markovian noise.  

Under the following correlation relations (below), and the Gaussian noise assumption, we see that LKFFB defines a stack of stochastic processes on a circle in \cref{eqn:cov_circle}, with the posterior Kalman state acting as the initial state for the next time step, such that $\nu = 0$, for each basis frequency:
 \begin{align}
 \ex{w_n} &= 0 \\
\ex{w_n,w_m} &= \sigma^2 \delta_{n,m}  \\
\ex{A^j_0} &=\ex{B^j_0} = 0 \\
\ex{A^j_n B^j_m} &= 0 \\
\ex{A^j_n A^j_m} &= \ex{B^j_n B^j_m} = \sigma_j^2 \delta_{n,m} \\
\ex{w_n A^j_m} &= \ex{w_n B^j_m} \equiv 0 \quad  \forall n, m \\
\end{align}
Consider a $j$-th substate, $x^j_n$, in the LKKFB, we obtain:
\begin{align}
\Phi(j \omega_0 \Delta t) &= \begin{bmatrix} \cos(j \omega_0 \Delta t) & -\sin(j \omega_0 \Delta t) \\ \sin(j \omega_0 \Delta t) & \cos(j \omega_0 \Delta t) \\ \end{bmatrix} \\
x^j_n & \equiv \begin{bmatrix} A^j_{n} \\ B^j_{n} \\ \end{bmatrix} = \Phi(j \omega_0 \Delta t) \left[\idn + \frac{w_{n-1}}{\sqrt{A^j_{n-1}{}^2 + B^j_{n-1}{}^2}} \right] \begin{bmatrix} A^j_{n-1} \\ B^j_{n-1} \\ \end{bmatrix} \\
\end{align}
\begin{align}
\implies \ex{x^j_n} &= 0 \\
\implies \ex{x^j_n x^j_m{}^T}_j & =   \Phi(j \omega_0 \Delta t) \ex{\begin{bmatrix} A^j_{n-1}A^j_{m-1} & A^j_{n-1}B^j_{m-1}\\ B^j_{n-1}A^j_{m-1} & B^j_{n-1}B^j_{m-1}\\ \end{bmatrix}} \Phi(j \omega_0 \Delta t)^T \label{eqn:cov_kf_term1}\\
& +   \Phi(j \omega_0 \Delta t) \left[\frac{w_{n-1}}{\sqrt{A^j_{n-1}{}^2 + B^j_{n-1}{}^2}} + \frac{w_{m-1}}{\sqrt{A^j_{m-1}{}^2 + B^j_{m-1}{}^2}} \right]\begin{bmatrix} A^j_{n-1}A^j_{m-1} & A^j_{n-1}B^j_{m-1}\\ B^j_{n-1}A^j_{m-1} & B^j_{n-1}B^j_{m-1}\\ \end{bmatrix} \Phi(j \omega_0 \Delta t)^T  \label{eqn:cov_kf_term2}\\
& +   \Phi(j \omega_0 \Delta t) \left[\frac{w_{n-1}w_{m-1}}{\sqrt{A^j_{n-1}{}^2 + B^j_{n-1}{}^2}\sqrt{A^j_{m-1}{}^2 + B^j_{m-1}{}^2}} \right]\begin{bmatrix} A^j_{n-1}A^j_{m-1} & A^j_{n-1}B^j_{m-1}\\ B^j_{n-1}A^j_{m-1} & B^j_{n-1}B^j_{m-1}\\ \end{bmatrix} \Phi(j \omega_0 \Delta t)^T \label{eqn:cov_kf_term3} \\
& = \sigma^2_j \delta_{n,m} \begin{bmatrix} 
1 & 0 \\ 
0 & 1  \\
\end{bmatrix} \label{eqn:cov_kf_term4}
 \end{align}

The cross correlation terms disappear under the temporal correlation functions so defined, namely, if assume $n \geq m$, then states $A^j_{m-1}, B^j_{m-1}$ at $m-1$ at most have a $w_{n-2}$ term (for the case $n=m$) and cannot be correlated with a future noise term $w_{n-1}$. This is a special case where $\nu=0$ for the simple stochastic process on a circle for $\nu \equiv 0$ \cite{karlin}.  The zero lag arises as the dynamical model propagates a random state $\Delta t$ ahead in time.

The dynamical trajectory in LKFFB is linearised for small $\Delta t$.  The linearisation is an approximation to a true, continuous time determinstic trajectory defining a stochastic process on a circle. We briefly visit this continous time trajectory to specify the link between LKFFB and GPR (periodic kernel). Let $t$ denote the continuous time deterministic dynamics for random initial state given by $a_j, b_j$, with zero mean, and mutually and serially uncorrelated properties as before:
\begin{align}
x^j(t) & \equiv \begin{bmatrix} A(t)^j \\ B(t)^j \\ \end{bmatrix} \equiv \begin{bmatrix} \cos(\omega_j t) & -\sin(\omega_j t) \\ \sin(\omega_j t) & \cos(\omega_j t) \\ \end{bmatrix} \begin{bmatrix} a_j \\ b_j \\ \end{bmatrix} \\
E[x^j(t)]&= 0 \\%\ex{\begin{bmatrix} a_j \\ b_j \\ \end{bmatrix}} = \begin{bmatrix} 0 \\ 0 \\ \end{bmatrix} \\
E[x^j(t) x^j(t'){}^T]&= \begin{bmatrix} \cos(\omega_j t') & -\sin(\omega_j t') \\ \sin(\omega_jt') & \cos(\omega_jt') \\ \end{bmatrix} \begin{bmatrix} a_j \\ b_j \\ \end{bmatrix} \begin{bmatrix} a_j & b_j \\ \end{bmatrix} \begin{bmatrix} \cos(\omega_j t) & -\sin(\omega_j t) \\ \sin(\omega_j t) & \cos(\omega_j t) \\ \end{bmatrix} \\ 
&=\sigma^2 \begin{bmatrix} 
\cos(\omega_j\nu) & 0 \\ 
0 & \cos(\omega_j\nu)  \\
\end{bmatrix}, \quad \nu \equiv |t'-t| \label{eqn:cov_circle}
\end{align}
We see that the initial state variables, $a_j, b_j$, must be zero mean and i.i.d. variables for $x^j(t)$ to be covariance stationary. If $a_j, b_j$ are Gaussian, then the joint distribution, $x^j(t)$, remains Gaussian. Hence, the continuous time limit of the dynamics in LKKFB for $J^{(B)}$  independent substates, $x^j(t)$, describe a process with the same first and second moments for a periodic kernel truncated at $J^{(B)}$. For Gaussian processes, this results in an approximate equivalent representation of classical Kalman filtering for $J^{(B)}$ stacked resonators with the periodic kernel. The link between that GPR (periodic kernel) and classical Kalman filtering, and the approximation error which arises from an arbitrary truncation of the periodic kernel, is fullly articulated in \cite{solin2014explicit}. 

While the formalism of LKFFB shares a common structure with GPR (periodic kernel) in a particular limit, the  physical interpretation of $A^j_{m-1}, B^j_{m-1}$ is that these are components of the Hilbert transform of the original signal \cite{livska2007}. In particular, we can calculate the instantaneous amplitude and phase associated with each basis oscillator. The efficacy of the Liska Kalman Filter in our application assumes an appropriate choice of the `Kalman basis' oscilaltors. 

We note that the choice of basis effects the interpretation of the state estimates. To illustrate, consider the choice of Basis A - C defined below. Basis A depicts a constant spacing above the Fourier resolution (e.g. $\omega_0^{(B)} \geq \frac{2\pi}{N_T \Delta t}$). Basis B  instroduces a minimum Fourier resolution and effectively creates an irregular spacing if one wishes to consider a basis frequency comb coarser than the experimentally established Fourier spacing over the course of the experiment. Basis C is identical to Basis B but allows a projection to arbitrarily low (zero)frequency components. 
\begin{align}
\text{Basis A: } & \equiv \{0, \omega_0^{(B)}, 2\omega_0^{(B)} \dots  J^{(B)} \omega_0^{(B)} \} \\
\text{Basis B: } & \equiv \{ \frac{2\pi}{N \Delta t}, \frac{2\pi}{N \Delta t} + \omega_0^{(B)} , \dots,   \frac{2\pi}{N \Delta t} + J^{(B)} \omega_0^{(B)} \} \\
\text{Basis C: } & \equiv \{ 0, \frac{2\pi}{N \Delta t}, \frac{2\pi}{N \Delta t} + \omega_0^{(B)},  \dots,   \frac{2\pi}{N \Delta t} + J^{(B)} \omega_0^{(B)} \} 
\end{align}
 While one can propagte LKFFB with zero gain, it may be advantageous for predictive control applications to generate predictions in one calculation rather than recursively. This means we sum constributions over all $j\in J^B$ oscillators and we reconstruct the signal for all future time values in one calculation, without having to propagate the filter recursively with zero gain. The interpretation of the predicted signal, $\hat{s}_n$, requires an additional (but time-constant) phase correction term $\psi_C$ that arises as a byproduct of the computational basis (i.e. Basis A, B or C).  The phase correction term corrects for a gradual mis-alignment between Fourier and computational grids which occurs if one specifies a non-regular spacing inherent in Basis B or C. Let $n_C$ denote the time-step at which instantaneous amplitudes $\norm{\hat{x}^j_{n_C}}$ and instantaneous phase $\theta_{\hat{x}^j_{n_C}}$ is extracted for the $j$-th oscillator:
\begin{align}
\hat{f} &= \sum_j\norm{\hat{x}^j_{n_C}} \cos(m\Delta t \omega_j + \theta_{\hat{x}^j_{n_C}} + \psi_C), \\
& \quad  n, n_C \in N_T, \quad m \in N_P \nonumber \\
\psi_C & \equiv \begin{cases}
0,  \quad \text{(Basis A)} \\
\equiv  \frac{2\pi}{\omega_0^{(B)} } (\omega_0^{(B)} - \frac{2\pi}{N \Delta t}), \quad \text{(Basis B or C)} \\
\end{cases}
\end{align}

Next, we define an analytical ratio to define the optimal training time, $n_C$, at which Kalman predictions should commence. 
\begin{align}
n_C &\equiv \frac{1}{\Delta t \omega_0^{(B)}} = \frac{f_s}{\omega_0^{(B)}} \label{eqn:sec:ap_liska_fixedbasis_nC}
\end{align}
Consider an arbitrarily chosen training period, $N_T \neq n_C $.  For $f_s$ fixed, our choice of $N_T > n_C $ means we are achieving a Fourier resolution which exceeds the resolution of the LKFFB basis. Now consider $N_T< n_C$. This means that we've extracted information prematurely, and we have not waited long enough to project on the smallest basis frequency, namely, $\omega_0^{(B)}$.  In the case where data is perfectly projected on our basis, this has no impact. For imperfect learning, we see that instantaneous amplitude and phase information slowly degrades for $N_T > n_C$; and trajectories for the smallest basis frequency have not stablised for $N_T < n_C$. 

Of these choices, Basis A for $\omega_0^{(B)} \equiv \frac{2\pi}{N_T \Delta t}$ is expected to yield best performance, at the expense of computational load, and this is confirmed in numerical experiments. All results in this manuscript are reported for Basis A with $N_T \equiv \frac{1}{\Delta t \omega_0^{(B)}} = \frac{f_s}{\omega_0^{(B)}} $.




\newpage
% ##############################################################################
\section{Quantised Kalman Filter \label{sec:app:qkf}}
% ##############################################################################

In this Appendix, we attempt to revist classical statistical theory of amplitude quantisation, and we propose a manner in which a coin flip measurement action can be incorporated into the existing classical frameowrk.  In the classical framework, we rely on the filtering and estimation theory outlined by \cite{karlsson2005} and we summarise their results here for convenience. We suggest a way of using this classical framework for deriving a CRLB applicable to the QKF. 

\subsection{QKF (Summary Definitions)}
The definition of the QKF and the accompanying the numerical results stand of their own accord in the main text and are not revisited here. 

As stated in the main text, we summarise the QKF definitons below for easy reference:
\begin{align}
d_n &\equiv \mathcal{Q}(\tilde{y})\\
\tilde{y}_n &= z_n + v_n \\
z_n & \equiv  h(x_n[0])  \equiv h(f_n) \equiv \frac{1}{2}\cos(\state_{n}) \\
x_n & = \Phi_n x_{n-1} + \Gamma_n w_n  \equiv  \begin{bmatrix} f_{n} \hdots f_{n-q+1} \end{bmatrix}^T \\
H_n &\equiv \frac{d h(\state_n)}{d\state_n} =  -\frac{1}{2}\sin(\state_{n}) \\
w_n & \sim \mathcal{N}(0, \sigma^2) \quad \forall n \\
v_n & \sim \mathcal{N}(0, R) \quad \forall n
\end{align}

The quantisation action, $\mathcal{Q}$, is performed by a binomial coin toss in QKF, where the bias on the coin in the QKF is given by $\tilde{y}_n$:
\begin{align}
\mathcal{Q}: Pr(d_n| \tilde{y}_n, \state_{n}, \tau) & \equiv \mathcal{B}(n_{\mathcal{B}}=1;p_{\mathcal{B}}= \tilde{y}_n + 0.5 ) \label{eqn:app:coinflipquantiser}
\end{align}

The numerical studies for establishing the performance of the QKF are detailed in the main text. 

\subsection{Classical Amplitude Quantisation and CRLB for Kalman Filters}

Many classical situations are give rise to scenarios where a continuous time, continuous amplitude signal is discretised in both time and amplitude. The discretisation in time is governed by well known Nyquist and Fourier domain sampling theory. Amplitude quantisation can occur for $2m = 2^b$ levels, where $b$ is the number of bits (namely, $b=1$ for our case). Examples include a classical sensor of $b$ bits generates quantised data measurements, where there are  $2^m$ values a measurement can take; or analogue information is digitised using an $b$ bit ADC. [XXX REFERENCES]

In classical probability theory, the underlying probability distribution of a true continous random process is discretised by the process of amplitude-quantisation. In particular, amplitude quantisation is seen as a \emph{linear} map by which the probability distribution of the underlying true, continous signal plus noise is divided into $2m$ sections, and the area under each section is calculated and condensed into a discrete (area-sampled) probability density function \cite{widrow1996,karlsson2005}.  This is given by Eq. (3)-(6) in \cite{karlsson2005}. We re-write these equations suggestively in our notation, with $\star$ denoting a convolution and $l(d)$ denotes a pulse train defined by $m$ and the Dirac-delta function, $\delta(\cdot)$:

\begin{align}
Pr_d(d) & \equiv l(y) (Pr_z \star Pr_U)(d) \\
l(d) & \equiv \sum_{i = -m}^{m-1} \delta (d - i2^{-b} +  \frac{2^{-b}}{2} ) \\
Pr_U & \equiv \begin{cases} 2^b, \quad -\frac{2^{-b}}{2} \leq d \leq  \frac{2^{-b}}{2} & 0, \quad \text{otherwise} \end{cases}
\end{align}
If amplitudes are quantised into $2m= 2^b$ levels, then the term $2^{-b}$ represents the size of the quantisation box in the dual domain, namely, it specifies the width of the uniform probability density function (area) by which to sample $Pr_z$.  The resulting error between the the true and quantised process is found to be zero mean and of variance $2^{-b} / 12$ \cite{widrow1996}. In the derivation of quantisation as area-sampling, the procedure is described purely in terms of the convolution of an arbitrary $Pr_z$ with a uniform distribution. Hence, amplitude quantisation is analalogised to Nyquist sampling and interpolation procedures in the time domain \cite{widrow1996}. 
Since a convolution operator and the multiplication with $l(d)$ are all linear transformations, the above relation describes a linear map between a continous $z$ and the quantised signal $y$. 

The statistical properties of $\mathcal{Q}$ are captured entirely by the defintions above. Hence, literature suggests that one can interpret amplitude quantisation as a linear map between probability distributions between a prior (for a continuous signal, $z$) and a posterior (for a quantised signal, $d$); just as Bayes rule defines a different type of map between probabability distribution of true and measured quantities of interest. 

For a typical Kalman filter,  the state, $x_n$ and its variance, $P_n$, is propagated by a set of equations. The equations for propagating the variance alone are known as the Ricatti equations \cite{grewal2001theory}. In information filtering, the quantity of interest is not $P_n$, but Fisher information, namely the inverse of $P_n$. When uncertainty about the state $x_n$ is high, working with information filtering is considered easier as one can model infinite variance in stable way by setting $P_n^{-1}$ to zero. The true theoretical CRLB binds the value of the Kalman covariance from below:

\begin{align}
%Cov(\state_n- \hat{\state}_n) & \equiv \ex{(\state_n- \hat{\state}_n)(\state_n- \hat{\state}_n)^T} \\
Cov(x_n- \hat{x}_n) & \equiv \ex{(x_n - \hat{x}_n)(x_n- \hat{x}_n)^T} \\
& \succcurlyeq P_n 
\end{align}

Where the variance propagation for $P_n$ can be decoupled from state propagation for $x_n$, the Ricatti equations will yield $P_n$ in the absence of measurement data based on the design of the filter dynamics ($\Phi_n$), measurement model, and noise strength covariances.  If these equations are recursively updated with the true noise parameters, then one obtains a theoretical CRLB [XXX References]. In the case where state and variance propagation are coupled, one must obtain the theoretical CRLB by performing the recursion below with, additionally, the true state $x_n$ information. 

The key aspect of deriving the CRLB for Kalman filters on quantised sensor information is the introduction of an extra `Fisher information term' purely due to quantisation in \cite{karlsson2005}. These information equations are analagous to the Ricatti equations except for the addition of an extra Fisher information term, $I_{m=1, n}$, for a $m=1$ bit quantisation procedure. We show this by re-writing Eqs. (48-53) in \cite{karlsson2005} in our notation:
\begin{align}
P_{n+1}^{-1} &= Q_n^{-1} + I_{m=1, n+1} - S_n^T(P^{-1}_n + V_n)^{-1} S^T_n \label{eqn:app:CRLB_recursion} \\
S_n &\equiv -\Phi_n^T Q_n^{-1} \\
&\equiv -\Phi^T Q^{-1}, \quad \forall n \\
V_n & \equiv \Phi_n Q_n^{-1} \Phi_n^T\\
& \equiv \Phi Q^{-1} \Phi^T , \quad \forall n 
\end{align} 
For time invariant linear dynamical models and time invariance of noise covariance matrices, the recursion simplies further to yield time invariant intermediary matrices, $S, V$. This means that the only time varying term in the specified recursion is $I_{m=1, n+1}$. The time dependence is  $I_{m=1, n+1}$ arises due to a non-linear measurement action, namely, via $h(x_n)$. In particular, a non-linear measurement action often couples state propogation to variance propagation, and the same effect applies to the calculation of the Fisher Information term, $I_{m=1, n+1}$. The form of $I_{m=1, n+1}$ as given by Theorum 5 of \cite{karlsson2005}, where $I(x)$ defines the total Fisher information for $N$ observations in a measurement record:
\begin{align}
I(x) & \equiv \sum_{n=1}^{N} I_{m=1, n} \\
I_{m=1, n} & \equiv  H_n^T I_{m=1}(z_n) H_n  \\
I_{m=1}(z_n) & \equiv - \ex{\frac{\partial^2}{\partial z_n^2} \log Pr(d_n|z_n)} \label{eqn:app:quantisedFisherinfo}
\end{align}

The classical quantiser, $\mathcal{Q}$, defined as area-sampling of probability distributions, has a corresponding effect such that there is an analytical from for $Pr(d_n|z_n)$. This form is derived in Theorum 3 in \cite{karlsson2005} for $m=1$ quantisation procedure. We re-state the results from Theorum 3 in our notation as:

\begin{align}
Pr(d_n| z_n) & \equiv \delta(k) \rho(-z_n/ \sqrt{R}) + \delta(k-1) (1- \rho(-z_n/ \sqrt{R})), \quad k = 0, 1 \label{eqn:app:classicalquantiser}\\
\rho(-z_n/ \sqrt{R}) & \equiv \int_{-\infty}^{-z_n/ \sqrt{R}} \frac{1}{\sqrt{2\pi}} e^{-\frac{v^2}{2 R}} dv \\
\delta(k) &\equiv \begin{cases} 1, &k=0 \\ 0, &k \neq 0 \end{cases}
\end{align}
Here, $\rho(-z_n/ \sqrt{R})$ is the \emph{erf} function, namely the normalised cumulative probability distribution function for Gaussian distributed variables with finite limits. The derivation assumes that the error  between the true signal and its quantised valu, $v_n$, is zero mean, white Gaussian distributed with variance $R$. The derivation proceeds by stating that if $z_n + v_n < 0$, then we are much more likely to quantise $z_n$ in the bottom level corresponding to $z_n = -1/2, k = 0$. For the conditional probability where $z_n$ is given, this is the same as the probability of seeing a value of the error such that $v_n < -z_n$.  Hence, the probability is given by summing the the Gaussian probability distribution to a finite upper limit given by value of $-z_n$, resulting in the definition of $\rho(-z_n/ \sqrt{R})$ as the normalised \emph{erf} function.

The definitions of the classical framework and the QKF state space system completely define the CRLB for the classical quantisation procedure, if a box-quantisation scheme (also known as a mid-riser / mid-tread quantisation in engineering) is used instead of the coin-flip measurement action defined by \cref{eqn:app:coinflipquantiser}. 

If we had a classical quantisation procedure, we would have used the state space and CRLB equations above to plot CRLB for a completely equivalent system. The procedure is as follows. For each realisation of true $\state$, one computes the theoretical CRLB (using true $x$ information).  Next, one computes root mean square error of residuals from Kalman filtering with a noisy measurement record (using Kalman $\hat{x}$ estimates). For both quantities, one takes the expectation over multiple runs over an ensemble of different realisations of the true $\state$ and Kalman residuals are used to compute the root mean square error (RMSE) for each time step, $n$. The ensemble averaged square root of the trace of $P_n^{-1}$ obtained the CRLB recursion defines the CRLB when one wishes to compare this to the RSME for a particular choice of the Kalman Filter. The ratio of these quantities, RMSE/ CRLB, is always larger than unity, and a value of unity is optimal for unbiased state estimation.

\subsection{QKF: Departure from Classical Quantisation}

Our key point of departure from classical quantisation is therefore given by \cref{eqn:app:coinflipquantiser}. This corressponds to a departure in the calculation of $Pr(d_n | z_n)$ in $I_{m=1}(z_n)$ for the CRLB recursion. In particular, the arguments for $I_{m=1, n}$ follows from general arguments about the additivity of information under appropriately linearised maps, and the addition of $I_{m=1, n}$ into information filtering equations appears to be a general argument about the influence of the measurement action on variance propagation (in this case, quantised) \cite{karlsson2005}. Further, the fundamental interpretation of quantisation as area-sampling appears to be unchanged based on an explicit consideration of the key derivation in \cite{widrow1996} and commentary in \cite{karlsson2005}. Namely, that the convolution with a uniform distribution followed by multiplication with a pulse train does not \emph{explicitly} appear to be related to the calculation of  $Pr(d_n | z_n)$ using a Gaussian error model for $v_n$ in \cite{karlsson2005}. The properties of the pulse train and the uniform disribution are set by the level spacing between quantised amplitude levels, namely, the spacing given by $2^b$ in the amplitude domain, and $2^{-b}$ spacing in the probability (dual) domain. These are fixed by the choice of application (i.e. the continous-amplitude axis being quantised) and the number of bits $b$ (i.e. the $2m=2^b$ equally spaced levels for the new quantised-amplitude axis).  Implicitly, the procedure above may be subject to re-interpretation and we leave this as a question for subsequent work. 

At present, we consider the calculation of $Pr(d_n | z_n)$, with the understanding that  a coin flip measurement given by \cref{eqn:app:coinflipquantiser} is the only point of departure from the classical amplitude quantisation framework. This is the subject of the section below.

\subsection{Discussion: Calculating $Pr(d_n | z_n)$ for CRLB using Coin-Flip Quantiser}

In the language of the classical framework outlined in \cite{karlsson2005, widrow1996}, we propose that $d_n$ is the quantised signal (with area-sampled posterior distribution) with the true continous amplitude process being given by $\{z_n \}$. The true signal, $\{z_n\}$, is a sequence of stochastically drifting `biases' derived from repeated applications of the Born rule. Born's rule governs the naturally quantised outcomes of a qubit and we are able to incorporate additional apriori information about the measurement and quantisation process in our filtering algorithm. These are appropriately rescaled to be zero mean and symmetric around zero for numeric purposes, via the quantity $z_n$. In the ideal case, the quantity $z_n$ is used to set the bias of a coin before performing a single coin toss, namely the quantised action $\mathcal{Q}$.  One may question whether the value of $z$ is continuous between $n$ and $n+1$ - we assert that this follows from the slow, non-Markovian drift assumption about the underlying covariance stationary process, $\delta \omega (t)$.  

During the quantisation, there is uncertainty in our knowledge of $z_n$ as this is a theoretically unobservable quantity and exists only in the limit of infinite coin toss experiments (in the frequentist sense). Hence, the bias  of the binomial distribution in QKF is set by $\tilde{y}_n$. Just quantisation errors  from a classical procedure were considered in \cite{karlsson2005} and yielded a model for $Pr(d_n | z_n)$ based on $\rho(-z_n/ \sqrt{R})$,  we \emph{define} the quantities $Pr(d_n | z_n, v_n)$ and $ Pr( \tilde{y}_n | z_n)$ and we use an appropriate marginalisation over our error model for $v_n$ to yield $Pr(d_n | z_n)$.


The first term, $Pr(d_n | z_n, v_n)$ is obtained by considering repeated applications of the Born rule. For each independent time step, $n$, the Born rule gives us the likelihood function for obtaining an outcome:
\begin{align}
Pr(d_n=1 | f_n, t, \tau) & \equiv \cos^2(\frac{f_n}{2}) 
\end{align}
Note that is a likelihood function. A typical inference procedure is to consider a map between probability distributions using Bayes Rule. This is implemented under the Kalman framework naturally through the Kalman update equations. At present, we wish to define a map \emph{within} the Kalman measurement action, namely,a map between the probability distributions of continous ideal measurements, $z_n$, and quantised measurements, $d_n$. In this formalism, we interpret $Pr(d_n=k | f_n, t, \tau)$ as a likelihood providing probabilities of different observed outcomes, labelled by $k$, for $k \in \{0, 1\}$. We shift the likelihood function for $k=1$ to obtain a zero mean process $z_n$ as follows: 
\begin{align} 
Pr(d_n=k=1 | f_n, t, \tau) &\equiv Pr(1 | f_n, t, \tau)\\
 & \equiv \cos^2(\frac{f_n}{2}) \\
& = \frac{1}{2} +  \frac{1}{2}\cos(f_n) \\
& \equiv \frac{1}{2} +  z_n  \\
Pr(0 | f_n, t, \tau) & = 1 - Pr(1 | f_n, t, \tau)\\
& = \frac{1}{2} - z_n \label{eqn:app:bornrule:up}
\end{align}

In actuality, we use $\tilde{y}_n$ to set the bias of the binomial distribution for a single coin toss experiment,that is, our quantisation procedure reflects our uncertainty in the knowledge of the true, unobservable bias. This means that the distribution arising for coin toss experiments is obtained by $z_n \to \tilde{y}_n$ in \cref{eqn:app:bornrule:up}, and the calculation for all outcomes labelled by $k \in \{0, 1\}$ is:
\begin{align}
Pr(d_n=k | z_n, v_n) &\equiv  \delta(k-1) (\tilde{y}_n + \frac{1}{2})  +  \delta(k) ( \frac{1}{2} - \tilde{y}_n) \\
& =  \tilde{y}_n \left( \delta(k-1) - \delta(k) \right)  + \frac{1}{2}\left(\delta(k-1) + \delta(k) \right) \\
& = \tilde{y}_n \left( \delta(k-1) - \delta(k) \right)  + \frac{1}{2}
\end{align}
The distribution above is conditioned both on $z_n$ (corresponding to stochastic qubit phase $f_n$) and the uncertainty $v_n$, such that $\tilde{y} = z_n + v_n$ is given.  In the first line above, the first term corresponds to the probability of observing the $\ket{k}, k=1$, and the second term denotes the probability of observing the qubit in state $k=0$. In the next line, the term $\frac{1}{2}\left(\delta(k-1) + \delta(k) \right)$ always contributes a $1/2$ factor and this simplification results in the final expression. Physically, this means that a qubit under no dephasing noise remains on the equator of the Bloch sphere with no stochastic phase accumulation, and an equal probability of being in either state, $\ket{0}, \ket{1}$. This is exactly what one obtains, $Pr(d_n=k | z_n=0, v_n=0) = 0.5 \quad \forall k$ under zero dephasing. 

Next, we consider the error model for incorporating the uncertainity in the knowledge of the bias. The simplest construction is to re-apply the Gaussian assumption for $v_n$ under which typical Kalman filtering and estimations problems are conducted, but without the traditional interpretation that this $v_n$ corresponds to additive white Gaussian measurement noise:
\begin{align}
Pr( \tilde{y}_n | z_n) & \equiv Pr( z_n + v_n | z_n) \\
& \equiv Pr(v_n) \\
Pr(v_n)  & \sim \mathcal{N}(0, R)
\end{align}

This model for $v_n$ is an insufficient description for  the uncertainty in the our knowledge of true $z_n$. In particular, the value of $z$ (ideal) is bounded by the Born rule to be between $[-0.5, 0.5]$. We assume that $v_n$ is Gaussian distributed additive white noise with mean zero. However, Gaussian noise is not bounded, whereas we wish to saturate the values of $\tilde{y}_n$ between $[-0.5, 0.5]$. These saturation effects are commonly encountered in studies of classical quantisation and a discussion of appropriately saturating the values of random variable without effecting positivity of the underlying distribution is borrowed from \cite{widrow1996}. Namely, the techique we use is to convolve with a uniform distribution defined between the limits $(a,b)$, where $a=-0.5, b=0.5$. In this manner, $Pr( \tilde{y}_n | z_n) \to Pr( \tilde{y}_n | z_n) \star \mathcal{U}(a,b)$, where the $\star$ denotes a convolution. 

Implicitly, this means the addition of an abstract uniformly distributed random variable to $\tilde{y}_n$. This uniformly distributed random variable is referred to as `dithering noise', and the addition of dithering noise to a system involves designing noise in a way such that the inference problem is left uncorrupted. While the addition of dithering noise is valid in many signal processing applications, where one can access the system before quantisation, qubit outcomes are naturally quantised. As yet, we do not assign a physical interpretation of our abstract dithering noise and we use a convolution with uniform distribution merely as an analytical  technique below to appropriately saturate a probability distribution for $v_n$ without losing positivity. We note that some information is lost in the saturation procedure, and the extent to which this approach modifies the inference problem for the ideal Kalman measurement, $z_n$, based on quantised observed values $d_n$, remains an open question for future work.  This convolution is computed below (with $\tau$ a dummy variable not to be confused with the Ramsey time earlier):
\begin{align}
Pr( \tilde{y}_n | z_n) \star \mathcal{U}(a,b) & =  Pr( v_n) \star \mathcal{U}(a,b) \\
&= \frac{1}{\sqrt{2\pi R}} \left( \frac{1}{b-a} \right)\int_{a}^{b} e^{\frac{-(v_n - \tau)^2}{2R}} d\tau \\
&= \frac{1}{\sqrt{2\pi R}} \left( \frac{1}{b-a} \right) \left( \int_{0}^{b} e^{\frac{-(v_n - \tau)^2}{2R}} d\tau - \int_{0}^{a} e^{\frac{-(v_n - \tau)^2}{2R}} d\tau \right) \\
&= \frac{1}{\sqrt{2\pi R}} \left( \frac{1}{b-a} \right) \left( \frac{\pi}{\sqrt{2}} \right) \left( erf(b - v_n) - erf(a - v_n)\right) 
\end{align}


We now use these calculations and obtain an expression for $Pr(d_n | z_n)$ under a coin flip measurement action. To do this, we marginalising over all values of $v_n$ as follows:
\begin{align}
Pr(d_n=k | z_n) & \equiv  \int_{-\infty}^{\infty} dv_n \quad  Pr(d_n=k | z_n, v_n) \left( Pr( \tilde{y}_n = z_n + v_n| z_n) \star \mathcal{U}(a,b) \right)  \\
 & \equiv  \int_{-\infty}^{\infty} dv_n \quad  Pr(d_n=k | z_n, v_n) \left( Pr( v_n) \star \mathcal{U}(a,b) \right)  \\
&=  \frac{1}{\sqrt{2\pi R}} \left( \frac{1}{b-a} \right) \left( \frac{\pi}{\sqrt{2}} \right) \int_{a}^{b} dv_n \quad  Pr(d_n=k | z_n, v_n) \left( \left( erf(b - v_n) - erf(a - v_n)\right)  \right) \\
&=  \frac{1}{2\sqrt{2\pi R}} \left( \frac{1}{b-a} \right) \left( \frac{\pi}{\sqrt{2}} \right) \int_{a}^{b} dv_n \quad  \left( \left( erf(b - v_n) - erf(a - v_n)\right)  \right) \nonumber \\
&+  \frac{\left( \delta(k-1) - \delta(k) \right)}{\sqrt{2\pi R}(b-a)}  \left( \frac{\pi}{\sqrt{2}} \right) \int_{a}^{b} dv_n \quad  \tilde{y}_n \left( \left( erf(b - v_n) - erf(a - v_n)\right)  \right) \label{eqn:app:proposedquantiser:1}\\
&=  \frac{1}{2\sqrt{2\pi R}} \left( \frac{1}{b-a} \right) \left( \frac{\pi}{\sqrt{2}} \right) \int_{a}^{b} dv_n \quad  \left( \left( erf(b - v_n) - erf(a - v_n)\right)  \right) \nonumber \\
&+  \frac{\left( \delta(k-1) - \delta(k) \right)}{\sqrt{2\pi R}(b-a)}  \left( \frac{\pi}{\sqrt{2}} \right) \int_{a}^{b} dv_n \quad  (z_n + v_n) \left( \left( erf(b - v_n) - erf(a - v_n)\right)  \right) \label{eqn:app:proposedquantiser:2}\\
&=  \frac{\frac{1}{2}+ z_n \left( \delta(k-1) - \delta(k) \right)}{\sqrt{2\pi R}(b-a)}  \left( \frac{\pi}{\sqrt{2}} \right) \int_{a}^{b} dv_n \quad  \left( \left( erf(b - v_n) - erf(a - v_n)\right)  \right) \nonumber \\
&+  \frac{\left( \delta(k-1) - \delta(k) \right)}{\sqrt{2\pi R}(b-a)}  \left( \frac{\pi}{\sqrt{2}} \right) \int_{a}^{b} dv_n \quad  v_n \left( \left( erf(b - v_n) - erf(a - v_n)\right)  \right) \label{eqn:app:proposedquantiser:3}
\end{align}
The first line of \cref{eqn:app:proposedquantiser:1} describes a contribution that exists irrespective of the value of $z_n$ or the outcome $k$. The second line of \cref{eqn:app:proposedquantiser:1} qubit behaviour in the presence of dephasing noise and it is distribution over the two possible outcomes that can be observed, namely, $k \in \{ 0, 1\}$. In \cref{eqn:app:proposedquantiser:2}, we expand $\tilde{y}_n$ in terms of $z_n, v_n$. Since $v_n$ is serially uncorrelated and independent of $z_n$, it is safe to treat $z_n$ outside the integral [CHECK]. This yields \cref{{eqn:app:proposedquantiser:3}} where terms have been regrouped in terms of factors preceeding the integrals. The output of each integral is a number and the value of this number does not depend on the outcome under consideration, $k$, or the conditioning random variable, $z_n$. Hence, we can hide the various constants and obtain a simpler form:
\begin{align}
Pr(d_n=k | z_n) & = \rho_0 \frac{1}{2} + (\rho_0 z_n + \rho_1)\left( \delta(k-1) - \delta(k) \right)  \label{eqn:app:proposedquantiser:4:main}\\ 
\rho_0 & \equiv  \frac{1}{\sqrt{2\pi R}} \left( \frac{1}{b-a} \right) \left( \frac{\pi}{\sqrt{2}} \right) \int_{a}^{b} dv_n \quad  \left( \left( erf(b - v_n) - erf(a - v_n)\right)  \right) \label{eqn:app:proposedquantiser:5}\\
\rho_1 &  \equiv \frac{1}{\sqrt{2\pi R}} \left( \frac{1}{b-a} \right) \left( \frac{\pi}{\sqrt{2}} \right)  \int_{a}^{b} dv_n \quad  v_n \left( \left( erf(b - v_n) - erf(a - v_n)\right)  \right) \label{eqn:app:proposedquantiser:6} 
\end{align}

The expressions given by \cref{eqn:app:proposedquantiser:4:main,eqn:app:proposedquantiser:5,eqn:app:proposedquantiser:6} (with $a= -0.5, b=0.5$) is the final result for this derivation and the proposed quantity to substitute into \cref{eqn:app:quantisedFisherinfo} for the calculation of a Fisher Information corresponding to a quantiser given a coin flip action. To reiterate, our present understanding is that the rest of the classical formalism can be implemented without modification. 

We compare \cref{eqn:app:proposedquantiser:4:main} to the classical counterpart given by  \cref{eqn:app:classicalquantiser}. While there are some structural similarties, the marginalisation procedure over $v_n$  yields essentially two numbers and the time dependence of this term arises from the dependence of $z_n$. The next step - the calculation of the Fisher information via \cref{eqn:app:quantisedFisherinfo} - is UNKNOWN OMG AND COULD ALL BE A LIE .

\subsection{Discussion: Quantised Fisher Information Term for QKF}

The calculation of the Fisher information term for quantised measurements, $I_{m=1, n}$, enables one to calculate the CRLB for the QKF with a coin-flip measurement action. We now calculate and plot this term. Our objective is to compare the classical quantisation CRLB recursion results with a CRLB from a coin-flip quantisation. We wish to sense-check the derivations and assumptions above; and attempt to gain a theoretical insight into the numerical results reported for the QKF. 

To proceed, we expand the argument of \cref{eqn:app:quantisedFisherinfo} using the quotient rule, and we expand the expectation value by summing over all possible values of $k \in \{0,1 \}$, as in \cite{karlsson2005}. 

\begin{align}
\frac{\partial}{\partial z_n} \log Pr(d_n|z_n) & \equiv \frac{1}{Pr(d_n|z_n)}  \frac{\partial}{\partial z_n} Pr(d_n|z_n) \\ 
\frac{\partial^2}{\partial z_n^2} \log Pr(d_n|z_n) & = \frac{Pr(d_n|z_n)  \frac{\partial^2}{\partial z_n^2}Pr(d_n|z_n) - \left(  \frac{\partial}{\partial z_n} Pr(d_n|z_n) \right)^2 }{Pr(d_n|z_n)^2} \\
I_{m=1}(z_n) & \equiv - \ex{\frac{\partial^2}{\partial z_n^2} \log Pr(d_n|z_n)} \\
& = - \sum_{k\in \{0, 1\}}  Pr(d_n=k|z_n)  \frac{\partial^2}{\partial z_n^2} \log Pr(d_n=k|z_n) \\
& = - \sum_{k\in \{0, 1\}}  Pr(d_n=k|z_n)  \frac{Pr(d_n=k|z_n)  \frac{\partial^2}{\partial z_n^2}Pr(d_n=k|z_n) - \left(  \frac{\partial}{\partial z_n} Pr(d_n=k|z_n) \right)^2 }{Pr(d_n=k|z_n)^2}\\
& = - \sum_{k\in \{0, 1\}}  \frac{\partial^2}{\partial z_n^2}Pr(d_n=k|z_n) - \frac{\left(  \frac{\partial}{\partial z_n} Pr(d_n=k|z_n) \right)^2 }{Pr(d_n=k|z_n)}
\end{align}

We now find expressions for the calculation of the partial derivatives with respect to $z_n$: 
\begin{align}
Pr(d_n=k|z_n) 
& = \rho_0 \frac{1}{2} + (\rho_0 z_n + \rho_1)\left( \delta(k-1) - \delta(k) \right) \\
\frac{\partial}{\partial z_n} Pr(d_n=k|z_n) & = \rho_0 \left( \delta(k-1) - \delta(k) \right)\\
\frac{\partial^2}{\partial z_n^2}Pr(d_n=k|z_n) & = 0
\end{align}

Substituting this value into the Fisher information term yields:

\begin{align}
I_{m=1}(z_n)  & = - \sum_{k\in \{0, 1\}}  0 - \frac{\left( \rho_0 \left( \delta(k-1) - \delta(k) \right) \right)^2 }{ \frac{\rho_0}{2} + (\rho_0 z_n + \rho_1)\left( \delta(k-1) - \delta(k) \right)} \\
& = \sum_{k\in \{0, 1\}} \frac{\rho_0^2}{ \frac{\rho_0}{2} + (\rho_0 z_n + \rho_1)\left( \delta(k-1) - \delta(k) \right)} \label{eqn:app:applied_fisherinfo:0}
\end{align}

The result follows after observing that $\left( \delta(k-1) - \delta(k) \right)$ yields $\pm 1 $ for $k \in \{ 0,1 \}$, and hence, $\left( \delta(k-1) - \delta(k) \right)^2 \equiv 1, \quad \forall k$. 

We sense this derivation by computing the coefficients $\rho_0, \rho_1$ for $a=-0.5, b = 0.5$ and substituting into \cref{eqn:app:proposedquantiser:5,eqn:app:proposedquantiser:6}:

\begin{align}
 \int_{a}^{b} dv_n \quad  \left( \left( erf(b - v_n) - erf(a - v_n)\right)  \right) & \approx 0.97213 \\
 \int_{a}^{b} dv_n \quad  v_n \left( \left( erf(b - v_n) - erf(a - v_n)\right)  \right) & = 0 \\
\implies \rho_0 & \approx \frac{0.97213 \sqrt{\pi}}{2\sqrt{R}} = \frac{0.861528}{\sqrt{R}} \label{eqn:app:applied_fisherinfo:1}\\
\implies \rho_1 &  = 0  \label{eqn:app:applied_fisherinfo:2}
\end{align}

The first integral suggests that approximately $ 0.97213 $ of the information is retained following the saturation of $v_n$ between allowed values of $a, b$. The second integral will always be zero for any symmetric choices of $|a|=b$, and this is necessarily $0.5$ in our case for the interpretation of $z_n + v_n$ as a probability of success in a coin toss experiment. Hence, for our application, the second term will always be zero. 

Substituting \cref{eqn:app:applied_fisherinfo:1,eqn:app:applied_fisherinfo:2} into \cref{eqn:app:applied_fisherinfo:0} yields the Fisher information in simplified approximate form as:
\begin{align}
I_{m=1}(z_n) & = \sum_{k\in \{0, 1\}} \frac{\rho_0}{ \frac{1}{2} + z_n \left( \delta(k-1) - \delta(k) \right)}\\
& = \frac{0.861528}{\sqrt{R}} \sum_{k\in \{0, 1\}} \frac{1}{ \frac{1}{2} + z_n \left( \delta(k-1) - \delta(k) \right)} \\
& = \frac{0.861528}{\sqrt{R}} \left( \frac{1}{ \frac{1}{2} + z_n} + \frac{1}{ \frac{1}{2} - z_n} \right) \\
& = \frac{0.861528}{\sqrt{R}} \frac{4}{ 1 - 4z_n^2} \label{eqn:app:applied_fisherinfo:final}
\end{align}
We see that the Fisher information should have units of $[z^-2]$ (inverse variance) and this is true since $\sqrt{R}$ is inherited as a unitless scaling factor from the Gaussian distributed error model for $v_n$. In the absence of dephasing noise, $z_n=0$, and the Fisher information contribution for $k=0$ or $k=1$ to the total sum does not depend on the measurement outcome $k$. The physical interpretation is that the qubit has an equal probability of being in both $k$ states in the absence of dephasing noise, and hence, the associated information contributions are indistinguishable. Lasltly, we sense check this term with the classical case in \cite{karlsson2005} given by \cref{eqn:app:classicalquantiser}. We observe that both $I_{m=1}(z_n)$ terms take as input parameters the value of $z_n$ at each time step, and the scaling of this term is proportional to the noise covariance strength $\propto 1/\sqrt{R}$. No other inputs are required for computation of this term during the recursion for the CRLB, and hence, there is structural agreement between the classical and coin-flip versions of the quantisation procedure. However, this formula diverges at the boundaries, namely at $|z_n| \equiv 0.5$

Lastly, we consider the zero dephasing limit by revisting $Pr(d_n=k | z_n)$  and  we check whether \cref{eqn:app:proposedquantiser:1} provides a sensible value for $|a|=b=0.5$. In the event that a qubit experiences zero dephasing, $z_n=0$, at $n$, we physically expect \cref{eqn:app:proposedquantiser:1} to be precisely $1/2$ for any choice of $k$. However, our description of the error model (namely, $v_n \neq 0$ even if $z_n=0$) will create an artefact given by:
\begin{align}
Pr(d_n=k | z_n=0) & = \left( \frac{0.97213}{2} \right) \left( \frac{\sqrt{\pi}}{2\sqrt{R}} \right) \\
\end{align}
The surviving constants on the first line of \cref{eqn:app:proposedquantiser:1} are depicted above. Tthe second line of  \cref{eqn:app:proposedquantiser:1} disappears entirely for $z_n=0$ and $\rho_1 \equiv 0$. Of the surving terms, we interpret the first bracket as $\frac{0.97213}{2} \to 1/2$ if our uncertainty in the bias diappears, namely, the distribution for $v_n$ was not saturated between $[-0.5, 0.5]$. The term inside the second bracket are scaling factors that arise from our error model, $v_n$. Since $\sqrt{\pi} / 2 \approx 1.77 /2 $ which is a number slightly below unity, we see that the net artefact created by our description depends on the modelled strength of the uncertainty in our knowledge of the bias, $R$. When non-Markovian drift is present, the tuning of this abstract term, $R$, during the training of the QKF will ensure physically interpretable results. 

If, however, no non-Markovian drift exists, and the Kalman tuning procedure for QKF fails to converge on $R\approx \frac{\pi \times 0.97213^2}{4} \approx 0.74$ such that $Pr(d_n=k | z_n=0) \approx 0.5$, then one risks introducing a systematic bias given our description. In this context, one may wish to develop or test a tuning procedure for a physical QKF  implementation for an engineered case where no dephasing drift exists, and a strong rationale is needed for an optimiser to move away from $R \neq 0.74$.

\subsection{Classical CRLB vs. Coin Flip CRLB in Numerical Results Fig(8a)}

We plot and compare the CRLB given by the recursion equation in \cref{eqn:app:CRLB_recursion} for the results presented in Fig. 8(a) for the QKF in the main text. This case corresponds to a perfectly learned dynamical model and noise parameters, such that the effect of the non-linear quantised measurement action is being tested for a non-Markovian, covariance stationary $\state'$. The oversampling regime is slowly reduced and we see a report reduction in prediction horizon in the main text. 

For the true $\state'$, we calculate the CRLB using a Fisher information term for coin-flip measurements, \cref{eqn:app:applied_fisherinfo:final}, and the classical counterpart using the Fisher information term for quantised measurements given by \cref{eqn:app:classicalquantiser}. In all cases, we discard data for the first $100$ points, as the AKF model needs to accumulate enough measurements before dynamic updates can begin sensibly. We the expectation values of the Kalman residuals [not shown] lie above the CRLB by 2 orders of magnitude.  

In \cref{fig:app:CRLB}, we plot the ratio of the classical CRLB / coin-flip CRLB, where a ratio of unity implies that both calculations are approximately identical in the mean square limit. There are three key observations: firstly, the coin flip CRLB always lies below the classical counterpart for the cases considered, namely, the ratio is greater than unity for all cases. Secondly, the ratio is constant, which means that both CRLB approximately behave the same with $n$ for the cases considered. Thirdly, the ratio tends to unity as oversampling reduces and QKF predictive performance decreases. The applicability of these findings to general implementations remains an open question. 

\begin{figure}[h!]
    \includegraphics[scale=1]{CRLB} 
    \caption{ \label{fig:app:CRLB} TBC}
\end{figure}
\FloatBarrier
\newpage
% ##############################################################################
\section{Experimental Verification Procedure \label{sec:app:exptverfication}}
% ##############################################################################

We show that a stochastic detuning is indistingushable from the derivative of phase noise in a carrier enabling experimental verification of algorithms in this manuscript.
This is considered in \cite{soare2014} and we provide an alternative description that enables direct access to equation of motion for qubit probability amplitudes during the application of engineered carrier phase noise. 

We define system and interaction Hamiltonians for a two level system with energy splitting corresponding to $\omega_A$  interacting with a magnetic field. 
\\
\begin{align}
\op{\mathcal{H}}_0 &= \frac{1}{2} \hbar \omega_A \p{z} \\
\tilde{g} & \equiv \vec{g} \cdot \op{z} = \bra{1} \op{d} \cdot \op{z} \ket{2} \\
\op{\mathcal{H}}_{AF} & = -\tilde{g} \Omega(t)  \cos(\omega_\mu t + \phi(t)) \p{-} \\
& - \tilde{g}^* \Omega(t)  \cos(\omega_\mu t + \phi(t)) \p{+}
\end{align} Here, $\op{\mathcal{H}}_0$ is the system Hamiltonian, $\op{d}$ is the dipole operator for a spin half particle in a magnetic field, $\vec{g}$ is a complex system-field coupling term that inherits the direction of the dipole operator, $\tilde{g}$ is a complex scalar that captures the coupling strength, and $\op{\mathcal{H}}_{AF}$ is the Schoedinger picture atom-field interaction Hamiltonian in the rotating wave approximation.

The carrier with noise is identically defined as in \cite{soare2014}:
\begin{align}
\vec{B} &\equiv \Omega(t) \cos(\omega_\mu t + \phi(t)) \op{z} \\
\phi(t) & \equiv \phi_C(t) + \phi_N(t)
\end{align}
with a real control amplitude $\Omega(t)$, carrier frequency $\omega_\mu$, controlled phase $\phi_C(t)$ and stochastic phase noise $\phi_N(t)$.  It is possible to add amplitude noise to the control, i.e. $\Omega(t) \equiv \Omega_C(t) + \Omega_N(t)$, however we set $\Omega_N(t) =0$ at present. 

Substituting $\op{\mathcal{H}}_{N}$ for a stochastic detuning, $\delta \omega (t)$ from \cref{sec:app:setup}, the equations of motion for probability amplitudes of our two state system under dephasing, for arbitrary $\ket{\psi} = c_1 \ket{1} + c_2\ket{2} $ are:
\begin{align}
\dr{\pjs{1}{\psi}} &\equiv \dot{c_1} \\
& = \frac{-i}{\hbar}\bra{1} \op{\mathcal{H}}_0 + \op{\mathcal{H}}_{N} + \op{\mathcal{H}}_{AF}^{RWA}  \ket{\psi} \\
& = \frac{i(\omega_A - \delta\omega(t)) }{2}c_1 - \frac{i}{\hbar}Z_0\dd e^{i\omega_\mu t}c_2\\
\dr{\pjs{2}{\psi}} &\equiv \dot{c_2} \\
& = \frac{-i}{\hbar}\bra{2} \op{\mathcal{H}}_0 + \op{\mathcal{H}}_{N} + \op{\mathcal{H}}_{AF}^{RWA}  \ket{\psi} \\
& = -\frac{i(\omega_A - \delta\omega(t)) }{2}c_2 - \frac{i}{\hbar}Z_0^*\dd e^{-i\omega_\mu t}c_1 \\
\quad Z_0\dd &\equiv \frac{\Omega\dd\tilde{g}}{2}e^{i\phi\dd}.
\end{align}

The first interaction picture transformation is with respect to the carrier, namely, we define an arbitrary transformation as:
\begin{align}
\tilde{c}_1 &= c_1e^{i\lambda t} \\
\tilde{c}_2 &= c_2e^{-i\lambda t}
\end{align}
The equations of motion transform as follows:
\begin{align}
\dot{\tilde{c}}_1 & = \dot{c}_1 e^{i \lambda t} + i \lambda \dot{\tilde{c}}_1 \\
& =  i(\frac{\omega_A - \delta\omega(t)}{2} + \lambda)\tilde{c}_1 - \frac{i}{\hbar}Z_0\dd e^{i(\omega_\mu +2\lambda)t}\tilde{c}_2\\
& \nonumber \\
\dot{\tilde{c}}_2 & = \dot{c}_2 e^{-i \lambda t} - i \lambda \dot{\tilde{c}}_2 \\
& =  -i(\frac{\omega_A - \delta\omega(t)}{2} + \lambda)\tilde{c}_2- \frac{i}{\hbar}Z_0^*\dd e^{-i(\omega_\mu +2\lambda)t}\tilde{c}_1 \\
\end{align}
We set $2\lambda \equiv -\omega_\mu$ and $\Delta \omega(t) \equiv \omega_A - \delta\omega(t) -\omega_\mu$ to remove the time dependence due to the carrier, resulting in:
\begin{align}
\dot{\tilde{c}}_1 & =  \frac{i\Delta \omega(t)}{2}\tilde{c}_1 - \frac{i}{\hbar}Z_0\dd \tilde{c}_2\\
& \nonumber \\
\dot{\tilde{c}}_2 & =  -\frac{i\Delta \omega(t) }{2}\tilde{c}_2- \frac{i}{\hbar}Z_0^*\dd \tilde{c}_1\\
 \Rightarrow \op{\mathcal{H}}_{\omega_\mu}^{I} &\equiv \frac{\hbar \Delta \omega(t)}{2}\p{z} + Z_0\dd\p{-} + Z_0^*\dd\p{+}.
\end{align}
In the last line, $\op{\mathcal{H}}_{\omega_\mu}^{I}$ is the effective interaction picture Hamiltonian acting on the transformed state $\ket{\psi} = \tilde{c}_1\ket{1} + \tilde{c}_2\ket{2}$ with transformed field amplitudes $Z_0\dd$.

The second interaction picture transformation is with respect to a classical phase noise field. While only attainable in engineered phase noise demonstrations during experiment, the resulting interaction picture Hamiltonian reveals the indistinguisbaility of engineered phase noise and environmental dephasing, as asserted in \cite{soare2014}. We define an arbitrary transformation with respect to a time-varying classical quantity $\lambda (t)$:
\begin{align}
\alpha_1 &= \tilde{c}_1e^{i\lambda\dd} \\
\alpha_2 &= \tilde{c}_2e^{-i\lambda\dd}
\end{align}
The equations of motion transform as follows:
\begin{align}
\dot{\alpha}_1 & = \dot{\tilde{c}}_1 e^{i \lambda\dd} + i \dot{\lambda}\dd \alpha_1 \nonumber \\
& =  i(\frac{\Delta \omega(t)}{2} + \dot{\lambda}\dd)\alpha_1 - \frac{i}{\hbar}Z_0\dd e^{i2\lambda\dd}\alpha_2 \\
& \nonumber \\
\dot{\alpha}_2 & = \dot{\tilde{c}}_2 e^{-i \lambda\dd} - i \dot{\lambda}\dd\dot{\alpha}_2 \nonumber \\
& =  -i(\frac{\Delta \omega(t)}{2} + \dot{\lambda}\dd)\alpha_2- \frac{i}{\hbar}Z_0^*\dd e^{-i2\lambda\dd}\alpha_1
\end{align}
We substitute $Z_0\dd$ and set $2\lambda\dd \equiv -\phi_N\dd$ resulting in:

%%%%% [SECTION BELOW IS CHANGED FROM ORIGINAL NOTES] 
\begin{align}
\dot{\alpha}_1 & =  i\frac{\Delta \omega(t) - \dot{\phi}_N \dd}{2}\alpha_1 \nonumber \\ %%%% ADDED PHI_N
& - \frac{i}{2\hbar}(\Omega\dd \tilde{g} e^{i\phi_C\dd}) \alpha_2  \\
\dot{\alpha}_2 & =  -i\frac{\Delta \omega(t) - \dot{\phi}_N\dd}{2}\alpha_2 \nonumber  \\ %%%% ADDED PHI_N
& - \frac{i}{2\hbar}(\Omega\dd \tilde{g}^* e^{-i\phi_C\dd})\alpha_1  \\
\Rightarrow \op{\mathcal{H}}^{I}_{\omega_\mu,\phi_N\dd} &\equiv \frac{\hbar}{2}(\Delta \omega(t)- \dot{\phi}_N\dd)\p{z} \nonumber \\
& + \frac{\Omega\dd}{2} (\tilde{g} e^{i\phi_C\dd}\p{-} + \tilde{g}^* e^{-i\phi_C\dd}\p{+}) \label{eqn:Hi}
\end{align}
In the last line, $\op{\mathcal{H}}^{I}_{\omega_\mu,\phi_N\dd}$ is the effective interaction picture Hamiltonian acting on the transformed state $\ket{\psi} = \alpha_1\ket{1} + \alpha_2\ket{2}$. For the case where we set $\tilde{g}^* = \tilde{g} \equiv 1$ (real coupling constant), we recover the effective interaction picture Hamiltonian in \cite{soare2014}.


\newpage
% ##############################################################################
\section{Derivation of LKFFB (alternative to \cite{livska2007})} \label{sec:ap_liska_deriv}
% ##############################################################################
The Liska Kalman Filter has adaptive noise matrix, $Q$. We ensure that the standard classical  Kalman Filter derivations are unchanged by the presence of the proposed $Q$ in \cite{livska2007}. To derive the Kalman algorithm, we adapt the approach outlined in \cite{grewal2001theory} for standard applications. 

To begin, we search for a linear predictor $\hat{x}_n(+)$ defined by the unknown weights, $\lambda_n, \gamma_n$, at each time-step $n$:
\begin{align}
\hat{U} & \equiv \{y_1, y_2,\dots, y_n\} \\
\hat{x}_n(+) & \equiv \lambda_n \hat{x}_n(-) + \gamma_n y_n \label{eqn:KF_predictor}\\
\hat{x}_n(-) & \equiv \Phi_{n-1} \hat{x}_{n-1}(+) \\
(+) &\equiv \text{Aposteriori state estimate at $n$}\\
(-) &\equiv \text{Apriori state estimate at $n$}
\end{align}
To find unknowns $\lambda_n, \gamma_n$, we impose orthogonality of estimator to the set of all known data by invoking the general linear prediction theorum for covariance stationary processes, as per standard textbooks, (e.g. \cite{grewal2001theory,karlin2012first}). This means that the following must be satisfied:
\begin{align}
\ex{(x_n - \hat{x}_n(+))U^T} &= 0 \label{eqn:KF_reln_1}\\
\implies \ex{(x_n - \hat{x}_n(+))y_i^T} &= 0, \quad i = 1,\dots, n-1 \label{eqn:KF_reln_2}\\
\implies \ex{(x_n - \hat{x}_n(+))y_n^T} &= 0 \label{eqn:KF_reln_3}
\end{align}
First, we state the following properties of process and measurement noise are true:
With $n,m$ denoting time indices, the properties of noise processes are:
\begin{align}
\ex{w_n} & = \ex{v_n}= 0 \quad \forall n \label{eqn:KF_stat_noisemean}\\
\ex{w_nw_m^T} &= \sigma^2 \delta(m-n),  \quad \forall n,m \label{eqn:KF_stat_noisepros}\\
\ex{v_n v_m^T}&= R\delta(m-n), \quad  \forall n,m  \label{eqn:KF_stat_noisemsmt} \\
\ex{w_n v_m} &= 0,  \quad \forall n,m \label{eqn:KF_stat_crosscorr}
\end{align} 
The process noise variance, $\sigma^2$ is a known scalar. The measurement noise variance $R$ is a known covariance matrix in general, but will turn out to be a scalar for our choice of measurement action $H$. This means we can interpret \ref{eqn:KF_stat_crosscorr} as product between scalars. Based on noise properties and the state space models defined in the main text, we conclude that the following correlation relations are true:
\begin{align}
\ex{x_{n-1}w_{n-1}} & = 0 \label{eqn:KF_reln_4}\\
\ex{w_{n-1} y_i} &= 0, \quad i = 1,2,\dots n-1 \label{eqn:KF_reln_5}\\
\ex{\frac{x_{n-1}}{\norm{x_{n-1}}}w_{n-1}} & = 0 \label{eqn:KF_reln_6}\\
\ex{x_{n-1} w_{n-1} x_{n-1}^T } & = 0 \label{eqn:KF_reln_7}\\
\ex{\frac{x_{n-1}}{\norm{x_{n-1}}} w_{n-1} x_{n-1}^T } & = 0 \label{eqn:KF_reln_8} \\
\ex{x_{n-1} w_{n-1} y_i } & = 0, \quad i = 1,\dots, n-1 \label{eqn:KF_reln_9}\\
\ex{\frac{x_{n-1}}{\norm{x_{n-1}}} w_{n-1} y_i } & = 0, \quad i = 1,\dots, n-1 \label{eqn:KF_reln_10}
\end{align} We physically interpret these relations by observing that $x_{n-1}$ or $y_i, i = 1,2,\dots n-1$ do not contain a process noise term of the form $w_{n-1}$. At most, the expansion of $x_{n-1}, y_{n-1}$ terms contain $w_{n-2}$. Where terms of the form $w_{n-k}\dots w_{n-2}w_{n-1}$ appear, we invoke \ref{eqn:KF_stat_noisepros} and set these terms to zero. Where terms of the form $\propto w_{n-2}w_{n-2}w_{n-1}$ appear, we invoke uncorrelated process noise and zero mean process noise and set these terms to zero. The $\norm{x_{n-1}}$ term depends on $x_{n-1}$, and we assert that a similar logic holds where physically, a normed state cannot be correlated with a future process noise term. The set of correlation relations thus far imply that:
\begin{align}
\ex{x_n} &= \Phi_{n-1}\ex{x_{n-1}} \label{eqn:KF_reln_11}\\
\ex{(x_n - \apx{n})x_0^T} &= 0 \label{eqn:KF_reln_12}\\
\ex{(x_n - \apx{n})\hat{y}_n(-)^T} &=0 \label{eqn:KF_reln_13} \\
\ex{(x_n - \amx{n})y_i^T} &=0, \quad i = 1,\dots, n-1\label{eqn:KF_reln_14}\\
\ex{\Gamma_n w_n \apx{n}^T} &=0 \label{eqn:KF_reln_16}\\
\ex{\Gamma_n w_n x_n^T} &=0 \label{eqn:KF_reln_17} 
\end{align}
These relations are used to find unknown weights, $\lambda_n$ and $\gamma_n$. In particular, satisfying an orthogonality condition (\cref{eqn:KF_reln_2}) at different time steps yields the former, and satisfying an orthogonality condition at same time steps (\cref{eqn:KF_reln_3}) yields the latter (Kalman gain). 

First, we find $\lambda_n$ using \cref{eqn:KF_stat_noisemsmt,eqn:KF_reln_10,eqn:KF_reln_11,eqn:KF_reln_2}:
\begin{align}
&\ex{(x_n - \apx{n})y_i^T} =0, \quad i = 1,\dots, n-1\label{eqn:KF_reln_15}\\ 
\implies 0 & = \ex{(x_n - \apx{n})y_i^T} \\
& = \ex{(\Phi_{n-1}x_{n-1}(\idn + \frac{w_{n-1}}{\norm{x_{n-1}}}) - \lambda_n \hat{x}_n(-) - \gamma_n y_n)y_i^T} \\
& = \Phi_{n-1}\ex{x_{n-1}y_i^T} + \Phi_{n-1}\ex{\frac{x_{n-1}w_{n-1}}{\norm{x_{n-1}}}y_i^T}  - \lambda_n \ex{ \hat{x}_n(-)y_i^T} - \gamma_n H_n \ex{ x_n y_i^T}  - \gamma_n \ex{ v_n y_i^T}  \\
&= \Phi_{n-1}\ex{x_{n-1}y_i^T} + \Phi_{n-1}\ex{\frac{x_{n-1}w_{n-1}}{\norm{x_{n-1}}}y_i^T} - \lambda_n \ex{ \hat{x}_n(-)y_i^T} - \gamma_n H_n\ex{ x_n y_i^T}, \quad \text{by \ref{eqn:KF_stat_noisemsmt} } \\
&= \ex{\Phi_{n-1}x_{n-1}y_i^T} - \lambda_n \ex{ \hat{x}_n(-)y_i^T} - \gamma_n H_n\ex{ x_n y_i^T}, \quad \text{by \ref{eqn:KF_reln_10} } \\
&=\ex{x_{n}y_i^T} - \lambda_n \ex{ \hat{x}_n(-)y_i^T} - \gamma_n H_n\ex{ x_n y_i^T}, \quad \text{by \ref{eqn:KF_reln_11}} \\
&= \ex{x_{n}y_i^T} + \lambda_n \ex{( x_n - \hat{x}_n(-))y_i^T} - \lambda_n \ex{ x_n y_i^T}  - \gamma_n H_n\ex{ x_n y_i^T} \label{eqn:app:LKFFB_addproxy}  \\
&= \ex{x_{n}y_i^T} - \lambda_n \ex{ x_n y_i^T} - \gamma_n H_n\ex{ x_n y_i^T}\\
& = \ex{(\idn - \lambda_n  - \gamma_n H_n)x_n y_i^T} \\
&\ex{x_n y_i^T} \neq 0 \implies \lambda_n = \idn  - \gamma_n H_n
\end{align} We add $\pm \lambda_n \ex{ x_n y_i^T}$ in \cref{eqn:app:LKFFB_addproxy} to obtain the final result. 
The discussion so far has not made explicit reference to properties of measurement noise. The calculation of the Kalman gain incorporates information about incoming noisy measurements, and we define:
\begin{align}
e_n & \equiv \amx{n} - x_n,   \label{eqn:app:LKFFB_resid}\\
\amp{n} & \equiv E[e_ne_n^T] \label{eqn:app:LKFFB_P}\\
R_n & \equiv E[v_nv_n^T] \label{eqn:app:LKFFB_R}\\
E[v_n e_n^T]&=0 \label{eqn:app:LKFFB_uncorr_resid}
\end{align}
The Kalman Gain $\gamma_n$ satisfies orthogonality conditions at the same time step:
\begin{align}
0 &= \ex{(x_n - \apx{n})(\hat{y}_n(-) - y_n)^T} \\ 
& = E \left[ (x -\lambda_n \amx{n} - \gamma_n H_n x_n - \gamma_n v_n) \right. \\
& \left. (H_n (\amx{n} - x_{k}) -v_n)^T\right]  \label{eqn:app:LKFFB_apriori_y}\\
&= E\left[(x_n - \amx{n} + \gamma_n H_n\amx{n} - \gamma_n H_n x_n - \gamma_n v_n)(H_n (\amx{n} - x_{k}) -v_n)^T\right]  \\
&= E\left[(x_n - \amx{n} + \gamma_n H_n\amx{n} - \gamma_n H_n x_n - \gamma_n v_n)(H_n (\amx{n} - x_{k}) -v_n)^T\right].  \\
&= E\left[(-e_n + \gamma_n H_ne_n - \gamma_n v_n)(H_n e_n -v_n)^T\right]  \\
&= E\left[(-e_ne_n^T H_n^T+ \gamma_n H_ne_ne_n^T H_n^T - \gamma_n v_ne_n^T H_n^T + e_nv_n^T - \gamma_n H_ne_nv_n^T + \gamma_n v_nv_n^T \right]  \\
&= -\amp{n} H_n^T+ \gamma_n H_n\amp{n}H_n^T + \gamma_n R_n \\
& \implies \gamma_n = \amp{n} H_n^T(H_n\amp{n}H_n^T + R_n)^{-1} \label{eqn:KF_update_gamma}
\end{align} 

We use \cref{eqn:KF_reln_3,eqn:KF_reln_13} to intiate the derivation. We expand terms according to the design of our apriori and aposterori predictors, as well as the measurement action in \cref{eqn:app:LKFFB_apriori_y}. Terms are regrouped such that we substitute the expression for the residuals defined in \cref{eqn:app:LKFFB_resid}, enabling a straightforward simplification using (\cref{eqn:app:LKFFB_resid,eqn:app:LKFFB_P,eqn:app:LKFFB_R,eqn:app:LKFFB_uncorr_resid}) to get to the final result. 

The first Kalman equation - the measurement update to the true Kalman state - is:
\begin{align}
\apx{n} &= (1 - \gamma_nH_n) \amx{n} + \gamma_ny_n \\
& =  \amx{n} + \gamma_n(y_n - H_n\amx{n}) \\
&= \amx{n} + \gamma_n (y_n - \hat{y}_n(-)) \label{eqn:KF_update_x+} 
\end{align}
 The second Kalman update equation - the  update to the uncertainty of the true Kalman state - is:
\begin{align}
\apx{n} &=  \amx{n} - \gamma_nH_n \amx{n} + \gamma_nH_n x_n + \gamma_nv_n  \\
e^{+}_n &\equiv \apx{n} - x_n \\
&=  \amx{n} - \gamma_nH_n \amx{n} + \gamma_nH_n x_n + \gamma_nv_n - x_n \nonumber \\
 &=  e_n - \gamma_nH_n e_n + \gamma_nv_n \\
& = \left[1 - \gamma_nH_n \right] e_n + \gamma_nv_n  \\
\app{n} &\equiv \ex{e_n^+ e_n^{+T}} \\
& = E \left[ \left[1 - \gamma_nH_n \right] e_n e_n^T \left[1 - \gamma_nH_n \right]^T + \gamma_nv_n e_n^T \left[1 - \gamma_nH_n \right]^T   +  \left[1 - \gamma_nH_n \right] e_n v_n^T \gamma^T + \gamma_nv_n v_n^T \gamma^T \right] \\
& = \left[1 - \gamma_n H_n \right] \amp{n}\left[1 - \gamma_n H_n \right]^T + \gamma_n R_n \gamma_n^T  \\
& =  \amp{n} - \amp{n}H_n^T\gamma_n^T - \gamma_n H_n \amp{n} + \gamma_n \left[ H_n \amp{n}H_n^T +  R_n \right] \gamma_n^T, \quad \text{using $\gamma_n$}\\
&= \amp{n} - \amp{n}H_n^T\gamma_n^T - \gamma_n H_n \amp{n} +  \amp{n}H_n^T \gamma_n^T  \\ 
&= \left[1  - \gamma_n H_n \right] \amp{n} \label{eqn:KF_update_p+}
\end{align}
While the true Kalman state is propagated in time via the dynamical model, we must also propagate the Kalman estimate of the uncertainity in the true state. Unlike a typical Kalman filter, true Kalman state estimation cannot be decoupled from state variance estimation due to $\Gamma_n$ in LKFFB. This means Kalman gains cannot be calculated in advance of data collection. We confirm below that Kalman uncertainty estimates can be propagated as:
\begin{align}
e_n &= \amx{n} - x_n \\
&= \Phi_{n-1}\left[\apx{n-1} - x_{n-1} \right] - \Gamma_{n-1}w_{n-1} \\
&= \Phi_{n-1}e^{+}_{n-1} - \Gamma_{n-1}w_{n-1} \\
\amp{n} &= E[e_ne_n^T] \\
&= \Phi_{n-1}E\left[e^{+}_{n-1}e^{+T}_{n-1}\right]\Phi_{n-1}^T  - \Phi_{n-1}E\left[e^{+}_{n-1}w_{n-1}^T\Gamma_{n-1}^T\right] - E\left[\Gamma_{n-1}w_{n-1}e^{+T}_{n-1}\right]\Phi_{n-1}^T  \\
& + E\left[\Gamma_{n-1}w_{n-1}w_{n-1}^T\Gamma_{n-1}^T\right] \\
&= \Phi_{n-1}E\left[\app{n-1}\right]\Phi_{n-1}^T  + E\left[\Gamma_{n-1}w_{n-1}w_{n-1}^T\Gamma_{n-1}^T\right] - \Phi_{n-1}E\left[(\apx{n-1} - x_{n-1})w_{n-1}^T\Gamma_{n-1}^T\right] \nonumber \\
& - E\left[\Gamma_{n-1}w_{n-1}(\apx{n-1} - x_{n-1})^T\right]\Phi_{n-1}^T  \nonumber  \\
&= \Phi_{n-1} \app{n-1} \Phi_{n-1}^T + E\left[\Gamma_{n-1}w_{n-1}w_{n-1}^T\Gamma_{n-1}^T\right], \label{eqn:KF_update_p-_0} \\
&= \Phi_{n-1} \app{n-1} \Phi_{n-1}^T + Q_{n-1}, \label{eqn:KF_update_p-} \\
Q_{n} & \equiv E\left[\Gamma_{n}w_{n}w_{n}^T\Gamma_{n}^T\right] 
\end{align}
We use \cref{eqn:KF_reln_16,eqn:KF_reln_17} to obtain \cref{eqn:KF_update_p-} in the last step. Hence, the standard Kalman predictor equations - \ref{eqn:KF_update_gamma}, \ref{eqn:KF_update_x+}, \ref{eqn:KF_update_p+}, \ref{eqn:KF_update_p-} - are valid for adaptive noise features given in \cite{livska2007}, as stated in the main text. 
\newpage
\subsection{Proofs \ref{eqn:KF_reln_4} - \ref{eqn:KF_reln_17}:}

\begin{align}
\ex{x_{n-1} w_{n-1}} & = \ex{\Phi_{n-2}x_{n-2}(\idn + \frac{w_{n-2}}{\norm{x_{n-2}}}) w_{n-1}} \\
& = \ex{\Phi_{n-2}x_{n-2}w_{n-1}} +\ex{ \frac{\Phi_{n-2}x_{n-2} w_{n-2}}{\norm{x_{n-2}}} w_{n-1}} \\
& = \ex{\Phi_{n-2}x_{n-2}w_{n-1}} \\
& = \ex{\Phi_{n-2}\Phi_{n-3}x_{n-3}(\idn + \frac{w_{n-3}}{\norm{x_{n-3}}})w_{n-1}} \\
& \vdots \\
& = \Phi_{n-2}\dots\Phi_{n-i+1}x_0\ex{w_{n-1}} \\
&= 0 \\
\nonumber \\ 
\ex{w_{n-1} y_i} &= \ex{w_{n-1}  H_{n-1} x_{n-1} + w_{n-1}v_{n-1}}  \\
 &= \ex{w_{n-1}  H_{n-1} x_{n-1}}, \quad\text{by \ref{eqn:KF_stat_crosscorr} } \\ 
 &= H_{n-1} \ex{w_{n-1} x_{n-1}} \\
 &= 0, \quad\text{by \ref{eqn:KF_reln_4}} \\
\nonumber \\ 
\ex{\frac{x_{n-1}}{\norm{x_{n-1}}}w_{n-1}} & = \ex{\frac{\Phi_{n-2}x_{n-2}(\idn + \frac{w_{n-2}}{\norm{x_{n-2}}})}{\norm{x_{n-1}}}w_{n-1}} \\
& = \ex{\frac{\Phi_{n-2}x_{n-2}w_{n-1} + \frac{w_{n-2}w_{n-1}}{\norm{x_{n-2}}}}{\norm{x_{n-1}}}} \\
& = \ex{\frac{\Phi_{n-2}x_{n-2}w_{n-1}}{\norm{x_{n-1}}} + \frac{w_{n-2}w_{n-1}}{\norm{x_{n-2}}\norm{x_{n-1}}}} \\
&\vdots \\
&= 0 \quad \text{since $x_{n-1, n-2, \dots}$, $\norm{x_{n-1,n-2, \dots}}$ cannot be correlated with future  noise, $w_{n-1}$.}\\
\nonumber \\ 
\ex{x_{n-1} w_{n-1} x_{n-1}^T} &= \ex{\Phi_{n-2}x_{n-2}(\idn + \frac{w_{n-2}}{\norm{x_{n-2}}}) w_{n-1} \Phi_{n-2}x_{n-2}(\idn + \frac{w_{n-2}}{\norm{x_{n-2}}})} \\
&= \ex{\Phi_{n-2}x_{n-2} w_{n-1} x_{n-2}^T\Phi_{n-2}^T(\idn + \frac{w_{n-2}}{\norm{x_{n-2}}})^2} \\
&= 0 \quad \text{since $x_{n-1, n-2, \dots}$, $\norm{x_{n-1,n-2, \dots}}$ cannot be correlated with future  noise, $w_{n-1}$.}
\end{align}
\ref{eqn:KF_reln_8} is justified identically to \ref{eqn:KF_reln_7}. Recognising that $y_i = H_i x_i + v_i$ and that $\ex{w_jv_i} = 0 \forall i,j$, we can justify \ref{eqn:KF_reln_9}, \ref{eqn:KF_reln_10} with the same reasoning as \ref{eqn:KF_reln_7}.
\begin{align}
\ex{x_n} &= \ex{\Phi_{n-1}x_{n-1}(\idn + \frac{w_{n-1}}{\norm{x_{n-1}}})} \\
&= \Phi_{n-1}\ex{x_{n-1}} + \Phi_{n-1}\ex{x_{n-1}\frac{w_{n-1}}{\norm{x_{n-1}}})} \\
&= \Phi_{n-1}\ex{x_{n-1}}, \quad \text{by \ref{eqn:KF_reln_6}} \\
\nonumber \\ 
\ex{(x_n - \apx{n})x_0} &= \ex{(x_n - \apx{n})y_1} \\
&= 0 \quad \text{by \ref{eqn:KF_reln_2}, for $i=1$.}
\end{align}
\begin{align}
\text{To prove } \ex{(x_n - \apx{n})\hat{y}_n(-)^T} &=0, \quad \text{note:} \\
\amx{n} & \equiv \Phi_{n-1} \apx{n-1} \label{eqn:amx}\\
\hat{y}_n(-) & \equiv H_n \amx{n} \\
&= H_n \Phi_{n-1} \apx{n-1} \\
&= H_n \Phi_{n-1} \lambda_{n-1} \amx{n-1} + H_n \Phi_{n-1} \gamma_{n-1} y_{n-1}, \quad \text{by \ref{eqn:KF_predictor}} \\
&= H_n \Phi_{n-1} \lambda_{n-1} \Phi_{n-2} \apx{n-2} + H_n \Phi_{n-1} \gamma_{n-1} y_{n-1} \\
&\vdots \nonumber \\
& = \beta_0 x_0 + \sum_{k=1}^{n-1} \beta_k y_k, \quad \text{$\beta_k$ is a deterministic coefficent $H, \Phi, \gamma, \lambda$ over $\{n\}$.}\\
\implies \ex{(x_n - \apx{n})\hat{y}_n(-)} &= \ex{(x_n - \apx{n})(x_0^T\beta_0^T  + \sum_{k=1}^{n-1} y_k^T \beta_k^T } \\
&= \ex{(x_n - \apx{n})x_0^T}\beta_0^T  + \sum_{k=1}^{n-1} \ex{(x_n - \apx{n})y_k^T} \beta_k^T  \\
&= \sum_{k=1}^{n-1} \ex{(x_n - \apx{n})y_k^T} \beta_k^T, \quad \text{by \ref{eqn:KF_reln_12}}\\
&=0, \quad \text{by \ref{eqn:KF_reln_2}} \\
\nonumber \\ 
\ex{(x_n - \amx{n})y_i^T} &= \ex{(x_n - \Phi_{n-1} \apx{n-1})y_i^T}, \quad i = 1,\dots,n-1 \\
&=0, \quad \text{by \ref{eqn:KF_reln_2}} \\
\nonumber \\
\ex{\Gamma_n w_n \apx{n}^T} &=\ex{\Gamma_n w_n  \hat{x}_n(-)^T \lambda_n^T} +  \ex{\Gamma_n w_n  y_n^T \gamma_n^T} \quad \text{by \ref{eqn:KF_predictor}}\\
&=\ex{\frac{\Phi_{n}x_{n}}{\norm{x_{n}}} w_n  \hat{x}_n(-)^T \lambda_n^T} +  \ex{\frac{\Phi_{n}x_{n}}{\norm{x_{n}}} w_n  y_n^T \gamma_n^T} \\
&=\ex{\frac{\Phi_{n}x_{n}}{\norm{x_{n}}} w_n  \hat{x}_n(-)^T \lambda_n^T} \quad \text{by \ref{eqn:KF_reln_10}} \\
&=\Phi_{n}\ex{\frac{x_{n}}{\norm{x_{n}}} w_n \apx{n-1}^T} \Phi_{n-1}^T \lambda_n^T \quad \text{by \ref{eqn:amx}} \\
&\vdots \quad \text{repeatedly apply \ref{eqn:KF_predictor}, \ref{eqn:KF_reln_10}, \ref{eqn:amx}} \\
&=0, \quad \text{past terms uncorrelated with future $w_n$} \\
\nonumber \\ 
\ex{\Gamma_n w_n x_n^T} &= \Phi_{n}\ex{\frac{x_{n}}{\norm{x_{n}}} w_n x_n^T} =0 \quad \text{by \ref{eqn:KF_reln_8} } 
\end{align}








\iffalse %%%%%%%%%% THIS BELONGS IN THE APPENDIX, NOT THE MAIN TEXT AS ITS A NUMERICAL DETAIL
\begin{align}
\sigma_k, R_k &\equiv \iota_0 10^{\iota_1} \\
\iota_0 & \sim \mathcal{U}[0, 1]\\
\iota_1 & \sim \mathcal{U}[\{ -\iota_{max}, -\iota_{max} + 1,  \hdots,  \iota_{min}\}]
\end{align}
Scale magnitudes are set by $\iota_1$, a random integer chosen with uniform probability over $\{ -\iota_{max}, \hdots, \iota_{min} \}$ where we set $\iota_{min} = 3, \iota_{max} = 8$  such that $10^{-\iota_{max}}$ is sufficiently high to avoid machine floating point errors from recursive calculations over $> 10^3$ measurements. Uniformly distributed floating points for $\sigma_k, R_k $ in each order of magnitude is set by $\iota_0$. 
\fi
   
% \end{widetext}   
\bibliographystyle{apsrev4-1}
\bibliography{./tex/Biblio_ML}  
\end{document}    



% Machine learning (ML) approaches demonstrate utility in ch dynamical quantum systems and we apply ML algorithms to track and predict a stochastically evolving qubit subject to non Markovian dephasing noise. 
% % Structure for an abstract
% - The abstract needs to open by stating your problem
% - Next you need to highlight what you do in this paper
% - Then details of what you compare
% - A bit more background
% - Then a bit more on detailed findings
% - Then a summary statement.

% Extrinsic interference is routinely faced in systems engineering, and a common solution is to rely on a broad class of filtering techniques to afford stability to intrinsically unstable systems or isolate particular signals from a noisy background. Experimentalists leading the development of a new generation of quantum-enabled technologies similarly encounter time-varying noise in realistic laboratory settings. They face substantial challenges in either suppressing such noise for high-fidelity quantum operations1 or controllably exploiting it in quantum-enhanced sensing2,3,4 or system identification tasks 5,6, due to a lack of efficient, validated approaches to understanding and predicting quantum dynamics in the presence of realistic time-varying noise. In this work we use the theory of quantum control engineering7,8 and experiments with trapped 171Yb+ ions to study the dynamics of controlled quantum systems. Our results provide the first experimental validation of generalized filter-transfer functions casting arbitrary quantum control operations on qubits as noise spectral filters9,10. We demonstrate the utility of these constructs for directly predicting the evolution of a quantum state in a realistic noisy environment as well as for developing novel robust control and sensing protocols. These experiments provide a significant advance in our understanding of the physics underlying controlled quantum dynamics, and unlock new capabilities for the emerging field of quantum systems engineering.

% The wide-ranging adoption of quantum technologies requires practical, high-performance advances in our ability to maintain quantum coherence while facing the challenge of state collapse under measurement. Here we use techniques from control theory and machine learning to predict the future evolution of a qubit’s state; we deploy this information to suppress stochastic, semiclassical decoherence, even when access to measurements is limited. First, we implement a time-division multiplexed approach, interleaving measurement periods with periods of unsupervised but stabilised operation during which qubits are available, for example, in quantum information experiments. Second, we employ predictive feedback during sequential but time delayed measurements to reduce the Dick effect as encountered in passive frequency standards. Both experiments demonstrate significant improvements in qubit-phase stability over ‘traditional’ measurement-based feedback approaches by exploiting time domain correlations in the noise processes. This technique requires no additional hardware and is applicable to all two-level quantum systems where projective measurements are possible.

% Look at the opening - the first line you read is about the data set you're analyzing - is that the KEY message for your readers?  The opening needs to set up your objective and key findings to gain readers attention and quickly communicate essential facts.

% \begin{abstract}
% We consider machine learning (ML) algorithms for predictive control of qubits. Our task is to use a past measurement record to predict future evolution of the qubit state in the absence of an apriori dynamical model for qubit dynamics. We compare predictive performance of ML approaches for qubits subject to non-Markovian dephasing noise. Our numerical investigations quantify the achievable prediction horizon, model robustness, and noise filtering capabilities for two Kalman Filter (KF) variants and a Gaussian Process Regression (GPR) algorithm. We track stochastic qubit dynamics using autoregressive processes and harmonic oscillators. Our study reveals that an autoregressive KF is model-robust compared to an oscillator-based KF for realistic scenarios. We modify an autoregressive KF to use only single shot (0 or 1) qubit data and we achieve a prediction horizon if errors during filtering remain small. Meanwhile, we find a GPR algorithm representing an infinite basis of oscillators enables interpolation but not forward prediction of qubit evolution. Our work numerically characterises real-time predictive estimation for qubits under non-Markovian dephasing for future experimental applications.
% \end{abstract}

% \begin{abstract}
% We apply machine learning (ML) algorithms to track and predict a stochastically evolving qubit state in real-time. In the absence of a dynamical model describing qubit evolution under non-Markovian dephasing noise, we use a past record of time-stamped qubit measurements to predict qubit evolution beyond the data.  In this manuscript, we design and numerically investigate the predictive performance of various ML algorithms. These algorithms proxy stochastic qubit dynamics by using autoregressive processes or a large collection of harmonic oscillators. We compare the achievable prediction horizon, model robustness, and noise filtering capabilities for Kalman Filters (KF) and a Gaussian Process Regression (GPR) algorithm. For KF algorithms, we find an autoregressive KF is model-robust compared to an oscillator-based KF in realistic operating conditions and we modify the former algorithm to enable qubit state predictions using only quantised (0 or 1) data. Meanwhile, a GPR algorithm learns qubit dynamics using an infinite basis of oscillators enabling interpolation but not forward prediction. These numerical investigations advance our understanding of algorithmic design decisions for deploying real-time predictive control strategies in physical experiments.
% \end{abstract}
