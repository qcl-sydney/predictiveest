\section{Set Up}\label{ap:ap_setup}
 We reproduce an alternative derivation to \cite{soare} to justify that a sequence of Ramsey measurements on a qubit interacting with a dephasing noise field will encode information about the underlying phase noise. This will also establish notation for the rest of the Appendices. 
 \\
 \\
Below, we start with the Hamiltonian for a two level system interacting with a classical field. We consider probability amplitude equations of motion for our two level energy eigenstates in the presence of a noisy classical field. We move to an interaction picture twice to see that the derivative of phase noise will affect the expectation value of a projective $\p{z}$ measurement. 
\\
\\
In a particular parameter regime, we recover the final result of \cite{soare}. We argue that performing a series of Ramsey measurements, then, is effectively discretising a true (unknown) continous time stochastic process, $\beta_z \dd $.

\subsection{Two Level System in Free Evolution}

We define a two level system interacting with a magnetic field in Fig. \ref{fig:set_up}, and expand the system Hamiltonian as: 
\\
\begin{align}
\op{H_0} &= E_1\ket{1}\bra{1} + E_2\ket{2}\bra{2} \\
\p{z} &\equiv \ket{2}\bra{2} - \ket{1}\bra{1} \\
\op{\mathcal{I}} & \equiv \ket{1}\bra{1} + \ket{2}\bra{2} \\
\Rightarrow \op{H_0} & = \frac{1}{2} (E_1\ket{1}\bra{1}+ E_2\ket{2}\bra{2}) \\
& + \frac{1}{2} [(E_2 - E_1)\p{z} + E_1 \ket{2}\bra{2} + E_2 \ket{1}\bra{1}]\\
& = \op{\mathcal{I}} \left( \frac{E_1 + E_2}{2} \right) + \p{z}(\frac{E_2 - E_1}{2}) \\
\text{Let: } \quad E_{1,2} &\equiv \mp \frac{1}{2} \hbar \omega_A, \quad \text{to result in:} \\
\op{H_0} &= \frac{1}{2} \hbar \omega_A \p{z}
\end{align}
\\
\\
\begin{figure}[h]
	\centering
	\includegraphics[width=0.5\textwidth]{two_level_system.png}
	\caption[Setup: Diagram of Two Level System]{Diagram of Two Level System}
	\label{fig:set_up}
\end{figure}
\\
\\
\subsection{Classical Field with Noise}

We define our magnetic field with noise identically to \cite{soare}, as:
\begin{align}
\vec{B} &\equiv \Omega(t) \cos(\omega_\mu t + \phi(t)) \op{z} \\
\phi(t) & \equiv \phi_C(t) + \phi_N(t)
\end{align}
with a real control amplitude $\Omega(t)$, carrier frequency $\omega_\mu$, controlled phase $\phi_C(t)$ and stochastic phase noise $\phi_N(t)$.  It is possible to add amplitude noise to the control, i.e. $\Omega(t) \equiv \Omega_C(t) + \Omega_N(t)$, however we set $\Omega_N(t) =0$ at present. 
\\
\\
\subsection{System-Field Interaction}

We define the dipole operator for a spin half particle in a magnetic field, $\op{d}$, as:
\begin{align}
\idn \op{d} \idn & = (\pj{1} + \pj{2})\op{d} (\pj{1} + \pj{2}) \\
& = \pj{1}\op{d}\pj{2} +  \pj{2}\op{d}\pj{1} \\
& = \vec{g}\p{-} + \vec{g}^*\p{+}
\end{align}

In the second line, the absence of a permanent magnetic dipole, the symmetry of states $\ket{1}, \ket{2}$ suggests that $\bra{1}\op{d}\ket{1} = \bra{2}\op{d}\ket{2} \equiv 0$. We note that $\op{d}$ has a direction which is inherited by the complex coupling term, $\vec{g}$ in the third line.

In the Schrodinger picture:
\begin{align}
\op{H}_{AF} &\equiv -\op{d} \cdot \vec{B} \\
\kappa & \equiv \vec{g} \cdot \op{z} = \bra{1} \op{d} \cdot \op{z} \ket{2} \\
\Rightarrow \op{H}_{AF} & = -\kappa \Omega(t)  \cos(\omega_\mu t + \phi(t)) \p{-} - \kappa^* \Omega(t)  \cos(\omega_\mu t + \phi(t)) \p{+}
\end{align} 
 
In the Heisenberg picture, the Heisenberg equations of motion for $\p{\pm}(t)_H \propto e^{\pm i \omega_A t} \p{\pm}(0)$, and we implement the rotating wave approximation by throwing away terms far detuned from our two level system. This is at the same level of accuracy as  implementing a two-level system approximation, and corresponds to throwing away the first line below:

\begin{align}
\op{H}_{AF}^H &= \frac{\Omega\dd}{2} \kappa^* e^{i(\omega_A + \omega_\mu)t}e^{i\phi\dd}\p{+}(0) + h.c. \quad \text{(drop in H-picture)}\\
&+ \frac{\Omega\dd}{2} \kappa^* e^{i(\omega_A - \omega_\mu )t}e^{-i\phi\dd}\p{+}(0) + h.c. \quad \text{(keep in H-picture)} \\
\Rightarrow  \op{H}_{AF}^{RWA, S} &= \frac{\Omega\dd}{2} \kappa^*
 e^{-i( \omega_\mu t + \phi\dd}\p{+} + \frac{\Omega\dd}{2} \kappa e^{i( \omega_\mu t + \phi\dd)}\p{-} \quad \text{(S-picture)}
\end{align}

Then equations of motion for probability amplitudes of our two state system, given arbitrary $\ket{\psi} = c_1 \ket{1} + c_2\ket{2} $ become:
\begin{align}
\dr{\pjs{1}{\psi}} &\equiv \dot{c_1} = \frac{-i}{\hbar}\bra{1} \op{H}_0 + \op{H}_{AF}^{RWA}
  \ket{\psi} = \frac{i\omega_A }{2}c_1 - \frac{i}{\hbar}Z_0\dd e^{i\omega_\mu t}c_2\\
\dr{\pjs{2}{\psi}} &\equiv \dot{c_2} = \frac{-i}{\hbar}\bra{2} \op{H}_0 + \op{H}_{AF}^{RWA}  \ket{\psi} = -\frac{i\omega_A }{2}c_2 - \frac{i}{\hbar}Z_0^*\dd e^{-i\omega_\mu t}c_1\\
\text{with} \quad Z_0\dd &\equiv \frac{\Omega\dd\kappa}{2}e^{i\phi\dd}.
\end{align}
\\
\\
\subsection{First Interaction Picture}

Define transformation as:
\begin{align}
\tilde{c}_1 &= c_1e^{i\lambda t} \\
\tilde{c}_2 &= c_2e^{-i\lambda t}
\end{align}
The equations of motion transform as follows:
\begin{align}
\dot{\tilde{c}}_1 & = \dot{c}_1 e^{i \lambda t} + i \lambda \dot{\tilde{c}}_1 \\
& =  i(\frac{\omega_A}{2} + \lambda)\tilde{c}_1 - \frac{i}{\hbar}Z_0\dd e^{i(\omega_\mu +2\lambda)t}\tilde{c}_2\\
& \nonumber \\
\dot{\tilde{c}}_2 & = \dot{c}_2 e^{-i \lambda t} - i \lambda \dot{\tilde{c}}_2 \\
& =  -i(\frac{\omega_A}{2} + \lambda)\tilde{c}_2- \frac{i}{\hbar}Z_0^*\dd e^{-i(\omega_\mu +2\lambda)t}\tilde{c}_1
\end{align}
We set $2\lambda \equiv -\omega_\mu$ and $\Delta \omega \equiv \omega_A -\omega_\mu$ to remove the time dependence due to the carrier, resulting in:
\begin{align}
\dot{\tilde{c}}_1 & =  \frac{i\Delta\omega}{2}\tilde{c}_1 - \frac{i}{\hbar}Z_0\dd \tilde{c}_2\\
& \nonumber \\
\dot{\tilde{c}}_2 & =  -\frac{i\Delta \omega }{2}\tilde{c}_2- \frac{i}{\hbar}Z_0^*\dd \tilde{c}_1\\
 \Rightarrow \op{H}_{\omega_\mu}^{I} &\equiv \frac{\hbar \Delta\omega}{2}\p{z} + Z_0\dd\p{-} + Z_0^*\dd\p{+}.
\end{align}
In the last line, $\op{H}_{\omega_\mu}^{I}$ is the effective interaction picture Hamiltonian acting on the transformed state $\ket{\psi} = \tilde{c}_1\ket{1} + \tilde{c}_2\ket{2}$ with transformed field amplitudes $Z_0\dd$.
\\
\\
\subsection{Second Interaction Picture}

Define transformation as:
\begin{align}
\alpha_1 &= \tilde{c}_1e^{i\lambda\dd} \\
\alpha_2 &= \tilde{c}_2e^{-i\lambda\dd}
\end{align}
The equations of motion transform as follows:
\begin{align}
\dot{\alpha}_1 & = \dot{\tilde{c}}_1 e^{i \lambda\dd} + i \dot{\lambda}\dd \alpha_1 \\
& =  i(\frac{\Delta \omega}{2} + \dot{\lambda}\dd)\alpha_1 - \frac{i}{\hbar}Z_0\dd e^{i2\lambda\dd}\alpha_2\\
& \nonumber \\
\dot{\alpha}_2 & = \dot{\tilde{c}}_2 e^{-i \lambda\dd} - i \dot{\lambda}\dd\dot{\alpha}_2 \\
& =  -i(\frac{\Delta \omega}{2} + \dot{\lambda}\dd)\alpha_2- \frac{i}{\hbar}Z_0^*\dd e^{-i2\lambda\dd}\alpha_1
\end{align}
We substitute $Z_0\dd$ and set $2\lambda\dd \equiv -\phi_N\dd$  as in \cite{soare}, resulting in:
\begin{align}
\dot{\alpha}_1 & =  i\frac{\Delta \omega - \dot{\phi}\dd}{2}\alpha_1 - \frac{i}{2\hbar}(\Omega\dd \kappa e^{i\phi_C\dd}) \alpha_2\\
& \nonumber \\
\dot{\alpha}_2 & =  -i\frac{\Delta \omega - \dot{\phi}\dd}{2}\alpha_2- \frac{i}{2\hbar}(\Omega\dd \kappa^* e^{-i\phi_C\dd})\alpha_1\\
&\nonumber \\
\Rightarrow \op{H}^{I}_{\omega_\mu,\phi_N\dd} &\equiv \frac{\hbar}{2}(\Delta\omega - \dot{\phi}_N\dd)\p{z} + \frac{\Omega\dd}{2} (\kappa e^{i\phi_C\dd}\p{-} + \kappa^* e^{-i\phi_C\dd}\p{+}) \label{eqn:Hi}
\end{align}
In the last line, $\op{H}^{I}_{\omega_\mu,\phi_N\dd}$ is the effective interaction picture Hamiltonian acting on the transformed state $\ket{\psi} = \alpha_1\ket{1} + \alpha_2\ket{2}$. For the case where we set $\kappa^* = \kappa \equiv 1$ (real coupling constant), we recover the effective interaction picture Hamiltonian in \cite{soare}:

\begin{align}
\op{H}^{I}_{\omega_\mu,\phi_N\dd} & = \frac{\hbar}{2}(\Delta\omega - \dot{\phi}_N\dd)\p{z} + \frac{\Omega\dd}{2} (e^{i\phi_C\dd}\frac{\p{x} - i\p{y}}{2} + e^{-i\phi_C\dd}\frac{\p{x} + i\p{y}}{2})\\
& = \frac{\hbar}{2}(\Delta\omega - \dot{\phi}_N\dd)\p{z} + \frac{\Omega\dd}{4} (\p{x} (e^{i\phi_C\dd}\ +  e^{-i\phi_C\dd})- i\p{y} ( e^{i\phi_C\dd} - e^{-i\phi_C\dd} ))\\
& = \frac{\hbar}{2}(\Delta\omega - \dot{\phi}_N\dd)\p{z} + \frac{\Omega\dd}{2} (\p{x} \cos(\phi_C\dd) + \p{y} \sin(\phi_C\dd)) \label{eqn:Hi_kappareal}
\end{align}
\\
\\
Without any changes to the approach above, we may also expand $\Delta \omega \equiv \omega_A -\omega_\mu -\delta_N$, where $\delta_N$, is a perfectly correlated but unknown experimental error in our detuning for any single run. Although it is impossible to transform to this interaction picture (phase noise is unknown), both \ref{eqn:Hi} or \ref{eqn:Hi_kappareal} confirm that our Ramsey measurements encode the derivatives of the phase noise field, and that measurements can also encode perfectly correlated but unknown detuning errors.
\\
\\
In Appendix \ref{sec:ap_randomprocess}, we justify we can learn noise correlations for an appropriate class of stochastic processes ($\phi_N\dd$); however, resolving unknown but perfectly correlated detuning errors ($\delta_N$) will be beyond the scope of our techniques. To this end, we set $\Delta \omega \equiv 0$.
\\
\\
\subsection{Ramsey Measurement Protocol} \label{sec:ap_setup:subsec:RamseyMeasurement}

Consider the state on the equator of the Bloch sphere evolving under $\op{H}^{I}_{\omega_\mu,\phi_N\dd}$ during a Ramsey experiment for duration $\tau$. Here, the initial control pulse with noisy phase has now been turned off ($ \Omega \dd = 0)$, but phase noise jitter appears as a constant detuning $ \beta_t \equiv  \dot{\phi}_N\dd |_t$ over short Ramsey wait times, $\tau$. (Qn: or does the noisy field stay on?)

\begin{align}
\op{H}^{I}_{\omega_\mu,\phi_N\dd} & = -\frac{\hbar}{2}\beta_t\p{z}
\end{align}

In the semiclassical approximation, this Hamiltonian commutes with itself at different $t$, and hence we can write the time evolution operator during the Ramsey wait time as:

\begin{align}
U(t_0, t_0 + \tau) & = \exp^{\frac{-i}{\hbar} \int_{t_0}^{t_0 + \tau} \op{H}^{I}_{\omega_\mu,\phi_N\dd} dt} \\
& = \exp^{\frac{i}{2} \beta_t \tau \p{z}}
\end{align}
We impose a condition that $\beta_t \tau < \pi$ such that accumulated phase over $\tau$ can be inferred from a projective measurement on the $\p{z}$ axis. 
\\
\\
Consider that the initial $\pi/2$ control pulse perfectly delivers the state vector to the equatorial plane of the Bloch sphere, for example, the  $\ket{\p{x} +}$ state and is allowed to evolve under $U(t_0, t_0 + \tau)$. Then, the probability of measuring in $\ket{\p{x} \pm}$ is:
\begin{align}
|\bra{\p{x} \pm} U(t_0, t_0 + \tau) \ket{\p{x} +}|^2= \begin{cases} \cos(\frac{\beta_t \tau}{2})^2 \quad \text{for $ \ket{\p{x} +}$} \\   \sin(\frac{\beta_t \tau}{2})^2  \quad \text{for $ \ket{\p{x} -}$} \end{cases}
\end{align}
The second $\pi/2$ control pulse rotates the state vector such that a measurement in $\p{z}$ basis is possible, and the probabilities corresspond to $\ket{\p{z} \pm}$
respectively. This describes the non linear measurement action on phase noise jitter, $\beta_t$, which discretises the continous time process $\dot{\phi}_N\dd$, at time $t$, for a number of $n= 0, 1, ..., N$ equally spaced measurements with $t = n \Delta t, \Delta t >> \tau$.  The properties of phase noise as a random process are outlined in \cref{sec:ap_randomprocess}.
