\section{Body}

Objective
\begin{itemize}
\item track a slowly drifting stochastic phase of a classical control field interacting with a qubit by learning noise correlations encoded within a sequence of projective measurements 
\item  predict noise evolution beyond the measurment record
\item maximise prediction horizon through choice of machine learning algorithm to enable future control strategies, e.g. interleaving periods of measurement with periods of unsupervised control. 
\end{itemize}

Unique challenge 
\begin{itemize}
\item non linear, quantised measurement model
\item absence of any apriori information about true noise dynamics 
\item true stochastic process being tracked is non-Markovian
\end{itemize}

Approach
\begin{itemize}
\item develop non linear, quantised measurement model but implement analysis in a simpler regime where measurement model is linear and quantised measurment outcomes are pre-processed
\item reframe the absence of noise dynamics as a design problem where a deterministic transformation `colors' an initial input of white noise
\item additionally corrupt experimental data with measurement noise; and seek model robust prediction tools in realistic noise regimes with complex power spectral densities 
\item test Kalman filtering with a fully nonlinear, quantised measurement model for a true state with simple dynamics
\end{itemize}

Structure of this document....

\subsection{Measurement Model}

Ramsey Measurement: phase noise jitter in the intitial Ramsey control pulse is assumed to manifest as a constant detuning $ \beta_t \equiv  \dot{\phi}_N\dd |_t$ at time $t$ for short wait time $\tau$. The procedure is repeated to obtain the next measurement, at $t + \Delta t$ for $\Delta t >> \tau$, thereby discretising the continous time process, $\dot{\phi}_N\dd$. 
\\
\\
The probability of obtaining a measurement outcome, $d \in [0,1]$ is:

\begin{align}
P(d | \beta_t, \tau, t) = \begin{cases} \cos(\frac{\beta_t \tau}{2})^2 \quad \text{for $d=1$} \\   \sin(\frac{\beta_t \tau}{2})^2  \quad \text{for $ d=0$} \end{cases}
\end{align}

where the state vector is delivered to the equatorial plane of the Bloch sphere evolves freely for short Ramsey wait times, $\tau$, before it is rotated via a second control pulse and a projective measurement is taken with respect to the $\p{z}$ axis.

\subsubsection{Non Linear, Quantised Measurement Model}

Bayesian analysis optimal for non linear measurement models but cannot be adopted 
\begin{itemize}
\item $P(d | \beta_t, \tau, t)$ defines the Bayes likelihood for single shot
\item $P(\beta_t| d, \tau, t)  \propto P(d | \beta_t, \tau, t) P(\beta_t| \tau, t)$ holds for each time step $t$, but cannot assume Markovianity for propagating to $t+1$;  
\end{itemize}

Consider $P(\beta_t| d, \tau, t)$ an abstract discrete time signal in state space. Then non linear measurement is captured in $ h(\beta_t)$:
\begin{align}
\beta_t &= \Phi_{t-1} \beta_{t-1} + \Gamma_{t-1} w_{t-1} \\
y_t &= h(\beta_t) + v_t 
\end{align}

Just as sampling holds in discrete time, we quantise the amplitude as:

\begin{align}
z_t &= \mathcal{Q}(y_t) = \mathcal{Q}(h(\beta_t) + v_t)
\end{align}

In simulations, the quantiser $\mathcal{Q}$ is drawing from a binomial distribution where the bias of coin flip at $t$ is given by a true (engineered) $P(d | \beta_t, \tau, t)$. In experiment, $\mathcal{Q}$ is a naturally quantised physical sensor, namely, a qubit.

\subsubsection{Approximately Linear Measurement Model}
Assume that experimental data is pre-processed such that the measurement record is the a set of estimates of $\{ \hat{\beta}_t \}$, not single shot outcomes. Achieved in one of two ways:

\begin{itemize}
\item Standard technqiue:  average single shots over many parallel runs for different wait times to obtain $\hat{P}(d | \beta_t, \tau, t)$ vs. $\tau$. One may deduce $\beta_t$ from a Fourier transform of $P(d | \beta_t, \tau, t) $ vs. $ \tau$; or condense all pre-processing into Bayesian treatment of single shots as in CITE, with the assumption that $\beta_t$ is constant over $M\tau$ measurements performed, $M\tau << \Delta t$. 
\item Alternative: $M$ single shots are performed with the same wait time, $\tau$, such that drifts in probability are due to $\beta_t$. This binary signal can be decimation filtered to yield $P(d | \beta_t, \tau, t) $ vs. $t/\tau$, where $ \tau \equiv \Delta t$, and the inversion $\beta_t = \frac{1}{\tau} (1- 2\hat{P}(d | \beta_t, \tau, t))$ holds as long as $\beta_t \tau < \pi $. 
\end{itemize}

FIG: Single Shot Outcomes to Decimation Filtering under Linear Measurement Models

\subsection{Stochastic Dynamics with Linear Measurement}
In the absence of a theoretical dynamical model for the evolution of $\beta_t$:
\begin{itemize} 
\item impose properties on the phase noise field and its derivative
\item under these properties, reconstruction of phase noise is enabled in the mean square limit by spectral decomposition using harmonic sums; and by Wold's decomposition, via autoregressive processes of finite order.
\item purpose of numerical analysis is to test whether our mean square approximate reconstructions enable experimentally sensible state tracking and prediction in the time domain
\end{itemize}

The approximate representations of covariance stationary processes informs the structure of learning algorithms. Hence, we cover each reconstruction and algorithmic performance in the sections below:

\subsubsection{Least Squares and Autoregressive Kalman Filter}

\begin{itemize}
\item Autoregressive (AR) processes of finite order: define time domain structure and power spectral density
\item LS Filter: for one step ahead, this is an AR process and AR coefficients solved via with gradient descent
\item AKF: recast AR coefficients as Kalman dynamical model to increase measurement noise filtering capabilities 
\begin{itemize}
\item List AKF state space dynamical and measurement model 
\end{itemize}
\end{itemize}

FIG: Predictive performance with increased measurement noise: LSF vs AKF

\subsubsection{Gaussian Process Regression}
\begin{itemize}
\item Periodic kernel represents trignometric polynomial with infinite terms to reconstruct phase noise process
\item If trigometric polynomial is truncated at finite $J$, then GPR with periodic kernel represents $J$-th order state space model in a classical Kalman filter
\begin{itemize}
\item List GPR predictive equations
\item List GPR Periodic Kernel
\end{itemize}
\item Covariance matrix in predictive equations (above) is $N$ periodic for a  measurement record with $N$ terms: pattern reconstruction is enabled by learning Fourier domain amplitudes via a Periodic Kernel, but time domain predictions do not make sense without active tracking of phase information.
\end{itemize}
FIG: GPR-P Predictions outside the zone of measurement data repeat the intial measurement record.
\begin{itemize}
\item Other kernels: excluded on the basis that their shape in the Fourier domain is narrowband; or difficulty of optimising hyper-parameters for our application 
\end{itemize}
\subsubsection{Liska Kalman Filter with Fixed Basis}
\begin{itemize}
\item Tracks amplitudes and phases of a fixed collection of resonators used to probe the noise 
\item Enables full knowledge of phase noise if perfect projection is possible; but predictive power deteriorates rapidly for realistic noise scenarios relative to the Autoregressive Kalman Filter.
\end{itemize} 
FIG: Predictive performance of AKF vs LKFFB for realistic noise scenarios

\subsection{Optimisation}
\begin{itemize}
\item Bayes Risk calculated during state estimation and prediction relative to true engineered noise
\item Optimisation problem breaks down for LKFFB as realistic noise scenarios are considered; but remains robust for AKF
\item Cost function contains pathologies for standard gradient and simplex algorithms - a simple random sampling is used here; but diagnostics suggest coordinate ascent and particle swarm techniques could be promising candidates for future research
\end{itemize}
\subsection{Stochastic Dynamics with Non Linear, Quantised Measurements}

We update the Kalman filtering framwork for quantised sensor information, where a one bit classical qunatiser is considered. Inputs into the Kalman filter are 0 or 1  outcomes from a binomial distribution with a stochastically drifitng bias. The interpretation of measurement noise (previously) is now additive white quantisation noise. We track a simple sum of sinusoids with random phases where perfect projection is possible for LKFFB, and compare RMS performance from single shots with equivalent performance from tracking pre-processed data.

\subsubsection{Autoregressive Kalman Filter}
NOT DONE 
\subsubsection{Liska Kalman Filter with Fixed Basis}
NOT DONE
\\
\\
FIG: NOT DONE. However, early demos suggest that errors from non linear measurement model make it sub-optimal to use Kalman filtering directly on single shot outcomes; and that  LKFFB fails more than AKF. 
\\
\\
Particle filtering may work better; but an implementation of particle fitlering for non Markovian stochastic dynamics (and non linear, quantised  measurments) are out of scope of this paper. 

\section{Appendices}
\subsection{Physical Set Up and Measurement Model}
\subsection{Stationary Stochastic Processes }
\subsection{Linear Predictors for Covariance Stationary Processes}
\subsection{Kernel Selection in Gaussian Process Regression}
\subsection{Optimisation Procedure for Tuning Filter Parameters}

\section{References}