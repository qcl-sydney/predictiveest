
\section{Figures and Captions}

\begin{figure}[h!]
    \includegraphics[scale=1]{Predive_control_Fig_overview_17_one} 
    \caption{ \label{fig:main:Predive_control_Fig_overview_17_one} Physical Setting: In (a), we define the Hamiltonian for stochastic qubit dynamics under arbitrary environmental dephasing using a covariance stationary, non Markovian detuning $\delta \omega(t)$ with an arbitrary power spectral density. A sequence of Ramsey experiments with fixed wait time $\tau$ yield single shot outcomes $\{ d_n \}$, with likelihood $P(d_n|\state_n, \tau,n)$,  conditioned on a mean-square ergodic sequence of true phases, $\{ f_n\}$, with $n \in [-N_T, 0]$ indexing time during data collection. Our objective is to maximise forward time $n \in [0, N_P]$ for which an algorithm uses measurement data to predict a future qubit state and incurs a lower Bayes prediction risk relative to predicting the mean value of the dephasing noise [dark gray shaded]. In (b), single shot outcomes are processed to yield noisy accumulated phase estimates, $\{ y_n\}$, corrupted by measurement noise $\{v_n\}$. The choice of $\{d_n\}$ or $\{y_n\}$ as datasets for predictive estimation corressponds to non-linear or linear measurement records in (b) and (c).}
\end{figure} 

\clearpage \newpage

\begin{widetext}
\begin{figure*}
    \includegraphics[scale=1]{Predive_control_Fig_overview_17_two} 
    \caption{ \label{fig:main:Predive_control_Fig_overview_17_two} Predictive Methodologies: (a) In GPR, a prior  distribution over true phase sequences $P(\state), \state \equiv \{ \state_n \}$ is constrained by a linear Bayesian likelihood of observed data, $\{ y_n\}$. The prior encodes dephasing noise correlations by defining covariance relations for the $i, j$-th time points using  $\Sigma_\state^{i, j}$ and optimising over its free parameters during training. The moments of the resulting predictive distribution $P(\state^*|y)$ are interpreted as pointwise predictions and their pointwise uncertainties when evaluated for $n>0$.  (b) In KF, the Kalman state and its variance correspond to moments of a Gaussian distribution propagated in time via $\Phi$, and filtered via the Kalman gain, $\gamma$ at timestep $n$. The design of $\Phi$ deterministically colors a white noise process $\{w_n \}$ and `encodes' an apriori structure for learning dephasing noise correlations. Prediction proceeds by propagating forwards with $\gamma_n=0, n>0$. Additive white Gaussian measurement noise $v_n$ corrupts all measurement records.}
\end{figure*}
\end{widetext}

\clearpage \newpage 

\begin{figure} [h]
    \includegraphics[scale=1]{Predive_control_Fig_overview_17_three}
    \caption{\label{Predive_control_Fig_overview_17_three} Apriori Structure for $\Phi$ : All Kalman $\Phi$ variants are mean square approximations to covariance stationary, mean square ergodic $f$. AKF and QKF define $\Phi$ as a weight sum of $q$ past measurements driven by process noise, $w$ [top]. Kalman $\Phi$ for LKFFB represents a collection of $J$ osscilators driven by process noise, $w$, where frequency of oscillators must span dephasing noise bandwidth. The instantaneous amplitude and phase of each basis oscillator can be derived from the Kalman state estimate $x_{j, n}$ at any $n$. Predictions combine learned amplitudes and phases for each basis oscillator and sum contributions over all $J$ [bottom].}
\end{figure}

\clearpage \newpage


\begin{figure}
    \includegraphics[scale=1.]{fig_data_gpr}. 
    \caption{\label{fig:main:fig_data_gpr} In (a)-(d), prediction points $\mu_{\state^*|y}$ [purple] are plotted against time steps, $n$. We plot the true phase sequence,  $f$, [black] and  $f$ at the begining of the run [red dotted]. Predictions are generated in a single run by a trained GPR model with a periodic kernel corressponding to a Fourier domain basis comb spacing, $\omega_0^B$. Data collection of $N_T$ measurements [not shown] ceases at $n=0$. For simplicity, the true $f$ is a deterministic sine with frequency, $\omega_0$. (a) Perfection projection is possible $\omega_0 / \omega_0^B \in Z$ natural numbers, $\omega_0 = 3$ Hz. Kernel resolution is exactly the longest time domain correlation in dataset, $2 \pi / \omega_0^B \equiv \Delta t N_T \implies \kappa = 0$.   (b) Imperfect projection, with $\omega_0 / \omega_0^B \notin Z$, $\omega_0 / 2 \pi = 3 \frac{1}{3}$ Hz, $\kappa=0$. (c) We increase kernel resolution to be arbitrarily high, $\kappa >> 0 $, such that $\omega_0 / \omega_0^B >> 0 \notin Z $ for original $ \omega_0 / 2 \pi = 3$ Hz. (d) We test (b) and (c) for $\kappa >>0$, $ \omega_0 / \omega_0^B \notin Z$, $\omega_0 / 2 \pi = 3 \frac{1}{3}$ Hz. For all (a)-(d), $N_T = 2000, N_P = 150$ steps, $\Delta t = 0.001s$ and applied measurement noise level $1\%$.} 
\end{figure}

\clearpage \newpage

\begin{figure}
    \includegraphics[scale=1.0]{fig_data_all}
    \caption{\label{fig:main:fig_data_all} We plot state predictions against time steps $n > -50$ obtained from trained AKF, LKFFB and LSF algorithms. We plot true $f$ [black] and measurement data [grey dots], where measurements for $n \in [-N_T, -50]$ are omitted [(left)]. A single run contributes to Bayes prediction risk over an ensemble of 50 runs normalised against predicting the mean, $\mu_\state$, of dephasing noise [right]. A normalised risk $<1$ for $n > 0$ defines a desirable forward prediction horizon. A single run phase sequence $f$ is drawn from a flat top spectrum with $J$ true Fourier components spaced $\omega_0$ apart and uniformly randomised phases $\in [0, 2\pi]$. A trained LKFFB is implemented with comb spacing $\omega_0^B / 2\pi = 0.5$ Hz and $J^B =100$ oscillators; while trained AKF / LSF models corresspond to high $q \approx 100$. Relative to LKFFB,  (a) and (b) corresspond to perfect projection $\omega_0 / \omega_0^B  \in Z $ for $J= 40, \omega_0 / 2\pi = 0.5$ Hz. In (c) and (d), we simulate realistic noise with $\omega_0 / \omega_0^B  \notin Z$, $J = 45000$, $\omega_0 / 2\pi = \frac{8}{9} \times 10^{-3}$ Hz such that $>500$ number of true components fall between adjacent LKFFB oscillators. For (a)-(d), $N_T = 2000, N_P = 100$ steps, $\Delta t = 0.001s$ such that we fulfill $r_{Nqy} >> 2$, $N_T / \Delta t < \omega_0/2\pi$. Measurement noise level is $10\%$.}
\end{figure} 

\begin{figure}
    \includegraphics[scale=1.0]{fig_data_specrecon}
    \caption{\label{fig:main:fig_data_specrecon} We compare the true power spectrum for $f$ with derived spectral estimates from LKFFB and AKF. From (a)-(d), we vary true $f$ cutoff relative to an apriori noise bandwidth assumption $f_B$ such that $\omega_0 / 2\pi = 0.5$ Hz, $J = 20, 40, 80, 200$. For LKFFB, we use learned amplitude information from a single run ($\propto ||x^j_n||^2 $) with $\omega_0^B / 2\pi = 0.497$ Hz for $j \in J^B = 100$ oscillators. For AKF, we plot [EQN REF]\cref{eqn:main:ap_ssp_ar_spectden} using optimally trained $\{\phi_{q' \leq q}\}$ and $\sigma^2$, with order $q \approx 100$. The zeroth Fourier component and its estimates are omitted to allow for log scaling; and $N_T = 2000, N_P = 50$ steps, $\Delta t = 0.001s, r_{Nqy}=20$ and measurement noise level is $1\%$.} 
\end{figure} 

\clearpage \newpage

\begin{widetext}
    \begin{figure*} 
    \includegraphics[scale=1.0]{figure_lkffb_path}
    \caption{\label{fig:main:figure_lkffb_path} 
    We compare LKFFB and AKF performance when a true phase sequence $f$ is generated from a flat top spectrum in (a)-(d) by varying  $\omega_0 / 2\pi = 0.5, 0.499, \frac{8}{9} \times 10^{-3}, \frac{8}{9} \times 10^{-3}$ Hz and $J = 80, 80, 45000, 80000$ respectively. For (a)-(d), we depict normalised Bayes prediction risk for LKFFB, AKF, and LSF against time steps $n>0$. For LKFFB, these regimes corresspond to perfect learning in (a); imperfect projection on basis in (b); finite computational Fourier resolution in (c); and a relaxed bandwidth assumption ($f_B < \omega_0 / 2\pi$) in (d). In the panels (e)-(l), we depict optimisation of Kalman noise parameters ($\sigma^2, R$) for LKFFB [top row] and AKF [bottom row] for the four regimes in (a)-(d). Low loss regions represent risk values $< 10\%$ of the median risk incurred during Kalman hyperparameter optimisation for 75 trials of of randomised ($\sigma^2, R$) pairs. Optimal ($\sigma^*, R^*$) minimise state estimation risk. For each trial, a risk point is an expectation over 50 runs of true $f$ and noisy datasets during state estimation ($n \in  [-N_{SE}, 0]$) or prediction ($n \in  [0, N_{PR}]$). We choose $ N_{PR}=N_{SE}=50$ such that the shape of total loss over time steps form sensible optimsation problems over the range of numerical experiments in this paper. A scan of $N_{PR}, N_{SE}$ values do not appear to simplify our Kalman optimisation problem. We plot optimisation results for LKFFB in (e)-(h) and AKF in (i)-(l). A KF filter is `tuned' if optimal ($\sigma^*, R^*$) lies in the overlap of low loss regions for state estimation and prediction. This condition is violated in (h). KF algorithms are set up with $q = 100$ for AKF; $J^B = 100, \omega_0^B / 2\pi = 0.5$ Hz for LKFFB, with $N_T = 2000, N_P = 100$ steps, $\Delta t = 0.001s, r_{Nqy}=20$ and applied measurement noise level $1\%$.}
    \end{figure*} 
\end{widetext}

\clearpage \newpage

\begin{figure}
    \includegraphics[scale=1.]{fig_data_akfvlsf}
    \caption{\label{fig:main:fig_data_akfvlsf} (a) We plot the ratio of normalised Bayes prediction risk from AKF to LSF against time steps $n>0$.  AKF and LSF share identical $\{ \phi_q \}$ and  a value below $<1$ indicates AKF outperforms LSF. In (i)-(iv), applied measurement noise level is increased from $0.1 - 25 \%$, where the noise level is defined as standard deviation of additive Gaussian measurement noise relative to the sample standard deviation of random variables one realisation of true $f$. (b) We plot normalised Bayes Risk against time steps $n>0$ for AKF and LKFFB corressponding to cases (i) -(iv) and confirm a desirable forward prediction horizon underpins ratios in (a). True $f$ is drawn from a flat top spectrum with $\omega_0 / 2\pi = \frac{8}{9} \times 10^{-3}$ Hz, $J = 45000$, $N_T = 2000, N_P = 100$ steps, $\Delta t = 0.001s, r_{Nqy}=20$ such that \cref{fig:main:figure_lkffb_path}(c) corressponds to case (ii) in this figure. }
\end{figure}

\begin{figure}[h!]
    \includegraphics[scale=1.]{fig_data_qkf}
    \caption{\label{fig:main:fig_data_qkf2} We plot Bayes prediction risk for QKF against time steps $n>0$. In (a)-(b), we vary true $f$ cutoff relative to an apriori noise bandwidth assumption such that $J f_0 / f_B = 0.2, 0.4, 0.6, 0.8$ for an initially generated true $f$ in \cref{fig:main:fig_data_specrecon} with $\omega_0/ 2\pi = 0.497 $ Hz, $J = 20, 40, 60, 80$. Measurement noise is incurred on $f$ at $1 \%$ level for the linear measurement record and on $z$ at $1\%$ level corressponding to the non linear measurement record. In (a), we obtain $\{\phi_{q' \leq q}\}, q=100$ coefficients from AKF/LSF acting on a linear measurement record generated from true $f$. We re-generate a new truth, $f'$, from an autoregressive process by setting $\{\phi_{q'\leq q}\}, q=100$ as true coefficents and by defining a known, true $\sigma$. We generate quantised measurements from $f'$ and data is corrupted by measurement noise of a true, known strength $R$. Hence, QKF in (a) incorporates true dynamics and noise parameters $\{\{\phi_{q' \leq q} \}, \sigma, R\}$ but acts on single shot qubit measurements. In (b), we use $\{\phi_{q' \leq q} \}, q=100$ coefficients from (a) but we generate quantised measurements from the original, true $f$. We auto-tune QKF noise design parameters in a focused region ($\sigma_{AKF}^* \leq \sigma_{QKF}$, $R_{AKF}^* \leq R_{QKF}$) . For (a)-(b), forward prediction horizons are shown with $N_T = 2000, N_P = 50$ steps, $\Delta t = 0.001s, r_{Nqy}>> 2$.}
\end{figure}

\clearpage \newpage 



\clearpage \newpage
cite something for latex build \cite{mavadia2017} 