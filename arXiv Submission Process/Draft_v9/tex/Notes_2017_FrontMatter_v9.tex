
\title{Machine learning for predictive estimation of qubit dynamics subject to dephasing}

\author{Riddhi Swaroop Gupta} 
\email{riddhi.sw@gmail.com}
\affiliation{ARC Centre of Excellence for Engineered Quantum Systems, School of Physics, The University of Sydney, New South Wales 2006, Australia}

\author{Michael J. Biercuk}
\affiliation{ARC Centre of Excellence for Engineered Quantum Systems, School of Physics, The University of Sydney, New South Wales 2006, Australia}

\begin{abstract}
Decoherence remains a major challenge in quantum computing hardware and a variety of physical-layer controls provide opportunities to mitigate the impact of this phenomenon through feedback and feedforward control. In this work, we compare a variety of machine learning algorithms derived from diverse fields for the task of state estimation (retrodiction) and forward prediction of future qubit state evolution for a single qubit subject to classical, non-Markovian dephasing. Our approaches involve the construction of a dynamical model capturing qubit dynamics via autoregressive or Fourier-type protocols using only a historical record of projective measurements.  A detailed comparison of achievable prediction horizons, model robustness, and measurement-noise-filtering capabilities for Kalman Filters (KF) and Gaussian Process Regression (GPR) algorithms is provided. We demonstrate superior performance from the autoregressive KF relative to Fourier-based KF approaches and focus on the role of filter optimization in achieving suitable performance. Finally, we examine several realizations of GPR using different kernels and discover that these approaches are generally not suitable for forward prediction.  We highlight the underlying failure mechanism in this application and identify ways in which the output of the algorithm may be misidentified numerical artefacts.
\end{abstract}

\maketitle
