
\mbox{}

% \section{Time Series Random Processes, Noise \& Probability Distributions}
\nomenclature{$\delta \omega(t)$}{Environmental dephasing} 
\nomenclature{$\delta \bar{\omega}_n$}{Environmental dephasing under a fast measurement action relative to slowly drifting dephasing field such that dephasing is constant value at $n = t/\Delta t$ for timescales on order of $\tau$} 
\nomenclature{$\state$}{True state i.e. a true sequence of stochastic phases governing qubit dynamics under environmental dephasing noise}
\nomenclature{$\state'$}{An approximation to the true covariance stationary $\state$ using autoregressive methods (rather than a periodic random signal) to generate realisations of stochastic phases under dephasing}
\nomenclature{$Pr(x), Pr_x(\mu_x, \Sigma_x)$}{Probability distribution of $x$, with mean $\mu_x$ and variance $\Sigma_x$} 
\nomenclature{$\hat{Pr}(d_n | \state_n, \tau, n\Delta t)$}{Likelihood of a single shot qubit outcome given by Born's rule }
\nomenclature{$w$}{Zero mean additive Gaussian white process noise in a Kalman framework}
\nomenclature{$\sigma^2$ $(\sigma^*)$}{True (optimised) process noise covariance strength in Kalman filtering; process variance strength in GPR}
\nomenclature{$v$}{Zero mean additive Gaussian white measurement noise}
\nomenclature{$R$ $(R^*)$ }{True (optimised) measurement noise covariance strength in GPR and Kalman Filtering}
\nomenclature{$N.L.$}{Applied measurement noise level for simulated datasets, defined as the ratio of true measurement noise strength $R$ and variance of the true $\state$ under dephasing}
\nomenclature{$\mathcal{B}$}{Binomial distribution}
\nomenclature{$p_{\mathcal{B}}$}{Binomial distribution parameter for the bias of a weighted coin toss}
\nomenclature{$n_{\mathcal{B}}$}{Binomial distribution parameter for number of tosses in a coin toss experiment}
\nomenclature{$\mathcal{U}$}{Uniform distribution}
\nomenclature{$\mathcal{I}$}{Identity operator}
\nomenclature{$a, b$}{Saturation of probability distributions for a random variable with allowed values between $[a,b]$}
\nomenclature{$c$}{Constant scalar variable; often summarises a fixed DC bias in a stochastic process; and/or the limit of a geometric series in the formalism of ARMA processes.}
\nomenclature{$\mu$}{Generic mean of a Gaussian distribution}
\nomenclature{$\Sigma$}{Generic covariance matrix of a Gaussian distribution}
\nomenclature{$(+)$}{A posteriori statistical quantity (output of a Bayesian estimator)}
\nomenclature{$(-)$}{A priori statistical quantity (input of a Bayesian estimator)}
\nomenclature{$g_2(\psi_{j}, n_1; \psi_{j'}, n_2)$}{Joint probability distribution of the random phases, $\psi$, at different times, $n_1, n_2$, and for different spectral components, $j, j'$)}
\nomenclature{$R(\nu)$}{Covariance function for a covariance stationary process, $\state$, where $\nu$ refers to the separation distance between random variables in the process $\state$ }

% \section{Classical and Quantum Operators}
\nomenclature{$\hat{\cdot}$}{Quantum mechanical operator; or denotes estimators based on finite sample data for classical random variables (evident from context)}
\nomenclature{$\mathcal{L}$}{Lag operator for an autoregressive process}
\nomenclature{$\mathcal{Q}$}{Quantiser defined for a Kalman state space framework such that the output is either $0$ or $1$. This corresponds to a statistical description of outcomes given by a likelihood $Pr(d_n | \state_{n}, \tau)$}
\nomenclature{$\op{U}$}{Unitary time evolution quantum operator}
\nomenclature{$\p{x}, \p{y}, \p{z}$}{Pauli operators (Pauli basis)}
\nomenclature{$E$}{Energy level of a qubit corressponding to eigenstates of the qubit system Hamiltonian under a Jayens Cummings model}
\nomenclature{$\mathcal{\op{H}}$}{Total Hamiltonian governing qubit dynamics ($\mathcal{\op{H}}(t)$) based on system ($\mathcal{\op{H}}_0$) and dephasing noise ($\mathcal{\op{H}}_N(t)$) }
\nomenclature{$\omega_{A}$}{Frequency corressponding to the total energy level difference ($\hbar \omega_A$) between two states of the qubit}
\nomenclature{$\ket{\cdot}$}{Quantum state of the qubit in Dirac notation, where $\cdot \equiv \psi(t)$ describes Schrodinger picture state under time evolution expanded over an arbitrary basis; $\cdot \equiv \tilde{\psi}(t)$ describes the interaction picture state under time evolution expanded over an arbitrary basis; and $\cdot \equiv  k \in \{ 0,1\}$ is reserved for Schrodinger picture stationary eigenstates in the $\p{z}$ (measurement) basis.}
\nomenclature{sym}{def}
\nomenclature{sym}{def}

% \section{Experimental Sampling and Data Collection}
\nomenclature{$t$}{Continuous time parameter tracking laboratory wall time}
\nomenclature{$\tau$}{Ramsey wait time}
\nomenclature{$\Delta t$}{Time step between measurements}
\nomenclature{$N_T$}{Number of training points}
\nomenclature{$N_P$}{Number of prediction points beyond the measurement record}
\nomenclature{$N$}{Total number of time steps ($N_T + N_P$)}
\nomenclature{$r$}{Nyquist multiplier}
\nomenclature{$f_{(B)}$}{Bandwidth assumption such that $f_{(S)} \equiv r_{Nqy} f_{(B)}$}
\nomenclature{$f_{(S)}$}{Experimentally controlled sampling rate $f_{(S)} \equiv 1/\Delta t$}
\nomenclature{$\omega_{(S)}$}{Experimentally controlled sampling rate $= 2 \pi f_{(S)} = 2 \pi / \Delta t$}
\nomenclature{$S(\omega)$ ($\hat{S}(\omega)$)}{True (estimated) power spectral density for true $\state$}
\nomenclature{$\omega^{(B)}$}{True noise bandwidth assumption $ \approx \omega_{(S)} / r_{Nqy} $ and equivalently, $\approx \omega_0^{(B)} J^{(B)}$ in LKFFB }

% \section{Algorithm Design}
\nomenclature{$f_0^{(B)}$}{Computational basis  frequency comb spacing (Hz) in LKFFB or GPR (Periodic Kernel)}
\nomenclature{$\omega_0^{(B)}$}{Computational basis  frequency comb spacing (rad) in LKFFB or GPR (Periodic Kernel)}

\nomenclature{$J^{(B)}$}{Computational basis - total number of basis oscillators in LKFFB}
\nomenclature{$h(\cdot)$}{A non linear state space measurement model }
\nomenclature{$H_n$}{Linear state space measurement model or Jacobian of a non-linear measurement model}
\nomenclature{$x$}{State space unobserved true state in Kalman Framework}
\nomenclature{$P$}{True uncertainty of the true Kalman state in the Kalman Framework}
\nomenclature{$\gamma_{n}$}{Kalman gain}
\nomenclature{$\Gamma_{n}$}{Kalman process noise features}
\nomenclature{$\Phi$}{State space dynamical model for the true state in the Kalman Framework}
\nomenclature{$\state^{\ddagger}$}{State estimation and/or predictions from GPR about a true state $\state$, for test points collected in a length $N^{\ddagger}$ vector.}
\nomenclature{$K({}\cdot{}, {}\cdot \cdot{})$}{A Gram Schmidt ($\cdot$) by ($\cdot \cdot$) matrix whose elements are specified by $R(\nu)$  in GPR}
\nomenclature{$\nu$}{The separation distance between any two time steps, $n_1, n_2$. }
\nomenclature{$l$}{Length scale for the periodic kernel}
\nomenclature{$\kappa$}{A deterministic time-step at which discontinuities in GPR can be predicted.}
\nomenclature{$n^*_C$}{[Appendix only] LKFFB: optimal  number of points for training before commencing predictions}

\nomenclature{$A^j_{n} $}{LKFFB - real part of true Kalman sub-state $x^j_n$; equivalently, zero mean serially and mutally uncorrelated random variable for the $j$th oscillator in a expansion of a covariance stationary process as harmonic sums}
\nomenclature{$B^j_{n}$}{LKFFB - imag part of true Kalman sub-state $x^j_n$; equivalently, zero mean serially and mutally uncorrelated random variable for the $j$th oscillator in a expansion of a covariance stationary process as harmonic sums}
\nomenclature{$ \norm{x^j_n}$}{LKFFB - norm of true Kalman sub-state $x^j_n$}
\nomenclature{$\theta^{j}_{n} $}{LKFFB - phase of true Kalman sub-state $x^j_n$}
\nomenclature{$\Theta(j\omega_0 \Delta t$)}{The $j$ state space dynamical model for the $j$ basis oscillator, equivalently, a sub-matrix of the full dynamical model $\Phi$ for LKFFB. }

% \section{Bayes Risk and Performance Analysis}
\nomenclature{$n^*$}{Time step denoting the maximal forward prediction horizon $n^* \in [0, N_P]$ for which an algorithm predicts better relative to predicting the mean behaviour of the qubit under dephasing}
\nomenclature{$I$}{A set of model design parameters (known a priori or optimised during algorithmic tuning)}
\nomenclature{$L_{BR}(n | I)$}{Bayes Risk value at $n$ defined as an expectation value of datasets $\mathcal{D}$ and conditioned on a set of model parameters $I$}
\nomenclature{$\normpr $}{Normalised Bayes Risk at $n$: Risk calculated over datasets $\mathcal{D}$ and normalised against qubit behaviour under mean dephasing}
\nomenclature{$L(I_k)$}{Loss value defined as the sum of Bayes Risk $L_{BR}(n | I_k)$ over a fixed number of time steps ($n_L$) during state estimation ($n_L \equiv N_{SE}$) or prediction ($n_L \equiv N_{PR}$) for $k$-th choice of a set of model parameters, $I_k$.}
\nomenclature{$n_L$}{Loss function parameter for algorithmic tuning, namely, $n_L \in \{ N_{SE}, N_{PR}\}$). This defines the number of time steps which contributes to the total prediction risk value, for a given choice of model parameters, $I$.}
\nomenclature{$N_{PR}$}{Loss function parameter for algorithmic tuning, namely, the number of time steps after $n=0$ which contributes to the total prediction risk value, for a given choice of model parameters, $I$.}
\nomenclature{$N_{SE}$}{Loss function parameter for algorithmic tuning, namely, the number of time steps before $n=0$ which contributes to the total state estimation risk value, for a given choice of model parameters, $I$.}
\nomenclature{$L_0$}{Loss threshold below which a `low loss region' $L(I_k)$ is defined in the space spanned by model parameters in $I$, using an ensemble of experiments with data $\mathcal{D}$}


% \section{Measurement Records} 
\nomenclature{$\{ y_n\}$}{Observations of a true state corrupted by measurement noise (not quantised) }
\nomenclature{$\{ d_n\}$}{Quantised zero or one measurement records}
\nomenclature{$z_n$}{A noiseless (ideal) non linear measurement of the true Kalman state, $h(x)$}
\nomenclature{$\mathcal{D}$}{A collection of noisy datasets corressponding to $M$ total runs of an experiment under different realisations of $\state$}


% \section{Indicies}
\nomenclature{$X$}{[Nomenclature only] Dummy index variable to define sets of \emph{index variables}}
\nomenclature{$\{n: -N_T, \hdots, N_P \}$}{Index representing the number of discrete time steps for wall time $t_n = n\Delta t$}
\nomenclature{$\{n^{\ddagger}_X: X = 1, 2, \hdots, N^{\ddagger} \}$}{Index of test points in GPR, namely, a label for points in time where a GPR state estimate or forward prediction is desired}
\nomenclature{$N^{\ddagger}$}{The length of a vector containing test points in GPR}
\nomenclature{$\{m: 1, 2, \hdots M\}$}{Index representing the number of trials of the same experiment }
\nomenclature{$\{k: 1, 2, \hdots K\}$}{Index representing the number of trials of a choice of model (hyper-parameters) for a given ensemble of experiments }
\nomenclature{$\{j: 1, 2, \hdots J \text{ or } J^{(B)}\}$}{Index representing the number of Fourier components in true dephasing noise ($J$) or in a computational Fourier protocol ($J^{(B)}$)}
\nomenclature{$\{i: 1, 2, \hdots N_P\}$}{Index representing the $i$-th step ahead prediction from $n=0$, where $i=1$ is the step ahead prediction in the Kalman filter}
\nomenclature{$\{q': 1, 2, \hdots q\}$}{Index representing the $q'$-th autoregressive term in an autoregressive model of order $q$}
\nomenclature{$\iota_X$}{A set of quantities generated for the purpose of randomly sampling parameter space associated with Kalman design parameters $(\sigma, R)$, for $\{ X: 0, 1, min, max\}$}
\nomenclature{$\mathcal{Z}$}{The set of natural numbers}

% Amplitude Quantisation
\nomenclature{$\vartheta$}{Number of levels for classical amplitude quantisation}
\nomenclature{$b$}{Number of bits employed for amplitude quantisation}
\nomenclature{$I_{\vartheta,n}(z)$}{Fisher information for an arbitrary true state $z$ ( where information refers to inverse of state variance) parameterised by $n$ (time-steps) and $\vartheta$ (number of amplitude quantisation levels)}
\nomenclature{$\rho$}{Number of levels for classical amplitude quantisation}

% Physical Setting
\nomenclature{$\tilde{g}$}{[Appendix only] Complex scalar representing system-field coupling strength for an atom-field Hamiltonian in the rotating wave approximation}
\nomenclature{$f_0$}{True noise frequency comb spacing (Hz)}
\nomenclature{$\omega_0$}{True noise frequency comb spacing (rad)}
\nomenclature{$J$}{True noise - total number of Fourier components}
\nomenclature{$\eta$}{Parameter to set power spectral density shape, such that $F(j) = j^{\frac{p}{2}-1} $}
\nomenclature{$p$}{Order of a Moving Average, MA($p$), process}
\nomenclature{$\Psi_{p'}$}{Moving average coefficient for the $p'$ term in an MA($p$) process, $p' \leq p$}
\nomenclature{$q$}{Order of a Autoregressive proccess, AR($p$), process}
\nomenclature{$\phi_{q'}$}{Autoregressive coefficient for the $q'$ term in an AR($q$) process, $q' \leq q$}
\nomenclature{$\lambda_{q'}$}{Eigenvalue associated with the $q'$ autoregressive coefficient in an AR($q$) process, where the set of eigenvalues correspond to the dynamical matrix $\Phi$ defined for the AKF}
\nomenclature{$\Lambda_{q'}$}{Inverse operator for the $q'$ root, $(1- \lambda_{q'})$ in an AR($q$) representation}
\nomenclature{$\xi_n$}{A Gaussian white noise sequence in a MA($p$) representation, corressponding to process noise $w_n$ in an AR($q$) representation}
\nomenclature{$F(j)$}{[Appendix only] true $\state$ -  Fourier amplitude for the $j$-th oscillator }
% \nomenclature{$\psi_j$}{[Appendix only] true $\state$ - uniformly distributed random phases }
\nomenclature{$\psi^{j}, \psi $}{Uniformly distributed random phase over one cycle $[0, \pi]$ for true noise engineering}
\nomenclature{$\alpha$}{[Appendix only] true $\state$ - arbitrary real, constant, scaling factor}



% \nomenclature{sym}{def}
% \nomenclature{sym}{def}
% \nomenclature{sym}{def}

\printnomenclature 
 



 